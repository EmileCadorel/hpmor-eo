\usepackage{lettrine}		% Used for the fancy caps at each start of each chapter.
\usepackage{xspace}		% Takes care of spaces after macros
\usepackage{amsmath}		% Provides the align environment, used in chapter 13 for the notes

\usepackage[protrusion=true]{microtype}

\usepackage{fontspec}		% For the many fonts
\usepackage{xunicode}

\usepackage{xstring}

\usepackage{eso-pic,picture}

\usepackage[bookmarks=true,unicode=true,pdfborder={0 0 0},
	pdftitle={Harry Potter kaj la Metodoj de la Racieco},
	pdfauthor={LessWrong}, breaklinks={true},
	pdfkeywords={Harry Potter, racieco},pdfencoding=auto
]{hyperref}

% Logical markup

% These commands should be used to help make the source easy to understand
% and consistently typeset.
% Search for them in the source files to see how to use them.

% Special types of text
\newcommand{\abbrev}[1]{\textsc{\MakeLowercase{#1}}\xspace}

% Common abbreviations
\newcommand{\am}{~a.m.\xspace}
\renewcommand{\pm}{~p.m.\xspace}
\newcommand{\SPHEW}{\abbrev{SPHEW}}
% Tone of voice
\newcommand{\shout}[1]{\textsc{#1}}
\newcommand{\scream}[1]{\MakeUppercase{#1}}
\newcommand{\prophesy}[1]{\shout{#1}}

% parsel
\newcommand{\parselify}[1]{%
  \StrSubstitute{#1}{ss}{ß}[\parselstring]%
  \StrSubstitute{\parselstring}{s}{ss}[\parsselstring]%
  \StrSubstitute{\parsselstring}{ß}{sss}[\parssselstring]%
}
% N.B. Other commands, such as \emph, cannot be used inside \parsel
\newcommand{\parsel}[1]{\parselify{\fontspec[ExternalLocation]{Parseltongue.ttf}#1}{\ptsansi\parssselstring}}

% \spell macro
\newcommand{\spell}[1]{{\Star}\emph{#1}{\Star}}

% Author’s notes
\newcommand{\authorsnotefootnotemark}{\footnotemark}
% \newcommand{\authorsnotetext}[1]{\footnotetext{Author’s note: #1}}
\newcommand{\authorsnotetext}[1]{}


% Newspaper headlines

\newcommand{\headline}[1]{%
\begin{center}%
\textsc{#1}%
\end{center}}

\newcommand{\inlineheadline}[1]{%
\textsc{#1}%
}

\newenvironment{headlines}{%
  \newcommand{\header}[1]{\begin{SingleSpace}\upshape ##1\end{SingleSpace}}
  \let\hmorSavedLabel\label
  \renewcommand{\label}[1]{{\upshape\itshape ##1}}%
  \begin{Spacing}{0.75}
  \begin{center}
  \scshape
  \nonzeroparskip
}{
  \end{center}
  \end{Spacing}
  \let\label\hmorSavedLabel
}


% Letters / writtenNote

\newenvironment{writtenNote}{%
\fontspec[ExternalLocation]{Graphe_Alpha_alt.ttf}\scriptsize%
  \renewcommand{\emph}{\uline}
  \vskip 1\baselineskip plus 1\baselineskip minus 1\baselineskip%
  \begin{adjustwidth}{\parindent}{\parindent}%
    \par\noindent%
    \itshape%
    }
  {%
  \end{adjustwidth}%
  \vskip 1\baselineskip plus 1\baselineskip minus 1\baselineskip%
}

% \letterAddress
\newcommand{\letterAddress}[1]{%
\pagebreak[1]\noindent{}\fontspec[ExternalLocation]{Graphe_Alpha_alt.ttf}%
\scriptsize#1\nopagebreak[4]\par%
}

% \letterClosing
\newcommand{\letterClosing}[2][\vskip 1\baselineskip]{%
\nopagebreak[4]\fontspec[ExternalLocation]{Graphe_Alpha_alt.ttf}%
\scriptsize#1%
\par\nopagebreak[5]\noindent#2%
}






% PartChapters
% \partchapter{The Stanford Prison Experiment}{TSPE}{XIII}{Aftermaths}
% TOC: TSPE part XIII: Aftermaths
% Page header: The Stanford Prison Experiment XIII: \\? Aftermaths
% Title: The Stanford Prison Experiment, Part XIII: \\? Aftermaths
\newcommand{\partchapter}[3][\relax]{%
	\chapter[\texorpdfstring{#2, \abbrev{part #3}}{#2, part #3}]%
		[#2 #3]{#2, Part~#3#1}}
\newcommand{\namedpartchapter}[5][\relax]{%
	\chapter[%
			\texorpdfstring{%
				\abbrev{#3, part #4}: #5}{%
				#3, part #4: #5}][%
			\mbox{#2 #4:} \mbox{#5}]{%
			#2, Part~#4:\protect\linebreak[1] #5#1}%
}
      
% Hanging paras for play scripts (used in Omake IV)
\newenvironment{playdialog}{\begin{hangparas}{2em}{1}}{\end{hangparas}}

% Chapter openings
% Definition of chapterOpeningAuthorNote when they are desired to be printed
% FIXME: Make the environment definition switchable with a flag
% \newenvironment{chapterOpeningAuthorNote}{%
% \par\noindent%
% E.~Y.:~
% }
% {%
% \newline\rule[1ex]{\textwidth}{.1pt}\newline%
% }
\excludecomment{chapterOpeningAuthorNote}

\newenvironment{chapterOpeningQuote}{%
\parindent=0pt%
\itshape}
{\newline\rule[1ex]{\textwidth}{.1pt}\newline}






% Stars and breaks

% Single “magic star” decoration
\newcommand{\Star}{{\fontspec[ExternalLocation]{Miscelanea.ttf}*}}

% Three “magic stars” decoration
\def\Stars{{\large\Star\kern-.6ex\lower1.3ex\hbox{\large\Star}\kern-.1ex\raise.2ex\hbox{\tiny\Star}\spacefactor1000}}

% Text break made of \Stars (only used to define other commands)
\makeatletter
\def\sbreak{\mbox{}\nobreak\hfill\mbox{}\allowbreak\rule{.60\textwidth}{.0pt}\par%
  \vskip 0pt plus 2\baselineskip\noindent{%
    \parbox[c][0pt][c]{\textwidth}{%
      \hfil \hbox{\lower14pt\hbox{\normalsize\Stars}}%
    }%
  }}

% A standalone break
\def\later{\sbreak%
  \vskip 0pt plus 2\baselineskip%
  \par\rule{.5\textwidth}{.0pt}\vskip1pt\noindent}

% A break followed by a new section
\newcommand{\latersection}[1]{\sbreak\section{#1}}
\makeatother

% This file includes all the generic formatting for HPatMoR. This mostly entails configuring
% the memoir package, though “configuring” on occasion means “completely messing it up”.

\RequirePackage{fmtcount}
\RequirePackage{calc}

% Fonts used generally (specific fonts used only once or twice are not here).
\usepackage{xltxtra}
\defaultfontfeatures{Ligatures={TeX}}
\setmainfont[
  Extension=.otf
, UprightFont=*-Regular
, ItalicFont=*-Italic
, BoldFont=*-Bold
, BoldItalicFont=*-BoldItalic
, SmallCapsFont=AlegreyaSC-Regular
]{Alegreya}

\newfontface\hpfont[ExternalLocation]{Lumos}
\newfontface\ptsansi[ExternalLocation]{AlegreyaSans-Italic}

% Drop-caps at start of chapters
\renewcommand{\LettrineFontHook}{\hpfont}
\renewcommand{\LettrineTextFont}{}
\renewcommand{\DefaultLoversize}{.2}
\renewcommand{\DefaultLraise}{0.1}


\newcommand{\numberstringnumesp}[1]{
\ifthenelse{\equal{#1}{1}} {unu}{\ifthenelse{\equal{#1}{2}} {du}{\ifthenelse{\equal{#1}{3}} {tri}{\ifthenelse{\equal{#1}{4}} {kvar}{\ifthenelse{\equal{#1}{5}} {kvin}{\ifthenelse{\equal{#1}{6}} {ses}{\ifthenelse{\equal{#1}{7}} {sep}{\ifthenelse{\equal{#1}{8}} {ok}{\ifthenelse{\equal{#1}{9}} {naŭ}{\ifthenelse{\equal{#1}{10}} {dek}{\ifthenelse{\equal{#1}{11}} {dek unu}{\ifthenelse{\equal{#1}{12}} {dek du}{\ifthenelse{\equal{#1}{13}} {dek tri}{\ifthenelse{\equal{#1}{14}} {dek kvar}{\ifthenelse{\equal{#1}{15}} {dek kvin}{\ifthenelse{\equal{#1}{16}} {dek ses}{\ifthenelse{\equal{#1}{17}} {dek sep}{\ifthenelse{\equal{#1}{18}} {dek ok}{\ifthenelse{\equal{#1}{19}} {dek naŭ}{\ifthenelse{\equal{#1}{20}} {dudek}{\ifthenelse{\equal{#1}{21}} {dudek unu}{\ifthenelse{\equal{#1}{22}} {dudek du}{}}}}}}}}}}}}}}}}}}}}}}}

\newcommand{\numberstringnumespup}[1]{
\ifthenelse{\equal{#1}{1}} {UNU}{\ifthenelse{\equal{#1}{2}} {DU}{\ifthenelse{\equal{#1}{3}} {TRI}{\ifthenelse{\equal{#1}{4}} {KVAR}{\ifthenelse{\equal{#1}{5}} {KVIN}{\ifthenelse{\equal{#1}{6}} {SES}{\ifthenelse{\equal{#1}{7}} {SEP}{\ifthenelse{\equal{#1}{8}} {OK}{\ifthenelse{\equal{#1}{9}} {NAŬ}{\ifthenelse{\equal{#1}{10}} {DEK}{\ifthenelse{\equal{#1}{11}} {DEK UNU}{\ifthenelse{\equal{#1}{12}} {DEK DU}{\ifthenelse{\equal{#1}{13}} {DEK TRI}{\ifthenelse{\equal{#1}{14}} {DEK KVAR}{\ifthenelse{\equal{#1}{15}} {DEK KVIN}{\ifthenelse{\equal{#1}{16}} {DEK SES}{\ifthenelse{\equal{#1}{17}} {DEK SEP}{\ifthenelse{\equal{#1}{18}} {DEK OK}{\ifthenelse{\equal{#1}{19}} {DEK NAŬ}{\ifthenelse{\equal{#1}{20}} {DUDEK}{\ifthenelse{\equal{#1}{21}} {DUDEK UNU}{\ifthenelse{\equal{#1}{22}} {DUDEK DU}{}}}}}}}}}}}}}}}}}}}}}}}


% Extend lettrine cutout when more than one paragraph goes alongside the drop-cap
% Copied, with different macro names, from
% https://tex.stackexchange.com/questions/369868/using-lettrine-with-short-paragraphs
\newcount\hplettrinecount
\makeatletter
\def\hplettrineextrapara{%
\ifnum\prevgraf<\c@L@lines
\hplettrinecount\z@
\loop
\ifnum\hplettrinecount<\prevgraf
\advance\hplettrinecount\@ne
\afterassignment\hplettrinedima\count@\L@parshape\relax
\repeat
\parshape\L@parshape
\fi}
\def\hplettrinedima{\afterassignment\hplettrinedimb\dimen@}
\def\hplettrinedimb{\afterassignment\hplettrinedef\dimen@}
\def\hplettrinedef#1\relax{\edef\L@parshape{\the\numexpr\count@-1\relax\space #1}}
\makeatother
\newcommand{\lettrinepara}[3][]{\lettrine[nindent=0pt,#1]{#2}{#3}}

% Allow linebreaks after em-dash and hyphens, except when they’re followed by punctuation
\newXeTeXintercharclass \punctuationClass

\XeTeXcharclass `\’ \punctuationClass
\XeTeXcharclass `\‘ \punctuationClass
\XeTeXcharclass `\“ \punctuationClass
\XeTeXcharclass `\” \punctuationClass
\XeTeXcharclass `\. \punctuationClass
\XeTeXcharclass `\, \punctuationClass
\XeTeXcharclass `\: \punctuationClass
\XeTeXcharclass `\? \punctuationClass
\XeTeXcharclass `\! \punctuationClass
\XeTeXcharclass `\: \punctuationClass

\newXeTeXintercharclass \digitClass
\XeTeXcharclass `\0 \digitClass
\XeTeXcharclass `\1 \digitClass
\XeTeXcharclass `\2 \digitClass
\XeTeXcharclass `\3 \digitClass
\XeTeXcharclass `\4 \digitClass
\XeTeXcharclass `\5 \digitClass
\XeTeXcharclass `\6 \digitClass
\XeTeXcharclass `\7 \digitClass
\XeTeXcharclass `\8 \digitClass
\XeTeXcharclass `\9 \digitClass

\newXeTeXintercharclass \dashClass
\XeTeXcharclass `\— \dashClass % em
\XeTeXcharclass `\– \dashClass % en
\XeTeXcharclass `\- \dashClass % hyphen
\XeTeXcharclass `\… \dashClass

\XeTeXinterchartokenstate = 1

\def\morhyphenpenalty{75}
\exhyphenpenalty=10000

\XeTeXinterchartoks \dashClass 0 = {\hskip 0pt\penalty \morhyphenpenalty}

% Adjust space around lists
\setlength{\topsep}{.5\baselineskip plus 1\baselineskip minus .5\baselineskip}
\setlength{\partopsep}{.5\baselineskip plus 1\baselineskip minus .5\baselineskip}

% Miscellaneous global typesetting parameters
\frenchspacing
\setlength{\emergencystretch}{.06\textwidth}

% Try to avoid excessive hyphens
\doublehyphendemerits=30000
\finalhyphendemerits=30000
\adjdemerits=10000
\brokenpenalty10000\relax

% Make it easier to manage hyphenation
\makeatletter
\newcommand{\emdashhyp}{\leavevmode%
\prw@zbreak—\discretionary{}{}{}\prw@zbreak}
\makeatother

% Avoid widows and orphans
% https://mailman.ntg.nl/pipermail/ntg-context/2013/074250.html
\widowpenalty 10000
\clubpenalty 10000

% Various packages used
\usepackage[normalem]{ulem}
\usepackage{xfrac}
\usepackage{censor}
\usepackage[useregional]{datetime2}
\usepackage[nopagecolor=white,pagecolor=white]{pagecolor}
\usepackage{afterpage}
\usepackage{gitinfo2}
\usepackage{hyphenat}



\hyphenation{Her-mi-o-ne Gran-ger bru-shes Gryf-fin-dor Le-strange 
some-where which-ev-er Hog-warts re-pli-cat-ed ran-dom sta-tis-ti-cal 
Wi-zen-gam-ot an-aly-se an-aly-sis remem-ber Sly-the-rin Sly-the-rins
Se-ve-rus Mc-Gon-agall}


% Redefine \textls for XeTeX
\usepackage{calc}
\newcounter{hpletterspacing}
\renewcommand{\textls}[2][100]{%
  \setcounter{hpletterspacing}{#1 / \real{10.0}}%
  \addfontfeature{LetterSpace=\thehpletterspacing}#2%
}

\renewcommand{\contentsname}{Enhavtabelo}
\renewcommand{\chaptername}{ĉapitro}