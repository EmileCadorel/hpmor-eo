\chapter{Raciigo}

\begin{center}\rule{3in}{0.4pt}\end{center}

Hermione Granger timis fariĝanta malbona.

La malsamo inter Bona kaj Malbona estis kutime facila por kompreni, ŝi
neniam komprenis kial aliaj homoj havis tiom da problemoj. En
Herpŭrko, ``Bona'' estis Profesoro Flirtiko kaj Profesorino McGonagall
kaj Profesorino Sproso. ``Malbona'' estis Profesoro Skoldo kaj
Profesoro Ciuro kaj Drako Malfojo. Harry Potter\ldots estis unu el
tiuj nekutimaj okazoj, kie vi ne povis diri nur per rigardi. Ŝi estis
ankoraŭ provanta malkovri tie, kie li apartenis.

Sed kiam temis pri \emph{ŝi mem}\ldots

Hermione havis \emph{tro da plezuro} per premegi Harry'n Potter.

Ŝi faris pli bone en ĉiuj la klasoj, kiujn ili sekvis. (Krom balaila
flugo, kiu estis kiel gimnastiko, ĝi ne kalkulas.) Ŝi akiris
\emph{realajn} Domajn poentojn ĉiutage dum sia unua semajno, ne pro
strangaj heroaj aĵoj, sed pro \emph{lertaj} aĵoj kiel lerni sorĉon
rapide, kaj helpi aliajn studentojn. Ŝi sciis ke tiaj Domaj poentoj
estis pli bonaj, kaj la plej bona parto estis ke Harry Potter ankaŭ
sciis tion. Ŝi povis vidi tion en liaj okuloj, ĉiufoje kiam ŝi gajnas
alian \emph{realan} Doman poenton.

Se vi estis Bona, vi ne estis supozita tiel ĝui gajni.

Tio komencis je la tago de la trajna vojaĝo, tamen ĝi prenis keklkajn
tempojn por ke la vento turnas. Ne estis antaŭ malfrue tiu ĉi nokto,
ke Hermione komencis konscii \emph{kiel multe} ŝi lasis tiun knabon
marŝi super si.

Antaŭ ol ŝi renkontis Harry'n Potter, ŝi ne havis iun ajn, kiun ŝi
volis premegi. Se iu ne estis farinta tiel bone kiel ŝi en klaso,
estis sia devo helpi rin, ne hontige montri rin. Tio estis tio, kion
esti Bona volis diri.

Kaj nun\ldots

\ldots nun ŝi estis gajnanta. Harry Potter eksaltis ĉiufoje kiam ŝi
akiris alian Doman poenton, kaj tio estis tiel \emph{amuza}. Ŝiaj
gepatroj avertis ŝin pri la drogoj kaj ŝi suspektis ke tio, estis pli
amuza ol drogo.

Ŝi ĉiam ŝatis la ridetojn, kiujn la Profesoroj donis al ŝi kiam ŝi
faris ion bone. Ŝi ĉiam ŝatis vidi la longan vicon de kontrolmarkoj
sur perfekte respondita testo. Sed nun kiam ŝi faris bone en klaso ŝi
volis indiferente ĉirkaŭrigardi kaj kapti videton de Harry Potter
kunpreminta siajn dentojn, kaj tio farigis ŝin voli eksplodi je kanto
kiel en filmo de Disney.

Tio estis Malbona, ne?

Hermione timis fariĝanta malbona.

Kaj tiam iu penso, kiu aperis al ŝi, forviŝis ĉiujn ŝiajn timojn.

Ŝi kaj Harry estis en Romanco! Evidente! Ĉiuj sciis, kio ĝi volis diri
kiam knabo kaj knabino komencis batali ĉiutempe. Ili estis
\emph{amindumanta} unu la alian! Ne estis io ajn malbona pri tio.

Tio ne povis esti ke ŝi nur ĝui forbati la vivantan lernejan taglumon
de la plej fama studento de la lernejo, iu kiu estis en libroj kaj
parolis kiel libroj, la knabo kiu iel venkis la Mastron de la Tenebroj
kaj eĉ premegis \emph{Profesoron Skoldo} kiel malgaja eta insekto, la
knabo kiu estis, kiel Profesoro Ciuro diris supera sur ĉiuj alia en
unua jaro de Korvungo krom Hermione Granger, kiu estis komplete
premeganta la Knabo'n-Kiu-Postvivis en ĉiuj kursoj krom balaila flugo.

Ĉar tio estintus Malbona.

Ne, Tio estis romanco. \emph{Tio} estis tiel. \emph{Tio} estis kial
ili estis batalanta.

Hermione estis gaja, ke ŝi malkovris tion ĝustatempe por hodiaŭ, kiam
Harry perdos ilian konkurson de librolego, pri kiu la tuta lernejo
aŭdis paroli, kaj ŝi volis komenci danci, pro la kruta superflua ĝojo
kiun tio liberigis.

Estis je 2a kaj 45 posttagmeze je Sabato kaj Harry Potter ankoraŭ la
dunonon de \emph{Historio de Magio} de Matilda Melloko por legi, kaj
ŝi estis rigardadanta sian poŝohorloĝo dum ĝi tiktatis kun terura
malrapideco al 2a kaj 47 posttagmeze.

Kaj la tuta Korvunga komuna ĉambro estis rigardanta.

Ne estis nur la Korvungoj de unua jaro, la novaĵo disvastiĝis kiel
disverŝita lakto kaj pli da la duono de Korvungo estis amasigita en la
ĉambro, premitaj sur la sofoj kaj apogiĝintaj sur libroŝrankoj kaj
sidante sur la brakapogiloj de la seĝoj. Ĉiuj ses prefektoj estis tie
ĉi, inkluzive la estrino de la prefektoj de Herpŭrko. Iu bezonis ĵeti
Aŭtomatan Ĉarmon AeroFreŝigantan nur por ke estis sufiĉe da
oksigeno. Kaj la bruego de la konversacioj mortigis en murmurojn, kiuj
estis nun velkintaj en kompletan silenton.

2a kaj 46 posttagmeze.

La streĉo estis netolerebla. Sed temis pri iu ajn alia, \emph{iu ajn}
alia, lia malvenko estintus antaŭdecidita konkludo.

Sed temis pri Harry Potter, kaj vi ne povis ekskludi la eblecon ke li,
iam je kelkaj sekundoj, levos sian manon kaj klakos siajn fingrojn.

Kun subita teruro, ŝi konsciis kiel Harry Potter simple povus kapabli
fari ekzakte tion. Estus ekzakte kiel li havi jam fini legi la duan
duonon de la libro pli frue\ldots

La vido de Hermione komencis malklariĝi. Ŝi provis farigi sin mem
spiri, sed malkovris ke ŝi simple ne povis.

Ankoraŭ dek sekundoj, kaj li ankoraŭ ne havis levi sian manon.

Ankoraŭ kvin sekundoj.

Harry Potter zorgeme metis legosignon en sian libron, fermis ĝin kaj
metis ĝin apud li.

``Mi ŝatus rimarkigi por la utilo de posteularo,'' diris la
Knabo-Kiu-Postvivis per klara voĉo, ``ke restis al mi nur unu duonon
da libro por legi, kaj ke mi renkontis multajn neantenditajn
malfruigojn—''

``\emph{Vi perdis!}'' kriis Hermione. ``Vi \emph{perdis}! Vi
\emph{perdis nian konkurson}!''

Estis kolektiva elspiro kiam ĉiuj rekomencis spiri.

Harry Potter ĵetis al ŝi Rigardon de Flama Fajro, sed ŝi estis
ŝvebanta en haloo de pura feliĉo kaj nenio povis atingi ŝin.

``\emph{Ĉu vi konscias kiel speco de semajno mi ĵus havis?}'' diris
Harry Potter. ``Iu ajn malpli bona estaĵo estintus forte dispremita
por legi ok librojn eĉ de D-ro Seuss''\footnote{Theodor Seuss Geisel
estas fama usona verkisto, plej konata pro sia infana literaturo
verkita sub la pseŭdonimo D-ro Seuss}.

``*Vi* elektis la tempolimon.''

La rigardo de Flama Fajro de Harry krekis ankoraŭ pli varma. ``Mi ne
havis iun ajn logikan manieron por scii ke mi devos savi la tutan
lernejon de Profesoro Skoldo, aŭ ke mi estos frapita en Defenca klaso,
kaj se mi dirus al vi kiel mi perdis la tutan tempon inter la 5a
posttagmeze kaj vespermanĝo je ĵaŭdo, vi pensus ke mi estas freneza—''

``Oooh, ŝajnas ke \emph{iu} kaptiĝis per la \emph{sofismo de
planado}.''

Kruda ŝoko montris sin sur la vizaĝo de Harry Potter.

``Oh, tio memorigas min ke mi finis legi la unuan aron de libroj,
kiujn vi pruntis al mi,'' Hermione diris kun la plej senkulpe esprimo,
kiun ŝi sukcesis fari. Kelkaj el ili estis malfacilaj libroj. Ŝi al si
demandis kiom da tempo ĝi prenis al \emph{li} por fini legi ilin.

``Iam,'' diris la Knabo-Kiu-Postvivis, ``kiam la malproksimaj
posteuloj de \emph{Homo sapiens} rigardos malantaŭen trans la historio
de la galaksio kaj scivolos kiel ĉio iris tiel malbone, ili konkludos
ke la originala eraro estis kiam iu instruis al Hermione Granger kiel
legi.''

``Sed vi tamen perdis,'' diris Hermione. Ŝi levis manon al sia mentono
kaj aspektis pripensema. ``Nun kion ekzakte vi devus perdi, mi min
demandas?''


``\emph{Kio?}''

``Vi perdis la veton,'' Hermione klarigis, ``do vi devas pagi ion.''

``Mi ne memoras konsenti al tio!''

``Vere?'' diris Hermione Granger. Ŝi metis penseman aspekton sur sia
vizaĝo. Tiam, kvazaŭ la ideo ĵus aperis al ŝi, ``Ni votos, tial. Ĉiuj
en Korvungo, kiu pensas ke Harry Potter devas pagi ion, levi vian
manon!''

``\emph{Kio?},'' kriŝis Harry denove.

Li turniĝis ĉirkaŭe kaj visi ke li estis ĉirkaŭigita per maro de
levitaj manoj.

Kaj se Harry Potter rigardintus \emph{pli atenteme}, li vidintus ke
terura numbro de la atestantoj ŝajnis esti knabinoj kaj ke preksaŭ
ĉiuj knabinoj en la ĉambro havis la manon levita.

``Haltu!'' ĝemegis Harry. ``Vi ne konas tion, kion ŝi estas
demandonta! Ĉu vi ne vidas tion, kion ŝi estas faranta? Ŝi farigas vin
fari antaŭan sindevigon nun, kaj poste la premo de kohereco perforte
farigos vin konsenti kun ĉion ajn ŝi diros!''

``Ne zorgu,'' diris la prefektino Penelopo Klarakvo. ``Se ŝi demandas
ion, kio ne estas akceptebla, ni povas simple ŝanĝi nian opinion. Ĉu
tio ne estas vera, ĉiuj?''

Kaj estis avidaj kapjesoj de ĉiuj la knabinoj, al kiuj Penelopo
Klarakvo parolis pri la plano de Hermione.

\begin{center}\rule{3in}{0.4pt}\end{center}

Silenta silueto kviete marŝis laŭ la marvamaj koridoroj de la karceroj
de Herpŭrko. Li devis ĉeesti en iu ĉambro je la 6a posttagmeze por
renkonti iun personon, kaj se tio estis ebla, estis pli bona iri
antaŭtempe por montri respekton.

Sed kiam lia mano turniĝis la pordotenilon kaj malfermis la pordon por
malkovri malhelan kaj neuzitan ĉambron silentan, estis silueto jam
staranta inter la vicoj de malnovaj skribotabloj polvaj. Silueto kiu
portis etan brile verdan stangon, ĵetanta palan lumon, kiu malfacile
iluminis eĉ la personon kiu tenis ĝin, sen paroli pri la ĉirkaŭa
ĉambro.

La lumo de la koridoro mortis dum la pordo fermiĝis kaj ŝlosis
malantaŭ li, kaj la okuloj de Drako komencis adaptiĝis al la malhela
brilo.

La silueto malrapide ŝanĝiĝis por rigardi lin, malkaŝanta ombritan
vizaĝon, nur parte iluminita per la tremiga verda lumo.

Drako jam ŝatis tiun ĉi kunvenon. Daŭrigu la malvarman lumon verdan,
farigu ilin ambaŭ pli alta, donu al ili kapuĉojn kaj maskojn, movigu
ilin de klasĉambron al tombejo, kaj ĝi estus ekzakte kiel la komenco
de la duono de la historioj, kiujn la amikoj de lia Patro rakontis pri
la Morto-Manĝantoj.

``Mi volas ke vi scias, Drako Malfojo,'' diris la silueto per tono de
mortiga kvieto, ``ke mi ne kulpigas vin pro mia freŝdata malvenko.''

Drako malfermis sian buŝon por senpense protesti, estis neniu ebla
kialo por ke li povis esti kulpigita—

``Tio estis pli pro mia stulteco ol io ajn alia,'' daŭris la ombrita
silueto. ``Estis multe da aliaj aĵoj, kiujn mi povintus fari, je ĉiuj
etapoj laŭ la vojo. Vi ne demandis al mi fari ekzakte tion kion mi
faris. Vi nur petis helpon al mi. Mi estis la sola kiu malsaĝe elektis
tiun partikularan metodon. Sed la fakto restas ke mi perdis la
konkurso de unu duono de libro. La agoj de via stulta dombesto, kaj la
favoro kiun vi petis al mi, kaj, jes, mia propra malsaĝeco, kaŭzis min
\emph{perdi tempon.} Pli da tempo, ol vi scias. Tempo, kiu finfine,
montriĝis esti kriza. La fakto restas, Drako Malfojo, ke se vi ne
petintus tiun favoron, mi gajnintus. Kaj ne\ldots anstataŭ\ldots
\emph{perdis}.''

Drako jam aŭdis pri la malvenko de Harry, kaj la pago kiun Hermione
reklamaciis de li. La novaĵo disvastiĝis pli rapide ol strigoj
povintus plenumi.

``Mi komprenas,'' Drako diris. ``Mi bedaŭras.'' Estis nenio alia, kion
li povis diri se li volis ke Harry amikiĝas kun li.

``Mi ne petas ke vi bedaŭras aŭ pardonpetas,'' diris la malhela
silueto, ankoraŭ kun mortiga kvieto. ``Sed mi ĵus pasigis du tutajn
horojn kun Hermione Granger, portante vestaĵojn kiujn oni donis al mi,
vizitante tiel fascinajn lokojn en Herpŭrko kiel eta guglanta
akvofalo, kiu aspektis laŭ mi kiel morvo, estante akompanata per multe
da aliaj knabinoj kiuj insistis por fari tiel helpemajn aktivecojn
kiel ĵetkovri nian vojon kun Transfiguritaj petaloj de rozo. Mi estis
je galanta rendevuo, ido de Malfojo. Mia unua galanta
rendevuo. \emph{Kaj kiam mi vokos tiun pagon de favoro, vi pagos
ĝin}.''

Drako kapjesis solema. Antaŭ ol alveni, li faris la saĝan antaŭzorgon
de lerni ĉiujn disponeblajn detalojn pri la rendevuo de Harry, por ke
li povis ridi unu bona fojo, antaŭ ilia kunveno, kaj por ke li ne
faros eraron per subridi senĉese ĝis li perdos konscion.

``Ĉu vi pensas,'' Drako diris, ``ke io malgaja devas okazi al la
Granger knabino—''

``Disĵeti la vorton en Serpentimo ke la Granger knabino estas mia, kaj
iu ajn kiu enmiskiĝas en miaj aferoj, havos rian restaĵon disvastigita
trans iu loko sufiĉe granda por enhavi dek du malsamajn lingvojn. Kaj
pro tio ke mi ne estas Grifindoro kaj mi prefere uzas ruzo anstataŭ
direktaj atakoj, ili ne devas paniki se mi estas vidita rideti al
ŝi.''


``Aŭ se vi estas vidita en dua rendevuo?'' Drako diris, permesante nur
etan noton de skeptiko en sia voĉo.

``\emph{Ne estos dua rendevuo,}'' diris la verde iluminita silueto per
voĉo tiel timeganta ke ĝi sonis, ne nur kiel Morto-Manĝanto, sed kiel
Amikus Rokvico tiu fojo ĵus antaŭ kiam Patro diris al li ke li haltu,
ke li ne estis la Mastro de la Tenebroj.

Evidente, ĝi estis ankoraŭ akuta voĉo de juna knabo, kaj kiam vi
kombinis ĝin kun la nunajn vortojn, tiam nu, ĝi nur ne funkciis. Se
Harry Potter fariĝus la sekva Mastro de la Tenebroj iam, Drako uzus
Reflekribrilon por konservi kopion de ĉi tiu memoraĵo ie sekura, kaj
Harry Potter neniam kuraĝus perfidi lin.

``Sed parolu ni pri pli gajaj aferoj,'' diris la verda ombrita
silueto. ``Parolu ni pri scio kaj povo. Drako Malfojo, parolu ni pri
Scienco.''

``Jes,'' diris Drako. ``Parolu ni.''

Drako al si demandis kiom sia vizaĝo estis videbla, kaj kiom estis en
la ombro, en tiu tremiga verda lumo.

Kaj kvankam Drako daŭrigis sian vizaĝon serioza, estis rideto en sia
koro.

Li estis finfine havanta reale plenkreskan konversacion.

``Mi proponos povon al vi,'' diris la ombrita silueto, ``kaj mi
prezentos tiun povon al vi, kaj ĝin koston. La povo venas de koni la
formon de la realo kaj tiel gajni povon sur ĝi. Tio, kion vi povas
kompreni, vi povas kontroli, kaj tiu povo sufiĉas por iri marŝi sur la
luno. La kosto de tiu povo estas ke vi devas lerni kiel fari demandon
al la Naturo, kaj multe pli malfacila, lerni kiel akcepti la
respondojn de la Naturo. Vi faros eksperimentojn, plenumos testojn kaj
vidos tion, kio okazos. Kaj vi devos akcepti tion, kion la rezulto
volas diri, kiam ili diras al vi ke vi eraris. Vi devos lerni
\emph{kiel perdi}, ne kontraŭ mi, sed kontraŭ la Naturo. Kiam vi
troviĝos argumenti kontraŭ la realo, vi devos lasi la Naturo gajni. Vi
trovos tion dolora, Drako Malfojo, kaj mi ne scias se vi estas sufiĉe
forta por fari tion. Konante la koston, ĉu vi ankoraŭ volas lerni la
human povon?''

Drako spiris profunde. Li jam pensis pri tio. Kaj estis malfacile vidi
kiel li povus respondi malsame. Oni postulis, ke li sekvu ĉiujn
irejojn gvidantaj al amikeco kun Harry Potter. Tio estis nur lerni, li
ne promesis fari ion ajn. Li povus ĉiam halti sekvi la lecionon iam
ajn\ldots

Estis certe multaj aĵoj, pri la situacio, kiuj farigis tion aspekti
kiel kaptilo, sed eĉ kun ĉio tio ĉi, Drako ne vidis kiel tio povus iri
malbone.

Cetere Drako iel volis estri la mondon.

``Jes,'' diris Drako.

``Bonege,'' diris la ombrita silueto. ``Mi estas havinta semajnon
relative plenan, kaj ĝi prenos tempon por plani vian instruplanon—''

``Mi havas ankaŭ multe da aĵojn por fari, kaj mi bezonas plifirmigi
mian influon en Serpentimo,'' diris Drako, ``por ne paroli pri miaj
hejmtaskoj. Eble ni povas komenci nur en oktobro?''

``Tio ŝajnas saĝa,'' diris la ombrita silueto, ``sed kion mi volis
diri per plani vian instruplanon, estas ke mi bezonas scii tion, kion
mi estos instrui al vi. Tri afero venas al mi. La unua estas ke mi
instruos al vi pri la huma cerbo kaj menso. La dua opcio estas ke mi
instruos al vi pri la fizika universo, tiujn artojn kiuj povas gvidi
al viziti la lunon. Tio implikas grandan kvanton da numbroj, sed por
iu speco de menso, tiuj nombroj estas pli belaj ol iu ajn alia kiun la
scienco povas instrui. Ĉu vi ŝatas numbrojn, Drako?''

Drako kapneis.

``Tiam forgesu ni tion. Vi finfine lernos matematikojn, sed ne tuj, mi
pensas. La tria opcio estas ke mi instruos al vi pri genetikoj kaj
evoluo kaj heredaĵo, tio kion vi nomas sangon—''

``Tiu ĉi,'' diris Drako.

La silueto kapjesis. ``Mi pensis ke vi diris tiel. Sed mi pensas ke
tio estos la plej dolora vojo por vi, Drako. Kio se via familio kaj
amikoj, la puristoj de sango, dirus iun aĵon kaj ke vi trovos ke la
eksperimento dirus ion alian?''

``Tiam mi trovos kiel farigi la eksperimenton diri la \emph{bonan}
respondon!''

Estis paŭzo, dum la ombrita silueto staris tie la buŝo malfermita
nelonge.

``Hum,'' diris la ombrita silueto. ``Tio vere ne funkcias tiel. Tio
estas pri kio, mi provis averti vin, Drako. Vi ne povas farigi la
respondon esti iun ajn, kiun vi ŝatas.''

``Vi ĉiam povas farigi la respondon esti kiel vi deziras,'' diris
Drako. Tio estis preskaŭ la unua aĵo, kiun sia tutoro instruis al
li. ``Tio estas nur rilate al trovi la bonajn argumentojn.''

``Ne,'' diris la ombrita silueto, la voĉo kreskante pro frustracio,
``ne, ne, ne! Tiel vi akiros la \emph{malbonan respondon}, kaj vi ne
povas iri sur la lunon tiel! Naturo ne estas persono, vi ne povas
farigi ĝin kredi ion malsimilan, se vi provas diri ke la luno estas
farata el fromaĝo, vi povas argumenti dum tagoj sed tio ne ŝanĝos la
lunon! Tio, pri kio vi estas parolanta, estas \emph{raciigo}, kiel
komenci kun peco de papero, iri direkte al la lasta linio, kaj skribi
'kaj \emph{tial}, la luno estas farata el fromaĝo', kaj post tio,
reiri al la unua linio por skribi ĉian lertan argumenton supre. Sed aŭ
la luno estas farata el fromaĝo aŭ gi ne estas. Je la momento kiam vi
skribas la lastan linion, tio estas jam prava aŭ jam malprava. La
fakto ke la tuta papero faras la bonan konkludon aŭ ne, estas decidita
je la momento kiam vi skribas la lastan linion. Se vi provas elekti
unu el du multe kostaj kamionoj, kaj ke vi ŝatas tiun kiu brilas, ne
gravas kiajn inteligentajn argumentojn vi havas por aĉeti ĝin, la
\emph{reala} regulo ke vi uzas por \emph{elekti unu kamionon} estas
'prenu tiun kiu brilas', kaj ne gravas kiel efika tiu regulo estas por
elekti kamionon, tiu estas la speco de kamiono, kiun vi havos. Racieco
ne povas esti uzita por argumenti kontraŭ fiksa flanko, ĝia sola ebla
uzo estas por decidi \emph{kun kiu flanko} argumenti. Scienco ne estas
farita por konviki iun ajn ke la puristoj de sango pravas. Tio estas
\emph{politiko}! La povo de la scienco venas de \emph{trovi la
manieron laŭ kiu la Naturo funkcias, kaj tio ne povas esti ŝanĝita per
argumentoj!}. Tio, kion scienco povas diri al ni, estas \emph{kiel
sango vere funkcias}, kiel sorĉistoj reale heredis iliajn povojn de
iliaj gepatroj, kaj se Mugle naskitaj sorĉistoj estas reale pli
malpotencaj, aŭ pli potencaj—''

``\emph{Pli potencaj!}'' diris Drako. Li estis provinta sekvi tion,
malridetante, li povis vidi ke ĝi havis ian sencon sed ĝi certe ne
similis al io ajn, kion li aŭdis antaŭe. Kaj tiam Harry Potter diris
ion, kion Drako ne povis lasi diri. ``Vi pensas ke kotsangoj estas pli
\emph{potenca?}''

``Mi pensas nenion,'' diris la ombrita silueto. ``Mi scias nenion. Mi
kredas nenion. Mia lasta linio ne jam estas skribita. Mi malkovros
kiel testi la magian potencon de la mugle naskituloj, kaj la magian
potencon de la pursanguloj. Se miaj testoj diras al mi ke Mugle
naskituloj estas malpli potencaj, mi kredos ke ili estas malpli
potencaj. Se miaj testoj diras al mi ke la Mugle naskituloj estas pli
potencaj, mi kredos ke ili estas pli potencaj. Per scii tion, kaj
aliajn verojn, mi gajnos iom da povo—''

``Kaj vi postulas, ke mi kredu ĉion, kion vi diros?'' Drako demandis
iomete koleriĝante.

``Mi postulas ke vi faru la testojn per vi mem,'' diris la ombrita
silueto kviete. ``Ĉu vi timas tion, kion \emph{vi} trovos?''

Drako fikse rigardis la ombritan silueton dum mallonga momento, kun la
okuloj duone malfermitaj. ``Bona kaptilo, Harry,'' li diris. ``Mi
devas memori tion, tio estas nova.''

La ombrita silueto balancis la kapon. ``Tio ne estas kaptilo,
Drako. Memoru—\emph{Mi ne scias} tion, kion ni malkovros. Sed vi
komprenas le universon per argumenti kun ĝi, aŭ diri al ĝi ke ĝi
revenu kun malsimilan respondon je la sekva fojo. Kiam vi metas la
veston de sciencisto, vi devas forgesi vian politikon, kaj ĉiujn viajn
argumentojn kaj viajn fakciojn kaj flankojn, silentigi la sonoradon de
via menso, kaj nur deziri aŭdi la respondon de la Naturo.'' La ombrita
silueto paŭzis. ``Plejparto de homoj ne povas fari tion. Tio estas
kial ĝi estas malfacila. Ĉu vi certas ke vi ne preferas nur lerni pri
la cerbo?''

``Kaj se mi diras al vi ke mi preferas lerni pri la cerbo,'' Drako
diris, sia voĉo nun dura, ``vi ĉirkaŭiros por diri al homoj ke mi
timis tion, kion mi malkovros.''

``Ne,'' diris la ombrita silueto. ``Mi ne faros tian aĵon.''

``Sed vi eble povas fari similan specon de testo per vi mem, se vi
akiras la malbonan respondon, mi ne estos tie por diri ion ajn, antaŭ
ol vi montros ĝin al iu alia.'' la voĉo de Drako estis ankoraŭ dura.

``Mi malgraŭ ĉio demandos al vi unue, Drako,'' la ombrita silueto diri
kviete.

Drako paŭzis. Li ne antaŭvidis tion, li pensis ke li vidus la kaptilon
sed\ldots ``Vi tion \emph{farus?}''

``Evidente. Kiel mi povus scii kiu minaci aŭ kio ni povus demandi al
ili? Drako, mi diras denove ke tio ne estas kaptilo, kiun mi kreis por
vi. Almenaŭ ne por vi persone. Se viaj politikaj opinioj estis
malsimilaj, mi estus diranta, kio se le testoj montras ke la
pursanguloj estas pli potenca.''

``Vere?''

``*Jes!* Tio estas la kosto, kiun ĉiuj devas pagi por fariĝi vera
sciencisto!''

Drako levis manon. Li devis pripensi.

La ombrita, verde iluminita silueto atendis.

Tamen, ne prenis longe por pripensi pri tio. Se vi forĵetis ĉiujn la
konfuzigantajn partojn\ldots tiam Harry Potter estis plananta krei
ĥaoson kun io kio povas kaŭzi gigantan politikan eksplodon, kaj estus
freneza nur foriri kaj lasi lin fari tion sole. ``Ni studos sangon,''
diris Drako.

``*Bonege*,'' diris la silueto, kiu ekridetis. ``Gratulon por estis
preta fari la bonajn demandojn.''

``Dankon,'' Drako diris, ne tute sukcesante gardi la ironion el sia
voĉo.

``He, ĉu vi pensas ke iri sur la lunon estis \emph{facila?} Ĝoju ke
tio nur implikas ke vi keklfoje ŝanĝas vian manieron de pensi, kaj ne
oferbuĉo de homo!''

``Oferbuĉo de homo estintus \emph{multe} pli facila!''

Estis mallonga paŭzo, kaj poste la silueto kapjesis. ``Argumento
valida.''

``Aŭskultu, Harry,'' diris Drako sen multe da espero, ``Mi pensis ke
la ideo estis preni ĉiujn la aĵojn, kiujn la Muglojn konas, kunmeti
ilin kun la aĵojn kiujn la sorĉistoj konas, kaj fariĝi la mastroj de
la du mondoj.  Ĉu ne estus pli facila nur lerni ĉiujn la aĵojn, kiujn
la Mugloj \emph{jam} malkovris, kiel la luno afero, kaj uzi tiun
povon—''

``\emph{Ne,}'' diris la silueto, kapneante akre, sendante verdajn
ombraĵojn movantaj ĉirkaŭ sia nazo kaj orleoj. Lia voĉo estis nun
minacanta. ``Se vi ne povas lerni la artojn de la sciencistoj per
akcepti la realon, tial mi ne devas prezenti al vi tion, kion tiu
akcepto malkovris. Tio estus kvazaŭ potenca sorĉisto parolus al vi pri
tiuj pordegoj, kiuj devas neniam esti malfermitaj, kaj pri tiuj
fermiloj, kiuj devas neniam esti rompitaj, antaŭ ol vi provis vian
inteligenton kaj disciplinon per postvivi malpli grandajn danĝerojn.''

Malvarmeco transiris la vertebraron de Drako, kaj li ektremis
senintence. Li sciis ke tio estintis videbla eĉ en la malhela
lumo. ``Mi komprenas.'' Lia patro al li diris tion multe da
fojoj. Kiam pli potenca sorĉisto diris al vi ion, pri kio vi ne estis
preta por koni, vi ne devis serĉi plu se vi volis vivi.

La silueto klinigis sian kapon. ``Ja. Sed estas io alia, kion vi devus
kompreni. La unuaj sciencistoj, estante Mugloj, malhavis viajn
tradiciojn. Je la komenco, ili simple ne komprenis la nocion de
danĝera scio, kaj pensis ke ĉiuj konataj aĵoj devis esti parolitaj
libere. Kiam iliaj esploroj fariĝis danĝeraj, ili parolis kun iliaj
politikistoj pri aĵo, kiun devintus daŭri sekreta—ne rigardu min kiel
tio, Drako, tio ne estis simpla stulteco. Ili devintis esti sufiĉe
inteligenta por malkovri tiun sekreton unue. Sed ili estis Mugloj,
estis la unua fojo ke ili malkovris ion \emph{vere} danĝera, kaj ili
ne komencis kun tradicio de sekreteco. Estis milito, kaj la
sciencistoj de iu flanko zorgis ke se li ne parolus, la sciencistoj de
la malamika lando parolus al iliaj politikistoj unue \ldots'' La voĉo
mallaŭtiĝis grave. ``Ili ne detruis la mondo. Sed estis proksime. Kaj
ni ne ripetos tiun ĉi eraron.''

``Bone,'' Drako diris, lia voĉo nun tre firma. ``*Ni* ne tion
faros. Ni estas sorĉistoj, kaj studi sciencon ne farigos nin Mugloj.''

``Kiel vi diras,'' diris la verde iluminita silueto. ``Ni establos
nian propran Sciencon, magia Scienco, kaj tiu Scienco havos pli
inteligentajn tradiciojn ekde la komenco.'' La voĉo kreskis pli
dura. ``La scio, kiun mi interŝanĝos kun vi, estos instruitaj paralele
al la disciplino de akcepti la veron, la nivelo de tiu scio estos
kunligita al via progreso en tiu disciplino, kaj vi interŝanĝos tiun
konaton kun neniu alia, kiu ne estas lerninta tiun disciplinon. Ĉu vi
akceptas tion?''

``Jes,'' diris Drako. Kion alian li estis supozita fari, diri ne?

``Bone. Kaj tio, kion vi malkovris per vi mem, vi gardos ĝin por vi
mem, almenaŭ ke vi pensas ke aliaj sciencistoj estas pretaj por scii
ĝin. Tio, kion ni interŝanĝos inter ni, ni ne diros ĝin al la mondo
almenaŭ ke ni konsentas ke estas sekura ke la mondo scias ĝin. Kaj ne
gravas niaj propraj politikaj opinioj, kaj fidelecoj, ni punos ĉiujn
homoj, kiuj malkaŝos danĝeran magion aŭ fordonos danĝerajn armilojn,
ne gravas la speco de milito kiu povas okazi. De tiu tago, tio estos
la tradicio kaj la leĝo de la scienco inter la sorĉistoj. Ĉu vi
konsentas kun tio?''

``Jes,'' diris Drako. Fakte, tio komencis ŝajni tute interesa. La
morto-manĝantoj provis preni povon per esti pli timigantaj ol iu ajn
alia, kaj ili ne fakte jam gajnis. Eble estis tempo por provi regi la
mondon uzante sekretecon anstataŭ. ``Kaj nia grupo daŭras kaŝita, tiel
eble kiel longe, kaj ĉiuj ajn en ĝi devas konsenti niajn regulojn.''

``Evidente, sendube.''

Estis mallonga paŭzo.

``Ni bezonos pli bonajn robojn,'' diris la ombrita silueto. ``kun
kapuŝo, kaj tiel plu—''

``Mi estis pensanta ekzkte tio,'' diris Drako. ``Ni ne bezonas novajn
robojn, tamen, kapuĉmantelo por meti super. Mi havas amikinon en
Serpentimo, ŝi prenos viajn dimensiojn—''

``Ne diru al ŝi, por kio, tio estas tamen —''

``Mi ne estas \emph{stulta!}''

``Kaj neniu masko por la momento, ne kiam nur estas vi kaj mi—'' diris
la ombrita silueto.

``Bone! Sed pli malfrue, ni devus havi ian specialan markon, kiun
ĉiujn niajn servistojn havos, la Marko de la Scienco, kiel serpentimo
manĝanta la lunon sur la dekstra brako—''

``Tio estas nomita doktoreco, kaj ĉu tio ne farigos tre facila
identigi niajn homojn?''

``Kio?''

``Mi volas diri, kio se iu diras, ``okej, nun ĉiuj elterigas la
manikon super sia dekstra brako' kaj ĉiuj niajn estos kiel 'ups, mi
bedaŭras, ŝajnas ke mi estas spiono'—''

``\emph{Forgesu ke mi diris ion,}'' diris Drako, ŝvito subite fluante
ĉie sur lia korpo. Li bezonis distraĵon, rapide— ``Kaj kiel ni nomos
nin? La Scienco-manĝantoj?''

``Ne,'' diris la ombrita silueto malrapide. ``Tio ne sonas
bone\ldots''

Drako viŝis la humideco de sia fronto kun la maniko. Kion la Mastro de
la Tenebroj \emph{pensis}? Patro diris ke la Mastro de la Tenebroj
estis \emph{inteligenta!}

``Mi trovis!'' diris la ombrita siluento subite. ``Vi ne komprenos
nuntempe, sed tio taŭgas, kredu min.''

Tiam Drako akceptintus 'Maĉanta Malfojo', dum ĝi ŝanĝingus la
subjekton. ``Kio ĝi estas?''

Kaj starante inter la malpuraj skribotabloj en la neuzata klasĉambro,
en la karceroj de Herpŭrko, la verde iluminita silueto de Harry Potter
etendis la brakojn kaj dramece diris, ``Tiu tago markos la aŭroro
de\ldots la \emph{Bajezia Konspiro.}''

\begin{center}\rule{3in}{0.4pt}\end{center}

Silenta silueto treniĝis lace trans la koridoroj de Herpŭrko je la
direkcio de Korvungohejmo.

Harry estis irinta direkte de la kunveno kun Drako al vespermanĝo, kaj
restis al vespermanĝo preskaŭ sufiĉe longe por preskaŭ sufokiĝi pro
kelkaj tro rapidaj englutoj de manĝaĵo antaŭ foriri por atingi sian
liton.

Ne estis eĉ ankoraŭ je la 7a, sed estas jam de longe je la tempo de
iri al lito por Harry. Li konsciis lastnokte ke li ne kapablos uzi la
Returnilon de Tempo je sabato antaŭ ol la konkurso de legado estis jam
fininta. Sed li povis uzi la Returnilon de Tempo vendrede nokte, kaj
gajni tempon tiel. Do Harry devigis sin mem maldormi ĝis la 9a
vendrede, kiam la protektan ŝelo malfermiĝis, kaj tiam li uzis la kvar
restantajn horojn de la Returnilo de Tempo por reiri ke la 5a
posttagmeze kaj subite ekdormi. Li vekiĝis ĉirkaŭ la 2a sabade matene,
kiel plani, kaj senpaŭze legis dum la sekvaj dudek horoj\ldots kaj tio
ne eĉ sufiĉis . Kaj nun Harry devus iri dormi relative frue dum kelkaj
sekvaj tagoj, ĝis lia dorma ciklo refariĝos bonorde.

La portreto sur la pordo al Harry faris tiel stulta enigmon destinitan
al dek unu jaraĝaj infanoj, ke li respondis sen ke la vortoj eĉ pasis
trans sia konscia menso, kaj tiam Harry marŝis ŝaceliĝante laŭ la
stuparo al sia dormejo, metis sian piĵamon kaj falis sur sian liton.

Al li ŝajnis ke sia kuseno estis relative malglata.

Harry ĝemis. Li sidiĝis kontraŭvole, turniĝis sur la lito, kaj levis
sian kusonon.

Ĝi malkaŝis noton, du Galionojn el oro, kaj libron titolita
\emph{Oklumencio: La Arto de Kaŝi}.

Harry prenis la noton kaj ĝin legis :

\emph{Ve, vi akiris problemojn kaj rapide. Via patro ne povintus
konkuri kun vi.}

\emph {Vi al vi faris potencan malamikon. Skoldo havas la fidelecon,
admiron, kaj la timon de la tuta Domo Serpentimo. Vi ne povas fidi iun
ajn de tiu ĉi Domo de nun, ne gravas ĉu ili venas al vi amike aŭ
timinde.}

\emph{Ekde nun vi neniam devas renkonti la rigardon de Skoldo. Li
estas Legilimencisto kaj povas legi vian menson se vi tion faras. Mi
aldonis libron, kiu eble povas helpi vin lerni kiel protekti vin mem,
tamen estas ne multe, kion vi povas lerni sen tutoro. Kvankam, vi
povas esperi almenaŭ detekti trudon.}

\emph{Por ke vi povas trovi iom ekstran tempon por studi Oklumencion,
mi estas aldoninta 2 Galionojn. Tio estas la kosto de la respondojn
kaj hejmtaskojn por la klaso de Historio de Magio de unua jaro
(Profesoro Lornno donis la samajn testojn kaj samajn taskojn de kiam
li mortis). Viaj novaj amikoj, la ĝemeloj Wizle, devus kapabli vendi
kopion al vi. Nedirinde, vi ne devas esti vidita havante ĝin.}

\emph{Pri Profesoro Ciuro, mi scias iomete. Li estas Serpentimo, kaj
Profesoro de Defenco, kaj tio estas du poentoj kontraŭ li. Konsideru
singardeme ĉiujn konsilojn, kiujn al vi li donas, kaj diru nenion al
li, kion vi ne volas farigi konata.}

\emph{Dumbledore nur pretendas esti freneza. Li estas ege inteligenta,
kaj se vi daŭras iri en ŝranko kaj malaperi, li certe deduktos ke vi
havas la Mantelon de Nevidebleco, se li ne jam deduktis tion. Evitu
lin se eble, kaŝu la Mantelon de Nevidebleco ie sekura, (NE en via
haŭtpoŝo) ĉiufoje kiam vi ne povas eviti lin, kaj agu tre singardeme
je lia ĉeesto.}

\emph{Bonvolu, estu pli singardema en la estonteco, Harry Potter.}

- \emph{Paĉjo Kristanasko}

Harry fiksrigardis la noton.

Tiu ŝajnis esti bonan konsilon. Evidente Harry ne fraŭdos en la klaso
de Historio, eĉ se ili donis al li mortan simion kiel profesoro. Sed
la Legilimencio de Severus\ldots kiu ajn sendis tiun noton, konis
multe da gravajn, sekretajn aĵojn kaj intencis paroli al Harry pri
ili. La noto estis ankoraŭ avertanta lin kontraŭ la volo de Dumbledore
ŝetli la Mantelon, sed nuntempe Harry honeste havis neniun signon ĉu
tio estis malbona signo. Tio povus nur esti komprenebla eraro.

Ŝajnis esti ia komploto en Herpŭrko. Eble se Harry komparis historion
inter Dumbledore kaj le sendinto de la noto, li povus krei kunigita
bildo, kiu estus akurata? Kiel se ili ambaŭ konsentis pri io,
tiel\ldots

\ldots ne gravas\ldots

Harry metis ĉiu tiun ĉi en sian haŭtpoŝon, kaj plialtigis la Kvietilo
kaj tiris la kovrilo super sian kapon, kaj mortis.


\begin{center}\rule{3in}{0.4pt}\end{center}

Estis dimanĉo matene, kaj Harry estis manĝanta patkukojn en la
Grandega Ĉambrego, mordante akre kaj rapide, ekrigardante nervoze sian
brakhorloĝon preskaŭ ĉiuj sekundoj.

Estis je la 8a kaj du minutoj matene, kaj en ekzakte du horoj kaj unu
minuto, jam de unu semajno li unue vidis la Wizle'j kaj pasis tra la
kolono al la Patformo Naŭ kaj Tri-Kvaronoj.

Kaj penso aperis al li\ldots Harry ne sciis se tio estis valida
maniero pensi pri la universo, li ne plu sciis ion ajn, sed tio ŝajnis
ebla\ldots

Ke\ldots

\emph{Ne sufiĉe da interesaj aĵoj okazis al li dum la lasta semajno.}

Harry planis, kiam li estos fininta manĝi, iri direkte al sia ĉambro
kaj kaŝi sin en la malsupra nivelo de sia trunko kaj ne paroli kun iu
ajn ĝis la 10a kaj 3 matene.

Kaj tiam Harry vidis la ĝemeloj Wizle marŝante al li. Unu el ili
estis portanta ion kaŝita malantaŭ lia dorso.

Li devus krii kaj forkuri.

Li devus krii kaj forkuri.

Kio ajn ĝi estis\ldots tio povus vere estis\ldots

\ldots la \emph{granda fino}\ldots

Li vere devus nur krii kaj forkuri.

Kun la rezignita senso ke la universo venos kaj kaptos lin
\emph{ĉiaokaze}, Harry daŭris tranĉi patkukon kun sia forko kaj
tranĉilo. Li ne povis arigi energion. Tio estis la malgaja vero. Harry
sciis nun kiel homoj sentas kiam ili estas tro laca por forkuri, tro
laca por provi eskapi la destinon, kaj ili nur falas sur la plankon
kaj lasas la terure ungega kaj tentakla demono de la plej malhela
absimo tiri ilin al ilia abomena destino.

La Wizle proksimiĝis.

Kaj ankoraŭ pli proksima.

Harry manĝis alian pecon de patkuko.

La Wizle ĝemeloj alvenis al li, ridetante hele.

``Bonan tagon, Fred,'' Harry diris malsprita. Unu el la ĝemeloj
balancis la kapon. ``Bonan tagon, George.'' la alia ĝemelo balancis la
kapon.

``Vi ŝajnas laca,'' diris George.

``Vi devus kuraĝiĝi,'' diris Fred.

``Rigardu tion, kion ni havas por vi!''

Kaj George prenis, de malantaŭ la dorso de Fred—


Kuko kun dek du flamaj kandeloj.

Estis paŭzo, dum la Korvunga tablo rigardis ilin.

``Vi eraras,'' diris iu. ``Harry Potter estas naskita je la tridek
unua de jul—''

``*Li estas alvenanta*,'' diris grandega kava voĉo, kiu rompis ĉiujn
la konversaciojn kiel glavo de glacio. ``\emph{LA UNU, KIU DISŜIROS LA
VERAN—}''

Dumbledore estis saltinta desur sia trono kaj kuranta sur la Ĉefa
Tablo kiam li ekkaptis firme la virinon kiu estis elparolanta tiujn
abomenajn vortojn. Faŭkse aperis per ekbrilo, kaj ĉiuj el tiu ĉi tri
malaperis en fendo de fajro.

Estis ŝokita paŭzo\ldots

\ldots sekvita per kapoj turniĝantaj al la direkcion de Harry Potter.

``Mi ne faris tion,'' Harry diris per laca voĉo.

``Tiu estis \emph{profetaĵo!}'' iu ĉe la tablo siblis. ``Kaj mi vetas
ke ĝi temas pri \emph{vi!}''

Harry suspiris.

Li stariĝis de sia seĝo, plilaŭtigis sian voĉo, kaj diris tre laŭte
super la konversacioj, kiuj estis komencintaj, ``\emph{Ĝi ne temas pri
mi! Kompreneble! Mi ne estas alvenanta, mi jam estas tie ĉi!}''

Harry residiĝis.

La homoj, kiuj estis rigardantaj lin, turniĝis de li.

Iu alia je la tablo diris, ``Tial pri kiu ĝi temas!''

Kaj malakre, malgaje Harry ekkonsciis kiu \emph{ne estis} jam en
Herpŭrko.

Nomu tion diveno, sed Harry havis le senton ke malmorta Mastro de la
Tenebroj povus montri sin iutage post nelonge.

La konversacioj daŭris ĉirkaŭ li.

``Por ne mencii, disŝiri la veran \emph{kio?}''

``Mi pensas ke mi aŭdis Trelaŭne komencis diri ion, kio komencis per
'S' ĵus antaŭ kiam la direktoro ekkaptis ŝin.''

``Kiel\ldots spirito? Suno?''

``Se iu disŝiros la sunon, ni \emph{vere} havas problemojn!''

Tio ŝajnis relative malverŝajna por Harry, almenaŭ ke la mondo enhavas
timantajn aĵojn, kiuj aŭdis pri la ideojn de David Criswell pri
kolektado de steloj.

``Do,'' Harru diris per laca tono, ``tio okazas ĉiu dimanĉo matene, ĉu
ne?''

``Ne,'' diris studento, kiu eble povis esti en sepa jaro, ridetante,
``tio estas malprava.''

Harry ŝultrolevis. ``Ne gravas. Ĉu iu volas iom da naskitĝtaga kuko?''

``Sed ne estas je via naskitĝtago!'' diris la sama studento, kiu
obĵetis pli frue.

Tiu estis la repliko, kiu farigis Fred'n kaj George'n komenci ridi,
evidente.

Eĉ Harry sukcesis rideti.

Kiam oni servis la unuan tranĉaĵon al li, Harry diris, ``Mi havintis
\emph{vere longan semajnon.}''

\begin{center}\rule{3in}{0.4pt}\end{center}

Harry estis sidanta en la kaverna nivelo de sia trunko, kiu estis
ŝovfermita kaj ŝlosita, tiel ke neniu povis eniri, kun kovrilo tirita
super sia kapo, atendante ke la semajno estu fininta.

10:01.

10:02.

10:03, nur por esti certa.

10:04 kaj la unua semjno estis fininta.

Harry ekspiris spiron de kvietigo, kaj delikate fortiris la kovrilon
desur sia kapo.

Post kelkaj momentoj, li estis irinta en la hela suna lumo de sia
dormejo.

Baldaŭ, li estis en la Korvunga komuna ĉambro. Kelkaj homoj rigardis
lin, sed neniu diris ion ajn, aŭ provis paroli al li.

Harry trovis belan larĝan skribotablon, tiris komfortan seĝon, kaj
sidiĝis. De lia haŭtpoŝo, li prenis folion el papero, kaj krajonon.

Panjo kaj Paĉjo diris al Harry, ke en la baldaŭa estonteco, kvankam
ili komprenis lian entuziasmon pri forlasi hejmon kaj iri malproksime
de lia gepatroj, li devis skribi al ili ĉiusemajne sen mankoj, nur por
ke ili scias ke li estas vivanta, senvunda, kaj ne en malliberejo.

Harry fikse rigardis, malsupren, la malplenan folion el papero. Vidu\ldots

Post lasi liajn gepatrojn al la fervoja stacidomo, li\ldots

\ldots li renkontis knabon edukita per Darth Vader, amikiĝis kun la
tri plej fifamaj ŝercemuloj en Herpŭrko, renkontis Hermione'n, kaj
poste estis incidento kun la Ordiganta Ĉapelo\ldots

Lundo oni al li donis tempan maŝinon por solvi siajn problemojn de
dormo. Li akiris legendan Mantelon de Nevidebleco de nekonita
bonfaranto, savi sep Huflopufojn per stariĝi kontraŭ kvin timigantaj
pli maljunaj knaboj inter kiu iu minacis rompi liajn fingrojn,
konsciis ke li havis misteran malhelan flankon, lernis kiel ĵeti
\emph{Malvermirgu} en Ĉarma klaso, kaj komencis sian konkuradon kun
Hermione\ldots Je mardo, oni prezentis Astronomio, instruita per
Profesoro Aŭroro Sinistro kiu estis agrabla, kaj Historio de la magio,
instruita per fantomo kiu devus esti ekzorcita kaj anstataŭita per
legilo de kasedoj\ldots Merkredo, li estis nomita la plej danĝera
studento de la klasĉambro\ldots Ĵaŭdo, ne eĉ pensi pri Ĵaŭdo\ldots
Vendredo, la incidento en Pocia klaso, sekvita per lia minaco al la
Direktoro, sekvita per la Defenca Profesoro farigante lin esti frapita
en klaso, sekvita per la Defenca Profesoro malkovriĝante esti la plej
bonega homo, kiu estas ankoraŭ marŝanta sur la tero\ldots Sabato, li
perdis veton kaj faris sian unuan galantan rendevuon kaj komencis
reaĉeti Drako'n\ldots kaj poste tiumatene Profesoro Trelaŭne malaŭdita
profetaĵo kiu eble povas aŭ ne povas indiki ke senmorta Mastro de la
Tenebroj estas atakonta Herpŭrko'n.

Harry mense organizis la enhavon, kaj komencis skribi.

\emph{Kara Panjo kaj Paĉjo:}

\emph{Herpŭrko estas multe da amuzaĵoj. Mi lernis kiel malrespekti la
Duan Leĝon de termodinamiko en Ĉarma klaso, kaj mi renkontis knabinon
nomita Hermione, kiu povas legi pli rapide ol mi.}

\emph{Mi farus pli bone halti tie.}

\emph{Via amanta filo,}

\emph{Harry James Potter-Evans-Verres.}








