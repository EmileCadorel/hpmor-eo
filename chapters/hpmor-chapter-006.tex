\chapter{La sofismo de planado}


\lettrine{I}uj infanoj atendintus ĝis \emph{post} ilia unua vizito en la Diagon Aleo.

``Sako da elemento 79,'' Harry diris, antaŭ ol li retiris sian manon,
malplena, el la haŭtpoŝo el Moke-Haŭto.

Plimulto da infanoj atendintus almenaŭ ke ili havas ilian
\emph{bastonon} unue.

``Sako da \emph{okane},'' diris Harry. La peza sako da oro aperis en sian manon.

Harry elprenis la sakon, kaj tiam remetis ĝin en la haŭtpoŝon. Li
retiris sian manon, por remeti ĝin en la haŭtpoŝon denove, kaj li
diris, ``Sako da objektoj de ekonomiaj interŝanĝoj.'' Ĉi-foje, lia
mano eliris malplene.

``Redonu al mi la sakon kiun mi ĵus enmetis.'' eliris la sako da oro
ankoraŭ unu fojon.

Harry James Potter-Evans-Verres estis metinta la manon sur almenaŭ
unu magia objekto. Kial atendi?

``Profesorino McGonagall,'' Harry diris al la perpleksa sorĉistino,
kiu promenis malantaŭ li. ``Ĉu vi povas doni al mi du vortojn, unu
vorto por oro, kaj unu vorto por io alia kio ne estas mono, en lingvo
kiun mi ne konas? Sed ne diru al mi kiu estas kiu.''

``*Ahava* kaj \emph{zahav},'' diris Profesorino McGonagall. ``Tiu
estas Hebreo, kaj la alia vorto volas diri amo.''

``Dankon, Profesorino. Sako da \emph{ahava}.'' Malplena.

``Sako da \emph{zahav}.'' Kaj ĝi aperis en sian manon.

``Zahav estas oro?'' Harry demandis, kaj la Profesorino McGonagall
kapjesis.

Harry pripensis siajn kolektitajn eksperimentajn datumojn. Ĝi estis
nur la plej kruda kaj prepara peco de klopodo, sed tio estis sufiĉa
por subteni almenaŭ unu konkludon.

``\emph{Aaaaaaarrrg tio havas neniom da senco!}''

La sorĉistino apud li levis alte siajn brovojn. ``Problemoj, S-iĉo
Potter?''

``Mi nur pruvis ke ĉiuj hipotezoj, kiujn mi havis, estis malpravaj!
Kiel ĝi povas scii ke 'sako da 115 Galionoj' estas ĝuste sed ne 'sako
da 90 plus 25 Galionoj'? Ĝi povas konti sed ĝi ne povas adicii? Ĝi
povas kompreni nomojn, sed ne nomajn frazojn kiuj havas la saman
signifon? La persono kiu konstruis tiun verŝajne ne sciis paroli
Japane kaj mi ne scias paroli Hebree, do ĝi ne uzas rian scion kaj ĝi
ne uzas mian scion—'' Harry skuis sian manon senespere. ``La reguloj
ŝajnas iomete koheraj sed ili volas diri nenion!  Kaj mi ne eĉ
komencis demandi pri kiel haŭtpoŝo povas havi voĉan rekonon kaj
naturan lingvan komprenon, kiam la plej bonaj programistoj de
artefarita intelekto ne povas igi ke la plej rapidaj komputiloj faru
ĝin post tridek kvin jaroj de malfacila laboro,'' Harry anhelis
spirante, ``sed \emph{kio} okazas?''

``Magio,'' diris Profesorino McGonagall.

``Tiu estas nur \emph{vorto}! Eĉ post ol vi diris al mi tion, mi ne
povas fari iun ajn novan antaŭdiron! Tio estas ĝuste kiel diri
'flogistono' aŭ 'vitala impulso' aŭ 'emerĝo' aŭ 'komplikeco'!''

La sorĉistino en nigra robo ridis laŭte. ``Sed tio \emph{estas} magio
, S-iĉo Potter.''

Harry kliniĝis iomete. ``Respekte, Profesorino McGonagall, mi ne
certas ke vi komprenas tion, kion mi provas fari tie.''

``Respekte, S-iĉo Potter, mi certas ke ne. Krom se—tio estas simple
konjekto—vi provas konkeri la mondon?''

``Ne! Mi volas diri jes—nu, \emph{ne!}''

``Mi opinias ke mi devus esti alarmita, pro ke vi havas problemojn por
respondi al la demando.''

Harry malafable pensis pri la konferenco de Dartmouth pri Artefarita
Intelekto je 1956. Ĝi estis la unua konferenco pri la temo, tiu kiu
kreis la frazon ``Artefarita Intelekto''. Ili indentigis ĉefajn
problemojn kiel igi la komputilon kompreni lingvon, lerni, kaj
plibonigi sin mem. Ili sugestis, perfekte serioze, ke signifaj
progresoj pri ĉi tiuj problemoj povus esti faritaj per dek sciencistoj
kunlaborantaj dum du monatoj.

\emph{Ne. Kuraĝiĝu. Vi nur} komencas \emph{rigardi la problemon de
malimpliki ĉiujn la sekretojn de la Magio. Vi fakte ne} scias \emph{ĉu
estos malfacila fari tion en la daŭro de du monatoj}.

``Kaj vi \emph{vere} neniam aŭdis pri aliaj sorĉistoj, kiuj pripensis
tiajn demandojn aŭ faris tiajn sciencajn eksperimentojn?''  Harry
demandis denove. Tio ŝajnis tiel evidenta al li.

Kaj ankoraŭ, necesis pli ol ducent jaroj post la invento de la scienca
metodo por ke kelkaj Muglaj sciencistoj pensu sisteme esplori kiajn
frazojn kvarjara knabo povis aŭ ne kompreni. La psikologio de la
evoluado de lingvistiko povintus esti malkovrita je la dekoka
jarcento, principe, sed neniu eĉ pensis rigardi ĝis la dudeka
jarcento. Do vi ne povis vere kulpigi la multe pli malgrandan sorĉan
mondon pro ke ili ne esploris la Ekstraktan Ĉarmon.

Profesorino McGonagall pintigis siajn lipojn, kaj poste levis siajn
ŝultrojn. ``Mi ankoraŭ ne certas pri tio, kion vi volas diri per
'scienca eksperimento', S-iĉo Potter. Kiel mi diris, mi vidis Mugle
naskitajn lernantojn kiuj provis funkciigi la Muglan sciencon je
Herpŭrko, kaj homoj inventas novajn Ĉarmojn kaj Pociojn ĉiujare.''

Harry kapneis. ``Teknologio tute ne estas la sama aĵo ol scienco. Kaj
provi multajn malsimilajn manierojn por fari ion ne estas la sama ol
eksperimenti por trovi la regulojn.'' Estis multe da homoj kiuj provis
inventi flugantajn maŝinojn per provi multajn aĵojn kun flugilojn, sed
nur la fratoj Wright konstruis ventotunelon por mezuri levoforton\ldots
``Hm, Kiom da Mugle edukitaj infanoj alvenas al Herpŭrko ĉiujare?''

``Eble, dek aŭ pli''

Harry mankis ŝtupon kaj preskaŭ stumblis sur sia propra piedo. ``\emph{Dek?}''

La Mugla mondo havis pli ol ses miliardoj da loĝantoj. Se vi estus
la sola en unu miliono da homoj, estus sep aliaj vi en Londono kaj mil
pliaj en Ĉinio. Estis fatala ke la Mugla popolo produktu kelkajn dek
unu jarajn infanojn, kiuj sciis kalkulon—Harry sciis ke li ne estis la
sola. Li renkontis mirinfanojn en matemakikaj konkursoj kiuj verŝajne
pasigis \emph{tutajn tagojn} praktiki matematikajn problemojn kaj kiuj
neniam legis ian ajn sciencfikcian libron kaj kiuj \emph{tute}
kolapsos antaŭ pubereco kaj neniam plenumos ion ajn dum iliaj estontaj
vivoj ĉar ili nur praktikis konatajn teknikojn anstataŭ lerni pensi
kreeme. (Harry estis iomete malbona venkito.)

Sed\ldots la sorĉa mondo\ldots

Dek Mugle edukitaj infanoj ĉiujare, kiuj ĉiuj finis ilian Muglan
edukadon je la aĝo de dek unu jaroj? Kaj la Profesorino McGonagall
eble povis esti partia, tamen ŝi pretendis ke Herpŭrko estis la
plej larĝa kaj la plej elstara sorĉa lernejo en la tuta mondo\ldots kaj ĝi nur
instruis ĝis la aĝo de dek sep jaroj.

Profesorino McGonagall sendube konis ĉiujn detalojn pri kiel oni
ŝanĝiĝis en katon. Sed ŝi ŝajnis laŭvorte neniam aŭdi pri la scienca
metodo. Por ŝi, tiu nur estis Mugla magio. Kaj ŝi eĉ ne ŝajnis kurioza
pri kiaj sekretoj eble povus esti kaŝitaj malantaŭ la kompreno de
natura lingvo de la Ekstrakta Ĉarmo.

Tio lasis du eblojn, vere.

Eblo unua: Magio estis tiel opaka, implikita, kaj nepenetrebla, ke
eĉ se sorĉistiĉoj kaj sorĉistinoj provintis kiel eble plej bone por
kompreni, ili malmulte progresis aŭ tute ne progresis kaj poste
rezignis; kaj Harry ne faros pli bone.

\emph{Aŭ}

Harry kraketigis siajn fingrojn decideme, sed ili nur faris ian kvietan
klakon, anstataŭ resoni malbonaŭgure kontraŭ la muroj de la Diagon
Aleo.

Eblo dua: Li konkeros la mondon.

Finfine. Eble ne tuj.

Tia afero kelkfoje daŭris pli ol du monatoj. La Mugla scienco ne iris al
la luno ĵus unua semajno post Galileo.

Sed Harry ne povis haltigi la gigentan rideton, kiu streĉigis liajn
lipojn tiel forte ke ili komencis dolorigi.

Harry ĉiam timis fini kiel unu el tiuj infanoj kiuj neniam plenumis
ion ajn kaj pasigis ilian vivon fanfaroni pri kiom anticipe ili estis je
la aĝo de dek jaroj. Sed plimulto da plenkreskulaj geniuloj neniam plenumis
ion ankaŭ. Estis probable mil homoj tiel inteligenta kiel Enstein por
ĉiu efektiva Enstein en la historio. Ĉar tiuj ĉi geniuloj ne havis en
iliaj manoj la unu aĵon, kiun vi tute bezonis por atingi
grandecon. Ili neniam trovis gravan problemon.

\emph{Vi estas mian nun,} Harry pensis al la muroj de la Diagon
Aleo, kaj al ĉiuj butikoj kaj al ĉiuj objektoj, ĉiuj komercistoj kaj
klientoj; kaj al ĉiuj landoj kaj homoj de la sorĉa Britio, kaj al la
tute pli granda sorĉa mondo; kaj al la multe pli granda universo pri
kiu la Muglaj sciencistoj komprenis tiom malpli ol ili
kredis. \emph{Mi, Harry James Potter-Evans-Verres, deklaras nun ke ĉi
tiu teritorio apartenas al mi por la nomo de la scienco.}

Fulmo kaj tondro tute malsukcesis ekbrili kaj eksplodi en la sennebula ĉielo.

``Kio ridetigas vin?'' enketis Profesorino McGonagall singarde kaj lace.

``Mi min demandis se estus sorĉo por krei fulman ekbrilon en la malantaŭplano
ĉiufoje kiam mi faras malbonaŭguran decidon,'' klarigis Harry. Li
estis zorge memoranta la ĝustajn vortojn de sia malbonaŭgura decido
por ke la futuraj historiaj libroj enhavu ĝin ĝuste.

``Mi havas la klaran senton, ke mi devus fari ion pri ĉi tio,''
suspiris Profesorino McGonagall.

``Ignoru tion, tio foriros. Ho, grandozia'' Harry metis siajn pensojn
pri la konkero de la mondo provizore flanke, kaj marŝis al butiko kun
malferma signo, kaj la Profesorino McGonagall sekvis.

\begin{center}\rule{3in}{0.4pt}\end{center}

Harry estis nun aĉetinta siajn pociajn ingrediencojn kaj sian
kaldronon, kaj, ho, kelkajn pliajn aferojn, kiel objektoj kiuj ŝajnis
esti bonaj aĵoj por enmeti en la haŭtpoŝon de Harry (aka Mokeo Supera
Haŭtpoŝo QX31 kun Etenda Ĉarmo, Ekstrakta Ĉarmo kaj Larĝigeblan
Lipon). Saĝaj, prudentaj aĉetaĵoj.

Harry reale ne komprenis kial Profesorino McGonagall aspektis
tiel \emph{suspektema}.

Tiam, Harry estis en butiko, kiu estis sufiĉe multekosta por situi
en la ĉefa torda strato de la Diagon Aleo.  La butiko havis
malfermitan fasadon kun varoj aranĝitaj sur oblikvaj lignaj vicoj,
garditaj nur per maldikaj grizaj briloj kaj vendistino aspektante juna
kaj kiu portis tre mallongigitan version de la robo de sorĉistino,
kiu elmontris ŝiajn genuojn kaj kubutojn.

Harry estis ekzamenanta la sorĉan ekvivalenton de ilaro de unua helpo,
nomita la Saniganta Ilaro de Krizo Plus. Estis du mem-streĉantaj
garotoj. Unu pocio de stabiligo, kiu malrapidigis sangan perdon kaj
preventis ŝokon. Unu injektilo, kies enhavo aspektis kiel likva fajro,
kaj kiu supozeble draste malrapidigis cirkuladon ĉirkaŭ traktita loko
dum ĝi daŭrigis oksigenadon de la sango ĝis tri minutoj, se vi bezonus
preventi ke veneno disvastiĝu tra la korpo. Kelkaj blankaj vestoj kiuj
povis esti uzitaj por envolvi parton de la korpo por provizore
sensentigi doloron.  Kaj kelkaj aliaj objektoj kiujn Harry tute
malsukcesis kompreni, kiel la ``Kuracado kontraŭ Ekspozicio al
Aberento$^{\textit{a:\ref{nomoj:aberento}}}$'', kiu aspektis kaj odoris kiel
ordinara ĉokolado. Aŭ la ``Ilo kontraŭ konfuziganta brido'', kiu
aspektis kiel eta tremanta ovo kaj portis afiŝon montrante kiel enmeti
ĝin en la nazotruon de iu.

``Nepra aĉeto por kvin Galionoj, ĉu vi konsentas ?'' Harry diris al
Profesorino McGonagall, kaj la adoleskanta vendistino vagante
proksime, kapjesis avide.

Harry atendis de la Profesorino McGonagall ke ŝi faru ian aprobantan
rimarkon pri lia prudento kaj preteco.

Tio, kion li akiris anstataŭe, povis nur esti priskribi kiel la Malbona Okulo.

``Kaj simple \emph{kial},'' Profesorino McGonagall diris forte
skeptika, ``Ĉu vi atendas bezoni sanigantan ilaron, junulo?'' (Post
la bedaŭrinda evento en la butiko de Pocioj, Profesorino McGonagall
provis eviti diri ``S-iĉo Potter'' kiam iu alia estis proksime.)

La buŝo de Harry malfermiĝis kaj refermiĝis. ``Mi ne atendas bezoni
ĝin! Tio estas nur por la okazo!''

``Por le okazo ke \emph{kio}?''

La okuloj de Harry plilarĝiĝis. ``Vi pensas ke mi planas fari ion
danĝeran, kaj ke \emph{tio estas} kial mi volas sanigantan ilaron?''

Iu rigardo de malagrabla suspekto kaj ironia nekredemo estis la respondo.

``Granda Skoto!'' diris Harry. (Tio estis esprimo, kion li lernis de
la freneza sciencisto Dok Brown en \emph{Reen en la Estonteco}) ``Ĉu vi
ankaŭ pensis tion kiam mi aĉetis la Pocion de Falanta Plumo, la
Brankherbon, kaj la botelon da manĝaĵo kaj akvo en piloloj?''

``Jes.''

Harry skuis sian kapon mirigite. ``Nur kian planon vi pensas, ke mi
\emph{faros}, tial?''

``Mi ne scias,'' Profesorino McGonagall diris malhele, ``sed ĝi finos
aŭ kun vi livranta tunon da arĝento al Gringoto, aŭ kun la dominado de
la mondo.''

``Dominado de la mondo estas tiel malbela frazo. Mi preferas nomi ĝin
optimumigo de la mondo.''

Tio rideginda ŝerco malsukcesis trangviligi la sorĉistinon, kiu ĵetis
al Harry la Rigardon de Pereo.

``Ŭaŭ,'' Harry diris, kiam li rimarkis ke ŝi estis serioza. ``Vi vere
pensas tion. Vi vere pensas ke mi planas fari ion danĝeran.''

``Jes.''

``Kiel se tio estis la sola kialo, kiu povus igi iun aĉeti sanigantan
ilaron? Ne prenu tion malĝuste, Profesorino McGonagall, sed \emph{kun
kiaj frenezaj infanoj vi kutimas trakti?}''

``Grifindoro'j,'' kraĉis la Profesorino McGonagall. La vorto portis
ŝarĝon da amareco kaj malespero, kiu falis kiel eterna malbeno al ĉiu
juna entouziasmo kaj bona humoro.

``Vicdirektorino Profesorino Minerva McGonagall,'' Harry diris,
metante siajn manojn firme kontraŭ la koksoj. ``Mi ne iros en
Grifindoro'n—~''

Tiam, la Vicdirektorino ekkriis ion pri ke se li \emph{estus} ordigita
en Grifindoro, ŝi malkovrus kiel mortigi ĉapelon, iu stranga rimarko,
kiun Harry lasis pasi sen komenti, kvankam la vendistino ŝajnis havi
subitan tusadon.

``—Mi estos en Korvungo. Kaj se vi vere opinias ke mi planas fari ion
danĝeran, tial, honeste, vi \emph{tute ne} komprenas min. Mi ne
\emph{ŝatas} danĝeron, tio estas timiga. Mi estas \emph{prudenta}. Mi
estas \emph{singardema}. Mi estas preparanta por neantaŭviditaj
eventualaĵoj. Kiel mia gepatroj kutimis kanti al mi: \emph{Estu
preparita! Tio estas la marŝkantado de Knaba Esploristo!  Estu
preparita! Kiel tra la vivo, vi marŝas! Ne estu nervoza, ne estu
konsternita, ne estu timiga—estu preparita!}''

(La gepatroj de Harry fakte nur kantis al li tiujn ĉi specifajn
liniojn de la kantado de Tom Lehrer, kaj Harry estis beate nekonscia
pri la resto.)

La sinteno de Profesorino McGonagall iomete moliĝis—kvankam precipe
kiam Harry diris ke li iros en Korvungo. ``Kontraŭ kiaj
\emph{eventualaĵoj} vi imagas ke tiu ilaro povas preparigi vin,
\emph{junulo}?''

``Kontraŭ ke iu el miaj samklasanoj estu mordita per horora monstro,
kaj dum mi freneze serĉas en mia haŭtpoŝo ion, kio povas helpi rin, ri
rigardas min malĝoje kaj kun siaj lastaj spiroj diras, 'Kial vi ne
estis preparita?' kaj poste ri mortas, kaj mi scias kiam siaj okuloj
fermiĝas ke ri neniam pardonos min—''

Harry aŭdis la vendistinon anheli, kaj li rigardis supren por vidi ke
ŝi fiksrigardis lin kun la lipoj premitaj kaj tremantaj. Kaj poste la
junulino turniĝis kaj foriris al pli profunda angulo de la butiko.

\emph{Kio\ldots?}

Profesorino McGonagall malleviĝis, kaj prenis la manon de Harry en la
sian, delikate sed firme, kaj tiris Harry'n eksteren sur la ĉefan straton
de la Diagon Aleo, kaj kondukis lin en strateton inter du butikoj
kiuj estis pavimitaj de malpuraj brikoj kaj finiĝis en muro el solida
nigra tero.

``*Kvietus*'' ŝi diris, kaj ekrano el silento aperis ĉirkaŭ ili,
blokante ĉiujn bruojn de la strato.

\emph{Kion mi misfaris\ldots}

Profesorino McGonagall turniĝis por rigardi Harry'n. Ŝi ne portis
plenkreskulan riproĉan vizaĝon, sed sia esprimo estis plata,
kontrolita. ``Vi devas memori, S-iĉo Potter,'' ŝi diris, ``ke estis
milito en tiu lando ĵus antaŭ dek jaroj. Ĉiuj perdis iun, kaj paroli
pri amikoj mortantaj en viaj brakoj—ne estis tre lerta ago.''

``Mi—Mi ne volis —'' La infereco falis kiel ŝtono sur la rimarkeble
vigla imago de Harry. Li parolis pri iu spiranta siajn lastajn
spirojn—kaj poste la vendistino forkuris—kaj la milito finis antaŭ
dek jaroj, do la knabino verŝajne havis maksimume ok aŭ naŭ jaroj,
kiam, kiam, ``Mi bedaŭras, mi ne volis\ldots'' Harry eklarmis, kaj
forturniĝis el la rigardo de la pli aĝa sorĉistino, sed estis muro de
tero kiu blokis la vojon kaj li ne ankoraŭ havis sian bastonon. ``Mi
bedaŭras, Mi bedaŭras, mi \emph{bedaŭras!}''

Peza spirado eliris malantaŭ li. ``Mi scias ke vi estas, S-iĉo
Potter.''

Harry aŭdacis rigardi malantaŭ si. Profesorino McGonagall nur ŝajnis
malĝoja, nun. ``Mi bedaŭras,'' Harry diris denove, sentante sin
mizera. ``Ĉu io kiel tio okazis al—'' kaj poste Harry fermis siajn
lipojn kaj kovris sian buŝon kun la mano por fari bonan mezuron.

La vizaĝo de la pli aĝa sorĉistino iĝis iom pli malĝoja. ``Vi devas
lerni pripensi antaŭ paroli, S-iĉo Potter, aŭ alie vi iros tra la vivo
sen multe da amikoj. Tio estas la sorto por multaj homoj en
Korvungo, kaj mi esperas ke tio ne estos la via.''

Harry nur volis forkuri. Li volis elpreni bastonon kaj malaperigi la
tutan aferon el la memoro de Profesorino McGonagall, retroviĝi ekster
la butiko kun ŝi denove, \emph{fari ke tio neniam okazu}—

``Sed la respondo al via demando, S-iĉo Potter, ne, nenio kiel
\emph{tio} iam ajn okazis al mi. Certe, mi vidis amikojn spirantajn
iliajn lastajn suspirojn, ses aŭ sepfoje. Sed neniu el ili iam ajn
malbenis min dum ili mortis, kaj neniam mi pensis ke ili ne pardonus
min. Kial vi \emph{diras} ion tiel, S-iĉo Potter? Kial vi eĉ
\emph{pensis} tion?''

``Mi, Mi, Mi,'' Harry glutis. ``Estas nur ke mi ĉiam provas imagi la
pli malbonan aĵon, kiu povas okazi,'' kaj eble ke li ankaŭ estis
ŝercanta iomete sed, li preferus mordi sian langon anstataŭ diri tion
nun.

``Kio?'' diris Profesorino McGonagall. ``Sed \emph{Kial}?''

``Tiel, mi povas malhelpi ĝin okazi!''

``S-iĉo Potter\ldots'' La voĉo de la pli aĝa sorĉistino
elreliĝis. Poste ŝi suspiris kaj genuiĝis apud li. ``S-iĉo Potter'' ŝi
diris delikate nun, ``Tio ne estas via respondeco zorgi pri la
lernantoj en Herpŭrko. Tio estas la mia. Mi ne lasos ion ajn okazi al
vi aŭ al iu ajn alia. Herpŭrko estas la plej sekura ejo por magiaj
infanoj en la tuta sorĉa mondo, kaj S-ino
Pimfito$^{\textit{n:\ref{nomoj:pimfito}}}$ havas kompletan oficejon de
kuracisto. Vi tute ne bezonos sanigantan ilaron, konservu viajn kvin
Galionojn.''

``Sed mi \emph{zorgas!}'' Harry ekkriis. ``*Nenie estas perfekte
sekura! kaj kio se miaj gepatroj havus koran infarkton aŭ estus prenitaj en
akcidento kiam mi reiros hejmen dum Kristnasko—S-ino Pimfito ne estos
tie, mi bezonas mian propran sanigantan ilaron—''

``*Kio!* en la nomo de Merlino\ldots'' Profesorino McGonagall
diris. Ŝi stariĝis, kaj rigardis malsupren al Harry, aspektante ŝirita
inter enuo kaj zorgo. ``Ne estas utile pensi pri tiel teruraj aferoj,
S-iĉo Potter!''

La esprimo de Harry ŝanĝiĝis maldolĉe, post kiam li aŭdis tion. ``Jes,
estas utile! Se vi pensas ke vi nur ne estos vundita, vi finos per vundi
aliajn homojn!''

Profesorino McGonagall malfermis sian buŝon, poste fermis ĝin. La
sorĉistino frotis la randon de sia nazo, aspektante
pripensanta. ``S-iĉo Potter\ldots Se mi proponus aŭskulti vin dum
momento\ldots ĉu estas io, pri kio vi volas paroli?''

``Kiel kio?''

``Kiel, kial vi estas konvinkita ke vi devu esti singarda kontraŭ
teruraj aferoj, kiuj povus okazi al vi.''

Harry fiksrigardis ŝin konfuze. Tio estis memevidenta
aksiomo. ``Nu\ldots'' Harry diris malrapide. Li provis organizi siajn
pensojn. Kiel li povis komprenigi siajn pensojn al sorĉa Profesorino,
kiam ŝi eĉ ne konis la bazojn? ``Muglaj esploritoj trovis ke homoj
estis tre optimismaj, en komparo kun la realo. Kiel kiam ili diras ke
io prenos du tagojn kaj ĝi fakte prenas dek tagojn, aŭ kiam ili diras
ke ĝi prenos du monatojn kaj ĝi prenos pli ol tridek kvin
jarojn. Ekzemple, en eksperimento, ili demandis al studentoj la
momenton post kiu ili certis al 50\%, 75\% kaj 99\% ke ili estos
fininta fari iliajn hejmtaskojn, kaj nur 13\%, 19\% kaj 45\% el la
studentoj finis post ĉi tiuj momentoj. Kaj ili malkovris ke la kialo
estis ke kiam ili demandis al iu grupo ilian taksadon de la plej bona
okazo, se ĉio okazis kiel eble plej bone, kaj al alia grupo ilian
taksadon de averaĝa okazo se ĉio okazis kiel kutime. Ili akiris
respondojn, kiuj estis statike nedistingeblaj. Vidu, se vi demandas al
iu kion ri atendas kiel la normala okazo, ri bildigos tion, kio
aspektas kiel la linio de maksimuma problablo je ĉiuj paŝoj laŭ la
vojo—ĉio iranta laŭ la plano, sen surprizo. Sed fakte, ĉar pli ol la
duono da la studentoj ne estis fininta je la ĝusta momento, post kiu
ili opiniis ke ili certe estus fininta je 99\%, la realo kutime
alportas rezulton iom pli malbona ol la `plej malbona scenaro'. Tio
estas nomata la sofismo de planado, kaj la plej bona maniero por
kontraŭi ĝin estas demandi kiom da tempo ĝi prenis kiam vi lastfoje
provis fari ĝin. Tio estas nomata uzi la eksteran vidon anstataŭ la
interna vido. Sed kiam vi faras ion novan kaj ne povas uzi eksteran
vidon, vi devas esti vere, vere, vere pesimisma. Kiel, tiel pesimisma
ke la realo fakte iĝos pli bona ol tio, kion vi atendis, tiel ofte
kiel ĝi iĝos malpli bona. Tio estas fakte \emph{vere malfacila} esti
\emph{tiel} pesimisma ke vi havu decan ŝancon troi la realon. Tial mi
faris tiun ĉi grandan klopodon por esti sinistra kaj mi imagis iun el
miaj samklasanoj mordita, sed tio kio fakte okazos, estas ke la
transvivantaj Morto-Manĝantoj atakos la tutan lernejon por kapti
min. Sed laŭ pli feliĉa noto—''

``Haltu,'' diris Profesorino McGonagall.

Harry haltis. Li estis dironta ke verdire ili sciis ke la Mastro de la
Tenebroj ne atakos pro tio ke li estis morta.

``Mi opinias ke mi eble ne estis klara,'' la sorĉistino diris, ŝia
preciza skota voĉo sonante eĉ pli zorga. ``Ĉu io ajn okazis al
\emph{vi persone}, io kio timigas vin, S-iĉo Potter?''

``Kio okazas al mi estas nur anekdota pruvo,'' Harry
klarigis. ``Tio ne alportas la saman pezon ol replikita, revizita inter
paroj, ĵurnala artikolo pri kontrolita esploro kun hazarda atribuo,
multaj subuloj, larĝa efika amplekso kaj forta statistika senco.''

Profesorino McGonagall pinĉis la pinton de sia nazo, enspiris, kaj
elspiris. ``Mi tamen ŝatus aŭdi tion,'' ŝi diris.

``Hm\ldots'' Harry diris. li prenis profundan spiron. ``Estis kelkaj
rabantuloj en nia kvartalo, kaj mia patrino petis de mi redoni
kaserolon, kiun ŝi pruntis de najbaro, de du stratoj for, kaj mi diris
ke mi ne volis ĉar mi eble estus rabita, kaj ŝi diris, 'Harry ne diru
aĵon kiel tio!' Kvazaŭ pensi pri tio povis fari ke ĝi okazu, do se mi
ne parolus pri tio, mi estus sekura. Mi provis klarigi kial mi ne
estis trankvila, kaj ŝi igis min redoni la kaserolon ĉiaokaze. Mi
estis tro juna por scii kiel statistike malverŝajna estis ke rabantulo
celis min, sed mi estis sufiĉe maljuna por scii ke ne pensi pri io ne
malhelpis ĝin okazi, do mi vere estis timanta.''

``Nenio alia?'' Profesorino McGonagall diris post mallonga paŭzo, kiam
estis klara ke Harry estis fininta. ``Ne estas io \emph{alia}, kio
okazis al vi?''

``Mi scias ke tio ne \emph{ŝajnas} multa,'' Harry defendis. ``Sed tio
estas nur unu el tiuj krizaj momentoj de la vivo, ĉu vi komprenas? Mi
volas diri, mi \emph{sciis} ke ne pensi pri io ne malhelpis ĝin okazi,
mi \emph{sciis} tion, sed mi povis vidi ke Panjo vere ne pensis
tiel.''  Harry haltis, batalante kontraŭ la kolero kiu komencis kreski
denove, kiam li pensis pri tio. ``Ŝi \emph{ne volis aŭskulti}. Mi
provis diri al ŝi, mi \emph{petis} ŝin ne sendi min eksteren, kaj ŝi
\emph{ridis}. Ŝi traktis ĉion kion mi diris, kiel ia granda
ŝerco\ldots'' Harry ree devigis la nigran koleregon
malplialtiĝi. ``Estis kiam mi konsciiĝis ke ĉiuj personoj, kiuj estis
supozitaj protekti min estis fakte frenezaj, kaj ke ili ne aŭskultos
min, ne gravas kiom multe mi petegis ilin, kaj ke mi neniam povos fidi
ilin por igi ĉion ĝuste.'' Kelkfoje, bonaj intencoj ne sufiĉis,
kelkfoje vi devis esti sana\ldots

Estis longa silento.

Harry prenis tempon por profunde spiri kaj trankviliĝi. Estis neniu
kialo fariĝi kolera.  \emph{Ĉiuj} gepatroj estis tiel, \emph{neniuj}
gepatroj malfierigus sin sufiĉe por meti sin je la sama nivelo kiel
infano kaj aŭskulti rin, siaj genetikaj gepatroj ne estintus
malsamaj. Racieco estis eta sparko en la nokto, infinitezima rara
escepto por regi la frenezecon, do estis senutila koleri.

Harry ne ŝatis sin, kiam li estis kolera.

``Dankon por konigi tion al mi, S-iĉo Potter,'' diris Profesorino
McGonagall post momento. Estis distrita esprimo sur ŝia vizaĝo
(preskaŭ ekzakte la sama esprimo kiu aperis sur la vizaĝo de Harry,
kiam li eksperimentis la haŭtpoŝon, se Harry vidintis sin en spegulo
por konscii tion). ``Mi devos pensi pri tio.'' Ŝi turnis sin al la
buŝo de la aleo, kaj levis sian bastonon—

``Hm,'' Harry diris, ``Ĉu ni povas iri akiri la resanigantan ilaron nun?''

La sorĉistino paŭzis, kaj rigardis malantaŭen al li konstante. ``Kaj
se mi diras ne—ke ĝi estas tro multekosta kaj ke vi ne bezonas
ĝin—tial kio?''

La vizaĝo de Harry tordiĝis amare. ``Ekzakte tio, kion vi pensas
Profesorino McGonagall. \emph{Ekzakte} tio, kion vi pensas. Mi
konkludos ke vi estas alia freneza plenkreskulo kun kiu mi ne povas
paroli, kaj mi komencos plani kiel akiri resanigantan ilaron
ĉiaokaze.''

``Mi estas la gardanto dum tiu ekskurso,'' Profesorino McGonagall
diris kun nuanco da danĝero. ``Mi ne \emph{permesos} ke vi puŝu
min.''

``Mi komprenas,'' Harry diris. Li gardis la rankoron ekster sia voĉo,
kaj ne diris ion ajn pri la aliaj aferoj kiuj eniris sian
spiriton. Profesorino McGonagall diris ke li pripensu antaŭ paroli. Li
verŝajne ne memoros tion morgaŭ, sed li povis almenaŭ tion memori dum
kvin minutoj.

La bastono de la sorĉistino faris maldikan cirklon en ŝia mano, kaj la
bruoj de la Diagon Aleo reaperis. ``Bonege, junulo,'' ŝi
diris. ``Ni eliru akiri tiun sanigantan ilaron.''

La makzelo de Harry falis pro surprizo. Poste li rapidis malantaŭ ŝi,
preskaŭ stumblante pro la subita hasto.

\begin{center}\rule{3in}{0.4pt}\end{center}

La butiko estis la sama ol kiam li eliris ĝin, rekoneblaj kaj
nerekoneblaj objektoj ankoraŭ estis kuŝitaj sur la oblikva ligna
prezentaĵo, la griza brilo ankoraŭ protektante kaj la vendistino estis
denove al sia malnova pozicio. La vendistino rigardis supren dum ili
proksmiĝis, sia vizaĝo prezentante surprizon.

``Mi bedaŭras,'' ŝi diris kiam ili estis proksimaj, kaj Harry parolis
proksimume je la sama momento, ``Mi pardonpetas por—''

Ili haltis kaj rigardis unu la alian, kaj post la vendistino ridis
iomete. ``Mi ne volis akirigi al vi malfacilaĵojn kun Profesorino
McGonagall,'' ŝi diris. Ŝia voĉo mallaŭtiĝis intence. ``Mi esperas ke
ŝi ne estis \emph{tro} abomena kun vi.''

``\emph{Della!}'' diris Profesorino McGonagall, sonante skandaliĝita.

``Sako da oro,'' Harry diris al sia haŭtpoŝo, kaj li rigardis denove
supren al la vendistino dum li kontis kvin Galionojn. ``Ne zorgu, mi
komprenas ke ŝi estas abomena kun mi ĉar ŝi amas min.''

Li kontis kvin Galionojn kaj donis ilin al la vendistino dum la
Profesorino McGonagall balbutis ion malgravan. ``Unu Sanigantan
Ilaron de Krizon Plus, bonvolu.''

Estis iom maltrankviliganta vidi kiel la Larĝigebla Lipo englutis la
medikan ilaron kies grando estis sama ol tiu de dokumentujo. Harry ne
povis malhelpi sin demandi kio okazus se li mem provis grimpi en la
haŭtpoŝon, konsiderante ke supozeble nur la persono, kiu metis ion
ene, kapablis eltiri ĝin.

Kiam la haŭtpoŝo estis fininta\ldots manĝi\ldots sian malfacile
gajnitan aĉetaĵon, Harry ĵuris ke li aŭdis etan ruktan sonon poste.
Tio \emph{devis} esti sorĉita intence. La alternativa hipotezo estis
tro terura por esti pripensita\ldots fakte Harry ne povis eĉ
\emph{pensi} pri iu ajn alternativa hipotezo. Harry rerigardis la
Profesorinon, dum ili komencis marŝi trans la Diagon Aleo unu fojon
plian. ``Kien ni iru sekve?''

Profesorino McGonagall fingrmontris butikon, kiu aspektis kvazaŭ ĝi estis
farita el karno anstataŭ brikoj kaj estis kovrita per felo anstataŭ
farbo. ``Etaj dombestoj estas permesitaj en Herpŭrko—vi povas havi
strigon por sendi leterojn, ekzemple—''

``Ĉu mi povas pagi Knuton aŭ ion kaj lui strigon kiam mi bezonas sendi
poŝtaĵon?''

``Jes,'' diris Profesorino McGonagall.

``Tial mi pensas emfaze \emph{ne}.''

Profesorino McGonagall kapjesis, kiel por marki punkton. ``Ĉu mi povas
demandi al vi kial ne?''

``Mi havis ŝtonan dombeston unufoje. Ĝi mortis.''

``Vi ne opinias ke vi povas varti dombeston?''

``Mi \emph{povas},'' Harry diris, ``sed mi finos obsedita la tutan
tagon pri ĉu mi memoris manĝigi ĝin ĉi tiun tagon, aŭ ĉu ĝi estas
malrapide mortanta pro malsato, demandante kie ĝia mastro estas kaj
kial estas neniu manĝaĵo.''

``Tiu kompatinda strigo,'' la pli aĝa sorĉistino diris kun dolĉa
voĉo. ``Delasita tiel. Mi min demandas, kion ĝi farus.''

``Nu, Mi imagas ke ĝi iĝos vere malsata kaj komancos provi eliri la
kaĝon per ungoj aŭ io alia, tamen ĝi verŝajne ne havus tiom da ŝanco
kun tio—'' Harry haltis rapide.

La sorĉistino daŭrigis, ankoraŭ kun dolĉa voĉo. ``Kaj kio okazos al ĝi
poste?''

``Pardonu min,'' Harry diris, kaj li atingis la profesorinon por preni
sian manon, delikate sed firme, kaj gvidis ŝin al alia aleo; ili
evitis tiom da bondezirantoj pli frue ke la procezo fariĝis
nepercetebla rutino. ``Bonvolu, ĵetu la silentigan sorĉon.''

``*Kvietus.*''

La voĉo de Harry estis tremanta. ``Tiu strigo \emph{ne} reprezentas
min, miaj gepatroj \emph{neniam} ŝlosis min en ŝranko kaj neniam lasis
min malsatiĝi, mi \emph{ne} havas timon de forlasiteco kaj mi
\emph{ne ŝatas la tendencon de viaj pensoj, Profesorino McGonagall!}''

La sorĉistino rigardis malsupren al li grave. ``Kaj kiaj pensoj, tiuj estus, S-iĉo Potter?''

``Vi pensas ke mi estis,'' Harry havis problemojn por diri tion, ``Mi
estis \emph{mistraktita}?''

``Ĉu vi estis?''

``\emph{Ne!} Harry ekkriis. ``Ne, Mi neniam estis! Ĉu vi pensas ke mi
estas \emph{stulta}?  Mi konas la koncepton de mistraktado de infanoj,
mi \emph{scias} pri nedeca tuŝo kaj ĉio tio, kaj se io ajn kiel tio
okazus al mi, mi vekos la policon. Kaj mi raportus tion al la
lernejestro! Kaj mi elserĉos sociajn servojn en la telefona libro. Kaj
mi diros al Avĉjo kaj Avinjo kaj al S-ino Figg! Sed miaj gepatroj
\emph{neniam} faris ion ajn kiel tio, neniam, \emph{neniam}! Kiel vi
\emph{aŭdacas} aludi tian aferon?!''

La pli aĝa sorĉistino fiksrigardis lin stabile. ``Estas mia devo
kiel Vicdirektorino enketi pri eblaj signoj de mistraktado de infanoj
kiuj estas sub mia prizorgado.''

La kolero de Harry iĝis neregebla en puran, nigran furiozon. ``Neniam
aŭdacu spiri vorton pri tiuj, tiuj aludoj al iu ajn
alia. \emph{Neniu}, ĉu vi aŭdis min, McGonagall?  Akuzo kiel tiu
povas ruingi homojn kaj detrui familiojn eĉ se la gepatroj estas
komplete senkulpaj! Mi legis pri tio en la ĵurnalo!'' La voĉo de Harry
estis suprenirinta en akutan krion. ``La sistemo ne scias kiel halti, ĝi
ne kredas la gepatrojn aŭ la infanon kiam ili diras ke nenio okazis!
\emph{Ne aŭdacu minaci mian familion per tio! Mi ne lasos vin detrui
mian hejmon.}''

``Harry,'' la pli alta sorĉistino diris dolĉe, kaj ŝi proksimiĝis sian
manon al li—.

Harry paŝis malantaŭen rapide, kaj sia mano forpuŝis kaj frapis la ŝian for.  

McGonagall glaciĝis, kaj poste ŝi retiris sian manon, kaj faris
retropaŝon. ``Harry ĉio estas bonorde,'' Ŝi diri. ``Mi kredas vin.''

``*Ĉu tio veras?*'' Harry sbilis. La furiozo ankoraŭ muĝanta en sia
sango. ``Aŭ, ĉu vi simple atendas esti for de mi, por ke vi povu
registri la paperojn?''

``Harry, mi vidis vian domon. Mi vidis vin kun viaj gepatroj. Ili amas
vin. Vi amas ilin. Mi kredas vin kiam vi diras ke viaj gepatroj ne
mistraktis vin. Sed mi \emph{devis} demandi, ĉar estas io stranga
aganta tie.''

Harry rigardis ŝin fride. ``Kiel kio?''

``Harry, mi vidis multajn mistraktitajn infanojn dum mi estis en
Herpŭrko, tio rompus vian koron scii kiom. Kaj, kiam vi estas feliĉa,
vi ne kondutas kiel iu el tiuj infanoj, \emph{tute} ne. Vi ridetas al
nekonatoj, vi brakumas homojn, mi metis mian manon sur vian ŝultron
kaj vi ne eksaltis. Sed kelkfoje, nur kelkfoje, vi diras aŭ faras ion
kio ŝajnas vere simila \ldots al iu kiu pasis siajn unuajn dek unu
jarojn ŝlosita en kelo. Ne en la amanta familio, kiun mi vidis.''
Profesorino McGonagall klinis sian kapon, sia esprimo iĝis
konsternita denove.

Harry prenis tion ene, kaj konsideris ĝin. La nigra furiozo komencis
foriri, kiam aperis al li ke li estis aŭskultita respekte, kaj ke
sia familio ne estis en danĝero.

``Kaj, kiel vi eksplikas viajn observadojn, Profesorino McGonagall?''

``Mi ne scias,'' ŝi diris. ``Sed eblas ke okazis al vi io, kion
vi ne memoras.''

Furiozo rekomencis kreski en Harry. Ĉio tio sonis tro simila al tio,
kion li legis en la ĵurnalaj historioj de frakasitaj
familioj. ``Subpremitaj memoroj estas parto de la
\emph{pseŭdoscienco}! Homoj ne subpremas traŭmatajn memorojn, ili
memoras ĉion, \emph{tro bone} dum la resto de ilia vivo!''

``Ne, S-iĉo Potter. Estas Ĉarmo nomita Forgeso.''

Harry frotiĝis tuj. ``Sorĉo kiu deviŝas memorojn?''

La pli alta sorĉistino kapjesis. ``Sed ne la tutan efekton de la
sperto, se vi vidas tion, kion mi volas diri, S-iĉo Potter.''

Malvarmo pasis trans lia spino. \emph{Tiu} hipotezo\ldots \emph{ne}
povis esti facile refutita. ``Sed miaj gepatroj ne povas fari tion!''

``Efektive ne,'' diris la Profesorino McGonagall. ``Tio demandintus
iun de la sorĉa mondo. Estas\ldots nenio maniero por esti certa, mi
timas.''

La racionala lerteco de Harry komencis restarti. ``Profesorino
McGonagall, kiom certa vi estas pri viaj observadoj, kaj kiuj
alternativaj eksplikoj povas esti?''

La sorĉistino malfermis siajn manojn, kiel por montri ilian
maplenecon. ``Certa? Mi certas pri \emph{nenio}, S-iĉo Potter. En mia
tuta vivo, mi neniam renkontis iun alian kiel vi. Kelkfoje, vi nur ne
ŝajnas kiel knabiĉo de dek unu jaroj, aŭ eĉ kiel \emph{homo}.''

La brovoj de harry leviĝis al la ĉielo—

``Mi bedaŭras!'' Profesorino McGonagall diris rapide. ``Mi tre
bedaŭras, S-iĉo Potter. Mi provis diri ion, sed mi timas ke ĝi eliris
tre malsama al tio kion mi havis en la kapo—''

``Male, Profesorino McGonagall,'' Harry diris, antaŭ rideti
malrapide. ``Mi devus preni tion kiel tre granda komplimento. Sed tio
ĉu ĝenas vin, se mi alportas alian eksplikon?''

``Bonvolu fari.''

``Infanoj ne estas intencitaj esti tro multe pli lerta ol iliaj
gepatroj,'' Harry diris. ``Aŭ tro multe pli racia, eble—mia patriĉo
povus verŝajne superruzi min, se li estus, vi scias, fakte
\emph{provanta}, anstataŭ uzi sian plenkreskulan inteligentecon
precipe por trovi novajn kialojn por ne ŝanĝi sian opinion—'' Harry
haltis. ``Mi estas tro lerta, Profesorino. Mi havas nenion por diri al
normalaj infanoj. Plenkreskuloj ne respektas min sufiĉe por vere
paroli al mi. Kaj sincere, eĉ se ili farus, ili ne sonus tiel lerta
kiel Richard Feynman, do mi farus pli bone per legi ion, kion Richard
Feynman skribis anstataŭe. Mi estas \emph{izolita}, Profesorino
McGonagall. Mi estis izolita dum mia tuta vivo. Eble, tio havas iom da
la sama efiko ol esti enŝlosita en kelo. Kaj, mi estas tro lerta por
rigardi miajn gepatrojn je la maniero, laŭ kiu la infanoj devus
rigardi iliajn gepatrojn. Miaj gepatroj amas min, sed ili ne pensas ke
ili devontigas sin respondi al racio, kaj kelkfoje mi sentas min
kvazaŭ ili estas la infanoj—infanoj kiuj \emph{ne aŭskultas} kaj havas
absolutan aŭtoritaton sur mia tuta ekzisto. Mi provas ne esti tro
amara pri tio, sed mi ankaŭ provas esti \emph{honesta} kun mi mem, do,
jes, mi estas amara. Kaj mi ankaŭ havas problemon pri kontrolado de
kolero, sed mi estas laboranta pri tio. Tio estas ĉio.''

``\emph{Tio estas ĉio?}''

Harry kajesis firme. ``Tio estas ĉio. Certe, Profesorino McGonagall,
eĉ en la sorĉa Britio, la normala ekspliko ĉiam valoras esti
\emph{konsiderita}, ĉu ne?''

\begin{center}\rule{3in}{0.4pt}\end{center}

Estis pli malfrue en la tago, la suno estis malleviĝanta en la somera
ĉielo, kaj aĉetantoj komencis foriri la straton. Iuj butikoj estis
jam fermitaj; Harry kaj Profesorino McGonagall estis aĉetinta liajn
lernolibroj en Flori kaj Ŝmiri tuj antaŭ la templimo. Kun nur eta
eksplodo, kiam Harry iris rekte al la ŝlosilvorto ``Aritmanco'' por
malkovri ke la libro de sep jaro rilatis al nenio pli matematike
altenivela ol trigonometrio.

En ĉi tiu momento, tamen, sonĝoj de malaltaj esploraj fruktoj estis
malproksimaj de la menso de Harry.

En ĉi tiu momento, ambaŭ el ili marŝis el la butiko de
Olivando$^{\textit{n:\ref{nomoj:olivando}}}$ kaj Harry estis rigardadanta sian
bastonon. Li skuis ĝin kaj produktis multkolorajn fajrerojn, kiuj vere
ne devintus esti tiel ekstre ŝokaj post ĉio alia, kion li vidis, sed
iel—.

\emph{Mi povas fari magion.}

\emph{Mi. Kiel en, mi persone. Mi estas magia; Mi estas sorĉistiĉo.}

Li \emph{sentis} la magion verŝanta en sian brakon, kaj tiam, li
ekkonsciis ke li ĉiam havis tiun senton, ke li posedis ĝin dum sia
tuta vivo. Iu sento kiu ne estis vido aŭ aŭdo aŭ flarado aŭ gusto aŭ
palpo sed simple magio. Kiel havi okulojn sed konservi ilin fermitaj
ĉiutempe, tiel ke vi eĉ ne konsciis ke vi nur vidis mallumon; kaj
poste iu tago, la okuloj malfermiĝu kaj vidu la mondon. La ŝoko de
tio verŝis trans li, tuŝante partojn de li mem, vekante ilin, kaj
poste mortis post malmultaj sekundoj. Lasante nur la certan scion ke
li estis sorĉistiĉo, kaj ke li ĉiam estis, kaj ke li eĉ, iel strange,
ĉiam sciis tion.

Kaj—

``Tio estas tre kurioza efektive ke destinas por vi ĉi tiun bastonon
kiam ĝia frato faris al vi ĉi tiun cikatron.''

Tio ne \emph{povis} esti koincido. Estis miloj da bastonoj en tiu
butiko. Nu, bone, fakte tio povis esti koincido, estis ses miliardoj
da homoj en la mondo kaj miloj da koincidoj kun probableco de mil
kontraŭ unu okazis ĉiutage. Sed la teoremo de Bajes diris ke nur
raciaj hipotezoj, kiuj ŝajnis pli verŝajnaj ol mil ebloj kontraŭ unu
ke li finu kun la frato de la bastono de la Mastro de la Tenebroj,
havis utilon.

Profesorino McGonagall simple diris \emph{'kiom bizara'} kaj lasis tion
tiel. Tio farigis Harry'n atingi staton de ŝoko pro la pura, abomena
\emph{senscivolemo} de la sorĉistoj. En neniu \emph{imagebla} mondo,
Harry estintus simple dirinta ``Hm'' kaj poste simple marŝanta el la
butiko sen eĉ \emph{provi} fari hipotezon pri kial tio okazis.

Lia maldekstra mano leviĝis kaj tuŝis sian cikatron.

Kio\ldots \emph{ekzakte}\ldots

``Vi estas kompleta sorĉistiĉo nun,'' diris Profesorino
McGonagall. ``Gratulon.''

Harry kapjesis.

``Kaj kion vi pensas pri la mondo de sorĉistoj?'' diris ŝi.

``Ĝi estas stranga,'' Harry diris. ``Mi devus pensi pri ĉion la magion,
kion mi vidis\ldots kaj pri ĉiu, kiun mi nun scias esti ebla, kaj
ankaŭ pri ĉiu kiun mi nun scias esti mensogo, kaj pri la tuta laboro
antaŭ ol mi komprenos ĝin. Anstataŭe, mi trovis min distrita pro
relative trivialaĵoj kiel,'' Harry mallaŭtigis sian voĉo. ``La tuta
Knabo-Kiu-Postvivis afero.''  Ŝajnis, ke neniu estas proksima, sed
sensencis tenti la sorton.

Profesorino McGonagall tusadis. ``Ĉu Vere? sen ŝerco''

Harry kapjesis. ``Jes. Tio estas nur\ldots \emph{bizara}. Malkovri ke
vi estis parto de la granda historio, la serĉado por venki la grandan
kaj teruran Mastron de la Tenebroj, kaj ĝi estas jam finita. Finita. Ĉio
tio estas tute finita. Kvazaŭ vi estus Frodo Baggins, kaj vi malkovrus
ke viaj gepatroj prenis vin al la Monto de Pereo, kaj igis vin ĵeti la
Ringon kiam vi havis nur unu jaron kaj vi eĉ ne memorus ĝin.''

La rideto de Profesorino McGonagall iom plifortiĝis.

``Vi scias, se mi estus iu alia, iu ajn alia, mi verŝajne estus sufiĉe
maltrankvila pri vivi post tiu komenco. \emph{Dio, Harry, kion vi faris
post vi venkis la Mastron de Tenebroj ? Via propra butiko de libroj?
Grandioze! Diru al mi, ĉu vi scias ke mi nomis mian infanon kiel
vi?} Sed mi ankoraŭ havas esperojn ke tio ne estos problemo.'' Harry
suspiris. ``Tamen\ldots ĝi estas preskaŭ sufiĉa por igi min esperi ke
restas kelkaj aĵoj por solvi en la serĉado, nur por ke mi povu diri ke mi
reale, vi scias, kunhelpis iel.''

``Ho?'' diris Profesorino McGonagall per stranga tono. ``Kion vi havas
en menso?''

``Nu, ekzemple, vi menciis ke miaj gepatroj estis perfiditaj. Kiu
perfidis ilin?''

``Sirius Nigro$^{\textit{n:\ref{nomoj:nigro}}}$,'' la sorĉistino diris, preskaŭ
siblante la nomon. ``Li estas en Azkaban. La sorĉa malliberejo.''

``Kiom probable estas ke tiu Sirius Nigro eskapu el la malliberejo kaj
ke mi devu ĉasi lin kaj venki lin dum ia spektakla duelo, aŭ ankoraŭ
pli bone meti grandan premion sur sia kapo kaj kaŝi min en aŭstralio
dum mi atendas la rezulton.''

Profesorino McGonagall palpebrumis. Du fojojn. ``Tre malmulte probabla. Neniu
iam ajn eskapis el Azkaban, kaj mi dubas ke \emph{li} estos la unua.''

Harry estis iom skeptika pri tiu ``*neniu* iam ajn eskapis el
Azkaban'' frazo. Tamen, eblis ke kun magio vi povis fakte proksimiĝi al la
100\% perfekta malliberejo, speciale kiam vi havis bastonon kaj ili
ne. La pli bona maniero eskapi estus tiel ke vi ne eniru unue.

``Konsentite,'' Harry diris. ``Ŝajnas ke tio estis agrable enpakita.''
Li suspiris, frotante sian manon kontraŭ sia kapo. ``Aŭ eble ke la Mastro
de Tenebroj ne vere mortis tiun nokton. Ne komplete. Lia spirito
restadas, flustras al homoj en koŝmaroj, kiuj superfluas al la maldorma
mondo, kaj serĉante manieron por reveni en la vivanta regiono, kiun li
ĵuris destrui, kaj nun, konforme al la malnova profetaĵo, li kaj mi
estas ŝlositaj en morta duelo, en kiu la venkinto devos perdi, kaj la
perdinto devos venki—''

La kapo de Profesorino McGonagall turniĝis, kaj ŝiaj okuloj rigardis
ĉirkaŭe, kvazaŭ por serĉi aŭskultantojn en la strato.

``Mi estas \emph{ŝercanta}, Profesorino,'' Harry diris kun iom da
agaco. Diable, kial ŝi ĉiam prenis ĉion tiel serioze—

Malrapide sento de sinkado komencis naskiĝi en la fundo de la stomako
de Harry.

Profesorino McGonagall rigardis Harry'n kun trankvila esprimo. Tre,
\emph{tre} trankvila esprimo. Poste rideto aperis sur ŝia
vizaĝo. ``Evidente, ke vi estas ŝercanta, S-iĉo Potter.''

\emph{Aĥ fek'}.

Se Harry bezonintus formaligi la senvortan inferencon kiu ĵus ekaperis
en sia menso, ĝi devintus esti io kiel, `Se mi taksas la probablon ke
Profesorino McGonagall faris tion, kion mi ĵus vidis kiel la rezulto
de zorga kontrolo de ŝi mem, kontraŭ la distribuo de probablo de ĉiuj
la aĵoj, kiujn ŝi normale farintus se mi farus malbonan ŝercon, tiu
konduto estas signifika pruvo ke ŝi kaŝas ion.'

Sed tio, kion Harry fakte pensis, estis \emph{Aĥ fek'}.

Harry turnis sian propran kapon, por skani la straton. Ne, neniu
proksime. ``Li ne estas morta, ĉu ne,'' Harry suspiris.

``S-iĉo Potter—''

``La Mastro de la Tenebroj estas vivanta. \emph{Evidente}, ke li estas
vivanta.  Tio estis \emph{ago} de absoluta \emph{optimismo}, ke mi eĉ
revis alimaniere. Mi sendube forlasis miajn \emph{sentojn}, mi ne
povas \emph{imagi} tion, kion mi estis pensanta. Nur ĉar \emph{iu}
diris ke lia korpo estis trovita brulita al cindroj, mi ne povas imagi
kial mi estis opinianta ke li estis \emph{morta}. \emph{Klare}, Mi
havas ankoraŭ multe por lerni pri la arto de \emph{pesimismo}.''

``S-iĉo Potter—''

``Almenaŭ diru al mi ke ne reale estas profetaĵo\ldots'' Profesorino
McGonagall estis ankoraŭ donanta al li tiun brilan fiksan
rideton. ``Ho, vi mokas min.''

``S-iĉo Potter, vi ne devus inventi aĵon, pri kiuj vi devas zorgi—''

``Ĉu vi vere estas diranta tion al min? Imagu mian reagon poste, kiam
mi malkovros ke estas io pri kio mi devas zorgi finfine.''

Ŝia rideto ŝanceliĝis.

La ŝultroj de Harry falis. ``Mi havas la tutan magian mondon por
analizi. Mi ne havas tempon por tio.''

Poste la du el ili silentiĝis, kiam geviro kun fluanta oranĝa robo
aperis sur la strato kaj malrapide paŝis apud ili; La okuloj de
Profesorino McGonagall sekvis rin, diskrete. La buŝo de Harry estis
movanta dum li maĉis forte sian lipon, kaj se oni atentive
rigardintus, oni rimarkus ke eta loko da sango aperis.

Kiam la ulo kun la oranĝa robo estis foririnta, Harry parolis denove,
per mallaŭta flustrado. ``Ĉu vi diros al mi la veron nun, Profesorino
McGonagall? Kaj ne vin ĝenu provi rakonti fabelon al mi, mi ne estas
stulta.''

``Vi havas \emph{dek unu jarojn}, S-iĉo Potter!'', ŝi diris per severa
flustro.

``Kaj, do sub-homo. Pardonu\ldots dum momento, mi forgesis.''

``Tiuj estas teruraj kaj gravaj aferoj! Ili estas sekretoj, S-iĉo
Potter! Tio estas \emph{katastrofo} ke vi, ankoraŭ infano, scias
tiom. Vi ne devas paroli pri tio al \emph{neniu}, ĉu vi komprenas?  Vi
nepre devas paroli al iu ajn!''

Kiel tio okazis kelkfoje, kiam Harry sufiĉe koleriĝis, sia sango iĝis
malvarma anstataŭ varma, kaj terure malhela klareco malsupreniris en
sian menson, mapante eblajn taktikojn kaj taksante iliajn
konsekvencojn kun fera realismo.

\emph{Montri ke vi rajtas scii: Fiasko. Dek unu jaraj infanoj ne
rajtas scii ion ajn, laŭ la okuloj de McGonagall.}

\emph{Diri ke vi ne plu estos amikoj: Fiasko. Ŝi ne donas sufiĉan valoron al via amikeco.}

\emph{Montri ke vi estos en danĝero se vi ne scias: Fiasko. Planoj jam
  estis faritaj, ĉiuj bazitaj sur via nescio. La certa malagrablaĵo de
  repensi ŝajnos multe pli malagrabla ol la nur necerta perspektivo,
  ke vi estos damaĝita.}

\emph{Justo, kaj racio estos ambaŭ fiasko. Vi devas trovi ion, kion vi
havas, kaj kion ŝi volas, aŭ trovi ion, kion vi povas fari, kaj kion
ŝi timas\ldots}

Ha.

``Nu, en tiu ĉi okazo, Profesorino,'' Harry diris per mallaŭta, glacia
tono, ``Ŝajnas kvazaŭ mi havas ion, kion vi volas. Vi povas, se vi
volas, diri la veron al mi, la tutan veron, kaj kompense mi konservos
vian sekreton. Aŭ vi povas provi daŭrigi kaŝi la veron al mi, tiel ke
vi povu uzi min kiel peono, en tiu okazo, mi ŝuldos nenion al vi.''

McGonagall haltis tuj sur la strato. Ŝiaj okuloj brilis kaj ŝia voĉo
falis en profundan siblon. ``Kiel vi aŭdacas?''

``\emph{Kiel vi aŭdacas?}'' li flustrante respondis al ŝi.

``Vi ĉantanĝus min?''

La lipoj de Harry tordiĝis. ``Mi proponas al vi \emph{favoron}. Mi
\emph{donas} al vi ŝancon por protekti \emph{vian} preciozan
sekreton. Se vi rifuzas, mi havos ĉiujn naturajn motivojn por serĉi
informojn aliloke, ne pro malamo kontraŭ vi, sed ĉar \emph{mi devas
scii}! Transpasu vian sensencan koleron al infano, kiu laŭ via opinio
devas obei vin, kaj vi ekkonscios ke iu ajn sana plenkreskulo faros la
saman aĵon! \emph{Rigardu tion de mia perspektivo! Kiel vi sentus vin,
se tiu estus VI}''

Harry rigardis McGonagall'n, observis ŝian malafablan respiron. Ŝajnis
al li, ke estis tempo por malaltigi la premon, kaj lasi ŝin boleti dum
momento. ``Vi ne estas devigata decidi tuj,'' Harry diris kun pli
normala tono. ``Mi komprenos se vi volas tempon por pensi pri mia
\emph{propono}\ldots sed mi avertas vin pri iu afero,'' Harry diris,
sia voĉo iĝante pli malvarma. ``Ne provu tiun Ĉarmon de Forgeso al
mi. En la pasinteco, mi kreis signalon, kaj mi jam sendis tiun
signalon al mi mem. Se mi trovas ĉi tiun signalon, kaj mi ne memoras
sendi ĝin\ldots'' Harry lasis sian voĉo malaŭtiĝi grave.

La vizaĝo de McGonagall moviĝis dum sia esprimo ŝanĝiĝis. ``Mi ne
estis pensanta pri Forgesigi vin, S-iĉo Potter\ldots sed kial vi
inventintus tian signalon, se vi ne sciis pri—''

``Mi pensis pri tio dum mi legis Muglan sciencfikcian libron, kaj mi
diris al mi mem, \emph{nu}, \emph{ĉiaokaze}\ldots Kaj ne, mi ne estas
dironta al vi la signalon, mi ne estas stulta.''

``Mi ne planis demandi,'' McGonagall diris. Ŝi ŝajnis kliniĝanta sur
si mem, kaj subite aspektis tre maljuna, kaj tre laca. ``Tio estis
elĉerpa tago, S-iĉo Potter. Ĉu ni povas igi vin akiri vian trunkon,
kaj sendi vin hejmen? Mi fidos vin por ne paroli al iu ajn pri tiu
afero ĝis mi havas tempon por pripensi. Tenu enmense, ke estas nur du
aliaj homoj en la mondo kiuj scias pri tiu afero, kaj ili estas la
Direktoriĉo Albus Dumbledore, kaj la Profesoriĉo Severus
Skoldo$^{\textit{n:\ref{nomoj:skoldo}}}$.''

Do. Novaj informoj; Tio estis paca propono. Harry kapjesis akceptante,
kaj turnis sian kapon por rigardi antaŭen, kaj komencis marŝi denove,
dum sia sango malrapide komencis varmiĝi unu fojon plian.

``Do nun, mi devas trovi ian manieron por mortigi senmortan Malhelan
Sorĉiston,'' Harry diris, kaj suspiris frustriĝante. ``Mi vere
dezirintus ke vi diris tion al mi antaŭ ol mi komencis butikumi.''

\begin{center}\rule{3in}{0.4pt}\end{center}

La butiko de trunkoj estis pli riĉe provizita ol iu ajn el la aliaj
butikoj, kiuj Harry vizitis; la drapiraĵoj estis abundaj kaj delikate
desegnitaj, la planko kaj la muroj estis faritaj el makulita kaj
polurita ligno, kaj la trunkoj okupis lokoj de honoroj sur poluritaj
eburaj estradoj. La vendistiĉo vestis sin per ornamita robo --kies
kvalito estis nur iomete malpli bona ol tiu de la robo de Lucius
Malfojo--, kaj parolis kun ravega, olea ĝentileco al ambaŭ Harry kaj
Profesorino McGonagall.

Harry faris siajn demandojn, kaj gravitis ĉirkaŭ trunko farita el
pezaspekta ligno, ne polurita sed varma kaj solida, gravurita kun
desegno de gardanta drako, kies okuloj turniĝis por rigardi iun ajn
proksime. Tiu trunko estis ĉarmita por esti malpeza, por ŝrumpi al
demando, kaj por elkreski ungajn tentaklojn el ĝian malsupron por ke
ĝi povu svingiĝi malantaŭ ĝia posedanto. Ĝi havis du kestojn, sur ĉiuj
kvar flankoj, kiuj ŝoviĝis por riveli kompartimenton tiel profundan
kiel la tuta trunko. Kovrilo kun kvar seruroj, kiuj malkaŝis malsamajn
spacojn ene. Kaj—tio estis la grava afero—manilo sur la malsupro, kiu
ŝovis kadron enhavanta ŝtuparon kiu kondukis malsupren en eta lumita
ĉambro, kiu povis enhavi, Harry estimis, ĉirkaŭ dek du bibliotekojn.

Se li faris pakaĵon kiel tiu, Harry ne sciis kial iu ajn ĝenus sin mem
per posedi domon.

Cent ok ora Galionoj. Tio estis la kosto de bona trunko, leĝere
uzita. Por ĉirkaŭ kvindek Britiaj pundoj po la Galiono, tio estis sufiĉe
por aĉeti brokantan aŭton. Tiu estis pli multekosta ol ĉiuj la ceteraj
aĵoj, kiujn Harry aĉetis en sia tuta vivo, kunigitaj.

Naŭdek sep Galionoj. Tio estas kiom da Galionoj restis en la sako da
oro, kiun Harry estis permesita retiri el Gringoto.

Profesorino McGonagall portis esprimon de ĉagreno sur la vizaĝo. Post
longa tago de butikumado, ŝi ne bezonis demandi al Harry kiom da oro
restis en sia sako kiam la vendisto anoncis la koston, tio volis diri
ka la Profesorino povis fari bonan mensan aritmetikon sen uzi plumon
kaj paperon. Ankoraŭ unu fojon, Harry memorigis sin, ke \emph{scienca
  analfabeteco} ne estis la sama aĵo ol \emph{stulteco}.

``Mi bedaŭras, juna viriĉo,'' diris Profesorino McGonagall. ``Tio tute
estas mia kulpo. Mi proponus reiri kun vi al Gringoto, sed la banko
malfermas sole por la kriza servo nun.''

Harry rigardis ŝin, scivolante\ldots

``Nu,'' suspiris Profesorino McGonagall, dum ŝi svingis sur unu
kalkano, ``ni povas do eliri, mi supozas.''

\ldots Ŝi ne tute perdis sin, kiam infano aŭdacis defii ŝin.  Ŝi ne
estis feliĉa, sed ŝi pripensis anstataŭ eksplodi pro furiozo. Tio eble
estis nur pro la fakto ke estis senmorta Mastro de la Tenebroj, kiu
devis esti venkita—kaj ke ŝi bezonis la bonvolo de Harry. Sed la
plimulto da plenkreskuloj ne estintus kapabla pripensi tiel multe; ili
tute ne konsiderintus estontajn konsekvencojn, se iu kun malpli granda
stato rifuzus obei ilin\ldots

``Profesorino?'' Harry diris.

La sorĉistino turnis sin malantaŭen kaj rigardis al li.

Harry prenis profundan enspiron. Li bezonis esti iom kolera por tio,
kion li volis provi nun. Ne ekzistis maniero por ke li havu la kuraĝon
fari tion alie. \emph{Ŝi ne aŭskultus min}, li farigis sin pensi,
\emph{Mi devintus preni pli da oro sed ŝi ne volis aŭskulti\ldots}
Enfokusigante sian tutan mondon sur McGonagall kaj sur la bezono
fleksi la konversacion laŭ sia volo, li parolis.

``Profesorino, vi pensis ke cent Galionoj estus pli ol sufiĉa por
aĉeti trunkon. Tio estis kial vi ne ĝenis vin pri averti min antaŭ ol
restis nur naŭdek sep Galionoj. Tio estas ekzakte la afero, kiun la
esplora studado montris—tio estas, kio okazas kiam homoj pensis ke ili
al si lasis \emph{etan} marĝenon por eraro. Ili ne estas sufiĉe
pesimismaj. Se ĝi dependus de mi, mi prenintus ducent Galionoj nur por
esti certa. Estis abundo da mono en tiu sekurĉambro, kaj mi povintus
reenmeti la ekstran poste. Sed mi pensis ke vi ne lasus min fari
tion. Mi pensis ke vi koleriĝus kontraŭ mi nur pro ke mi estus
demandinta. Ĉu mi estis malprava?''

``Mi supozas, ke mi devas konfesi ke vi pravas,'' diris
Profesorino McGonagall. ``Sed, juna viriĉo—''

``Tiu estas la kialo, pro kiu mi havas penon por fidi
plenkreskulojn.'' Iel, Harry gardis sian voĉon regule. ``Ĉar ili
koleriĝas se vi eĉ provas argumenti kun ili. Por ili, tio estas
malobeo kaj impertinenteco kaj defio al ilia supera triba stato. Se vi
provas paroli al ili, ili \emph{koleriĝas}.  Do, se mi havus ion ajn
vere grava por fari, mi ne estus kapababla fidi vin. Eĉ se vi
aŭskultus kun profunda intereso ĉion, kion mi dirus—ĉar tio estas
ankaŭ parto de la \emph{rolo} de iu, kiu aktoras kiel zorganta
plenkreskulo—vi neniam ŝanĝus viajn agojn, vi ne fakte konduktus
malsame, pro ĉio, kion mi diris.''

La vendisto estis rigardanta ambaŭ ilin senhonte fascinita.

``Mi povas kompreni vian vidpunkton,'' Profesorino McGonagall diris,
finfine. ``Se mi kelkfoje ŝajnas tro severa, bonvolu memori ke mi
deĵoras kiel Direktorino de la Domo Grifindoro de tiel longe, ke tio
ŝajnas kiel pluraj miloj da jaroj antaŭe.''

Harry kapjesis, kaj daŭris. ``Do—supozu ke mi havas manieron por akiri
pliajn Galionojn el mia sekurĉambro \emph{sen} ke ni reiru al
Gringoto, sed tio implikas ke mi malobservu la rolon de obeema
infano. Ĉu mi povas fidi vin pri tio, eĉ se vi devas eliri el via
propra rolo de Profesorino McGonagall tiel ke oni povu preni avantaĝon
de tio?''

``\emph{Kio?}'' diris Profesorino McGonagall.

``Por diri tion alimaniere, se mi povas igi la hodiaŭon okazi malsame,
kaj fari ke ni \emph{ne prenu} nesufiĉan kvanton da mono kun ni, ĉu
tio estus orde kvankam tio implikus ke infano estu insolenta al
plenkreskulo retrospektive?''

``Mi\ldots supozas\ldots'' la sorĉistino diris, aspektante tute skeptika.

Harry elprenis sian haŭtpoŝon, kaj diris, ``dek unu Galionoj venantaj
de mia familia sekurĉambro.''

Kaj estis oro en la mano de Harry.

Dum momento la buŝo de Profesorino McGonagall gapis larĝe, sed poste
sia makzelo ekfermis kaj siaj okuloj mallarĝiĝis. La sorĉistino diris
akre, ``*Kie* vi akiris tion—''

``En mia familia sekurĉambro, kiel mi diris.''

``\emph{Kiel?}''

``Magio.''

``Tiu estas malfacile respondo!'' klakis la Profesorino McGonagall,
ĵus antaŭ halti, palpebrumante.

``Ne, tiu ne estas, ĉu ne? Mi devus pretendi ke tio estas ĉar mi
eksperimente malkovris la veran sektreton pri kiel la haŭtpoŝo
funkscias kaj ke ĝi fakte povas rekuperi objektojn el ie ajn, ne nur
el ĝia propa interno, se mi formulas la demandon korekte. Sed fakte,
mi akiris ilin, kiam mi falis sur la amaso da oro, kaj mi kaŝis
kelkajn Galionojn en mia poŝo. Iu ajn, kiu komprenas pesimismon, scias
ke mono estas io kion vi povus bezoni rapide kaj sen multe da
averto. Do, nun, ĉu vi estas kolera kontraŭ mi pro ke mi defiis vian
aŭtoritaton? Aŭ vi ĝojas ke ni sukcesas nian gravan mision?''

La okuloj de la vendisto estis tiel larĝaj kiel subtasoj.

Kaj la alta sorĉistino staris tie, silente.

``Disciplino en Herpŭrko devas esti observata,'' ŝi diris, post tuta
minuto. ``Por la bonfarto de ĉiuj lernantoj. Kaj tio devas kompreni
vian ĝentilecon kaj vian obeemon al ĉiuj Profersoroj.''

``Mi komprenas, Profesorino McGonagall.''

``Bone. Nun ni aĉetu tiun trunkon, kaj ni reiru hejmen.''

Harry sentis sin kvazaŭ li volis vomi, aŭ hurai, aŭ sveni, aŭ
\emph{io}. Estis la unua fojo ke lia atenta rezonado funkciis sur
\emph{iu ajn}. Eble, ĉar estis la unua fojo ke li havis ion vere
serioza, kion plenkreskulo bezonis de li, sed tamen—

Minerva McGonagall, +1 poento.

Harry kliniĝis, kaj metis la sakon da oro kaj la aldonajn dek unu
galionojn en la manon de McGonagall. ``Multan dankon, Profesorino. Ĉu
vi povas fini tiun aĉeton por mi? Mi devas iri al la necesejo.''

La vendistiĉo, miela ankoraŭ unun fojon, indikis pordon lokita en la
muro kun ora manilo. Dum Harry komencis marŝi for, li aŭdis la
vendistiĉon demandi kun sia olea voĉo, ``Ĉu mi povas scii kiu li
estas, Sinjorino McGonagall? Mi supozas ke li estas Serpentimo—en tria
jaro, eble?—kaj de elstara familio, sed mi ne rekonis—''

La klako de la pordo de la necesejo tranĉis liajn vortojn, kaj post ol
Harry identigis la seruron kaj fermis ĝin, li kaptis la magian
memlavantan mantukon, kaj kun necertaj manoj, viŝis la humidon desur
sia frunto. La tuta korpo de Harry estis englutita per ŝvito, kiu
klare trapasis siajn muglajn vestojn, kvankam almenaŭ tio ne estis
videbla tra la sorĉista robo.

\begin{center}\rule{3in}{0.4pt}\end{center}

La suno estis kuŝonta, kaj estis tre malfrue efektive, kiam ili
troviĝis denove en la korto de la Likanta Kaldrono, la silenta
interfaco kovrita per folioj inter la Diagon Aleo de la magia Britio
kaj la tuta Mugla mondo. (Tiu estis terure disparigita
ekonomio\ldots). Harry iros al telefono por voki sian patriĉon, kiam
li estos je la alia flanko. Li ne bezonis zorgi pri ke oni ŝtelas sian
bagaĝon ŝajne. Lia trunko havis la staton de ĉefa magia objekto, io ke
plimulto da Mugloj ne rimarkus; tio estas io, kion vi povis akiri en
la sorĉa mondo, se vi estis preta por pagi le koston de brokanta aŭto.

``Do, tie ni disiras, por kelka tempo,'' Profesorino McGonagall
diris. Ŝi skuis sian kapon ĝoje. ``Tiu estis la pli stranga tago de
mia vivo ekde\ldots multe da jaroj. Edke la tago kiam mi lernis ke
infano venkis Vi-Scias-Kiu'n. Mi demandas al mi nun, retrorigarde, se
tiu estis la lasta racia tago de la mondo.''

Ho, kiel se \emph{ŝi} havis ion pri kio plendi. \emph{Vi pensas ke via
  tago estis nereala? Provu la mian.}

``Vi impresis min hodiaŭ,'' Harry diris al ŝi. ``Mi devintus memori
gratuli vin laŭte, mi premiis poentojn al vi en mia kapo kaj ĉio.''

``Dankon, S-iĉo Potter,'' diris la Profesorino McGonagall. ``Se vi
estus jam ordigita en Domo, mi forprenus tiom da poentoj ke viaj nepoj
perdontus ankoraŭ la Pokalon de Domoj.''

``Dankon al \emph{vi}, Profesorino.'' Estis verŝajne tro frue por nomi
ŝin Mini.

Tiu virino eble estis la pli sana plenkreskulo, kiun Harry iam ajn
renkontis, malgraŭ ŝia manko de scienca kono. Harry estis eĉ
konsideranta proponi al ŝi esti la numero du de iu ajn grupo, kiun li
formos por batali kontraŭ la Mastro de la Tenebroj, kvankam li ne
estis sufiĉe stulta por diri tion laŭte. \emph{Nun, tio, kion mi
  bezonas, estas bona nomo por ĝi\ldots? La Morto-Manĝanto-Manĝantoj?}

``Mi revidos vin baldaŭ, kiam la lernejo komencos,'' Profesorino
McGonagall diris. ``Kaj, S-iĉo Potter, pri via bastono—''

``Mi scias tion, kion vi estas demandonta,'' Harry diris. Li elprenis
sian preciozan bastonon kaj, kun profunda interna pinĉo, renversis ĝin
en sia mano, kaj prezentis la tenilo al ŝi. ``Prenu ĝin. Mi ne planis
fari ion ajn, tute nenion, sed mi ne volas ke vi havu koŝmarojn pri mi
eksplodante mian hejmon.''

Profesorino McGonagall kapneis rapide. ``Ho ne, S-iĉo Potter! Oni ne
faras tion. Mi nur volis averti vin pri ne uzi vian bastonon hejme,
ĉar la Ministerio povas detekti neplenaĝan magion kaj tio estas
malpermesata sen kontrolo.''

``Ah,'' Harry diris. ``Tio ŝajnas tre saĝa regulo. Mi estas feliĉa
vidi ke la magia mondo prenu tiajn aĵojn serioze.''

Profesorino McGonagall fiksrigardis lin forte. ``Vi vere opinias tion?''

``Jes,'' Harry diris. ``Mi komprenas. Magio estas danĝera, kaj la
reguloj estas tie pro bonaj kialoj. Kelkaj aliaj aferoj estas ankaŭ
danĝeraj. Mi komprenas tion ankaŭ. Memoru ke mi ne estas stulta.''

``Estas malverŝajne ke mi iam forgesos tion. Dankon, Harry, tio igas
min senti pli bone pri fidi vin kun keklaj aĵoj. Adiaŭ, por la
momento.''

Harry turnis sin por foriri, en la Likantan Kaldronon kaj eksteren en
la Mugla mondo.

Kiam sia mano tuŝis la manilon de la malantaŭa pordo, li aŭdis lastan
flustron malantaŭ li.

``Hermione Granger.''

``Kio?'' Harry diris, sia mano ankoraŭ sur la pordo.

``Serĉu knabinon de unua jaro, nomita Hermione Granger en la trajno
por Herpŭrko.''

``Kiu ŝi estas?''

Ne estis respondo, kaj kiam Harry returniĝis, Profesorino McGonagall
estis foririnta.

\begin{center}\rule{3in}{0.4pt}\end{center}

\emph{Post la faktoj:}

Direktoriĉo Albus Dumbledore apogis sin antaŭen sur sia skribotablo. Liaj
scintilantaj okuloj fiksrigardis Minerva'n. ``Do, kara, kiel vi
trovis Harry'n?''

Minerva malfermis sian buŝon. Poste ŝi fermis sian buŝon. Poste ŝi
malfermis sian buŝon denove. Neniu vorto eliris.

``Mi vidas,'' Albus diris peze. ``Dankon por la raporto, Minerva, vi
povas eliri.''
