\chapter{Teoremo de Bajes}

\begin{center}\rule{3in}{0.4pt}\end{center}

Harry fiksrigardis supren al la griza plafono de la eta ĉambro, de kie
li kuŝis sur la portebla lito mola, kiu estis placita tie ĉi. Li estis
manĝinta multe da la manĝetoj de Profesoro Ciuro —komplikaj dolĉaĵoj
de ĉokolado kaj aliaj substancoj, polvitaj kun briletaj etaj aĵoj kaj
etaj gemoj el sukero, aspektante tre kosta kaj montrigante sin, fakte,
esti tute gusta. Harry tute ne sentis sin kulpa pri tio, \emph{tion
ĉi} li estis \emph{gajninta}.

Li ne provis dormi. Harry havis la senton ke li ne ŝatus tion, kio
okazus se li fermus la okulojn.

Li ne provis legi. Li estintus ne kapabla koncentri sin.

Amuze, kiel la cerbo de Harry ĵus ŝajnis daŭri funkcii kaj funkcii,
neniam halti ne gravas kiel laca ĝi estas. Ĝi fariĝas pli stulta sed
ĝi rifuzas \emph{elŝaltiĝi}.

Sed li havis, li vere kaj reale havis senton de triumfo.

+1 poento al la programo de Kontraŭ-Harry-La-Mastro-De-La-Tenebroj,
ne eĉ iomete priskribis ĝin. Harry sin demandis kion la Ordiganta
Ĉapelo dirintus nun, se li povus meti ĝin sur sian kapon.

Ne surprize Profesoro Ciuro akuzis Harry'n direktigi sin laŭ la vojo
irante al la Mastro de la Tenebroj. Harry ne komprenis tro malrapide,
li devintus vidi la paralelon direkete—

\emph{Komprenu ke la Mastro de la Tenebroj ne gajnis en tiu ĉi tago. Lia
  celo estis lerni luktajn artojn, kaj li eliris sen eĉ unu leciono.}

Harry eniris la Pocia klason kun le intenco de lerni Pociojn. Li
eliris sen eĉ unu leciono.

Kaj Profesoro Ciuro aŭdis, kaj komprenis kun timiganta akurateco,
etendis la manon kaj retiris Harry'n de la vojo, la vojo kiu kondukas
al estiĝi kopion de Vi-Scias-Kiu.

Estis frapeto kontraŭ la pordo. ``La kurso estas fininta,'' diris la
kvieta voĉo de Profesoro Ciuro.

Harry proksimiĝis la pordon kaj troviĝis subite malkvieta. La tencio
malpliiĝis dum li aŭdis la paŝojn de la Profesoro malproskimiĝi la
pordon.

\emph{Pri kio sur la tero temas? Ĉu ĝi estas tio, kio farigos lin
  maldungita?}

Harry malfermis la pordon, kaj vidis ke Profesoro Ciuro estis
atendanta al kelkaj metroj malproksime.

\emph{Ĉu Profesoro Ciuro sentas ĝin ankaŭ?}

Ili marŝi laŭ la nin-malplena scenejo al la skribotablo de Profesoro
Ciuro, sur kiu Profesoro kliniĝis; kaj Harry, kiel antaŭe, haltis for
antaŭe la podio.

``Do,'' Profesoro Ciuro diris. Estis amika senso pri li iel, eĉ se lia
vizaĝo ankoraŭ gardis sian kutiman seriozeco. ``Kio estis tio, pri kio
vi volis paroli kun mi, S-ro Potter.''

\emph{Mi havas misteran makhelan flankon.} Sed Harry ne povis nur
elbabili tion simple tiel.

``Profesoro Ciuro,'' Harry diris, ``ĉu mi estas el la vojo de fariĝi
Mastron de la Tenebroj, nun?''

Profesoro Ciuro rigaridis Harry'n. ``S-ro Potter,'' li diris solane,
kun nur leĝera grimaco, ``konsila vorto. Estas aĵoj kiel tro perfekta
plenumo. Realaj homoj, kiuj ĵus estis frapitaj kaj humiligitaj dum dek
kvin minutoj, ne starigas kaj gracie pardonas iliajn malamikojn. Tio
estas la speco de aĵo, kion oni faras kiam oni provas konvinkiĉiujn ke
oni ne estas malhela, ne—''

\emph{``Mi ne povas kredi tion! Vi ne povas farigi ke ĉiuj eblaj observoj
  povas konfirmi vian teorion!''}

``Kaj tio estis \emph{iomete} tro da indigno.''

\emph{``Kion sur la tero, mi devas fari por konvinki vin?''}

``Por konvinki min ke vi ne havas intencojn de fariĝi Mastron de la
Tenebroj?'' diris Profesoro Ciuro, nun aspektante tute amuzita. ``Mi
supozas ke vi povas simple levi vian dekstran manon.''

``Kio?'' Harry diris svage. ``Sed mi povas levi mian manon se mi
volas—'' Harry haltis, sentante relative stulta.

``Ja,'' diris Profesoro Ciuro. ``Vi povas fari ĝin same facile, ne
gravas via intenco. Estas nenio, kion vi povas fari por konvinki min,
ĉar mi tiam scias ke ĝi estis ekzakte tion, kion vi provis fari. Kaj
se ni volas esti pli preciza, tiam dum mi supozas ke tio estas apenaŭ
ebla ke perfektaj bonaj homoj ekzistas eĉ se mi neniam renkontis iun,
ĝi estas malgraŭe neverŝajna ke iu, kiu estis frapita dum dek kvin
minutoj kaj poste stariĝis, sentis subitan ondon de afabla pardoneco
por liaj atakantoj. Alie, estas \emph{malpli} verŝajna ke juna infano
imagus tion, kiel la \emph{rolo por roli}, por konvinki sian
instruiston kaj samklasanojn ke li ne estos la sekvanta Mastro de la
Tenebroj. La graveco pri ago ne estas en tio, al kio la ago similas
\emph{je la surfaco}, S-ro Potter, sed en la stato de menso, kiu
farigas tiun agon pli aŭ malpli verŝajna.''

Harry palpebrumis. Li ĵus aŭdis la dikotomion inter la reprezentada
heŭristiko kaj la Bajezia difino de pruvo priskribita al li per
sorĉisto.

``Sed denove,'' diris Profesoro Ciuro, ``ĉiuj ajn povas voli impresi
siajn amikojn. Tio ne bezonas malhelon. Do sen ke tio estas ia ajn
konfeso , S-ro Potter, diru al mi honeste. Kiu penso estis en via
menso je la momento kiam vi malpermesis ian ajn venĝon. Ĉu tiu penso
estis vera impulso aŭ pardono? Aŭ ĉu tio estis konscio de kiel viaj
samklasanoj vidus la agon?''

\emph{Kelkfoje, ni faras nian propran kanton de fenikso.}

Sed Harry ne diris ĝin laŭte. Estis klara ke la Profesoro Ciuro ne
kredintus lin, kaj verŝajne respektintus lin malpli por provi dissendi
tiel travideblan mensogon.

Post kelkaj momentoj de silento, Profesoro Ciuro ridetis kun
kontentigo. ``Kredu ĝin aŭ ne, S-ro Potter,'' diris la Profesoro, ``vi
ne bezonas timi ke mi malkovris vian sekreton. Mi ne diros al vi
rezigni fariĝi la sekvan Mastron de la Tenebroj. Se mi povus returne
turnigi la montrilojn de la tempo, kaj iel eligi la ambicion de la
menso de la infana mi mem, la mi mem de hodiaŭ ne profitus pro la
ŝanĝo. De kiam mi povas memori, tio estis mia celo, ĝi gvidis min por
studi kaj lerni kaj bonigi min mem kaj iĝi pli forta. Ni fariĝas tion,
kion ni devas iĝi, per sekvi niajn dezirojn, kien ajn ili kondukas
nin. Tio estas la kompreno de Salazaro. Demandu al mi gvidi vin al la
biblioteka sekcio, kiu havas la samajn librojn, kiujn mi legis kiam mi
havis dek tri jarojn, kaj mi feliĉe montros al vi la vojon.''

``Por la amo de malsaĝeco,'' Harry diris, kaj li sidiĝis sur la
malmolan plankon, kaj poste kuŝiĝis sur la dorso, rigardante
supren la malproksimajn arkojn je la plafono. Tio estis eble plej
proksime kiel li povis kolapsi pro malespero sen vundi sin mem.

``Ankoraŭ tro da indigno,'' observis Profesoro Ciuro. Harry ne
rigardis sed li povis aŭdi la eligitan ridadon en la voĉo.

Kaj Harry ekkonsciis.

``Fakte, mi pensas ke mi scias tion, kio konfuzas vin tie ĉi.'' Harry
diris. ``Tio estas pri kio mi volas paroli kun vi, fakte. Profesoro
Ciuro, mi pensas ke tio, kion vi estas vidanta estas mia mistera
malhela flanko.''

Estis paŭzo.

``Via\ldots malhela flanko\ldots''

Harry sidiĝis. Profesoro Ciuro estis rigardanta lin, kun unu el la
plej stranga esprimo, kiun Harry jam vidis sur la vizaĝo de iu, sen
paroli de iu ajn tiel digna kiel Profesoro Ciuro.

``Tio okazas kiam mi koleriĝas,'' Harry klarigis. ``Mia sango iĝas
malvarma, ĉioj iĝas malvarmaj, ĉioj ŝajnas perfekte klaraj\ldots
Retrospektive, tio estas kun mi ekde longe—je mia unua jaro en Mugla
lernejo, iu provis ŝteli mian pilkon, kaj mi tenis ĝin malantaŭ mia
dorso, kaj piedpatis lin en la suna plekso, ĉar mi legis ke ĝi estis
malforta punkto, kaj la aliaj infanoj ne enuis min post tio. Sed tio
estas nur freŝdate ke mi estis sub sufiĉa streso por rimarki ke ĝi
estas reala, vi scias, mistera malhela flanko, kaj ne nur problemo de
kolera kontrolo kiel la lerneja psikologiisto diris. Kaj mi ne havas
ian ajn supran magian povon, kiam tio okazas, tio estas unu el la
unuaj aĵoj kiujn mi kontrolis.''

Profesoro Ciuro frotis sian nazon. ``Lasu min pripensi pri tio,'' li diris.

Harry atendis silente dum tuta minuto. Li uzis tiu tempo por stariĝi,
kio estis pli malfacila ol li anticipis.

``Nu,'' Profesoro Ciuro diris post momento. ``Mi supozas ke estas io,
kion vi povus diri al mi, kio konvinkus min.''

``Mi jam \emph{divenis} ke mia malhela flanko estas reale nur alia
parto de mi kaj ke la respondo ne estas neniam koleriĝi sed lerni
resti sub kontrolo per akcepti ĝin, mi ne estas stulta aŭ io ajn, kaj
mi vidis tiun ĉi historion sufiĉe da fojoj por scii tien, kien ĝi
iras, sed tio estas malfacila kaj vi ŝajnas esti iu persono kiu
povas helpi min.''

``Nu\ldots jes\ldots tre akrevida, S-ro Potter, mi devas diri\ldots
tiu ĉi flanko de vi, kiel vi ŝajnas jam konjektis, via intenco de
mortigi, kiu estas kiel vi diris parto de vi\ldots''

``Kaj bezonas esti trejnita,'' Harry diris por kompletigi la skemon.

``Kaj bezonas esti trejnita, jes.'' La stranga esprimo estis ankoraŭ
sur la vizaĝo de Profesoro Ciuro. ``S-ro Potter, se vi vere ne deziras
iĝi la sekva Mastro de la Tenebroj, tial kiu estis la ambicio, kiun la
Ordiganta Ĉapelo provis konviki vin forlasi, la ambicio pro kiu vi
estis ordigita al Serpentimo?''

``Mi estis ordigita en \emph{Korvungo!}''

``S-ro Potter,'' diris Profesoro Ciuro, nun kun multe pli kutime
aspekta rideto seka, ``Mi scias ke vi kutimas ke ĉiuj ĉirkaŭ vi estas
stultaj, sed bonvolu ne konsideru min kiel unu el ili. La ŝanco ke la
Ordiganta Ĉapelo faras sian unuan bubaĵon post ok cent jaroj kiam ĝi
estis sur via kapo estas tiel eta ke ĝi ne valoras esti
konsiderita. Mi supozas ke estas apenaŭ ebla ke vi klakis viajn
fingrojn kaj inventis iun simplan kaj lertan manieron por venki la
sorĉo kontraŭ kolero en la ĉapelo, kvankam mi mem povas pensi pri nenian
manieron. Sed de for, la plej probabla klarigo estas ke Dumbledore
decidis ke li ne estis ĝoja kun la elekto de la Ĉapelo por la
Knabo-Kiu-Postvivis. Tio estas evidenta por iu ajn kun la plej leĝera
kvanto da komuna saĝo, do via sekreto estas sekura en Herpŭrko.''

Harry malfermis la buŝon, kaj poste fermis ĝin denove, kune la sento
de kompleta senhelpeco. Profesoro Ciuro malpravis, sed malpravis laŭ
iu tiel konvinkanta maniero ke Harry komencis pensi ke ĝi simple
\emph{estis} la racia juĝo, donate la atestaĵojn disponeblajn al
Profesoro Ciuro. Estis momentoj, neniam \emph{antaŭvideblaj} momentoj
sed tamen kelkaj momentoj, kiam vi akiras neprobablajn atestaĵojn kaj
kiam la plej bona sciebla diveno estas malĝusta. Se vi havus medicinan
teston kiu estis malĝusta unu fojo el mil, kelkfoje ĝi tamen estus
malĝusta ĉiaokaze.

``Ĉu mi povas demandi al vi neniam ripeti tion, kion mi estas dironta al vi?'' diris Harry.

``Absolute,'' diris Profesoro Ciuro. ``Konsideru min demanditan''

Harry ankaŭ ne estis stulta. ``Ĉiu mi povas konsideri ke vi respondis jes?''

``Tre bone, S-ro Potter. Vi povas efektive tiel konsideri.''

``*Profesoro Ciuro—*''

``Mi ne ripetos tion, kion vi estas dironta al mi,'' Profesoro Ciuro diris, ridetante.

Ili ambaŭ ridis, poste Harry fariĝis serioze denove. ``La Ordiganta
Ĉapelo ŝajnis pensi ke mi fariĝos Mastro de la Tenebroj
almenaŭ ke mi irintus al Huflopufo,'' Harry diris. ``Sed mi ne volas fariĝi iu.''

``S-ro Potter\ldots'' diris Profesoro Ciuro. ``Ne prenu tion
malĝuste. Mi promesas ke via respondo ne estos juĝita. Mi nur volas
scii tion, kion vi respondos honeste. Kial ne?''

Harry havis tiun malhelpecan senton denove. \emph{Vi ne devas fariĝi
Mastro de la Tenebroj} estis tiel evidenta teoremo en la morala
sistemo ke ĝi estis malfacila por priskribi la realajn etapojn de
pruvo. ``Hum, homoj estus vunditaj?''

``Certe vi volis vundi homojn,'' diris Profesoro Ciuro. ``Vi volis
vundi tiujn ĉi brutalojn hodiaŭ. Esti Mastro de la Tenebroj volas diri
ke la homoj, kiujn vi volas vundi, fariĝas vunditaj.''


Mankis vortojn al Harry, kaj poste li decidis simple uzi la
evidentan. ``Unue, nur ĉar mi volas vundi iun ne volas diri ke tio
estas ĝusta—'' 

``Kio farigas ion ĝusta, se ne vi volante ĝin?''

``Ha,'' Harry diris, ``utilismo per prefero.''

``Pardonu min?'' diris Profesoro Ciuro.

``Ĝi estas la etika teorio ke la bono estas tio, kio kontentigas la preferon de la plejparto da homoj—''

``Ne,'' Profesoro Ciuro diris. Liaj fingroj frotis la randon de sia
nazo. ``Mi ne pensas ke ĝi estas komplete tio, kion mi provis
diri. S-ro Potter, finfine ĉiuj homoj faras tion, kion li volas
fari. Kelkfoje homoj donas nomojn kiel 'ĝusta' al aĵoj, kiujn li volas
fari, sed kiel oni eble povus fari aĵon, kiu ne estas onia volo?''

``Nu, evidente,'' Harry diris. ``Mi ne povas agi laŭ moralaj
konsideroj se ili mankas potencon por movigi min. Sed tio ne volas
diri ke mia volo de vundi tiujn ĉi Serpentimojn havas pli da potenco
por movigi min ol moralaj konsideroj!''

Profesoro Ciuro palpebrumis.

``Por ne mencii,'' Harry diris, ``esti Mastro de la Tenebroj volus
diri ke multe da senkulpaj spektantoj estus vunditaj ankaŭ!''

``Kial tio gravas por vi?'' Profesoro Ciuro diris. ``Kion ili faris por vi?''

Harry ridis. ``Ho, nun tio estas tiel subtila ol \emph{La Ribelo de
Atlas}.''

``Pardonu min?'' Profesoro Ciuro diris denove.

``Ĝi estas libro, kiun miaj gepatroj malpermesis min por legi, ĉar ili
pensis ke ĝi koruptos min, do evidente mi legis ĝin ĉiaokaze kaj mi
estis ofendita pro ke ili pensis ke mi falus en tiel evidentaj
kaptiloj. Blah blah blah, oni ĉarmas mian sencon de supereco, oni
diras ke iliaj provas tiri min malsupren, blah blah blah.''

``Do ĉu vi estas diranta ke mi bezonas fari miajn kaptilojn malpli
evidentaj?'' diris Profesoro Ciuro. Li frapetis sian fingron sur sia
vango, aspektante pripensema. ``Mi povas labori sur tio.''

Ambaŭ ili ridis.

``Sed por daŭri kun la aktuala demando,'' diris Profesoro Ciuro, ``kion ili ĉiuj faris por vi?''

``Aliaj homoj faris multaj grandajn aĵojn por mi!'' Harry
diris. ``Miaj gepatroj adoptis min kiam miaj gepatroj mortis ĉar ili
estis bonaj homoj, kaj fariĝi Mastro de la Tenebroj perfidos tion!''

Profesoro Ciuro estis silenta dum momento.

``Mi konfesas,'' diris Profesoro Ciuro kviete, ``kiam mi havis vian
aĝon, tia penso ne povus iam ajn aperi al mi.''

``Mi bedaŭras,'' Harry diris.

``Ne fari tion,'' diris Profesoro Ciuro. ``Tio estas de longe, kaj mi
solvis miajn gepatrajn problemojn laŭ maniero kiu ŝatas al mi. Do ĉu
vi estas tirita malantaŭen per la penso de malaprobo de viaj gepatroj?
Ĉu tio volas diri ke se ili mortas en akcidento, estus nenio por
haltigi vin iĝi—''

``Ne,'' Harry diris, ``Nur ne. Tio estas ilia impulso de afablo kiu
protektas min. Tiu impulso ne estas nur en miaj gepatroj. Kaj tiu
impulso estus tio, kio estus perfidita.''

``Ĉiaokaze, S-ro Potter, vi ne respondis al mia originala respondo,''
diris Profesoro Ciuro. ``Kio estas via ambicio?''

``ho,'' diris Harry. ``Hum\ldots'' Li organizis siajn
pensojn. ``Kompreni ĉiujn gravajn aĵojn, kiuj estas por scii pri la
universo, apliki tiun scion por fariĝi ĉiopova, kaj uzi tiun potencon
por reskribi la realon ĉar mi havas kelkajn malkonsentojn pri la
maniero laŭ kiu ĉio funkcias nuntempe.''

Estis malonga silento.

``Pardonu min, se tiu demando estas stulta, S-ro Potter,'' diris
Profesoro Ciuro. ``sed ĉu vi certas ke vi ne nur konfesis ke vi volas
fariĝi la sekva Mastro de la Tenebroj?''

``Tio estas nur se vi uzas viajn povojn por malbono,'' klarigis
Harry. ``Se vi uzas la povon por la bono, vi estas Mastro de la
Lumo.''

``Mi vidas,'' Profesoro Ciuro diris. Li frapetis la alian vangon kun
fingro. ``Mi supozas ke mi povas akcepti tion. Sed S-ro Potter, dum la
amplekso de via ambicio estas digna de Salazaro li mem, kiel ekzakte
vi propozas fari tion? Ĉu etapo unu estas iĝi granda batalanta
sorĉisto, aŭ Direktoro de Neparolebla, aŭ Ministro de Magio, aŭ—''

``Etapo unu estas iĝi scientisto.''

Profesoro Ciuro estis rigardanta Harry'n kvazaŭ li ĵus ŝanĝiĝis en katon.

``Sciencisto,'' Profesoro diris post kelkaj tempoj.

Harry kapjesis.

``Iu \emph{sciencisto?}'' ripetis Profesoro Ciuro.

``Jes,'' Harry diris. ``Mi devas plenumi miajn celojn per la
povo\ldots de la \emph{Scienco!}''

``Iu \emph{sciencisto!}'' diris Profesoro Ciuro. Estis aŭtentika
indigno sur lia vizaĝo, kaj lia voĉo kreskis pli akra kaj seka. ``Vi
povas esti le plej bona el miaj studentoj! La plej granda batalanta
sorĉisto kiu venis en Herpŭrko ekde kvin jardekoj! Mi ne povas imagi
ke vi malŝparas vian tempon en blanka laboratoria vesto farante
senutilajn aĵojn al ratoj!''

``He!'' diris Harry. ``La scienco estas pli ol nur tio! Ne ke estas io
\emph{malbona} kun eksperimentoj sur ratoj, evidente. Sed scienco
\emph{estas} kiel vi faras por kompreni kaj kontroli la universon—''

``Stultulo,'' diris Profesoro Ciuro, kun voĉo de kvieta kaj amara
intenceso. ``Vi estas stultulo, Harry Potter.'' Li pasis sian manon
kontraŭ sia vizaĝo, kaj kiam lia mano estis pasinta, lia vizaĝo estis
pli kvieta. ``Aŭ pli volonte, vi ne jam trovis vian veran ambicion. Ĉu
mi povas forte rekomendi al vi provi fariĝi Mastro de la Tenebroj
anstataŭ? Mi faros ĉion, kion mi povas fari por helpi vin kiel afero
de publika servico.''

``Vi ne ŝatas sciencon,'' Harry diris malrapide. ``Kial ne?''

``Tiuj stultaj Mugloj mortigos nin iam!'' la voĉo de Profesoro Ciuro
estis kreskinta pli laŭte. ``Ili finos ĉion! Ĉion finos ili!''

Harry sentis sin iom perdita tie. ``Pri kio oni estas parolanta tie ĉi, nukleaj armiloj?''

``*Jes*, nukleaj armiloj!'' Profesoro Ciuro estis preskaŭ krianta
nun. ``Eĉ Li-Kiu-Oni-Ne-Devas-Nomi neniam uzis tiun, eble ĉar li ne
volis regi stakon de cindroj! Ili neniam devintus esti farita! Kaj tio
fariĝos pli malbona kun la tempo!'' Profesoro Ciuro estis staranta
rekte anstataŭ klini sur sia skribotablo. ``Estas pordoj, kiujn vi ne
devas malfermi, estas fermiloj, kiujn vi ne devas rompi! La stultuloj,
kiuj ne povas malhelpi sin enmiskiĝi estas mortigitaj per malgrandaj
danĝeroj multe tro frue, kaj la postvivantoj ĉiuj scias ke estas
sekretoj, kiujn vi ne devas interŝanĝi kun iu ajn, al kiu mankas
inteligento kaj disciplino por malkovri ilin per ili mem! Ĉiuj
potencaj sorĉistoj scias tion! Eĉ la plej teruraj Malhelaj Sorĉistoj
scias tion!  Kaj tiuj stultaj Mugloj ne ŝajnas kapabli malkovri tion!
La avidaj etaj stultuloj kiu malkovris la sekreton de nuklea armilo,
ne gardis ĝin por li mem, ili diris ĝin al iliaj stultaj politikistoj
kaj nun \emph{ni} devas vivi sub la konstanta minaco de anihilacio!''

Tiu estis relative malsama maniero de rigardi aĵojn ol tiu, kun kiu
Harry kreskis. Ĝi neniam aperis al li ke nukleaj fizistikoj devintus
formi konspiron de silento por gardi la sekreton de nuklea armilo de
iu ajn, kiu ne estis sufiĉe inteligenta por esti nuklea scientisto. La
penso estis interesiga, se ne pli. Ĉu ili havintus sekretajn
pasvortojn? Ĉu ili havintus maskojn?


(Fakte, laŭ ĉio, kion Harry sciis, estis multaj specoj de nekredeble
ruinigaj sekretoj, kiujn sciencistoj gardis por ili mem, kaj la
sekreto de nuklea armilo estis la sola kiu eskapis en la naturon. La
mondo aspektus same al li.)

``Mi devos pripensi pri tio,'' Harry diris al Profesoro Ciuro. ``Tio
estas nova ideo por mi. Kaj unu el la \emph{kaŝitaj} sekretoj de
scienco, transdonita de kelkaj raraj instruistoj al iliaj studentoj,
estas kiel eviti forjeti novajn ideojn je la momento kiam via aŭdis
unu kiun vi ne amas.''


Profesoro Ciuro palpebrumis denove.

``Ĉi estas iu speco de scienco, kiun vi aprobas?'' diris Harry,
``Medicino, eble?''

``Spaca vojaĝado,'' diris Profesoro Ciuro. ``Sed la Mugloj ŝajnas
treni iliajn piedojn sur la sola projekto kiu eble povus lasi la
sorĉistoj eskapi tiun ĉi planedon antaŭ ili ekplodas ĝin.''

Harry kapjesis. ``Mi ankaŭ estas granda admiranto de spaca
programo. Almenaŭ ni havas tion komune.''

Profesoro Ciuro rigardis Harry'n. Io tremis en la okuloj de la
Profesoro. ``Mi havos vian vorton, vi promesos kaj ĵuros ke vi neniam
parolos pri tio, kio estas sekvonta, al iu ajn.''

``Vi havas ĝin,'' Harry diris tuj.

``Gardu vin de rompi vian ĵuron, aŭ vi ne ŝatos la rezultojn,'' diris
Profesoro Ciuro. ``Mi nun ĵetos raran kaj potencan sorĉon, ne sur vi,
sed sur la klasĉambro ĉirkaŭ ni. Staru rekte, por ke vi ne tuŝu la
limojn de la sorĉo unufoje kiam ĝi estos ĵetita. Vi ne devas interagi
kun la magio, kiun mi estas daŭriganta. Nur rigardu. Alie mi ĉesigos
la sorĉon.'' Profesoro Ciuro faris paŭzon. ``Kaj provu ne fali.''

Harry kapjesis, konfuze kaj iom maltrankvile.

Profesoro Ciuro levigis sian bastonon kaj diris ion, kion la oreloj
kaj mensoj de Harry tute ne sukcesis kompreni, vortojn kiuj
preterpasis konscio kaj neniiĝis al abismo.

Eta cirklo de marmoro ĉirkaŭ la piedoj de Harry daŭris konstanta. Ĉiu
la marmoro de la planko malaperis, la muroj kaj plafono neniiĝis.

Harry staris sur la eta cirklo el blanka marmoro en la mezo de senfina
kampo de steloj, brulante terure helaj kaj neŝanceliĝaj. Ne estis
tero, ne estis luno, kaj neniu suno, kiujn Harry rekonis. Profesoro
Ciuro restis je la sama loko, kiel antaŭe, ŝvebanta en la mezo de la
stelokampo. La Lakta Vojo estis klare videbla kiel granda bano de lumo
kaj ĝi kreskis pli hela dum la vido de Harry adaptis al la malhelo.

La vido tordis la koron de Harry kiel nenio, kion Harry iam ajn vidis.

``Ĉu ni estas\ldots en spaco\ldots?''

``Ne,'' diris Profesoro Ciuro. Lia voĉo estis malĝaja, kaj pia. ``Sed tio estas vera bildo.''

Larmoj aperis en la okuloj de Harry. Li viŝis ilin for panike, li ne
volis manki tion ĉi pro iu sutlta akvo kiu malklarigis lian vidon.

La steloj ne plu estis eta juveloj metita en giganta velura kupolo,
kiel ili estis en la nokta ĉielo de la tero. Tieĉi estis ne ĉielo
super, ne ĉirkaŭiranta sfero. Nur punktoj el perfekta lumo kontraŭ
perfekta nigreco, senfina malpleno kun nenombreblaj etaj truoj trans
kiuj brilis la brilanco el kelkaj malproksimaj reĝlandoj neimageblaj.

En la spaco, la steloj \emph{aspektis} terure, terure, terure malproksime.

Harry daŭris viŝi siajn okulojn, ankoraŭ kaj ankoraŭ.

``Kelkfoje,'' Profesoro Ciuro diris per voĉo tiel kvieta ke ĝi preskaŭ
ne estis tie, ``kiam tiu ĉi difekta mondo ŝajnas nekutime malaminda,
mi demandas al mi, se povas esti iu alia loko, tre malproksime, kie mi
devintus esti. Mi ne ŝajnas povi imagi ke tiu ĉi loko povas ekzisti,
kaj se mi ne eĉ povas imagi ĝin, tial kiel mi povas kredi ke ĝi
ekzistas? Kaj tamen, la universo ests tre, tre larĝa, kaj eble ke ĝi
estas malgraŭ ĉio?  Sed la steloj estas tiel malproksime, ĝi prenos
longan, longan tempon por alveni tie, eĉ se mi konus la vojon. Kaj mi
scivolas pri kio mi sonĝos, se mi dormus dum longan, longan
tempo\ldots''

Kvankam tio sonis kiel sakrilegio, Harry sukcesis murmuri. ``Bonvolu
lasu min resti tie dum momento.''

Profesoro Ciuro kapjesis, de kie li staris sensubtene kontraŭ la
steloj.

Tio estis facila forgesi la etan cirklon el marmoro sur kiu, vi staris
kaj via propra korpo, kaj iĝi punkto de konscio kiu eble povis esti
senmove, aŭ povus esti move. Kun ĉiuj nekalkuleblaj distancoj, estis
neniu maniero por diri.

Estis tempo de ne tempo.

Kaj poste la steloj malaperis, kaj la klasĉambro revenis.

``Mi bedaŭras,'' diris Profesoro Ciuro, ``sed ne estas havonta viziton.''

``Tio estas bonorde,'' Harry murmuris. ``Tio sufiĉas.'' Li neniam
forgesos tiun ĉi tagon, kaj ne pro la negravaj aĵoj kiuj okazis pli
frue. Li lernos kiel ĵeti tiun sorĉon eĉ se tio estis la lasta aĵo,
kiun li lernos.

Tiam la peza pordo el kverko de la klasĉambro eksplodis el lia ĉarniro
kaj glitis sur la marmora planko kun tre laŭta bruo.

``\emph{CIRINUS! KIEL VI AŬDACAS!}''

Kiel granda ŝtorma nubo, maljuna kaj potenca sorĉisto ekpenetris la
ĉambron, kun aspekto de tiel inkadeska kolero sur la vizaĝo, ke la
severa rigardo kiun Harry vidis pli frue ŝajnis kiel nenio kompare.

Estis malorientiganta ŝlosilo en la menso de Harry turnigante parton
de li en lokon, kiu povis elporti la ŝokon, dum alia parto de li volis
kriante forkuri la plej timiganta aĵon, kiun li iam ajn vidis.

Neniuj partoj de Harry estis feliĉaj pri ke iliaj rigardoj de steloj
estis fininta. ``Direktoro Albus Percival—'' Harry komencis diris je
malvarma voĉo.

\emph{BUM}. La mano de Profesoro Ciuro forte frapis la
skribotablo. ``\emph{S-ro Potter!}'' bojis Profesoro Ciuro. ``Tiu
estas la \emph{Direktoro de Herpŭrko} kaj vi estas ordinara studento!
Vi parolos al li konvene!''

Harry rigardis Profesoron Ciuro.

Profesoro Ciuro estis donanta al Harry severan rigardon.

Neniu el ili ridetis.

La longaj paŝoj de Dumbledore hatlis antaŭ li atingis tien, kie Harry
staris antaŭ la podio kaj Profesoro Ciuro staris apud sia
skribotablo. La Direktoro fikserigardis ambaŭ ilin ŝokite.

``Mi bedaŭras,'' Harry diris per humiligita tono
bonmaniera. ``Direktoro, dankon por voli protekti min, sed Profesoro
Ciuro faris la bonan aĵon.''

Malrapide, la esprimo de Dumbledore ŝanĝiĝis de io, kio povis vaporigi
ŝtalon, en ion simple kolera. ``Mi aŭdis studentojn diri ke tiu ĉi
viro farigis pli maljunajn Serpentimojn mistraktis vin! Ke li
malpermesis vin defendi vin mem!''

Harry kapjesis. ``Li sciis ekzakte tion, kio estas malbona kun mi kaj
li montris al mi kiel ripari ĝin.''

``Harry, \emph{pri kio vi estas parolanta?}''

``Mi lernis al li kiel perdi,'' Profesoro Ciuro diris seke. ``Tiu estas vivgrava kompetento.''

Estis videbla ke Dumbledore ankoraŭ ne komprenis, sed lia voĉo estis
pli kvieta. ``Harry\ldots'' li diris malrapide. ``Se estas ia ajn
minaco, kiun la Profesoro de Defenco faris por malhelpi vin plendi—''

\emph{Vi lunatiko, post hodiaŭ inter ĉiuj tagoj, ĉu vi vere pensas ke mi—}

``Direktoro,'' Harry diris, provante aspekti embarasita, ``tio kio
estas malbona kun mi, ne estas ke mi restas kvieta pri ofendaj
instruistoj.''

Profesoro Ciuro subridis. ``Ne perfekte, S-ro Potter, sed sufiĉe bone
por via unua tago. Direktoro, ĉu vi restis sufiĉe da tempo por aŭdi
pri la kvindek unu poentoj por Korvungo, aŭ ĉu vi rapidiĝis tuj ekde
vi aŭdis la unuan parton?''

Rapida esprimo de konfuzo pasis trans la vizaĝo de Dumbledore, sekvita
per surpizo. ``Kvindek unu poentoj por Korvungo?''

Profesoro Ciuro kapjesis. ``Li ne atendis ilin, sed tio ŝajnis
konvena. Diru al Profesorino McGonagall ke mi pensas ke la afero trans
kiu S-ro Potter ĵus pasis, por regajni la perditajn poentojn, sufiĉe
konvenos por ke ŝi akceptas. Ne, Direktoro, S-ro Potter ne diris iun
ajn al mi. Estas facila vidi kiun parton de la eventoj de tiu ĉi tago
estas ŝia laboro, ekzakte kiel mi scias ke la fina kompromiso estis de
via propra sugesto. Tamen, mi min demandas, kiel sur la Tero, S-ro
Potter kapablis gajni superecon sur ambaŭ vi kaj Skoldo kaj poste kiel
Profesorino McGonagall kapablis gajni superecon sur li.''

Iel Harry sukcesis kontroli sian vizaĝon. Ĉu tio estis tiel evidenta
por vera Serpentimo?

Dumbledore venis pli proksime al Harry, ekzamene. ``Via koloro
aspektas iom pala, Harry,'' la maljuna sorĉisto diris. fikse rigardis
proksime la vizaĝon de Harry. ``Kion vi havis por hodiaŭ tagmanĝo?''


``Kio?'' Harry diris, sia menso ŝanceliĝante en subitan konfuzon. Kial
Dumbledore farus demandon pri frititaj ŝafidaĵoj kaj mallarĝaj tranĉoj
de brokolo, kiam tio estis nur la pli malverŝajna kaŭzo de—


La maljuna sorĉisto rektiĝis. ``Tio ne gravas, mi pensas ke vi fartas bone.''

Profesoro Ciuro tusis, laŭte kaj intence. Harry rigardis al la
Profesoro, kaj vidis ke Profesoro Ciuro estis rigardanta akre
Dumbledore'n.

``\emph{Ah-hem!}'' Profesoro Ciuro diris denove.

Dumbledore kaj Profesoro Ciuro rigardis unu la alian en la okuloj, kaj io ŝajnis pasi inter ili.

``Se tion vi ne diros al li,'' Profesoro Ciuro tiam diris, ``Mi tion faros, eĉ se vi maldungas min pro tio.''

Dumbledore suspiris kaj turniĝis al Harry. ``Mi pardonpetas por invadi
vian mensan privatecon, S-ro Potter,'' la Direktoro diris
formale. ``Mi ne havis celon, krom determini se Profesoro Ciuro faris
same antaŭe.''

\emph{Kio?}

La konfuzo restis ekzakte same longe kiel ĝi prenis al Harry por
kompreni tion, kio ĵus okazis.

``\emph{Vi—!}''

``Afable, S-ro Potter,'' diris Profesoro Ciuro. Lia vizaĝo estis
severa, sed li estis rigardanta Dumbledore'n.

``Legilimencio estas kelkfoje maldinstingita disde kumona saĝo,''
diris la Direktoro. ``Sed ĝi lasas postsignojn, kiujn alia potenca
Legilimenanto povas detekti. Ĝi estis tio, kion mi estis serĉanta,
S-ro Potter, kaj mi faris nerilatan demandon por certigi ke vi ne
pensis pri iu ajn grava aĵo dum mi rigardis.''

``\emph{Vi devintus demandi al mi unue!}''

Profesoro Ciuro balancis la kapon. ``Ne, S-ro Potter, la Direktoro
havis iun motivadon por liaj zorgoj, kaj se li havintus peti vian
permeson, vi estus pensanta pri ekzakte tiuj aĵoj, kiujn vi ne dezirus
lin por scii.'' La voĉo de Profesoro Ciuro kreskis iom pli akra. ``Mi
estis relative pli malkvieta, Direktoro, pro ke vi ne vidis bezonon
por al li diri ĝin poste!''

``Vi farigis pli mafacila konfirmi sian mensan privatecon je estontaj
okazoj,'' Dumbledore diris. Li ĵetis malvarman rigardon al Profesoro
Ciuro. ``Ĉu tio estis intenca, mi pri tio min demandas?''

La esprimo de Profesoro Ciuro daŭris malkompatema. ``Estas tro da
Legilimenantoj en tiu ĉi lernejo. Mi insistas ke S-ro Potter ricevu
instruadon de Oklumencio. Ĉu vi permesos min esti lian instruiston?''

``Absolute ne,'' Dumbledore diris direkte.

``Mi ne pensis ke vi akceptos.  Do pro ke vi senhavigis lin je mia
senkosta servado, vi pagos la instruiadon de S-ro Potter, disdonita
per agnositka instruisto de Oklumencio.''

``Tiaj servadoj ne estas malmulte kostaj,'' Dumbledore diris,
rigardante Profesoro'n Ciuro, kelke surprizita. ``Tamen, mi konas
kelkajn homoj, kiuj povas—''

Profesoro Ciuro kapneis firme. ``Ne, S-ro Potter petos al la
respondeculo de sia bankokonto je Gringoto, ke al li rekomendas
neŭtralan instruiston. Kun respekto, Direktoro Dumbledore, post la
eventoj de ĉi tiu mateno, mi devas protesti kontraŭ la ebleco ke vi aŭ
viaj amikoj povas atingi la mensojn de S-ro Potter. Mi devas ankaŭ
instisti, ke tiu ĉi instruisto faru Nerompeblan Voton pri ne riveli
iun ajn, kaj ke li konsentu esti Forgesigita je ĉiuj seancoj tuj
poste.''

Dumbledore estis malridetanta. ``Tiaj servadoj estas ege malmulte
kostaj, kiel vi scias, kaj mi ne povas malhelpi min scivoli kial
\emph{vi} opinias ke ili estas necesaj.''

``Se la mono estas la problemo,'' Harry diris laŭte, ``Mi havas
kelkajn ideojn por fari grandaj kvantoj da mono rapide—''

``Dankon al vi Cirinus, via saĝeco estas nun tute evidenta kaj mi
bedaŭras pri kontesti ĝin. Viaj zorgoj pri Harry, ankaŭ honoras vin.''

``Nedankinde,'' diris Profesoro Ciuro. ``Mi esperas ke ne havos iun
ajn protestoj, se mi donas al li atenton tute particularan.'' La
vizaĝo de Profesoro Ciuro estis nun tre serioza kaj tre senmova.

Dumbledore rigardis Harry'n.

``Tio estas egale mia volo,'' Harry diris.

``Do tiel tio estos\ldots'' la maljuna sorĉisto diris. Io stranga
pasis trans lia vizaĝo. ``Harry\ldots vi devas kompreni ke se vi
elektas tiun viron kiel via instruisto kaj kiel via amiko, via unua
mentoro, tiam laŭ iu maniero aŭ alia, vi perdos lin, kaj la maniero
per kiu vi perdos lin, povas aŭ ne povas permesi vin retrovi lin
denove.''

Tio ne aperintis al Harry. Sed estis malfeliĉo sur la ofico de
Profesoro de Defenco\ldots iu kiu verŝajne funkciis kun perfekta
reguleco dum jardekoj\ldots

``Probable,'' diris Profesoro Ciuro mallaŭte, ``sed mi restos je sia
tuta dispono dum mi vivtenas min.''

Dumbledore suspiris. ``Mi supozas ke tio estas ŝparema, almenaŭ, pro
tio ke kiel Profesoro de Defenco vi estas \emph{jam} kondamnita je ia
nekonata maniero.''

Harry devis labori malfacile por devigi sian vizaĝon daŭri senesprime,
kiam li konsciis tion, kion Dumbledore estis reale implicanta.

``Mi informos al S-ino Pinĉo ke S-o Potter estas permesita akiri librojn pri Oklumencio,'' diris Dumbledore.

``Estas antaŭprepara trejnado, kiun vi devas fari sole per vi mem,''
diris Profesoro Ciuro al Harry. ``Kaj mi sugestas ke vi ekrapidu kaj
faru ĝin.''

Harry kapjesis.

``Mi estas lasonta vin, tiam,'' diris Dumbledore. Li balancis la kapon
je la direkto de ambaŭ Harry kaj Profesoro Ciuro kaj foriris, marĉante
iom malrapide.

``Ĉu vi povas ĵeti la sorĉon denove?'' Harry diris ĵus post ol
Dumbledore estis foririnta.

``Ne hodiaŭ,'' diris Profesoro Ciuro kviete, ``kaj ankaŭ ne morgaŭ, mi
tion timas. Ĝi lacigas multe por esti ĵetita, tamen malpli por daŭri,
do kutime mi preferas subteni ĝin kiel eble plej longe. Tiufoje mi
ĵetis ĝin pro la momenta impulso. Mi devintus pensi, kaj konscii ke ni
povis estis interrompita—''

Dumbledore estis nun la malpli preferata persono de Harry en la tuta mondo.

Ili ambaŭ suspiris.

``Eĉ se tio estis la sola fojo, mi vidus ĝin,'' Harry diris, ``mi neniam haltos esti dankema al vi.''

Profesoro Ciuro kapjesis.

``Ĉu vi aŭdis pri la Pioniro programo?'' Harry diris. ``Estis sondoj,
kiuj flugis al malsamaj planedoj kaj prenis fotojn. Du el la sondoj
finis per iri laŭ trajektorioj, kiuj gvidis ilin el la Suna Sistemo
kaj en la interstelan spacon. Do, ili metis oran ŝildon sur la sondoj,
kun bildo de viro, kaj de virino kaj montris kie trovi nian Sunon en
la galaksio.''

Profesoro Ciuro daŭris silento dum momento, kaj poste ridetis. ``Diru
al mi, S-ro Potter, ĉu vi povas diveni tiun penson, kiu pasis trans
mia menso kiam mi finis krei la tridek sep aĵojn de mia listo de
aĵoj, kiujn mi neniam devus fari kiel Mastro de la Tenebroj? Metu vin
mem en miaj ŝuoj—imagu vin mem je mia loko—kaj divenu.'' 

Harry imagis sin mem rigardanta la liston de tridek sep aĵojn kiujn vi
ne devas fari unufoje kiam vi fariĝis Mastro de la Tenebroj.

``Vi decidis ke se vi devis sekvi la \emph{tutan} liston ĉiam
ĉiaokaze, tiam ne estis vera utileco por fariĝi Mastro de la
Tenebroj,'' Harry diris.


``*Precize*,'' diris Profesoro Ciuro. Li estis ridete grimacanta. ``Do
mi malrespektos la regulon nombro du—kiu estis simple 'ne
fanfaronu'—kaj diri al vi pri io, kion mi faris. Mi ne vidas kiel la
kono povas fari iun ajn damaĝon. Kaj mi forte suspektas ke vi
malkovros ĝin ĉiaokaze, unufoje kiam oni konos unu la alian
sufiĉe. Malgraŭe\ldots mi devas havi vian ĵuron ke vi neniam parolos
pri tio, kion mi estas dironta.''

``Vi havas ĝin!'' Harry havis la senton, ke tio estis estonta nekredeble bone.

``Mi abonis Muglan bultenon, kiu gardis min informata pri la progreso
de Spaca Vojaĝo. Mi ne aŭdis pri Pioniro 10 ĝis ili raportis ĝian
lanĉon. Sed kiam mi malkovris ke Pioniro 11 ankaŭ foriros la Sunon
Sistemon eterne,'' Profesoro Ciuro diris, lia rideto kreskis al la
plej larĝa, kiun Harry iam vidis sur lia vizaĝo, ``mi ŝteliris NASA'n,
jes mi faris tion, kaj mi ĵetis etan sorĉon sur tiu amata ora ŝildo, kiu
farigos ĝin daŭri multe pli longe ol ĝi estintus alie.''

\ldots

\ldots

\ldots

``Jes,'' Profesoro Ciuro diris, kiu nun ŝajnis iĝinta ĉirkaŭ kvindek
piedojn pli alta, ``Mi antaŭpensis ke vi reagos tiel.''

\ldots

\ldots

\ldots

``S-ro Potter?''

``\ldots Mi ne povas pensi pri io ajn por diri.''

``'Vi gajnas' ŝajnas adekvata,'' diris Profesoro Ciuro.

``Vi gajnas,'' Harry diris tuj.

``Vidu?'' diris Profesoro Ciuro. ``Ni povas nur imagi tiun gigantan
amason de problemoj, kiun vi povintus akiri se vi ne sukcesintus diri
tion.''

Ili ambaŭ ridis.

Alia penso aperis al Harry. ``Vi ne aldonis iun ekstran informon al la
ŝildo, ĉu ne?''

``Ekstra informo?'' diris Profesoro Ciuro, sonante kiel se la ideo
neniam aperis al li antaŭ kaj ke tio tute interesis lin.

Tio farigis Harry'n iom suspekta, kiam oni konsideris ke tio prenis
malpli ol unu minuto por ke \emph{Harry} pensis pri tio.

``Eble vi aldonis holografian mesaĝon kiel en \emph{Star Wars?}''
diris Harry. ``Aŭ\ldots mm. Portreto ŝajnas povi stoki la saman
kvanton da informoj ol tuta cerbo de iu homo\ldots vi ne povis aldoni
iun ajn ekstran pezon al la sondo, sed eble vi povintus ŝanĝigi
antaŭekzistan parton en portreton de vi mem? Aŭ vi trovis volontulon
mortantan pro definitiva malsano, ŝtelirigis rin en NASA'n, kaj ĵetis
sorĉon por certigi ke ria fantomo finis en la ŝildon—''

``S-ro Potter,'' Profesoro Ciuro diris, sia voĉo subite akra, ``sorĉo
necesiganta homan morton estus certe klasifikata per la Ministerio kiel
Malhela Arto, senkonsidere la cirkonstancoj. Studentoj ne devus
esti aŭdita parolante pri tiaj aĵoj.''

Kaj la mirinda aĵo pri la maniero per kiu Profesoro Ciuro diris tion,
estis kiel perfekte ĝi daŭrigis kredindan malkonfeson. Ĝi estis dirita
per la ekzakte konvena tono por iu, kiu ne estis volonta diskuti tiajn
aĵojn kaj pensis ke studentoj devus stiriĝi for de ili. Harry honeste
\emph{ne sciis} se Profesoro Ciuro estis nur atendante paroli pri tio
post kiam Harry estos lerninta protekti siajn mensojn.

``Konsentite,'' Harry diris. ``Mi ne parolos pri tiu ideo al iu ajn alia.''

``Bonvolu estu diskreta pri la tuta afero, S-ro Potter,'' Profesoro
Ciuro diris. ``Mi preferas vivi mian vivon se altiri publikan
atenton. Vi ne trovos ion ajn en la ĵurnaloj pri Cirinus Ciuro ĝis mi
decidis ke estis la bona momento por instrui Defencon ĉe Herpŭrko.''


Tio ŝajnis iomete malgaja, sed Harry komprenis. Kaj poste Harry
komprenis la implikojn. ``Do kiom da aliaj nekredeblaj aĵoj vi estas
farinta, pri kiuj neniu iam ajn aŭdis eĉ ion ajn—'' 

``Oh, kelkaj,'' diris Profesoro Ciuro. ``Sed mi pensas ke tio estas
tute sufiĉa por hodiaŭ, S-ro Potter, mi konfesas ke mi sentas min iom
laca—''

``Mi komprenas. Kaj dankon al vi. Por ĉio.''

Profesoro Ciuro kapjesis, sed li estis kliniĝanta pli forte sur sia skribotablo.

Harry rapide foriris.



