\chapter{Ĉio, kion mi kredas, estas falsa}

\lettrinepara[ante=“]{B}{one}, nur por esti klara'', diris Harry, ``Se la
profesorino ŝvebigas vin, Paĉjo, kiam vi scias, ke vi ne estas ligita al iu ajn
drato, tio estos sufiĉa pruvo. Vi ne ŝanĝos vian opinion, kaj ne diros, ke tio
estas nur ruzo de magiisto. Tio ne estus lojale ludi. Se vi vin sentas tiel, vi
devas diri tion nun, kaj ni povos eltrovi alian eksperimenton anstataŭan.''

La patriĉo de Harry, Profesoriĉo Mikael Verres-Evans, rulis la okulojn. ``Jes, Harry.''

``Kaj vi, Panjo, via teorio diras, ke la profesorino devus kapabli fari tion,
kaj se tio ne okazas, vi rekonos, ke vi estis eraranta. Nenio pri tio, ke la
magio ne funkcias kiam homoj estas skeptikaj pri ĝi, aŭ io ajn simila.''

Vicdirektorino Minerva McGonagall rigardis Harry'n kun perpleksa aspekto. Ŝi
aspektis tute sorĉe en sia nigra robo kaj sub sia pinta ĉapelo, sed kiam ŝi
parolis, ŝi sonis formale kaj skote, kio tute ne taŭgis al la stilo. Unuavide,
ŝi aspektis kiel iu, kiu rikanas kaj metas bebojn en kaldronon, sed la tuta
efiko estis ruinigita tuj kiam ŝi malfermis la buŝon. ``Ĉu tio sufiĉas, S-iĉo
Potter?'', ŝi diris. ``Ĉu mi devas komenci kaj demonstri?''

``\emph{Sufiĉas}? Probable ne,'' diris Harry. ``Sed almenaŭ tio helpos. Bonvolu
komenci, Vicdirektorino.''

``Simpla profesorino sufiĉos'', ŝi diris, kaj poste: ``\emph{Flugapika
Leviosu$^{\textit{a:\ref{nomoj:leviosu}}}$}.''

Harry turnis sian rigardon al sia patriĉo.

``Ho'', diris Harry.

Lia patriĉo turnis sian rigardon malsupren al li. ``Ho'', eĥis lia patriĉo.

Poste, la profesoriĉo Verres-Evans rigardis malsupren al Profesorino McGonagall.
``Konsentite, vi povas delasi min nun.''

La profesorino zorge mallevis la patriĉon de Harry sur la teron.

Harry hirtigis sian hararon per unu mano. Eble tio estis la stranga parto en li,
kiu estis jam konvinkita, sed\ldots ``Tio estas iom kontraŭklimaksa,'' Harry
diris. ``Oni povus pensi, ke estus ia pli drama mensa okazaĵo asociita kun la
ĝisdatigo per observado de infinitezima probablo—'' Harry haltis. Panjo, la
sorĉistino, kaj eĉ lia paĉjo ĵetis al li \emph{tiun rigardon}, denove. ``Mi
volas diri, kun ekscii ke ĉio, kion mi kredas, estas malĝusta.''

Serioze, tio devintus esti pli drama. Lia cerbo devintus forĵeti sian nunan
stokon da hipotezoj pri la universo; neniu el tiuj permesis, ke tio okazu. Sed
anstataŭ tio, lia cerbo simple ŝajnis daŭrigi normale. \emph{Konsentite, mi
vidis profesorinon de Herpŭrko skui sian bastonon kaj levigi mian patriĉon en la
aero\ldots{}, kaj kio nun ?}

La sorĉistino estis ridetanta bonvole al ili, aspektante tute amuzita. ``Ĉu vi
ŝatas plian demonstron, S-iĉo Potter?''

``Vi ne estas devigata'', diris Harry. ``Ni plenumis definitivan eksperimenton.
Sed\ldots''. Harry hezitis. Li ne povis malhelpi sin; estis justa kaj deca esti
scivola. ``Kion alian vi \emph{povas} fari?''

Profesorino McGonagall transformiĝis en katon.

Harry retropaŝis sen pripensi, kontraŭpedalante tiel rapide, ke li stumblis sur
vaga pilo da libroj, kaj poste severe alteriĝis sur sia pugo kun klako. Liaj
manoj iris malsupren por kapti lin, sen sufiĉe sukcesi, kaj li sentis varmegan
doloron en sia ŝultro kiam la pezo falis sen subteno.

Tuj, la eta katino reŝanĝiĝis en virinon en robo. ``Mi bedaŭras, S-iĉo Potter,''
diris la sorĉistino, sonante sincere, kvankam la angulo de ŝiaj lipoj ektremis
supren. ``Mi devintus averti vin''

Harry prenis mallongajn anhelojn. Lia voĉo eliris sufoke: ``\emph{Vi ne povas FARI tion!}''

``Tio simple estas Transfiguro$^{\textit{a:\ref{nomoj:transfiguro}}}$,'' diris
Profesorino McGonagall. ``Animagus-transformo, por esti ekzakte.''

``Vi ŝanĝis vin en katon! Etan katon! Vi malrespektis la principon de Konservado
de Energio! Ĝi ne estas arbitra principo, ĝi estas implikita per la formo de la
Hamiltoniana kvantumo! Forĵeti ĝin detruas unitecon, kaj tiel vi finas kun
superlumecaj signaloj. Kaj katoj estas komplikaj! Homa menso simple ne povas
bildigi la tutan anatomion de kato, kaj, kaj la tutan biokemion de kato, kaj kio
pri la \emph{neŭrologio}? Kiel vi povas daŭrigi \emph{pensi} kun cerbo kies
grandeco egalas tiun de kata cerbo?''

La lipoj de Profesorino McGonagall tremis pli forte nun. ``Magio.''

``Magio \emph{ne sufiĉas} por fari tion! Vi devas esti dio!''

Profesorino McGonagall palpebrumis. ``Tio estas la unua fojo, ke iu nomas min \emph{tiel}.''

La vidado de Harry iĝis malklara, kaj lia cerbo komencis kompreni tion, kio ĵus
rompiĝis. La tuta ideo de unuigita universo kun simplaj matematikaj principoj,
ĉio tio estis ĵetita en la rubujon; la tuta nocio de \emph{fiziko}. Tri mil
jaroj da solvado de komplikaj aĵoj en pli malgrandajn pecojn por malkovri, ke la
muziko de planedoj estis la sama melodio kiel falanta pomo, kaj por trovi, ke la
veraj leĝoj estis perfekte universalaj, kaj ke ne ekzistis escepto ie ajn, kaj
ke ili prenis la formon de simplaj matematikaj principoj regantaj la malgrandajn
partojn, \emph{por ne mencii la fakton}, ke menso estis la cerbo, kaj la cerbo
estis farita el neŭronoj, kaj ke la cerbo estis tio, kio homo estas—

Kaj poste, virino ŝanĝiĝis en katon, des pli malbone por ĉio tio.

Centoj da demandoj batalis por la prioritato de la lipoj de Harry, kaj la
gajninto verŝis: ``Kaj, kaj kia sorĉkanto \emph{Flugapika Leviosu} estas? Kiu
elpensas la vortojn de tiu sorĉo, bebo?''

``Tio sufiĉas, S-iĉo Potter,'' Profesorino McGonagall diris akre, kvankam ŝiaj
okuloj brilis kun subpremita plezuro. ``Se vi deziras lerni pri magio, mi
sugestas, ke ni kompletigu paperojn, por ke vi povu iri al Herpŭrko.''

``Bone,'' Harry diris, iom konfuza. Li ordigis siajn pensojn. La Marŝado de
Racio simple devis rekomenci, nur tio. Ili ankoraŭ havis la eksperimentan
metodon, kaj tio estis la grava afero.

``Do, kiel mi iras al Herpŭrko?''

Sufokita ridado eskapis el la buŝo de Profesorino McGonagall, kvazaŭ ĝi estus
eltirita el ŝi per pinĉilo.

``Ne tiel rapide, Harry'', diris lia patriĉo. ``Ĉu vi memoras kial vi ĝis nun ne
iris al la lernejo? Kio pri via stato?''

Profesorino McGonagall turniĝis por alfronti Mikael'n. ``Lia stato? Kio estas?''

``Mi ne dormas bone'', diris Harry. Li skuis senpove siajn manojn. ``Mia
dormciklo daŭras dudek ses horojn; mi ĉiam ekdormas du horojn pli malfrue
ĉiutage. Mi ne povas ekdormi pli frue ol \emph{tio}. Je la deka vespere, je la
noktomezo, je la dua nokte, je la kvara nokte\ldots{}, ĝis mia horloĝo faras la
tutan cirklon. Eĉ se mi provas vekiĝi frue, tio ne faras diferencon, kaj mi
estas vrako dum la tuta tago. Jen kial mi ne iris al normala lernejo ĝis nun''

``Unu el la kialoj'' diris lia patrino. Harry ĵetis al ŝi aĉan ekrigardon.

McGonagall faris longan \emph{hmmmm}. ``Mi ne memoras antaŭe aŭdi tian
aferon\ldots{}'', ŝi diris malrapide. ``Mi kontrolos kun S-ino
Pimfito$^{\textit{n:\ref{nomoj:pimfito}}}$ por vidi, ĉu ŝi konas rimedon''.
Post, ŝia vizaĝo lumiĝis. ``Ne, mi estas certa, ke tio ne estos problemo — Mi
trovos solvon ĝustatempe. Nun,'' Kaj ŝia rigardo akriĝis denove, ``Kiuj estas la
\emph{aliaj} kialoj?''

Harry ĵetis rigardon al siaj gepatroj. ``Mi estas konscienca oponanto kontraŭ la
deviga principo por infanoj, ĉar mi ne devas suferi pro la nefunkcianta lerneja
sistemo, kiu ne provizas instruistojn aŭ studajn materialojn eĉ de minimume
taŭga kvalito.''

Ambaŭ gepatroj de Harry hurlis pro ridado, kvazaŭ ili pensus, ke tio estas tute
granda ŝerco. ``Ho'' diris la patriĉo de Harry, kun brilaj okuloj, ``Ĉu tio
estas kial vi mordis vian matematikan instruistinon en via tria studjaro.''

``\emph{Ŝi ne sciis, kio estas logaritmo!}''

``Evidente'', subtenis la patrino de Harry. ``Mordi ŝin estis tre matura respondo al tio.''

La patriĉo de Harry kapjesis. ``Bone konsiderita politika ago por trakti la
problemon de instruistoj, kiuj ne komprenas logaritmon.''

``\emph{Mi havis sep jarojn!} Kiom longe vi daŭrigos paroli pri tio?''

``Mi scias'', diris sia patrino simpatie, ``Vi mordis \emph{unu} matematikan
instruiston, kaj ili neniam lasos vin forgesi tion, ĉu ne?''

Harry turniĝis al Profesorino McGonagall. ``Jen! Vi vidas kion mi devas toleri?''

``Pardonu min,'' diris Petunia, antaŭ ol ŝi forkuris tra la malantaŭa pordo al
la ĝardeno, de kie ŝia ridado estis klare aŭdebla.

``Estos, ha, estos,'', Profesorino McGonagall ŝajnis havi kelkajn problemojn por
paroli pro iu kialo aŭ alia. ``Ne estos mordoj de instruistoj en Herpŭrko, ĉu
tio estas klara, S-iĉo Potter?''

Harry kapjesis dum li rigardis ŝin. ``Bone, mi ne mordos iun, kiu ne mordis min antaŭe''

Profesoro Mikael Verres-Evans ankaŭ devis forlasi la ĉambron nelonge post kiam
li aŭdis ĉi tion.

``Nu,'' Profesorino McGonagall suspiris post kiam la gepatroj de Harry reakiris
siajn sensojn kaj revenis. ``Nu, mi opinias, ke en ĉi tiuj cirkonstancoj, mi
devus eviti preni vin por aĉeti viajn studmaterialojn ĝis unu tago aŭ du antaŭ
la komenco de la klasoj.''

``Kio? Kial? La aliaj infanoj jam konas la magion, ĉu ne? Mi devas tuj komenci
rekapti mian malfruon!''

``Estu trankvila, S-iĉo Potter,'' respondis Profesorino McGonagall, ``Herpŭrko
tute kapablas instrui la bazojn. Kaj mi suspektas, S-iĉo Potter, ke se mi lasus
vin sola dum du monatoj kun viaj studaj libroj, eĉ sen bastono, kiam mi revenos
al ĉi tiu domo, mi trovus nur krateron el kiu liberiĝas purpura fumo, senhoman
urbon ĉirkaŭ ĝi, kaj plagon da flamantaj zebroj, terurantaj tion, kio restos de
Anglujo.''

La patrino kaj la patriĉo de Harry jesis en perfekta unisono.

``\emph{Panjo!, Paĉjo!}''
