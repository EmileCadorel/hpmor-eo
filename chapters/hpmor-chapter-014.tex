\chapter{La Nekonata kaj la Nekonebla}

\begin{center}\rule{3in}{0.4pt}\end{center}

\emph{``Estis misteriaj demandoj, sed misteria respondo estas oksimoro.''}  

\begin{center}\rule{3in}{0.4pt}\end{center}

``Eniru,'' diris la nesonora voĉo de Profesorino McGonagall.

Harry faris tiel.

Le oficejo de la Vicdirektorino estis pura kaj bone orda; sur la muro
direkte apuda de la skribtablo estis labirinto el lignaj fakoj de ĉiuj
formoj kaj altoj, la plejparto kun pluraj pergamenaj ruloj puŝitaj en
ili, kaj estis iel vere klara ke Profesorino McGonagall sciis ekzakte
kio ĉiuj fakoj enhavis, eĉ se neniu alia sciis. Malantaŭ la skribtablo
estis fermita pordo ŝlosita per pluraj seruroj.

Profesorino McGonagall sidis sur sendorsa tabureto malantaŭ la
skribtablo, aspektante konfuza—Ŝiaj okuloj estis larĝigitaj, kun eble
leĝera noto de anksieco, kiam ŝi vidis Harry'n.

``S-ro Potter?'' diris Profesorino McGonagall. ``Pri kio temas?''

La menso de Harry iĝis malplena. Li estis instruita per la ludo de iri
tien, li atendis ke ŝi havos ion mense\ldots

``S-ro Potter?'' diris Profesorino McGonagall, komencante aspekti
iomete tedita.

Feliĉe, la panikanta cerbo de Harry memoris tiam, ke li \emph{havis}
ion, pri kio li planis diskuti kun Profesorino McGonagall. Io grava
kaj kiu bone valoris ŝian tempon.

``Hum\ldots'' Harry diris. ``Se estas iu ajn sorĉo kiun vi povas ĵeti,
por certigi ke neniu aŭskultas nin\ldots''

Profesorino McGonagall stariĝis el sia seĝo, firme fermis la eksteran
pordon, kaj komencis preni sian bastonon kaj diri sorĉojn.

Estis je tiu momento ke Harry ekkonsciis ke li estis en ĉeesto de
netaskebla kaj oportuneco relative neanstataŭebla de oferi Komed-Teon
al Profesorino McGonagall, kaj li ne povis kredi ke li estis serioze
pensanta ke ĉio iros bone, ke la sodakvo malaperos ĵus post kelkaj
sekundoj, kaj li diris al tiu parto de li mem \emph{silenti}.

Ĝi faris, kaj Harry komencis ordoni mentale kion li estis dironta. Li
ne planis havi ĉi tiun konversacion tiel frue, sed pro tio ke li estis
tie\ldots

Profesorino McGonagall finis ĵeti sorĉon kiu sonis multe pli aĝa ol
Latino, kaj tiam sidiĝis denove.

``Konsentite,'' ŝi diris per kvieta voĉo. ``Neniu aŭskultas.'' Ŝia
vizaĝo estis relative streĉa.

\emph{Ho, jes, ŝi estas atendanta ke mi minacos ŝin por informoj pri la profetaĵo.}

Erh, Harry zorgos pri tio iu alia tago.

``Temas pri la Incidento kun la Ordiganta Ĉapelo,'' Harry
diris. (Profesorino McGonagall papelbrumis.) ``Hum\ldots Mi pensas ke
estas ekstra sorĉo en la Ordiganta Ĉapelo, io kion la Ordiganta Ĉapelo
ĝi mem ne konsciis, io kio estas provoka kiam la Ordiganta Ĉapelo
diras 'Serpentimo'. Mi aŭdis mesaĝon kiun, mi relative certas,
Korvungoj ne estas supozata aŭdi.  Ĝi venis kiam la Ordiganta Ĉapelo
estis eltirita de mian kapon, kaj kiam mi sentis la konkto
sirompanta. Ĝi sonis kiel siblo kaj kiel Angla samtempe.'' Estis akra
enspiro per Profesorino McGonagall, ``kaj ĝi diris: Saluton de
Serpentimo al Serpentimo, se vi volas trovi miajn sekretojn, parolu al
mian serpenton.''

Profesorino McGonagall sidis tie kun sia buŝo malfermita, starante al
Harry kiel se li kreskigis du aliajn kapojn.

``Do\ldots'' Profesorino McGonagall diris malrapide, kvazaŭ ŝi ne
povis kredi la vortoj kiujn estis elironta de sian proprajn lipojn,
``vi decidis veni al mi direkte kaj paroli al mi pri tio.''

``Nu, jes, evidente,'' Harry diris. Ne estis neniu bezono konfesi kiom
da tempo tio al li prenis por pri tio pensi. ``Konstrate al, diru ni,
provi serĉi ĝin mi mem, aŭ diri ĝin al iu ajn alia infano.'' 

``Mi\ldots vidas,'' Profesorino McGonagall diris. ``Kaj se, eble, vi
malkovros la enirejo de la legenda Ĉambro de Sekretoj de Salazaro
Serpentimo, enirejo kiun nur vi povas malfermi\ldots''

``Mi fermos la enirejon kaj raporti al vi senprokraste por ke teamo de
magiaj arkeologoj spertaj povas esti kunvenita,'' Harry diris
prompte. ``Poste mi malfermos la enirejon denove kaj ili eniris tre
zorgeme por certigi ke ne estas iu ajn danĝera. Mi eble iros pli
malfrue por ĉirkaŭrigardi, aŭ se ili bezonas min por malfermi ion alia,
sed tio estos post kiam la ejo estas deklarita klara kaj kiam ili
fotis kiel ĉiu aspektas antaŭ homoj komencas promeni en la hitorika
loko senpreza.''

Profesorino McGonagall sidis tie kun ŝia buŝo malfermita, rigardante
lin kiel se li ĵus aliformiĝis en katon.

``Tio estas evidenta kiam vi ne estas iu Grifindoro,'' Harry diris afable.

``Mi pensas,'' Profesorino diris kun relative sufokanta voĉo, ``ke vi
\emph{tre} subestimi la malofteco de komuna saĝo, S-ro Potter.''

Tio sonis preskaŭ prava. Kvankam\ldots ``Huflopufo dirintus la saman aĵon.''

McGonagall paŭzis, emociita. ``*Tio* pravas.''

``La Ordiganta Ĉapelo proponis Huflopufo al mi.''

Ŝi papelbrumis kvazaŭ ŝi ne povis kredi siajn proprajn orelojn. ``Ĉu
li \emph{vere} faris?''

``Jes.''

``S-ro Potter,'' McGonagall diris, kaj nun ŝia voĉo estis mallaŭta,
``kvin jarcentoj pli frue estis la lasta fojo ke studento mortis inter
la muroj de Herpŭrko, kaj mi estas nun certa ke kvin jarcentoj pli
frue estis la lasta fojo ke iu aŭdis ĉi tiun mesaĝon.''

Malvermo pasis tra Harry. ``Tial mi ege certigos ke mi ne faros kian
ajn agon pri tiu afero sen konsulti vin, Profesorino McGonagall.'' Li
paŭzis. ``Kaj ĉu mi povas, sugesti ke vi kunigas la plej bonaj homoj
tial ke vi povas trovi kaj vidi ĉu tio estas ebla eltiri la ekstran
sorĉon de la Ordiganta Ĉapelo\ldots kaj se vi ne povas, eble metas
\emph{alian} sorĉon, Kvietus kiu mallonge aktivigas ĵus kiam ka Ĉapelo
estas eltirita de la kapon de la studento, tio eble povas funkcii kiel
fliko. Jen, neniu plu mortigitaj studentoj.'' Harry kapjesis pro
kontentigo.

Profesorino McGonagall aspektis ankoraŭ pli konsternita, se tia aĵo
estis imagebla. ``Mi ne povas premii vin kun sufiĉe da poentoj por tio
se tute doni la Doman Pokalon al vi.''

``Hum,'' Harry diris. ``Hum, Mi prefere ne gajnos tiom da Doma
poentoj.''

Nun Profesorino McGonagall donis al li strangan rigardon. ``Kial ne?''

Harry havis iometan problemon por meti tion vorte. ``Ĉar tio estus
malgaja, vi scias? Kiel\ldots kiel kiam mi ankoraŭ provis iri al la
lernejo en la Mugla mondo, kaj kiam ajn estis grupa projekto, mi iris
rekte kaj faris la tutan aĵon mi mem, ĉar la aliaj estintus nur
malrapidiganta min. Mi konsentas pri gajni multe da poentoj, eĉ pli ol
iu ajn, sed se mi gajnas sufiĉe de poentoj por havi decidan efikon sur
la venko de la Doma Kopalo nur per mi mem, tial tio estus kvazaŭ mi
portas la Domon Korvungo sur mia dorso kaj tio estas tro malgaja.''

``Mi vidas\ldots'' McGonagall diris heziteme. Tio estis evidenta, ke
tiu maniero de pensi neniam venis al ŝi ``Supozu ke mi nur premias vin
kun kvindek poentoj, tial?''

Harry kapneis denove. ``Tio ne estas justa al la aliaj infanoj, se mi
gajnas multe poentojn pro adultaj aĵoj al kiuj mi povas partopreni kaj
ili ne povas. Kiel Terio Boto\footnote{\emph{Terry Boot}} estas
supozata gajni kvindek poentojn por raporti murmuron kiun li aŭdis de
la Ordiganta Ĉapelo? Tio estus kompleta maljusta.''

``Mi vidas kial la Ordiganta Ĉapelo proponis Huflopufon al vi,'' diris
Profesorino McGonagall. Ŝi estis rigardanta lin kun strange aspektante.

Tio igis Harry'n sufoki iomete. Li honeste pensis ke li ne estis digna
por Huflopufo. Ke la Ordiganta Ĉapelo estis simple provanta ŝovi lin
ie ajn krom Korvungo, en Domon kies virtoj kiujn li ne havis\ldots

Profesorino McGonagall ridetis nun. ``Kaj se mi provas doni \emph{dek}
poentojn al vi\ldots?''

``Ĉu vi klarigos de kien tiuj poentoj venis, se iu demandas? Estos
verŝajne multe da Serpentimoj, kaj mi ne volas diri infanoj en
Herpŭrko, kiuj estus vere \emph{vere} koleraj se ili scius pri la
sorĉo estante tirita el la Ordiganta Ĉapelo, kaj malkovrus ke mi estas
koncerna. Do mi pensas ke absoluta sekreteco estas la plej granda
parto de braveco. Ne bezonas danki min, sinjorino, virto estas sia
propra premio.''

``Tiel tio estas,'' Profesorino McGonagall diris, ``sed mi havas vere
speciala aĵo alia por doni al vi. Mi vidas ke mi ege mapravis pri vi
mense, S-ro Potter. Bonvolu atendu tie.''

Ŝi stariĝis, iris al la ŝlosita pordo, levis sian bastonon kaj speco
de malklara kurteno venis ĉirkaŭ ŝin. Harry povis nek vidi nek aŭdi kion
estis okazanta. Kelkaj minutoj pli malfrue la malklaro malaperis kaj
Profesorino McGonagall staris tie, frontanta lin, kun la pordo
malantaŭ ŝi aspektante kvazaŭ ĝi neniam estis malfermita.

Kaj Profesorino McGonagall tenis en sia mano kolĉenon, maldika ĉeno el
oro portanta en sia centro arĝentan cirklon, en kiu estis aparato kiel
sablhorloĝo. En ŝia alia mano estis faldita broŝuro. ``Tio estas por
vi,'' ŝi diris.

Ŭaŭ! Li estis akironta iun specon de belega artikolo magia kiel serĉa
premio! Verŝajne tiu afero pri rifuzi proponon de monaj premioj ĝis vi
akiras magia artikolo fakte funkciis en vera vivo, ne nur komputila
ludo.

Harry akceptis sian novan kolĉeno, ridetante. ``Kio tiu estas?''

Profesorino McGonagall enspiris. ``S-ro Potter, tiu estas objekto kiu
estas ordinare pruntedonas nur al infanoj kiuj jam montris ilin alte
respondecema, por helpi ilin kun lernejaj planoj malfacilaj.''
McGonagall hezitis, kvazaŭ estonte aldni ion alia. ``Mi \emph{devas}
emfazi, S-ro Potter, la vera naturo de tiu objekto estas sekreta kaj
vi \emph{ne} devas paroli pri tiu al aliaj studentoj, aŭ lasi ilin
vidi vin uzanta ĝin. Se tio ne estas akceptebla al vi, tial vi povas
redoni ĝin al mi nun.''

``Mi povas gardi sekretojn,'' Harry diris. ``Do kion ĝi faras?''

``Do laŭ la scio de la aliaj infanoj, tio estas Turniĝa pordeto kaj ĝi
estas uzata por kuraci raran kaj nekontaĝan malsanon magia nomata
Spontanan Duobligo. Vi portas ĝin sub viaj vestoj, kaj kiam vi havas
neniun kialon montri ĝin al iu ajn, vi ankaŭ havas neniun kialon
trakti ĝin kiel abomeninda sekreto. Turniĝa pordeto ne estas
interesa. Ĉu vi komprenas, S-ro Potter?''

Harry kapjesis, sia rideto larĝaĝante. Li sentis la laboron de
kompetenta Serpentimo. ``Kaj kion ĝi \emph{reale} faras?''

``Tiu estas Returnilo de Tempo. Ĉiuj spino de la sablhorloĝo sendas
vin unu horo reen en la tempo. Tiel se vi uzas ĝin por reiri du horon
ĉiuj tagoj, vi devus ĉiam esti kapabla endormiĝi je la sama horo.''

La suspensio de nekredemo de Harry tute eksplodis tra la fenestro.

\emph{Vi donas al mi tempomaŝino por kuraci miajn malordojn de dormado.}

\emph{Vi donas al mi TEMPOMAŜINO por kuraci MIAJN MALORDOJN DE DORMADO.}

\emph{Vi donas al mi \textbf{TEMPOMAŜINO} por kuraci \textbf{MIAJN MALORDOJN DE DORMADO}.}

``Ehehehehhheheh\ldots'' La buŝo de Harry diris. Li estis nun tenanta
la kolĉeno for de li kvazaŭ ĝi estis horloĝbombo. Fakte, ne, ne kiel
se ĝi estis horloĝbombo, tio ne \emph{komencis} priskribi la severeco
de la situacio. Harry tenis la kolĉeno for de li kvazaŭ ĝi estis
tempomaŝino.

\emph{Diru, Profesorino McGonagall, ĉu vi scias ke ordina materio
reirigita en tempo aspektas ekzakte kiel kontraŭmaterio? Jes, tio
pravas! Ĉu vi scias ke unu kilogramo de kontraŭmaterio renkontanta unu
kilogramo de materio neiigos per eksplodo ekvivalenta de 43 milionoj
de TNT? Ĉu vi konscias ke mi mem pezas 41 kilogramoj kaj ke la
rezultinta eksplodo lasus GIGANTA KRATERO FUMANTA TIE KIE KUTIMIS ESTI
SKOTLANDO?}

``Pardonu min,'' Harry sukcesis diri, ``sed tio ŝajnas reale reale
\emph{reale REALE DANĜERA}!'' La voĉo de Harry ne tute kreskis al
kriĉo, li ne povis kriis sufiĉe laŭte por honori la situacion, do ne
utilas provi.


Profesorino McGonagall rigardis lin kun tolerema afekcio. ``Mi feliĉas
ke vi prenas tion serioze, S-ro Potter, sed Returnilo de Tempo ne
estas \emph{tiel} danĝera. Ni ne donus ilin al infanoj se ili estus.''

``Vere,'' Harry diris. ``Ahahahaha. Evidente ke vi ne donus
tempomaŝinon al infano se ili estus danĝera, kion mi \emph{estis}
pensanta? Do nur por esti klara, terni sur la maŝino \emph{ne} sendos
min al la mezepoko kie mi surveturis Gutenberg'n kun ĉevala ĉaro kaj
preventi la kelrismo? Ĉar, vi scias, mi malamas kiam tio okazas al
mi.''

La lipoj de McGonagall ektremis je la manerio kiel kiam ŝi provas ne
rideti. Ŝi oferis al Harry la pamfleton kiun ŝi tenis, sed Harry estis
zorgeme tenanta la kolĉeno kun ambaŭ manoj kaj rigardegante al la
sablhorloĝo por certigi ke ĝi ne esis turnonta. ``Ne zorgu, McGonagall
diris post momenta paŭzo, kiam iĝis klara ke Harry ne moviĝos, ``Tio
ne povas okazi, S-ro Potter. La Returnilo de Tempo ne povas esti uzata
por movi al pli ol ses horoj antaŭen. Ĝi ne povas esti uzata pli ol
ses fojoj ĉiu tago.''

``Ho, bona, tre bona, tio. Kaj se iu kolizus min la Returnilo de Tempo
ne rompos kaj ne ĉirkaŭbaros la tutan kastelon Herpŭrko en senfina
buklo de ĵaŭdoj.''

``Verdire, ili povas esti fragila\ldots'' diris McGonagall. ``Kaj mi
pensas ke mi aŭdis pri strangajn aĵojn kiuj okazis se ili rompas. Sed
nenio \emph{tiel}!''

``Eble,'' Harry diris kiam li povis denove paroli, ``vi devus aldoni
al viaj tempomaŝinoj iun specon de \emph{protekta karapaco}, anstataŭ
\emph{lasu la vitro senŝirma}, por \emph{malhelpi tion okazi.}''

McGonagall aspektis tute frapita. ``Tio estas bonega ideo, S-ro
Potter. Mi devus informi la Ministerio pri tio.''

\emph{Jen, tio estas oficiala nun, ili ratifis tion en Parlamento,
  ĉiuj en la sorĉa mondo estis komplete stulta.}

``Kvankam, mi malamas iĝi tuta \emph{FILOZOFIEMA},'' Harry senespere
provis mallaŭtigis sian voĉon al io sub kriĉo, ``ĉu iu pensis pri la
\emph{IMPLIKACIOJ} de reveni ses horoj antaŭen kaj fari ion kio ŝanĝas
tempo, tio kio relative \emph{EKSIGOS ĈIUJN LA HOMOJN AFEKTAJN} kaj
\emph{ANSTATAŬIGOS ILIN KUN MALSAMAJ VERSIOJ—}''

``Ho, vi ne povas \emph{ŝanĝi} tempon!'' Profesorino McGonagall
interrompis. ``Dio mia, S-ro Potter, ĉu vi pensas ke tio estus
permesata al studentoj se tio estus ebla? Kio se iu provas ŝanĝi siajn
rezultojn de testo?''

Harry prenis momenton por trakti tion. Liaj manoj malstreĉis, nur
iomete, de ilia blanka tenilo de la ĉeno de la sablhorloĝo. Kiel se li
ne estis tenanta tempomaŝinon, sed nur nuklean ogivon.

``Do\ldots'' Harry diris malrapide. ``Homoj nur trovis ke la
universo\ldots estas mem kohera, iel, eĉ se ĝi havas tempan vojaĝon en
ĝi. Se mi kaj mia estonta mi interagas, tial mi vidos la saman aĵon por
ambaŭ de mi, eĉ se, dum mia unua fojo, mia estonta mi jam agas
kun tuta scio de aĵoj kiuj, de mia perspektivo, ne jam okazis\ldots''
la voĉo de Harry mallaŭtiĝis pro la nesufiĉo de Anglalingvo.

``Ĝusta, mi pensas,'' diris Profesorino McGonagall. ``Kvankam,
sorĉistoj estas konsilitaj eviti esti vidita de ilia estinta si
mem. Se vi ĉeestas du kursojn je la sama tempo kaj via vojo bezonas
kruciĝi tiun de la alia vi, tial ekzemple, la unua versio de vi devus
flanke paŝi kaj fermi siajn okulojn je konata tempo—vi jam havas
brakhorloĝon, bone—por ke la estonta vi povas pasi. Ĉio estas en la
pamfleto.''

``Ahahahaa. Kaj kio okazas kiam iu \emph{ignoras} tiun konsilon?''

Profesorino McGonagall pintigis siajn lipojn. ``Mi komprenas ke tio
povas esti sufiĉe konfuziganta.''

``Kaj ĝi ne, diru ni, kreas paradokson kiu detruos la universo.''

Ŝi ridetis tolereme. ``S-ro Potter, mi pensas ke mi memorus aŭdi pri
tio, se tio jam okazis.''

``*TIO NE ESTAS TRANKVILIGANTA! ĈU VI NENIAM AŬDIS PRI LA ANTROPA
PRINCIPO? KAJ KIA STULTULO KONSTRUIS UNU EL TIUJ AĴOJ POR LA UNUA
FOJO?*''

Profesorino McGonagall reale ridis. Tiu estis agrabla, ĝoja sono kiu
ŝajnis surprize nekonvene sur tiu severa vizaĝo. ``Vi estas havanta
alian 'vi ŝanĝiĝis en katon' momenton, ĉu ne, S-ro Potter. Vi propable
ne volas aŭdi tion, sed tio estas tute ameme ĉarma.''

``Ŝanĝiĝi en katon ne eĉ estas \emph{IOMETE} komparebla al tio. Vi
scias, ĝis tiu momento mi havis tiun retenatan penson teruran ie en la
malantaŭ de mia menso ke la sola restanta respondo estis ke mia tuta
universo estis komputila simulado kiel el la libro \emph{Simulakron 3}
sed nun \emph{eĉ tio estas ekskludita} ĉar tiu ĉi eta ludilo \emph{NE
ESTAS TURINGA KOMPUTEBLA!} Turinga maŝino povas simuli iri malantaŭen
al definita momento en la pasinteco kaj kalkuli malsaman estonton de
tie, iu oraklo maŝino povas dependi de la halta konduto de maŝinoj de
malpli ordeno, sed kio vi estas diranta estas ke realaĵo iel mem
kohere kalkulas je unu fojo uzante informoj kiuj\ldots ne ankoraŭ\ldots
okazis\ldots''

Ekkonscio frapis Harry kun forto de malkonstrua martelo.

Ĉio faris senson nun. Ĉio finfine faris senson.

``*DO ESTIS KIEL LA KOMED-TEO FUNKCIAS! \emph{Evidente}! La sorĉo ne
devigas komikajn eventojn okazi, ĝi nur igas vin \emph{senti impulson
de trinki} direkte antaŭ ke komikaj aĵoj estas okazonta ĉiaokaze! Mi
estas stulta, mi devintus ekkonscii kiam mi sentis la impulson de trinki
la Komed-Teon antaŭ la dua parolado de Dumbledore, ne trinkis ĝin, kaj
tiam sufokis sur mia propra salivo antastaŭ—trinki la Komed-Teon ne
okazigas la komikaĵon, la komikaĵo okazigas vin trinki la Komed-Teon!
Mi vidis ke la du eventoj estis asociata kaj supozis ke la Komed-Teon
devis esti la kaŭzo kaj la komikaĵo devis esti la efiko ĉar mi pensis
ke la tempora ordo bridis kaŭzado kaj kaŭza grafeo devis esti necikla
SED ĈIO FARAS SENSON UNUFOJE KE VI DESEGNAS LA KAŬZAJN SAGOJN IRANTE
\emph{MALANTAŬEN EN LA TEMPO!}'' 

\emph{Dua} ekkonscio frapis Harry kun forto de malkonstrua martelo.

Ĉi tiu li sukcesis gardi ĝin kvieta, faranta nur eta strangola sono
kiel mortonta katido kiam li ekkonsciis kiu metis la noton sur lia
lito ĉi tiu mateno.

La okuloj de Profesorino McGonagall estis fajra. ``Post kiam vi
diplomita, aŭ eble eĉ antaŭ, vi vere \emph{devus} instrui iom da tiuj
Muglaj teorioj en Herpŭrko, S-ro Potter. Ili ŝajnas tute fascinaj, eĉ
se ili estas ĉiuj malpravaj.''

``Glehhahhh\ldots''

Profesorino McGonagall ofertis kelkajn ŝercojn pliajn, demandis
kelkajn promesojn pliajn al kiuj Harry kapjesis, diris ion pri ne
paroli al serpentoj kiam iu povus aŭdi lin, memorigis lin legi la
pamfleton, kaj poste iel Harry trovis lin mem staranta ekstere de ŝia
oficejo kun la pordo firme fermita malantaŭ li.

``Gaahhhrrrraa\ldots'' Harry diris.

Jes, lia menso \emph{estis} blovita.

Ne mapli pro la fakto ke, se ne estis la bubaĵo, li povus tute simple
neniam akiri la Returnilon de Tempo.

Aŭ ĉu Profesorino McGonagall donintus ĝin al li ĉiuokaze, nur pli
malfrue en la tago, kiam ajn li pasintus por demandi pri lia dorma
malordo aŭ paroli al ŝi pri la mesaĝo de la Ordiganta Ĉapelo? Kaj ĉu li,
je tiu momento, volintus fari bubaĵon al si mem, kiu kondukintus lin
akiri la Returnilon de Tempo \emph{pli frue}? Tiel ke la sola
\emph{mem-kohera} eblo estis la unu kie la bubaĵo komencis antaŭ kiam
li vekiĝis en la mateno\ldots?

Harry trovis sin konsideranta, por la unua fojo en sia vivo, ke la
respondo al sia demando povus estis laŭlitere
\emph{nekonjektebla}. Tio pro ke si propra cerbo enhavis neŭronoj kiu
nur iris antaŭen en la tempo, estis \emph{nenio} kio lia cerbo povis
fari, neniu operacio kiun li povis plenumi, kiu estus simila al tiu de
Returnilo de Tempo.

Ĝis tiu momento, Harry vivis kun la direktivo de E. T. Jaynes kiu
estis ke se vi estis senscia pri fenomeno, tio estis fakto de via
propra mensostato, kaj tio ne estis fakto pri la fenomeno ĝi mem; via
necerto estis fakto pri vi, ne fakto pri kio ajn vi estis necerta; la
malklereco ekzistis en via menso, ne en la realaĵo; malplena mapo ne
korespondis al malplena teritorio. Estis misteriaj demandoj, sed
misteria respondo estas oksimoro. Fenomeno povas esti mistera
\emph{al} iu speciala persono, sed ne povis esti fenomeno mistera per
ĝi mem. Adori sanktan misteron estis nur adori sian propran malklerecon.

Do Harry estis rigardinta supren la magion kaj rifuzinta esti
timigita. Homoj havas neniun senson de historio, ili lernas pri kemio
kaj biologio kaj astronomio kaj pensas ke tiuj aferoj ĉiam estis la
vera fundo de la scienco, ke ili \emph{neniam estis} misteraj. La
steloj iam estis misteraj. Lordo Kelvin unufoje nomis la naturon de
vivo kaj biologio—la respondo de muskoloj al la volo de Homo kaj la
kreado de arboj el semoj—misteroj \emph{senfine preter} la atingo de
scienco. (Ne nur iomete preter, atentu, sed \emph{senfine}
preter. Lordo Kelvin certe senti emocian pezon gigantan pro \emph{ne
scii ion}.) Ĉiu mistero iam solvita estis puzlo ekde la aŭroro de la
specio de Homo ĝis iu solvas ĝin. 

Nun, por la unua fojo, li frontis perspektivon de mistero, kiu minacis
esti \emph{permanenta}. Se Tempo ne funkciis per sencikla kaŭzreto
tial Harry ne komprenis kion signifis kaŭzoj kaj efikoj, kaj tial li
ne komprenis el kia aĵo la realaĵo estis farita anstataŭ; kaj estis
tute ebla ke la homa menso neniam \emph{povus} kompreni, ĉar lia cerbo
estis farita el la \emph{malmodernaj neŭronoj en lineara tempo}, kaj
tio malkovriĝis esti iu malriĉa subaro de la realaĵo.

Laŭ optimismo, la Komed-Teo, kiu unufoje ŝajnis ĉiopova kaj tute
nekredebla, malkovriĝis havanta multe pli simpla eksplikon. Kiun li
mankis \emph{ekskluzive} ĉar la vero estis komplete ekstere el sia
hipoteza spaco aŭ io ajn kion sia cerbo disvolvis por kompreni. Sed
nun li fakte komprenis ĝin, verŝajne. Kio estis iel kuraĝiga. Iel.

Harry rigardis malsupren sian brakhorloĝon. Estis preskaŭ la 11a, li
sukcecis ekdormi je la 1a lastnokte, do laŭ la natura situacio li
devus iri dormi ĉi-nokte je la 3a. Do por iri dormi je la 10a vespere,
kaj vekiĝi je la 7a matene, li devus iri malantaŭen de kvin horoj
entute. Kiu volis diri ke se li volis reiri al sia dormejo ĉirkaŭ la 6a,
antaŭ iu ajn estas vekiĝita, li prefere hasti kaj\ldots  

Eĉ \emph{retrospektive}, Harry ne komprenis kiel li akiris
\emph{duonon} de la aĵoj involvitaj en la bubaĵo. De kien la
\emph{torto} venis?

Harry komencis timi serioze tempan vojaĝon.

Aliflanke, le devis agnoski ke tio \emph{estis} neanstataŭebla
oportuneco. Bubaĵo kiun vi povis fari al vi mem nur unufoje dum
vivdaŭro, en la sekvaj ses horoj post kiam vi malkovris Returnilojn de
Tempo.

Fakte, tio estis eĉ pli enigma, kiam Harry pensis pri tio. Tempo
prezentis la finitan bubaĵon kiel \emph{plenumita fakto}, kaj jam tio
estis tute klara, lia propra manfaritaĵo. Koncepto kaj plenumo kaj
skriba stilo. Ĉiuj partoj, eĉ tiuj kiujn li ankoraŭ ne komprenis.

Nu, li estis perdanta sian tempon, kaj estis maksimume tridek horoj en
tago. Harry sciis kelkajn pri tio, kion li estis supozita fari, kaj
verŝajne malkovros le restaĵon, kiel la torto, dum li
laboros. Senutilas prokrasti. Li ne povis ekzakte plenumi ion ajn,
kaptita tie en la \emph{estonteco}.

\begin{center}\rule{3in}{0.4pt}\end{center}

Kvin horoj pli frue, Harry ŝteliris en sian dormejon kun siaj roboj
eltirigitaj ĉirkaŭ sia kapo kiel maldika speco de kaŝvesto, en la
okazo ke iu jam vekiĝus kaj vidus lin kaj Harry'n ripozanta en sia
lito je la sama tempo. Li ne volis devi klarigi al iu ajn sian
etan problemon medicinan de Spontana Duobligo.

Feliĉe ŝajnis ke ĉiuj estis ankoraŭ dormantaj.

Kaj ankaŭ ŝajnis ke esti skatolo, envolvita en ruĝa kaj verda papero
kun hela rubando ora, ripozanta sub lia lito. La perfekta stereotipa
bildo de kristanaska donaco, kvankam ne estis kristanasko.

Harry rampis enen la plej mallaŭte kiel li povis, nur por la okazo ke
iu havis sian Kvietigilo turnita malalta.

Estis koverto ligita al la skatolo, fermita per senkolora vakso sen
sigelo presita.

Harry zorge malfermis la koverton, kaj eltiris le leteron.

La letero diris :

\emph{Tiu estas la Mantelo de Nevidebleco de Ignotus Peverell,
transdonita al siaj posteuloj la Potter'j. Malkiel mapliaj manteloj
kaj sorĉoj, ĝi havas la povon de gardi vin kaŝita, ne nur
nevidebla. Via patro pruntis ĝin al mi por studi nelonge antaŭ li
mortis, kaj mi konfesas ke mi multe uzi ĝin dum jaroj.}

\emph{En la estonteco, mi devos kontentiĝi kun
Senrevigado\footnote{\emph{Disillusionment}}, mi tion timas. Estas
tempo ke la Mantelo estu returnita al vi, ĝia heredanto. Mi pensis
fari el ĝi kristanaskan donacon, sed ĝi deziris reiri al via mano antaŭ
tiam. Ĝi ŝajnas atendi ke vi bezonas ĝin. Uzu ĝin bone.}

\emph{Sendube, vi jam pensas al ĉiuj manieroj de mirinda bubaĵo, kiel
via patro faris dum sia tempo. Se ĉiuj siaj misfaroj estus konataj
hodiaŭ, ĉiuj virinoj en Grifindoro kuniĝus por profani lian tombon. Mi
ne provos malhelpi ke la historio ripetiĝas, sed estu la PLEJ singarda
por ne riveli vin. Se Dumbledore vidas ŝancon de posedi unu el la
Sanktaĵoj de la Morto, li neniam lasos eskapi sian kapton ĝis la tago
de sia morto.}

\emph{Vere ĝojan Kristanaskon al vi.}

La noto estis nesubskribita.

\begin{center}\rule{3in}{0.4pt}\end{center}

``Atendu,'' Harry diris, haltanta kiam la aliaj knaboj estis elironta
la Korvungan dormejon. ``Pardonu min, estas io alia kion mi devas fari
kun mia trunko. Mi alvenos al matenmanĝo en keklaj minutoj.''

Terio Boto grumblis al Harry. ``Vi prefere ne planu traserĉi iun ajn
de niaj aĵoj.''

Harry levis unu el siaj manoj. ``Mi ĵuras ke mi ne intencas fari ian
ajn al iu ajn el viaj aĵoj, ke mi nur intencas atingi objektojn kiun
mi mem posedas, ke mi ne havas bubaĵan aŭ alian diskuteblan intencon
al iu ajn el vi, kaj ke mi ne antaŭvidas tiujn intencojn ŝanĝi antaŭ
kiam mi havas matenmanĝon en la Granda Ĉambrego.''

Terio malridetis. ``Atendu, ĉu tio estas—''

``Ne zorgu,'' diris Penelopo Klarakvo, kiu estis tie por gvidi
ilin. ``Ne estis evitilo. Bone dirita, Potter, vi devus esti
advokato.''

Harry papelbrumis. Ha, jes, Korvunga \emph{prefektino}. ``Dankon,'' li
diris. ``Mi imagas.''

``Kiam vi provos trovi la Grandan Ĉambregon, vi perdiĝos.'' Penelopo
deklaris tion kun la tono de kruda, neargumentebla fakto. ``Tuj kiam
tio okazos, demandu al pentraĵo kiel iri al la unua etaĝo. Demandu al
alia pentraĵo, \emph{tuj} kiam vi suspektas ke vi povus perdiĝi
denove. \emph{Speciale} se ŝajnas kiel vi iras pli alta, kaj pli
alta. Se vi estas pli alta ol la tuta kastelo devus esti, \emph{haltu}
kaj atendu savteamo. Alimaniere, ni vidos vin denove kvar monatoj pli
malfrue, kaj vi estos kvin monatoj pli aĝa kaj vestanta zontukon kaj
kovrita per neĝo kaj \emph{tio estas se vi restas en la kastelo.}''

``Komprenita,'' diris Harry, glutanta forte. ``Hum, ĉu vi ne devus
diri ĉiujn aĵojn al studentoj tuj?''

Penelopo suspiris. ``Kio, \emph{ĉiu}? Tio prenus semajnojn. Vi lernos
iom post iom.''  Ŝi turniĝis por eliri, sekvita per la aliaj
studentoj. ``Se mi ne vidas vin al matenmanĝo en tridek minutoj,
Potter, mi komencigis la serĉon.''

Unufoje kiam ĉiuj estis elirintaj, Harry ligis la noton al lia lito—li
jam skribis ĝin kaj ĉiujn la aliajn notojn, laborinta en sia kaverna
etaĝo antaŭ kiam ĉiuj aliaj estas vekiĝintaj. Poste li zorge atingis
inter la kampon de la Kvietigilo kaj elmetis la Mantelon de
Nevidebleco desur la ankoraŭ dormanta formo de Harry 1.

Kaj nur por la bonfarto de la petolaĵo, Harry metis la Mantelon en la
haŭtpoŝon de Harry 1, sciante ke ĝi estus tiel jam en la sia.

\begin{center}\rule{3in}{0.4pt}\end{center}

``Mi povas vidi ke la mesaĝo devas esti transdonota al Kornelio
Fluberŭalto,'' diris la pentraĵo de viro kun aristokrata imponeco kaj,
fakte, perfekta normala nazo. ``Sed ĉu mi povas demandi de kien ĝi
origine venis?''

Harry ŝultrolevis kun artifika senhelpeco. ``Oni diris al mi, ke ĝi
estis dirita per kava voĉo kiu belis el fendo en la aero ĝi mem, fendo
kiu malfermiĝis sur fajra abismo.''

\begin{center}\rule{3in}{0.4pt}\end{center}

``Hey!'' Hermione diris kun tono de indigno de sia loko ĉe la alia
flanko de la matenmanĝa tablo. ``Tiu estas la deserto de \emph{ĉiuj}!
Vi ne povas preni unu tutan torton kaj meti ĝin en vian haŭtpoŝon!''

``Mi ne prenas unu torton, mi prenas du. Pardonu min, ĉiuj, mi devas forkuri
nun!'' Harry ignoris la kriojn de kolerego kaj eliris la Grandan
Ĉambregon. Li bezonis alveni en Herbologian kurson, iom frue.

\begin{center}\rule{3in}{0.4pt}\end{center}

Profesorino Sproso rigardis lin akre. ``Kaj kiel \emph{vi} scias tion
kion la Serpentimoj planas fari?''

``Mi ne povas nomi mian informanton,'' Harry diris. ``Fakte, mi devas
demandi al vi pretendi ke tiu konversacio neniam okazis. Nur agu
kvazaŭ vi renkontas ilin nature kiam vi estis faranta tasko, aŭ io. Mi
kuros tien tuj kiam la Herbologio estos fininta. Mi pensas ke mi povas distri
la Serpentimojn ĝis vi atingos tien. Oni ne facile timigas aŭ
turmentas min, kaj mi ne pensas ke ili aŭdacos serioze dolorigi la
Knabon-Kiu-Postvivis. Tamen\ldots mi ne demandas al vi kuri en la
koridoroj, sed mi ŝatus ke vi ne lantu laŭ la vojo.''

Profesorino Sproso rigardis lin dum longa tempo, poste sia esprimo
mildiĝis. ``Bonvolu esti zorgema kun vi mem, Harry Potter. Kaj\ldots
dankon.''

``Nur certigu ne alveni tro malfrue,'' Harry diris. ``Kaj memoru, kiam
vi atingos tien, vi ne atendos vidi min kaj tiu konversacio neniam
okazis.''

\begin{center}\rule{3in}{0.4pt}\end{center}

Estis abomena, rigardi sin mem retiranta Nevilo'n el la cirklo de
Serpentimoj. Nevilo pravis, li uzis tro da forto, multe tro da forto.

``Bonan tagon,'' Harry diris malvarme. ``Mi estas la Knabo-Kiu-Postvivis.''

Ok knaboj de unua jaro, relative de la sama alto. Unu el ili havis
cikatron sur sia frunto kaj ne agis kiel la aliaj.

\begin{center}
  \emph{Ho, Ĉu iu potenco al ni la talenton donus?\\
    Tiu de vidi sin mem kiel aliaj nin vidus! \\
    De multaj eraregoj nin ĝi liberigus, \\
    Kaj malsaĝa nocio—}\footnote{Tiuj versoj estas provo de traduko de malnova
    anglalingvo. Ili origine venis de la poemo \emph{To a Louse} de Robert
    Burns}
\end{center}  


Profesorino McGonagall pravis. La Ordiganta Ĉapelo pravis. Estis klara unufoje kiam vi vidis tion de ekstere.

Estis io malbona kun Harry Potter.











