\chapter{La fundamenta eraro de atribuado}

\lettrine{L}a butiko de Mokeo estis kurioza eta butiko (oni povus eĉ
diri ĉarma) kaŝita malantaŭ vegetaĵa budo, kiu estis malantaŭ butiko
de magiaj gantoj, kiu estis en strateto flanke de la Diagon
Aleo. Seniluziige, la komercistino ne estis malglata maljuna
sorĉistino; nur juna virino kiu ŝajnis nervoza kaj portis palan flavan
robon. Tiam, ŝi estis tenanta Superan Poŝon el MokeHaŭto QX31, kies
venda atuto estis ke ĝi havis Larĝigeblan Lipon kaj ankaŭ
Nedetekteblan Etendan Ĉarmon: vi povis meti grandajn aĵojn en ĝin,
tamen la totala volumo havis limojn.

Harry insistis por ke ili venu ĉi tien rekte, unue—li insistis tiel
forte kiel li povis, sen ke la Profesorino McGonagall fariĝu
suspektema. Harry havis ion, kion li bezonis meti en la haŭtpoŝon kiel
eble plej frue. Tiu ne estis la sako da Galionoj, kiun Profesorino
McGonagall permesis lin depreni el Gringoto. Tiu estis ĉiuj aliaj
Galionoj, kiujn Harry kaŝe ŝovis en sian poŝon post kiam li falis sur
la amason da oraj moneroj. Tio \emph{estis} vera akcidento, sed Harry
ne emis manki oportunon\ldots Tamen, tio vere estis pli kiel impulsiva
momento. Ekde tiam, Harry mallerte portis la permesitan sakon da
Galionoj apud sia poŝo de pantalono, por ke iu ajn tinto ŝajnis
proveni de la bona loko.

Tio tamen lasis la demandon pri kiel li efektive metos la aliajn
Galionojn en la haŭtpoŝon sen esti kaptita. La oraj moneroj povus
esti sian, sed ili estis tamen ŝtelitaj—ŝtelita al si mem?
Aŭto-ŝtelita?

Harry levis la rigardon de la Supera Poŝo el MokeHaŭto QX31 al la
giĉeto antaŭ li.  ``Ĉu mi povas testi ĝin iom? Por vidi ĉu ĝi
funkcias, hum, fidinde?'' Li larĝigis la okulojn por atingi esprimon
de knabeca kaj ludema senkulpeco.

Harry faligis la sakon da oro en la haŭtpoŝon per sia
\emph{maldekstra} mano; sia \emph{dekstra} mano, ĝi eliris el sia poŝo
firme tenante iom da la oraj moneroj, eniris en la haŭtpoŝon, faligis
la ŝtelitajn Galionojn, kaj (kun flustro de ``sako da oro'') retrovis
la originalan sakon. Post kio, la sako reiris en sian
\emph{maldekstran} manon, por esti faligita ene denove, kaj la
\emph{dekstra} mano de Harry reiris en sian poŝon\ldots

Profesorino McGonagall rigardis lin unufoje, sed Harry sukcesis eviti
frostiĝi aŭ tremi, kaj ŝi ne ŝajnis rimarki ion ajn. Tamen, vi
\emph{neniam} tute sciis, kun plenkreskuloj kiuj havis humuron. Prenis
tri iteraciojn por fini la laboron, kaj Harry supozis ke li sukcesis
ŝteli proksimume tridek Galionojn al si mem.

Harry levis sian manon, por viŝi iom da ŝvito desur sia frunto, kaj
elspiris. ``Mi ŝatas ĉi tiun, bonvolu.''

De dek kvin Galionojn pli malpeza (du fojoj la kosto de bastono de
sorĉisto verŝajne), kaj de unu Supera Poŝo el MokeHaŭto QX31 pli peza,
Harry kaj Profesorino McGonagall eliris tra la pordo. La pordo formis
manon, kaj svingis adiaŭon al ili kiam ili eliris, kaj premis la brakon
de Harry per ia maniero, kiu sentigis Harry'n iomete malkomforta.

Kaj poste, bedaŭrinde\ldots

``Ĉu vi \emph{vere} estas Harry Potter?'' flustris maljuna viriĉo, kun
unu grandega larmo glitante al siaj dentoj. ``Vi ne mensogus pri tio,
ĉu ne?  Tamen, mi aŭdis famojn, kiuj diris ke vi ne \emph{vere}
postvivis la Mortigan Malbenon kaj ke tio estis kial neniu plu aŭdis
pri vi denove.''

\ldots ŝajnas ke la maskovesta sorĉo de Profesorino McGonagall estis
malpli perfekte efika kontraŭ pli sperta praktikanto de magio.

Profesorino McGonagall metis sian manon sur la ŝultron de Harry kaj
tiris lin al la plej proksima strateto ĝuste kiam ŝi aŭdis ``Harry
Potter?''. La maljuna viro sekvis, sed almenaŭ ŝajnis ke neniu alia
aŭdis.

Harry pripensis al la demando. Ĉu li vere \emph{estis} Harry Potter?
``Mi nur scias ke aliaj homoj diris tion al mi,'' Harry diris. ``ne
estas kiel se mi memoras mian naskiĝon.'' Lia mano dispuŝis sian
hararon el sia frunto. ``Mi havas ĉi tiun cikatron tiom longe, kiom mi
povas memori, kaj oni diris al mi ke mia nomo estas Harry Potter ankaŭ
tiom longe kiom mi povas memori. Sed,'' Harry diris pripense, ``Se
ekzistas sufiĉe da kialoj por necesigi intrigon, estas neniu kialo por
ke ili ne nur trovu alian orfon kaj eduku lin por ke li kredu ke
\emph{li} estas Harry Potter—''

Profesorino McGonagall pasigis sian manon sur sian vizaĝon,
incitegite. ``Vi aspektas ĝuste simila al via patro, James, je la jaro
kiam li unue frekventis Herpŭrko'n. Kaj mi povas atesti unike per la
bazo de \emph{karaktero} ke vi estas rilata al la plago de
Grifindoro.''

``*Ŝi* ankaŭ povas partopreni en la komploto,'' Harry observis.

``Ne,'' la maljuna viro diris per tremanta voĉo. ``Ŝi pravas. Vi havas
la samajn okulojn ol via patrino.''

``Hmm,'' Harry malridetis. ``Mi supozas ke \emph{vi} povus partopreni en ĝi
ankaŭ—''

``Sufiĉe! S-iĉo Potter.''

La maljuna viriĉo levis manon kvazaŭ li volis tuŝi Harry'n, sed poste li
lasis ĝin fali. ``Mi estas simple ĝoja ke vi estas viva,'' li
murmuris. ``Dankon, Harry Potter. Dankon por tio, kion vi faris\ldots
mi lasas vin sola nun.''

Kaj la bruo de lia lambastono frapante la grundon, malrapide foriris
el la strateto al la ĉefa strato de Diagon Aleo.

La Profesorino ĉirkaŭrigardis, aspektante stresa kaj kruda. Harry
ankaŭ aŭtomate ĉirkaŭrigardis. Sed la strateto ŝajnis malplena da ĉio,
krom maljunaj folioj, kaj la vojeto kondukanta al la Diagon Aleo.
Povis esti viditaj nur pasantoj, kiuj marŝis rapide.

Finfine Profesorino McGonagall ŝajnis malstreĉiĝanta. ``Tio ne estis
tre lerta,'' ŝi diris per mallaŭta voĉo. ``Mi scias ke vi estas kutima
pri tio, S-ro Potter, sed homoj zorgas pri vi. Bonvolu esti afabla al
ili''

Harry rigardis malsupren al siaj ŝuoj. ``Ili ne devus,'' Li diris kun
nuanco de amareco. ``Zorgi pri mi, mi volas diri.''

``Vi savis ilin de Vi-Scias-Kiun,'' diris Profesorino
McGonagall. ``Kiel ili povus ne zorgi?''

Harry rigardis supren al la strikta esprimo de la sorĉistino, kiu
estis malsupre ŝia pinta ĉapelo, kaj suspiris. ``Mi supozas ke ne
estas io ajn ŝanco por ke se mi dirus \emph{fundamentan eraron de
atribuado}, vi havus iun ajn ideon, pri kio mi volus paroli.''

``Ne,'' diris Profesorino McGonagall kun sia preciza skota
akcento, ``sed bonvolu klarigi, S-iĉo Potter, se vi povus havi la
afablecon.''

``Nu\ldots'' Harry diris, provante eltrovi kiel priskribi tion
specialan pecon de Mugla scienco. ``Supozu ke vi alvenas al laborejo
kaj vidas vian kolegon piedbatante sian skribotablon. Vi pensus, `Kia
kolera persono, li estas'. Via kolego estus pensanta pri kiel iu
puŝegis lin kontraŭ la muro en la vojo al la laborejo kaj poste kriis
al li. Ĉiuj estus koleraj pro tio, li pensus. Kiam oni rigardas aliajn
homojn, oni vidas personecajn trajtojn, kiuj klarigas iliajn
kondutojn, sed kiam oni rigardas sin mem, oni vidas cirkonstancojn
kiuj klarigas nian konduton.  Nia propra historio havas mensan sencon
por ni, de interne, sed oni ne vidas historiojn de aliaj homoj
trenante malantaŭ ili en la aero. Oni nur vidas ilin je unu situacio,
kaj oni ne vidas tion, al kio li aspektus je malsama situacio. Do
fundamenta eraro de atribuado estas tio, kion oni klarigas per
permanentaj, daŭraj trajtoj, tamen tio estus pli bone klarigata per
cirkonstancoj kaj kunteksto.'' Estis kelkaj elegantaj eksperimentoj
kiuj konfirmis tion, sed Harry ne intencis paroli pri ili.

La brovoj de la sorĉistino leviĝis kaj malaperis malantaŭ la rando
de sia ĉapelo. ``Mi pensas ke mi komprenas\ldots'' Profesorino
McGonagall diris malrapide. ``Sed kio estas la rilato kun vi?''

Harry piedbatis la brikan muron de la strateto tiel forte ke sia piedo
doloris lin. ``Homoj pensas ke mi savis ilin de Vi-Scias-Kiun ĉar mi
havas ian aĵon de granda batalanto de la lumo.''

``Tiu kun la povo por venki la Mastron de Tenebroj\ldots'' murmuris
la sorĉistino, stranga ironio eligante sian voĉon.

``Jes,'' Harry diris, enhavante ĝenon kaj ĉagrenon, ``Kvazaŭ mi detruis
la Mastron de Tenebroj ĉar mi havas ian permanentan kaj daŭran trajton
de detruanto de Mastro de Tenebroj. Mi havis dek kvin monatojn
tiutempe! Mi ne scias tion, kio okazis sed mi supozas, ke rilatas
al, kiel diras la diro, substantivaj mediaj cirkonstancoj. Kaj certe
nenio, kio rilatas al mia personeco. Homoj ne zorgas pri \emph{mi},
ili eĉ ne atentas pri \emph{mi}, ili volas premi la manon de
\emph{iu malbona klarigo}.''  Harry paŭzis, kaj rigardis
McGonagall'n. ``Ĉu vi scias tion, kio vere okazis?''

``Mi \emph{formis} ideon\ldots'' diris la Profesorino
McGonagall. ``Post ol mi vin renkontis, kio estas ke\ldots''

``Jes?''

``Vi triumfis super la Mastro de Tenebroj per esti pli abomena ol
\emph{li}, kaj postvivis la Mortigan Malbenon per esti pli terura ol
la Morto.''

``Ha. Ha. Ha.'' Harry piedbatis la muron denove.

Profesorino McGonagall subridis. ``Iru ni al Sinjorino Malkin
sekve. Mi timas ke viaj Muglaj vestoj povu altiri la atenton.''

Ili renkontis du pliajn bonintenculojn laŭ la vojo.

La butiko de roboj de Sinjorino Malkin havis vere enuigan fronton
el ruĝa ordinara briko, kaj kun glasaj fenestroj montrante multe da
nigraj roboj ene. Ne roboj, kiuj brilis aŭ ŝanĝiĝis, aŭ ŝpinis, aŭ
radiis strangajn radiojn, kiuj ŝajnis iri rekte tra via ĉemizo kaj
tiklis vin. Nur klaraj nigraj roboj, tion vi povis vidi tra la
fenestroj. La pordo estis larĝe malfermita, kvazaŭ por anonci ke ne
ekzistis sekretoj ĉi tie kaj nenio kaŝa.

``Mi foriras nelonge dum vi estos mezurita por viaj roboj,''
diris Profesorino McGonagall. ``Ĉu vi konsentas kun tio, sinjoriĉo
Potter?''

Harry kapjesis. Li malamis butikumi por vestoj kun fajra pasio, kaj ne
povis kulpigi la pli maljunan sorĉistinon pro ke ŝi sentis sin same.

La bastono de Profesorino McGonagall eliris de sia maniko,
kaj frapetis la kapon de Harry malpeze. ``Kaj ĉar vi devos esti tute klara
al la sentoj de sinjorino Malkin, mi forigas la Malklarigadon.''

``Uh\ldots'' Harry diris. Tio iom ĝenis lin; li ankoraŭ ne kutimis la
‘Harry Potter’ aferon.

``Mi estis en Herpŭrko kun Sinjorino Malkin,'' McGonagall diris. ``Eĉ
tiam, ŝi estis unu el la plej \emph{sinregaj} homoj, kiun mi konis. Ŝi
restus senplende eĉ se Vi-Scias-Kiun li mem enirus en ŝian butikon.''  La
voĉo de McGonagall estis rememoranta, kaj vere aprobanta. ``Sinjorino
Malkin ne tedos vin, kaj lasos neniun tedi vin.''

``Kien vi \emph{iros}?'' Harry demandis. ``Nur por la okazo, vi scias,
ke io \emph{fakte} okazas.''

McGonagall donis al Harry intensan rigardon. ``Mi iros \emph{ĉi
tien},'' ŝi diris montrante konstruaĵon je la alia flanko de la strato
kiun vidigis signon de ligna bareleto. ``kaj aĉetos trinkaĵon, kiun mi
senespere bezonas. \emph{Vi} farigos vin mezurita por viaj roboj,
kaj faros \emph{nenion alian}. Mi revenos por kontroli \emph{post
nelonge}, kaj mi atendas trovi la butikon de Sinjorino Malkin staranta
kaj neniel brulanta.''

Sinjorino Malkin estis movoplena maljuna virino, kiu ne diris iun ajn
vorton pri Harry kiam ŝi vidis la cikatron sur lia frunto, kaj ŝi
ĵetis sekan rigardon al sia asistanto kiam tiu ĉi knabino ŝajnis
dironta ion. Sinjorino Malkin elpoŝigis kompleton de vivaj
kaj tordiĝantaj pecoj de tuko, kiuj ŝajnis uzitaj kiel mezurrubandoj, kaj
eklaboris ekzamenante la celon de sia arto.

Apud Harry, estis pala juna knabiĉo kun pinta vizaĝo kaj
\emph{mojosaj} blond-blankaj haroj. Li ŝajne estis je la fina etapo de
la sama procezo. Unu el la du asistantoj de Malkin estis ekzamenanta
la knabiĉon kun blond-blanka hararo kaj la kvadratitan robon, kiun li
portis; kelkfoje ŝi frapetis angulon de la robo per sia bastono, kaj
la robo streĉiĝis.

``Bonan tagon,'' diris la knabiĉo. ``Herpŭrko, ankaŭ?''

Harry povis antaŭdiri kien tiu konversacio estis ironta, kaj li
decidis en kvarono da sekundo, ke sufiĉa estis sufiĉa.

``Bonan ĉielon,'' flustris Harry, ``ne povas esti.'' Li lasis siajn
okulojn larĝiĝi. ``Via\ldots nomo, sinjoriĉo?''

``Drako Malfojo,'' diris Drako Malfojo, aspektante iom skeptika.

``Tiu \emph{estas} vi! Drako Malfojo. Mi—Mi neniam pensis ke mi estos
tiel honorita, sinjoriĉo.'' Harry deziris ke li povu farigi larmojn
eliri el siaj okuloj. Aliaj ĝenerale komencis plori ĉirkaŭ tiu momento.

``Ho,'' diris Drako, sonante iom konfuzita. Poste, liaj lipoj etendiĝis en
kontentan rideton. ``Estas bone renkonti iun, kiu scias sian pozicion.''

Unu el la asistantoj, tiu, kiu ŝajnis esti rekoninta Harry'n, faris
nesonoran sufokiĝan sonon.

Harry balbutis. ``Mi ĝojas renkonti vin, S-iĉo Malfojo. Nur netuŝeble
ĝojanta. Kaj frekventi Herpŭrko'n je la sama jaro ol vi! Tio igas mian
koron svenigi.''

Oj. Tiu lasta parto povus soni iom strange, kvazaŭ li flirtis kun
Drako aŭ io kiel tio.

``Kaj, \emph{mi} estas kontenta eklerni ke mi estos traktita kun la
respekto kiun meritas la familio de Malfojo.'' la alia knabiĉo
respondis, akompanita de ia rideto, kiel tia, kian la plej granda el
la reĝoj povus doni al la plej malalta el siaj subuloj, se la subulo
estis honesta, tamen malriĉa.

He, diable, Harry havis problemojn por trovi sian sekvan replikon. Nu,
ĉiuj volis premi la manon al Harry Potter, do— ``Kiam miaj vestoj
estos finitaj, sinjoriĉo, ĉu vi bonvolos premi la manon al mi? Nenio
alia povas farigi tiun ĉi tagon pli bona, ne, tiun monaton, fakte mian
tutan vivon.''

La knabiĉo kun blond-blanka hararo gapis al li kiel respondo. ``Kaj kion
\emph{vi} faris por la Malfojo'j, kio farigas vin meriti tian favoron?''

\emph{Ho, mi tute provos tiun replikon sur la sekva persono kiu volas
premi la manon al mi.} Harry klinis sian kapon. ``Ne, ne sinjoriĉo, mi
komprenas. Mi bedaŭras demandi. Mi devus esti honorata purigi
viajn botojn, anstataŭe''

``Efektive,'' diris la alia knabiĉo acerbe. Lia severa vizaĝo heliĝis
iom. ``Diru al mi, En kiu Domo vi pensas ke vi estos ordigita? Mi
estas devigita al la Domo Serpentimo, evidente, kiel mia patriĉo Lucius
antaŭ mi. Kaj por vi, mi supozas ke ĝi estos la Domo Huflopufo, aŭ eble
la Domo de Elfoj.''

Harry ridetis time. ``Profesorino McGonagall diras ke mi estas la
plej Korvunga persono, kiun ŝi iam vidis aŭ pri kiu ŝi aŭdis paroli eĉ
per legendo, tiom ke Koŭena$^{\textit{n:\ref{nomoj:korvungo}}}$ ŝin mem estus dirinta al mi eliri pli multe,
kio ajn, \emph{tio} volas diri, kaj do ke mi sendube finos en la Domo
Korvungo se la ĉapelo ne krias tro laŭte, por ke la ceteraj el ni ne
povu kompreni iun ajn vorton, fino de citaĵo.''

``Ŭaŭ,'' diris Drako Malfojo, sonante iom impresita. La knabiĉo faris
ian pensantan suspiron. ``Via flato estis bonega, aŭ mi pensis tiel,
ĉiaokaze—vi ankaŭ estus bone en la Domo Serpentimo. Ordinare, nur mia
patriĉo ricevas tiajn rampaĉojn. Mi \emph{esperas} ke la aliaj
Serpentimo'j lakeos por mi nun ke mi estas en Herpŭrko\ldots Mi imagas
ke tio estas bona signo, do.''

Harry tusis. ``Fakte, pardonu min, sed mi ne havas iun ajn ideon pri
kiu vi estas, vere.''

``\emph{Ne, sed serioze!}'' diris la knabiĉo kun feroca elreviĝo. ``Kial
vi faris tion, tiam?'' La okuloj de Drako larĝiĝis kun subita
suspekto. ``Kaj, kiel vi ne konas la Malfojo'jn? Kaj kiaj estas
tiaj vestoj, kiujn vi portas? Ĉu viaj gepatroj estas
\emph{Mugloj}?''

``Ambaŭ miaj gepatroj estas mortaj,'' Harry diris. Lia koro
tordiĝis, kiam li diris tion tiel— ``Miaj aliaj du gepatroj estas
Mugloj, kaj ili estas tiuj, kiuj edukis min.''

``\emph{Kio?}'' diris Drako. ``Kiu vi \emph{estas}?''

``Harry Potter, mi ĝojas ekkoni vin''

``\emph{Harry Potter?}'' spiregis Drako. ``*La* Harry—'' kaj la knabiĉo
interrompis sin abrupte.

Estis mallonga silento.

Poste, kun hela entuziasmo, ``Harry Potter? \emph{La} Harry Potter?
Diable, mi ĉiam volis renkonti vin!''

La servisto de Drako faris sonon kvazaŭ ŝi estis strangolanta sed
daŭrigis sian laboron, kaj levis la brakojn de Drako por zorge depreni
la kvadratitan robon.

``Silentu,'' Harry sugestis.

``Ĉu mi povas havi aŭtografon? Ne, atendu, Mi volas bildon kun vi
unue!''

``Fermu la \emph{buŝon}, fermu la \emph{buŝon}, fermu la \emph{buŝon}.''

``Mi simple estas \emph{rava} renkonti vin!''

``Eksplodu en flamon, kaj mortu.''

``Sed vi estas Harry Potter, la glora savanto de la sorĉa mondo! La
heroo de ĉiuj, Harry Potter! Mi ĉiam volis esti ekzakte kiel vi, kiam
mi kreskos, tiel ke mi povu—''

Drako haltis je la mezo de sia frazo, sia vizaĝo frostiĝante pro plena
teruro.

Alta viriĉo kun blanka hararo, fride eleganta en nigra robo de la plej
fajna kvalito. Unu mano kaptante manipulitan kanon el arĝento, kiu
donis al la karaktero, ian de mortiga armilo, simple per esti en lia
mano. Liaj okuloj rigardis la ĉambron kun la senpasia kvalito de
ekzekutisto, viro por kiu mortigi ne estis dolora, aŭ eĉ bonguste
malpermesita, sed nur rutina agado kiel spiradi.

Tiu estis la viro, kiu ĵus eniris tra la malfermita pordo.


``Drako,'' diris la viro, mallaŭte kaj tre kolere, ``*kion* vi estas
\emph{diranta}?''

En frakcio de sekundo de bonkora paniko, Harry ellaboris savan planon.

``Lucius Malfojo!'' spiregis Harry Potter. ``*La* Lucius Malfojo?''

Unu el la asistantoj de Malkin bezonis turni sin for kaj alfrontis la muron.

Malvarmetaj murdemaj okuloj rigardis lin. ``Harry Potter.''

``Mi estas, tiel ĝoja renkonti vin!''

La malhelaj okuloj larĝiĝis, estis ŝokita surprizo anstataŭanta mortigan
minacon.

``Via filiĉo estis diranta \emph{ĉion} pri vi.'' Harry ŝprucis,
preskaŭ ne sciante tion, kio eliros el sia buŝo sed nur parolante kiel
eble plej rapide. ``Sed evidente, mi sciis ĉion pri vi antaŭ tiam,
ĉiuj scias pri vi, la granda Lucius Malfojo! La plej honorata laŭreato
de la Domo Serpentimo, mi mem pensis provi eniri en la Domon
Serpentimo, nur pro ke mi aŭdis ke vi estis en ĝi kiam vi estis
infano—''

``\emph{Kion vi estas diranta, S-iĉo Potter?}'' venis preskaŭ krio de
ekster la butiko, kaj Profesorino McGonagall uragane eniris tuj
post sekundo.

La aspekto sur ŝia vizaĝo estis tiel pure abomena ke la buŝo de Harry
malfermiĝis aŭtomate, kaj poste blokis sin por diri nenion plu.

``Profesorino McGonagall!'', kriis Drako. ``Ĉu tiu vere estas vi? Mi
aŭdis tiom pri vi de mia patriĉo, mi pensis provi eniri en Grifindoro'n
tiel ke mi povu—''

``\emph{Kio?}'' muĝis Lucius Malfojo kaj Profesorino McGonagall en
perkekta unisono, starante unu apud la alia. Iliaj kapoj turniĝis por
rigardi sin reciproke per duoblaj mocioj, kaj tiam la du retropaŝis
por foriĝi de unu la alia kvazaŭ ili estis farantaj sinkronan
dancon.

Okazis subita eksplodo da agoj, kiam Lucius ekprenis Drako'n kaj
trenis lin ekster la butikon.

Kaj tiam estis silento.

En la maldekstra mano de Profesorino McGonagall troviĝis eta glaso,
kiu kliniĝis je unu flanko dum la hasto, kaj malrapide gutigis gutojn de
alkoholo en la etan flakon el ruĝa vino, kiu aperis sur la planko.

Profesorino McGonagall paŝis antaŭen en la butikon ĝis ŝi alfrontis
Sinjorino Malkin.

``Sinjorino Malkin,'' diris Profesorino McGonagall per trankvila
voĉo. ``Kio okazis tie?''

Sinjorino Malkin rigardis ŝin silente dum kvar sekundoj, kaj poste ŝi
krakis. Ŝi falis kontraŭ la muro, ridegante, kaj ridigis ambaŭ siajn
asistantojn, unu el tiuj falis en siajn manojn kaj genuis sur la
plankon, ridante histerie.

Profesorino McGonagall malrapide turnis sin por rigardi Harry'n, kun
malvarmeta mieno . ``Mi lasas vin sola dum ses minutoj. Ses minutoj,
S-iĉo Potter, precize.''

``Mi estis ŝercanta iomete,'' Harry protestis, dum la sono de la
histeria ridado daŭris proksime.

``*Drako Malfojo diris antaŭ sia patriĉo ke li volis esti ordigita en
Grifindoro'n!* Ŝerci iomete \emph{ne estas sufiĉa} por \emph{fari}
tion!'' Profesorino McGonagall paŭzis, videble prenante
spirojn. ``Kiun parton de 'esti mezurita por via robo' sonas por vi
kiel \emph{bonvolu ĵeti Ĉarmon de Konfuzo al la tuta universo!}''

``Li estis en situacia kunteksto, en kiu tiuj ĉi agoj havis internan
sencon—''

``Ne, ne klarigu. Mi ne volas scii tion, kio okazis tie, neniam. Ne
gravas kia malhela povo loĝas en vi, ĝi estas \emph{kontaĝa}, kaj mi ne
volas fini kiel tiu kompatinda Drako Malflojo, tiu kompatinda
Sinjorino Malkin kaj siaj du kompatindaj asistantoj.''

Harry flustris. Estis klara ke Profesorino McGonagall ne estis en
la humoro por aŭskulti raciajn klarigojn. Li rigardas Sinjorinon
Malkin kiu ankoraŭ ridegis kontraŭ la muro, kaj la du asistantojn de
Malkin kiuj nun ambaŭ falis sur la genuojn, antaŭ lia korpo kovrita de
mezurrubandoj.

``Mi ne tute finis esti mezurita por miaj vestoj,'' Harry diris
afable. ``Kial vi ne irus preni alian trinkaĵon?''


