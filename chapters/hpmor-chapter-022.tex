\chapter{La Scienca Metodo}

\lettrine{E}ta studejo, proksima sed ne en la Korvunga dormejo, iu el
la multe neuzitaj ĉambroj de Herpŭrko. Planko el grizaj ŝtonoj, muroj
el ruĝaj brikoj, plafono el malhela ligno makulita, kvar brilantaj
globoj el vitro lokitaj en la kvar muroj de la ĉambro. Cirkla tablo
kiu aspektis kiel larĝa slabo el nigra marmoro, sida sur dikaj kruroj
el nigra marmoro, sed kiu montriĝis esti vere leĝera (laŭ ambaŭ pezo
kaj maso) kaj ne estis malfacila preni ĝin por movi ĝin se necesis. Du
komfortaj remburitaj seĝoj kiuj ŝajnis unue esti fiskitaj al la planko
je malkonvenaj lokoj, sed kiuj povus, ili malkovris finfine, ekkuris
tien kie vi troviĝis ekde kiam vi kliniĝis al sinteno kiu aspektis
kiel se vi estis sidiĝonta.


Estis ankaŭ videble granda nombro da vespertoj flugantaj ĉirkaŭe en la
ĉambro.

Tio estis kie, estontaj historiistoj iam memorus -- \emph{se} la tuta
projekto finfine vere kondukus ien-- kiel la loko kie scienca studado
de la magio komencis, kun du junaj studentoj de unua jaro en Herpŭrko.

Harry James Potter-Evans-Verres, teoriisto.

Kaj Hermione Jean Granger, eksperimentisto kaj testa subjekto.

Harry estis pli sukcesa en klasoj nun, almenaŭ el la klasoj kiujn li
konsideris interesaj. Li legis kelkajn librojn, kaj ne nur libroj por
dek unu jaraĝaj infanoj. Li trejnis sin je
Transfiguro$^{\textit{a:\ref{nomoj:transfiguro}}}$ ankoraŭ kaj ankoraŭ
dum siaj ekstraj horoj ĉiutage, uzante la aliajn horojn por komenci
Oklumencion$^{\textit{a:\ref{nomoj:oklumencio}}}$. Li prenis serioze
la klasojn kiuj valoris sian tempon, ne nur per fari siajn hemtaskojn
ĉiutage, sed per uzi sian libran tempon por lerni pli ol tio kio estis
demandita, per legi aliajn librojn preter la klasaj libroj, per voli
domini la temon, ne per nur memori kelkajn respondojn de testo, por
superregi. Vi ne tro povis vidi tion el Korvungo. Kaj nun \emph{eĉ} en
Korvungo, siaj solaj konkurantoj estis Padma Patil (kies gepatroj
venis de kulturo ne anglalingva, kaj do edukis ŝin per etiko de
laboro), Antonio Orŝtano (iu el la tre mallarĝa etno kiu gajnis 25\%
de la premioj de Nobel), kaj evidente, estante tre alte super ĉiuj
aliaj kiel Titano promenante trans grupo de hundidoj, Hermione
Granger.

Por plenumi tiun partikularan eksperimenton, vi bezonis ke la testa
subjekto lernu dek ses novajn sorĉkantojn, per si mem, sen helpo aŭ
korekto. Tio volis diri ke la testa subjekto estis Hermione. Punkto.

Probable oni devus mencii nun ke la vespertoj flugantaj ĉirkaŭ la
ĉambro ne estis \emph{brilantaj}.

Harry havis problemojn por akcepti la implicon de tio.

``\emph{Albela Malbelga!}$^{\textit{a:\ref{nomoj:malbelga}}}$''
Hermione diris unu plian fojon.

Denove, je la pinto de la bastono de Hermione, vesperto abrupte
aperis. Je iu momento, malplena aero. La sekva momento, vesperto. Ĝiaj
flugiloj ŝajnis jam movantaj kiam ĝi aperis.

Kaj denove, \emph{ĝi ne brilis.}

``Ĉu mi povas halti nun?'' diris Hermione.

``Ĉu vi certas,'' Harry diris tra tio kio ŝajnis bloki sian gorĝon,
``ke eble kun iom pli da trejno, vi ne povus farigi ĝin brili?'' Li
estis malrespektanta la eksperimentan procedon kiun li skribis antaŭ
ol ili komencis, kaj tio estis peko. Kaj li estis malrespektanta ĝin ĉar
li ne ŝatis la rezultojn kiujn li akiris, kaj tio estis morala peko,
vi povus iri al la scienca infero pro tio, sed tio ne ŝajnis gravi
ĉiuokaze.

``Kion vi ŝanĝis tiun fojon?'' Hermione diris, sonante iom laca.

``La daŭro de la \emph{aa, e}, kaj \emph{al} sonoj. Tio devus esti 3 por 2 por 2, ne 3 por 1 por 1.''

``\emph{Albela Malbelga!}'' diris Hermione.

La vesperto materiiĝis sed kun nur unu flugilo kaj ĉirkaŭiris mizere
al la planko, falante je cirkloj sur la grizajn ŝtonojn.

``Nu, kiel tio devas esti vere?'' diris Hermione.

``3 por 1 por 2.''

Kaj nun la vesperto materiiĝis kaj ĝi ja flugis supren al la plafono, sana kaj brilanta per hela verdlumo.

Hermione kapjesis pro satisfakcio. ``Konsentite, kio poste?''

Estis longa paŭzo.

``\emph{Serioze?} vi \emph{serioze} bezonas diri \emph{Albela
  Malbelga} kun la daŭroj de \emph{aa, e} kaj \emph{al} sonoj havante
propociojn de daŭro de 3 por 1 por 2, aŭ la vesperto ne brilas?
\emph{Kial? Kial? por la amo de ĉio kio estas sankta, Kial?}''

``Kial ne?''

“\emph{AAAAAAAAARRRRRRGHHHH!}”

\emph{Pok, Pok, Pok.}

Harry pripensis pri la naturo de la magio dum kelka tempo, kaj tiam
desegnis seriojn da eksperimentoj bazitaj sur la premiso ke virtuale
ĉio kion magiistoj kredis pri magio estis malvera.

Estis malebla ke vi \emph{vere} bezonis diri ``\emph{Flugapika
  Leviosu}'' je la ekzakta maniero por farigi aĵon flugi, ĉar, ve,
``Flugapika Leviosu''?  La universo certigus ke vi diris ``Flugapika
Leviosu'' je la ekzakta maniero aŭ alie ĝi ne farigus la plumon flugi?

Ne. Evidente ne, kiam vi pripensis pri tio serioze. Iu, tute eble
reala antaŭlerneja knabo, sed esperantista magiisto, kiu pensis ke
``Flugapika Leviosu'' sonis tute klakanta kaj fluganta, origine diris
tiujn vortojn dum ri ĵetis la sorĉkanton la unuan fojon. Kaj poste iu
diris al ĉiuj ke ili estas necesaj.

Sed (Harry al si diris) tio ne \emph{bezonis} esti tiel, tio ne estis
konstruita en la universo, tio estis konstruita en \emph{vi}.

Estis maljuna historio transdonita inter sciencistoj, singardema
rakonto, la historio de Blondlot kaj la N-Radioj.

Baldaŭ post la malkovro de la Ikso-Radioj, eminenta franca fizikisto
nomita Prosper-René Blondlot--li estis la unua kiu mezuris la rapidon
de radiaj ondoj, kaj kiu montris ke ili propagis je la sama rapido ol
la lumo-- anoncis la malkovron de nekredebla fenomeno nova, N-radioj,
kiu povis briligi ekranon tre malforte. Vi devis rigardi atentege por
vidi ĝin, sed ĝi estis tie. N-Radioj havis ĉiajn specojn de interesaj
ecoj. Ili estis kurbigitaj per aluminio, kaj povis esti ekfokusigitaj
per aluminia prismo por celi fadeneton el kadmia sulfato, kiu tiam
brilus malforte en la malheleco\ldots

Post nelonge, dekduoj da sciencistoj konfirmis la rezultojn de
Blondlot, precipe en Francio.

Sed estis ankoraŭ kelkaj sciencistoj, en Anglio kaj Germanio, kiuj
diris ke ili ne certis ke ili povis vidi la malfortan brilon.

Blondlot diris ke ili verŝajne misreguligis la aparaton.

Iun tagon Blondlot faris demonstradon de la N-radioj. La lumoj estis
elŝaltitaj, kaj lia asistanto anoncis la heligojn kaj malheligojn dum
li faris sian manipuladon.

Tio estis normala demonstrado, kaj ĉiuj la rezultoj estis tiuj kiuj
estis atenditaj.

Eĉ se Amerikana sciencisto nomita Robert Wood kviete ŝtelis la
aluminian prismon de la centro de la aparato de Blondlot.

Kaj tio estis la fino de la N-radioj.

\emph{Realo,} Philip K. Dick unufoje diris, \emph{estas tio kio, kiam vi haltas kredi je ĝi, ne malaperas.}

La peko de Blondlot estis evidenta retrospektive. Li devintus ne diri
al sia asistanto kion li estis faranta. Blondlot devintus certigi ke
sia asistanto \emph{ne} sciis kion li estis provanta aŭ kiam li estis
provanta ĝin, antaŭ ol demandi al ri priskribi la helecon de la
ekrano. Tio povintus esti tiel simpla.

Hodiaŭ tio estas nomata ``blinda eksperimento'' kaj tio estas unu el
la aĵoj kiujn modernaj sciencistoj prenas kiel evidenta. Se vi faras
psikologia eksperimento por vidi ĉu homoj fariĝas pli kolera kiam vi
frapas ilin je la kapo per ruĝa vergo ol per verda vergo, vi ne povas
rigardi la subjektojn vi mem kaj decidi kiom ``kolera'' ili
aspektas. Vi prenus fotojn de ili, post ol ili estis frapitaj per la
vergo, kaj sendus la fotojn al aro da taksantoj, kiu taksus je gamo de
1 al 10 kiom koleraj ĉiuj personoj aspektas, evidente \emph{sen} diri
kun kiu koloro de vergo ili estis frapitaj. Efektive, ne estis iu ajn
bona kialo diri al la taksantoj la temon de la eksperimento, neniu
kialo. Vi \emph{certe} ne volus diri al la eksperimentaj subjektoj ke
\emph{vi pensis} ke ili estus pli koleraj kiam frapitaj per la ruĝa
vergo. Vi nur oferus 20 pundojn, logus ilin en testan ĉambron, frapus
ilin per vergo, kies koloro estus hazarde elektita evidente, kaj
fotus. Fakte la frapo per la vergo kaj la kapto de la foto estus
faritaj per asistanto al kiu oni ne diris la hipotezojn, tiele ri ne
povus aspekti atentema, frapi pli forte, aŭ foti je la preciza
momento.

Blondlot detruis sian reputacion per ia eraro kia akirus malsukcesan
noton kaj verŝajne ridegojn de la instruista asistanto en kurso de
unua jaro de universtitato pri eksperimenta koncepto\ldots en la jaro
1991.

Sed tio estis iom pli malfrua, en 1904, kaj do prenis monatojn antaŭ
ol Robert Wood formulis la evidentan hipotezon alternativan kaj
malkovris kiel testi ĝin, kaj dekduoj da sciencistoj estis jam
ĉerpitaj.

Pli ol du jarcentoj post la komenco de la scienco. Tiel malfrua en la
scienca historio, kaj tio ne estis ankoraŭ evidenta.

Kiu igis \emph{tute} verŝajne ke la malgrandega sorĉa mondo, kie
scienco ne ŝajnis esti konata, tute ne, neniu iam ajn testis la unuan,
la plej simplan, la plej evidentan aĵon kiun modernaj sciencistoj
pensus testi.

Libroj estis plenaj da komplikaj instrukcioj por ĉiuj la aĵoj kiujn vi
devis fari \emph{ekzakte korekte} por ĵeti sorĉon. Kaj, Harry faris la
hipotezon ke la procezo pri obei tiujn instrukciojn, pri certigi ke vi
sekvis ilin korekte, verŝajne \emph{faris} ion. Ĝi \emph{devigis vin
  koncentriĝi je la sorĉkanto}. Se oni simple diris al vi skui vian
bastonon kaj deziri, tio probable \emph{ne funkcius} ankaŭe. Kaj de kiam vi
kredis ke la sorĉkanto estis supozita funkcii laŭ difinita maniero, de
kiam vi praktikis ĝin laŭ tiu maniero, vi eble ne estis kapabla
konvinki vin mem ke ĝi fakte povus funkcii laŭ aliaj manieroj\ldots

\ldots{}se vi faris la simplan sed malveran aĵon, kaj provis testi
alternativan formon \emph{per vi mem.}

Sed kio okazus se vi \emph{ne sciis} kiel la origina sorĉkanto estis?

Kio se vi donis al Hermione liston da sorĉoj, kiujn ŝi ne ankoraŭ
studis, prenitaj el libro de petolaj sorĉkantoj stultaj de la
biblioteko de Herpŭrko, kaj kelkaj el tiuj sorĉkantoj havis la
korektajn kaj originajn instrukciojn kiam la ceteraj havis unu ŝanĝitan
geston, unu ŝanĝitan vorton? Kio se vi gardis la instrukciojn senŝanĝe,
sed diris al ŝi ke la sorĉkanto kiu estis supozita krei ruĝan vermon,
kreis bluan vermon anstataŭe?

Nu, en tiu okazo, aperis ke\ldots{}

\ldots{}Harry havis problemon por kredi la rezultojn tie\ldots{}

\ldots{}se vi dirus al Hermione ke ŝi diru ``Albela Malbelga'' kun
proporcio de la vokala daŭro de 3 por 1 por 1, anstataŭ la vera
proporcio de 3 por 1 por 2, vi ankoraŭ akirus vesperton sed ĝi ne plu
brilus.

Ne ke kredo estis \emph{malkoncerna} tie. Ne ke \emph{nur} vortoj kaj bastonaj gestoj gravis.

Se vi donus al Hermione komplete malbonajn informojn pri tio kion la
sorĉkanto estis supozita fari, ĝi simple ne funkcius.

Se vi ne diris al ŝi tion, kion la sorĉkanto estis supozita fari, ĝi ne funkcius.

Se ŝi scius laŭ svagaj termoj kion la sorĉkanto estis supozita fari,
aŭ se ŝi estus nur parte malprava, tiam la sorĉkanto funkcius kiel
origine skribita en la libro, ne laŭ la maniero laŭ kiu oni diris al
ŝi.

Harry estis, je tiu ekzakta momento, laŭlitere frapanta sian kapon
kontraŭ la brika muro. Ne forte. Li ne volis damaĝi sian preciozan
cerbon. Sed se li ne havis elirejon por sia frustro, li spontanee
fajrus.

\emph{Pok. Pok. Pok.}

Ŝajnis ke la universo fakte \emph{volis} ke vi diru ``Flugapika
Leviosu'' kaj ĝi volis ke vi diru ĝin laŭ difinita maniero ekzakta kaj
ĝi ne pli zorgis pri kiel \emph{vi} pensis ke la prononco devus esti,
ol li zorgis pri kiel vi sentis vin pri gravito.

\emph{\shout{Kiaaaaaaaaaaaaaaaaaaal?}}

La plej malbona parto de tio estis la memkontenta, amuzita aspekto sur
la vizaĝo de Hermione.

Hermione ne \emph{akceptis} resti side kaj obeeme sekvi la instrukciojn
de Harry sen ke oni diris al ŝi kial.

Do Harry klarigis al ŝi kion ili estis testonta.

Harry klarigis kial ili estis testonta tion.

Harry klarigis kial probable neniu sorĉisto neniam provis tion antaŭ ili.

Harry klarigis kial li estis relative memkonfida pri siaj progonozoj.

Ĉar, Harry diris, estis neniu \emph{ŝanco} ke la universo vere volis ke vi diru ``Flugapika Leviosu''.

Hermione rimarkigis ke tio ne estis kion siaj libroj diris. Hermione
demandis al Harry ĉu li vere pensis ke li estis pli inteligenta, je la
aĝo de dek unu jaroj kaj nur post unu monato de edukado en Herpŭrko,
ol ĉiuj la aliaj sorĉistoj en la mondo kiuj ne konsentis kun li.

Harry diris la sekvan vorton precize:

``Evidente.''

Nun Harry estis rigardanta la ruĝajn brikojn direkte fronte al li kaj
kalkulanta kiom forte li devus frapi sian kapon por doni al si mem
cerboskuon, kiu interferus kun la formado de longatempa memoro kaj
malhelpus lin memori tion pli malfrue. Hermione ne ridis, sed li povis
senti ŝian \emph{volon ridi} radiante de malantaŭ li kiel terura premo
sur sia haŭto, kvazaŭ scii ke vi estis sekvata de seria murdisto, sed
\emph{pli malbone}.

``Diru ĝin,'' Harry diris.

``Mi ne estis tion \emph{faronta},'' diris la agrabla voĉo de Hermione
Granger. ``Tio ne ŝajnis afabla.''

``Ni finu do,'' diris Harry.

``Konsentite! Do vi donis al mi tiun \emph{tutan lecionon longan} pri
kiel malfacila estas fari bazan sciencon kaj kiel oni povis bezoni
resti sur la sama problemo dum \emph{tridek kvin jaroj}, kaj tiam vi
komencis kaj atendis ke ni faru la plej grandan malkovron de la
historio de la magio dum la unua horo kiam ni laboris kune. Vi ne nur
esperis, vi vere atendis tion. Vi estas frivola.''

``Dankon. Nun--''














