\chapter{La Scienca Metodo}

\lettrine{E}ta studejo, proksima sed ne en la Korvunga dormejo, iu el
la multe neuzitaj ĉambroj de Herpŭrko. Planko el grizaj ŝtonoj, muroj
el ruĝaj brikoj, plafono el malhela ligno makulita, kvar brilantaj
globoj el vitro lokitaj en la kvar muroj de la ĉambro. Cirkla tablo
kiu aspektis kiel larĝa slabo el nigra marmoro, sida sur dikaj kruroj
el nigra marmoro, sed kiu montriĝis esti vere leĝera (laŭ ambaŭ pezo
kaj maso) kaj ne estis malfacila preni ĝin por movi ĝin se necesis. Du
komfortaj remburitaj seĝoj kiuj ŝajnis unue esti fiskitaj al la planko
je malkonvenaj lokoj, sed kiuj povus, ili malkovris finfine, ekkuris
tien kie vi troviĝis ekde kiam vi kliniĝis al sinteno kiu aspektis
kiel se vi estis sidiĝonta.


Estis ankaŭ videble granda nombro de vespertoj flugantaj ĉirkaŭe en la
ĉambro.

Tio estis kie, estontaj historiistoj iam memorus -- \emph{se} la tuta
projekto finfine vere kondukus ien-- kiel la loko kie scienca studado
de la magio komencis, kun du junaj studentoj de unua jaro en Herpŭrko.

Harry James Potter-Evans-Verres, teoriisto.

Kaj Hermione Jean Granger, eksperimentisto kaj testa subjekto.

Harry estis pli sukcesa en klasoj nun, almenaŭ el la klasoj kiujn li
konsideris interesaj. Li legis kelkajn librojn, kaj ne nur libroj por
dek unu jaraĝaj infanoj. Li trejnis sin je
Transfiguro$^{\textit{a:\ref{nomoj:transfiguro}}}$ ankoraŭ kaj ankoraŭ
dum siaj ekstraj horoj ĉiutage, uzante la aliajn horojn por komenci
Oklumencion$^{\textit{a:\ref{nomoj:oklumencio}}}$. Li prenis serioze
la klasojn kiuj valoris sian tempon, ne nur per fari siajn hemtaskojn
ĉiutage, sed per uzi sian libran tempon por lerni pli ol tio kio estis
demandita, per legi aliajn librojn preter la klasaj libroj, per voli
domini la temon, ne per nur memori kelkajn respondojn de testo,
por superregi. Vi ne tro povis vidi tion el Korvungo. Kaj nun
\emph{eĉ} en Korvungo, siaj solaj konkurantoj estis Padma Patil (kies
gepatroj venis de ne angle parolanta kulturo, kaj do edukis ŝin per
etiko de laboro), Antonio Orŝtano (iu el la tre mallarĝa etno kiu
gajnis 25\% de la premioj de Nobel), kaj evidente, estante tre alte
super ĉiuj aliaj kiel Titano promenante trans grupo de hundidoj,
Hermione Granger.

Por plenumi tiun partikularan eksperimenton, vi bezonis ke la testa
subjekto lernu dek ses novajn sorĉkantojn, per si mem, sen helpo aŭ
korekto. Tio volis diri ke la testa subjekto estis Hermione. Punkto.

Probable oni devus mencii nun ke la vespertoj flugantaj ĉirkaŭ la
ĉambro ne estis \emph{brilantaj}.

Harry havis problemojn por akcepti la implicon de tio.

``\emph{Albela Malbelga!}$^{\textit{a:\ref{nomoj:malbelga}}}$''
Hermione diris unu plian fojon.

Denove, je la pinto de la bastono de Hermione, vesperto abrupte
aperis. Je iu momento, malplena aero. La sekva momento, vesperto. Ĝiaj
flugiloj ŝajnis jam movantaj kiam ĝi aperis.

Kaj denove, \emph{ĝi ne brilis.}

``Ĉu mi povas halti nun?'' diris Hermione.

``Ĉu vi certas,'' Harry diris tra tio kio ŝajnis bloki sian gorĝon,
``ke eble kun iom pli da trejno, vi ne povus farigi ĝin brili?'' Li
estis malrespektanta la eksperimentan procedon kiun li skribis antaŭ
ol ili komencis, kaj tio estis peko. Kaj li estis malrespektanta ĝin ĉar
li ne ŝatis la rezultojn kiujn li akiris, kaj tio estis morala peko,
vi povus iri al la scienca infero pro tio, sed tio ne ŝajnis gravi
ĉiuokaze.

``Kion vi ŝanĝis tiun fojon?'' Hermione diris, sonante iom laca.

``La daŭro de la \emph{aa, e}, and \emph{al} sonoj. Tio devus esti 3 por 2 por 2, ne 3 por 1 por 1.''

``\emph{Albela Malbelga!}'' diris Hermione.

La vesperto materiiĝis sed kun nur unu flugilo kaj ĉirkaŭiris mizere
al la planko, falante je cirkloj sur la grizajn ŝtonojn.

``Nu, kiel tio devas esti vere?'' diris Hermione.

``3 por 1 por 2.''

Kaj nun la vesperto materiiĝis kaj ĝi ja flugis supren al la plafono, sana kaj brilanta per hela verdlumo.

Hermione kapjesis pro satisfakcio. ``Konsentite, kio poste?''

Estis longa paŭzo.

``\emph{Serioze?} vi \emph{serioze} bezonas diri \emph{Albela
  Malbelga} kun la daŭroj de \emph{aa, e} kaj \emph{al} sonoj havante
propociojn de daŭro de 3 por 1 por 2, aŭ la vesperto ne brilas?
\emph{Kial? Kial? por la amo de ĉio kio estas sankta, Kial?}''

``Kial ne?''

“\emph{AAAAAAAAARRRRRRGHHHH!}”

\emph{Pok, Pok, Pok.}

Harry pripensis pri la naturo de la magio dum kelka tempo, kaj tiam
desegnis seriojn da eksperimentoj bazitaj sur la premiso ke virtuale
ĉio kion magiistoj kredis pri magio estis malvera.

Estis malebla ke vi \emph{vere} bezonis diri ``\emph{Flugapika
  Leviosu}'' je la ekzakta maniero por farigi aĵon flugi, ĉar, ve,
``Flugapika Leviosu''?  La universo certigus ke vi diris ``Flugapika
Leviosu'' je la ekzakta maniero aŭ alie ĝi ne farigus la plumon flugi?

Ne. Evidente ne, kiam vi pripensis pri tio serioze. Iu, tute eble
reala antaŭlerneja knabo, sed esperantista magiisto, kiu pensis ke
``Flugapika Leviosu'' sonis tute klakanta kaj fluganta, origine diris
tiujn vortojn dum ri ĵetis la sorĉkanton la unuan fojon. Kaj poste iu
diris al ĉiuj ke ili estis necesaj.












