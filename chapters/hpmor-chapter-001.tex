\chapter{Tago de tre malalta probableco}

\lettrine{Ĉ}iuj coloj de la muroj estis kovritaj per bibliotekoj. Ĉiuj
bibliotekoj havis ses bretojn, kiuj preskaŭ atingis ĝis la
plafono. Kelkaj bretoj estis plenplenaj da libroj kun rigidaj kovriloj
: scienco, matematiko, historio kaj ĉio cetera. Aliaj bretoj enhavis
du tavolojn da sciencfikciaj poŝlibroj, kie la malantaŭa tavolo
apogis sur poŝtukaj ujoj aŭ tabuloj, tiel ke oni povis vidi la malantaŭan
tavolon da libroj super la antaŭaj libroj. Kaj tamen tio ne
sufiĉis. Libroj superfluis sur la tabloj kaj la sofoj, kaj formis
malgrandajn amasojn sub la fenestroj.

Tiu estis la salono okupita per la eminenta Profesoriĉo Mikael
Verres-Evans, kaj sia edzino, S-ino Petunia Evans-Verres, kaj
ilia adopta filiĉo, Harry James Potter-Evans-Verres. 

Letero kuŝis sur la tablo de la salono, kune kun ne stampita koverto
el flava pergameno, adresita al \emph{S-iĉo. H. Potter} per smeralda
inko.

La Profesoriĉo kaj sia edzino parolis seke unu al la alia, sed ili ne
kriis. La Profesoriĉo konsideris ke krii ne estis civilizata.

``Vi ŝercas,'' Mikael diris al Petunia. Lia tono indikis ke li tre
timis ke ŝi estis serioza.

``Mia fratino estis sorĉistino'', Petunia rediris. Ŝi aspektis timigita,
sed tenis sian pozicion. ``Ŝia edziĉo estis sorĉisto''.

``Tio estas absurda!'' Mikael diris seke. ``Ili ĉeestis en nia geedziĝo — Ili
vizitis nin dum Kristnasko —''

``Mi diris al ili, ke vi ne devis scii,'' Petunia flustris. ``Sed, tio
estas vera. Mi vidis aĵojn—''

La profesoriĉo rulis la okulojn. ``Kara, mi komprenas ke vi ne bone
konas skeptikan literaturon. Vi eble ne rimarkas, kiom facile estas por
trejnita magiisto falsi ion, kio ŝajnas neebla. Memoru kiel mi
instruis al Harry fleksi kuleron? Se ŝajnas ke ili povis ĉiam diveni
tion, pri kio vi pensis, tiam tio nomiĝas malvarma legado—''

``Tio ne temis pri fleksitaj kuleroj''

``Nu, kio estis?''

Petunia mordis siajn lipojn. ``Mi simple ne povas tion diri al vi. Vi
pensus, ke mi —''. Ŝi glutis. ``Aŭskultu, Mikael, mi ne estis — ĉiam
kiel tio —'' Ŝi gestis al si mem, kiel por indiki sian sveltan
silueton. ``Lily faris tion, ĉar mi—, ĉar mi petegis ŝin. Dum jaroj,
mi petegis ŝin. Lily ĉiam estis pli bela ol mi, kaj mi estis\ldots{}
malafabla al ŝi, pro tio, kaj tiam ŝi ekhavis magion, ĉu vi povas
imagi kiel mi sentis min? Kaj mi petegis ŝin uzi iom da tiu magio al
mi, tiel ke mi ankaŭ povu esti bela, kvankam mi ne povis havi ŝian
magion, almenaŭ mi povus esti bela.''

Larmoj kolektiĝis en la okuloj de Petunia.

``Kaj Lily diris al mi ne, kaj faris la plej ridindajn ekskuzojn, kiel
ke la mondo finiĝus se ŝi estus agrabla al sia fratino, aŭ ke centaŭro
diris al ŝi, ke ŝi ne devu fari tion —. La plej ridindaj aĵoj, kaj mi
malamis ŝin pro tio. Kaj, kiam, mi ĵus diplomiĝis el universitato, mi
eliris kun tiu knabiĉo, Vernon Dursle, li estis grasa kaj li estis la
sola knabiĉo kiu bone volis paroli al mi. Kaj, li diris ke li volis
infanojn, kaj ke sia unua filiĉo nomiĝos Dudle. Kaj, mi diris al mi,
\emph{kiaj gepatroj nomas sian infanon Dudle Dursle?} Tio estis kvazaŭ
mi vidis mian tutan estontan vivon etendiĝanta antaŭ mi, kaj mi ne
povis elteni ĝin. Mi skribis al mia fratino, kaj diris al ŝi ke se ŝi
ne helpus min, mi preferus simple—''

Petunia haltis.

``Ĉiaokaze'', Petunia diris kun malforta voĉo. ``Ŝi cedis. Ŝi diris al
mi ke tio estis danĝera, kaj mi diris ke mi ne plu zorgis, kaj mi
trinkis tiun pocion, kaj mi iĝis malsana dum semajnoj, sed kiam mi
resaniĝis, mia haŭto estis puriĝinta kaj mi finfine estis akirinta
pezon\ldots{} kaj mi estis bela, homoj estis afablaj al mi.'' Ŝia voĉo
rompis, ``kaj post tiam, mi ne plu povis malami mian fratinon,
speciale kiam mi lernis tion, kion ŝia magio finfine alportis al
ŝi—''.

``Kara'', Mikael diris milde, ``Vi malsaniĝis, vi akiris iom da pezo
ripoze en lito, kaj via haŭto puriĝis sin mem. Aŭ esti
malsana igis vin ŝanĝi vian dieton—.''

``Ŝi estis sorĉistino,'' Petunia rediris. ``Mi vidis tion.''

``Petunia,'' Mikael diris. Agaco penetris lian voĉon. ``\emph{Vi scias} ke tio
ne povas esti vera. Ĉu mi vere devas ekspliki kial ?''

Petunia tordis siajn manojn. Ŝi ŝajnis ploronta. ``Mia amo, Mi
scias ke mi ne povas gajni argumenton kun vi, sed bonvolu, vi devas
fidi min pri tio—.''

``\emph{Paĉjo, Panjo!}''

Ambaŭ haltis kaj rigardis Harry'n kvazaŭ ili forgesis ke estis tria
persono en la ĉambro.

Harry profunde elspiris. ``Panjo, \emph{viaj} gepatroj ne havis
magion, ĉu ne?''

``Ne'', Petunia diris aspektante konfuzite.

``Do, neniu en via familio sciis pri magio, kiam Lily ricevis sian
leteron. Kiel ili konvinkiĝis?''

``Ah\ldots{}'' Petunia diris. ``Ili ne nur sendis leteron. Ili sendis
instruiston de Herpŭrko$^{\textit{n:\ref{nomoj:herpŭrko}}}$. Ri—'' La okuloj de Petunia
turniĝis al Mikael. ``Ri montris al ni iom da magio.''

``Do, vi ne bezonas batali pri tio''. Harry diris firme. Esperante ke
ĉi-foje, nur ĉi-foje, ili aŭskultos lin. ``Se tio estas vera, ni povas
simple venigi instruiston de Herpŭrko ĉi tien, kaj vidi la magion per
ni mem, kaj Paĉjo agnoskos ke tio estas vera. Kaj se ne, tial Panjo
agnoskos ke tio estas malvera. Tio estas kial la eksperimenta metodo
estas farita, tiel ke oni ne devas solvi ĉiujn aferojn nur
argumente.''

La profesoriĉo turnis sin kaj rigardis lin malsupren, arogante kiel
kutime.  ``Ho, nu, Harry. Vere, \emph{magio?} Mi pensis ke almenaŭ vi scius
ne preni tion serioze, filiĉo, eĉ se vi havas nur dek jarojn. Ne estas
malpli scienca afero ol magio!''

La buŝo de Harry tordiĝis amare. Li estis traktita bone, probable pli
bone ol la plimulto da genetikaj patriĉoj traktis iliajn proprajn
infanojn. Harry estis sendita al la plej bona bazlernejo— kaj kiam
tio ne plu funkciis, li ricevis instruistojn el la senfina grupo da
malsataj studentoj. Harry estis kuraĝigita studi ĉion, kio kaptis sian
atenton, aĉeti ĉiujn librojn, kiuj kaptis sian fantazion, kaj estis
sponsorita en ĉiuj matematikaj aŭ sciencaj konkursoj, kiujn li
eniris. Al li oni donis ĉion ajn racian, kion li volis, krom eble
la plej etan peceton da respekto. Oni ne povis tro multe atendi ke
doktoriĉo, kiu instruis biokemion en Oksfordo, aŭskultu opiniojn de
eta knabiĉo. Vi aŭskultus montrante intereson evidente; estas tio,
kion bonaj gepatroj farus, kaj tial, se vi estis konvinkita ke vi
estis bona gepatro, vi farus tion. Sed konsideri \emph{serioze}
dek-jaraĝan knabon? Malfacile.

Kelkfoje Harry volis krii al sia patriĉo.

``Panjo'', Harry diris. ``Se vi volas gajni tiun ĉi argumenton kontraŭ
Paĉjo, rigardu en la dua ĉapitro de la unua libro de lekcioj de
Feynman pri fiziko. Estas citaĵo tie, pri kiel filozofoj diris multe
pri la aĵoj, kiujn scienco absolute postulas, kaj ĝi estas tute
malprava, ĉar la sola regulo en scienco estas ke la fina arbitracio
estas observado—Vi nur devas rigardi la mondon kaj raporti tion, kion
vi vidis. Hm\ldots{}, mi ne sukcesas memori kie trovi ion pri kial
estas idealo de scienco solvi aferojn per eksperimentoj anstataŭ
argumentoj—''

Lia patrino rigardis lin malsupren kaj ridetis. ``Dankon,
Harry. Sed—'', ŝia kapo leviĝis por rigardi sian edziĉon. ``Mi ne volas
gajni argumenton kontraŭ via patriĉo. Mi nur volas ke li aŭskultu sian
edzinon, kiu amas lin, kaj ke li fidas ŝin nur unu fojon—''

Harry fermis siajn okulojn nelonge. \emph{Senesperaj}. Ambaŭ siaj
gepatroj estis nur senesperaj.

Nun liaj gepatroj denove ekhavis iun el tiuj argumentadoj, iu en kiu
lia patrino provis sentigi lian patriĉon kulpa, kaj en kiu lia patriĉo
provis sentigi lian patrinon stulta.

``Mi iras en mian dormĉambron''. Harry anoncis. Lia voĉo iom
tremis. ``Bonvolu provi ne tro batali pri tio, Panjo, Paĉjo, ni scios
sufiĉe frue ĉu tio estas prava, ĉu ne?''.

``Kompreneble, Harry'', diris lia patriĉo, kaj lia patrino donis al li
trankviligan kison, kaj ili denove ekbatalis tuj post kiam Harry estis
grimpinta la ŝtuparon al sia dormĉambro.

Li fermis la pordon malantaŭ li kaj provis pripensi.

La ridinda afero estis, ke li devintus konsenti kun sia patriĉo. Neniu
vidis iun ajn ateston de magio, kaj laŭ Panjo, estis tuta magia
mondo ekstere. Kiel iu povus kaŝi ion tian? Pli da magio? Tio ŝajnis
esti ekskuzo iom suspektinda.

Tio devintus esti simpla problemo jam solvita : aŭ ŝerco de Panjo, aŭ
mensogo aŭ frenezo, per kreskanta ordo de timigo. Se Panjo mem sendis
la leteron, tial tio klarigis kiel ĝi alvenis en la leterkeston sen
poŝtmarko. Iom da frenezo estis multe, multe malpli neprobabla ol ke
la universo vere funkciis tiele.

Malgraŭ tio, iu parto de Harry estis tute konvinkita ke la magio estis
reala, kaj tio ekde kiam li vidis la leteron supozeble de Herpŭrko la
lernejo de sorĉado.

Harry frotis sian frunton, grimacante. \emph{Ne kredu ĉion, kion vi pensas}
diris unu el liaj libroj.

Sed, tiu stranga certeco \ldots{} Harry trovis sin atendanta ke, jes,
instruisto de Herpŭrko prezentos sin al ni kaj skuos bastonon kaj
magio aperos. La stranga certeco ne provis gardi sin kontraŭ falsoj—ĝi
ne havis ekskuzon anticipe, por pravigi la fakton, ke ne estus
instruisto, aŭ ke la instruisto nur povus tordi kulerojn.

\emph{De kie vi venas, stranga eta antaŭdiro?} Harry direktis la penson al sia
cerbo. \emph{Kial mi kredas tion, kion mi kredas?}

Kutime, Harry estis sufiĉe bona por respondi tiun ĉi demandon, sed en
tiu speciala okazo, li havis neniom da indico pri tio, kion lia cerbo
estis pensanta.

Harry levis la ŝultrojn mense. Tiel, kiel plato el metalo sur pordo
estis farita por esti puŝita, kaj kiel tenilo estis farita por esti
tirita, testo estis farita por esti testita.

Li prenis folion el papero el sia skribotablo, kaj komencis skribi. \\

\emph{Kara Vicdirektorino} \\

Harry paŭzis pripensante; kaj poste li forĵetis la folion por preni
alian, kaj li elirigis alian milimetron da grafito el sia meĥanika
krajono. Ĉi tiu meritis grafikan penon pli detalan. \\

\emph{Kara Vicdirektorino, Minerva McGonagall},

\emph{Aŭ al kiu ajn, kiun tio koncernas} : 

\emph{Mi antaŭ nelonge ricevis vian leteron de akcepto al Herpŭrko,
adresita al S-iĉo H. Potter. Verŝajne, vi ne scias ke miaj genetikaj
gepatroj, James Potter kaj Lily Potter (antaŭe Lily Evans) estas
mortaj.  Mi estis adopta de la fratino de Lily, Petunia Evans-Verre,
kaj ŝia edziĉo, Mikael Verres-Evans.}

\emph{Mi estas ege interesita por ĉeesti Herpŭrko'n, se tia loko
efektive ekzistas. Nur mia patrino Petunia diris ke ŝi scias pri
magio, kaj ŝi mem ne povas uzi ĝin. Mia patriĉo estas tre skeptika. Mi
mem estas necerta. Krome, mi ne scias kie akiri la librojn kaj la
ekipaĵojn, kiuj estas listigitaj en via letero de akcepto.}

\emph{Patrino menciis ke vi sendis reprezentaton de Herpŭrko al Lily
Potter (Lily Evans tiutempe) por montri al ŝia familio ke la magio
ekzistas, kaj mi supozas, helpi Lily'n por akiri ŝian lernejan
ekipaĵon. Se vi povus fari tion por mia propra familio, tio estus ege
helpa.}

\emph{Sincere},

\emph{Harry James Potter-Evans-Verres}.\\


Harry aldonis ilian aktualan adreson, kaj poste faldis la leteron, kaj
enmetis ĝin en koverton, kiun li adresis al Herpŭrko. Plia konsidero
farigis lin akiri kandelon, kaj gutigi vakson sur la klapo de la
koverto, kaj uzante plumo-pinton li gravuris la inicialojn
H.J.P.E.V. Se li irus en la frenezon, li irus glore.

Poste, li malfermis sian pordon, kaj reiris malsupren. Lia patriĉo
sidis en la salono, kaj estis leganta libron pri altaj matematikoj,
por montri kiom inteligenta li estis; Kaj sia patrino estis en la
kuirejo preparante iun el la favorataj manĝaĵoj de lia patriĉo, por
montri kiom amanta ŝi estis. Tiel timiga kiel argumentadoj povis esti,
ne argumentadi povis esti iel pli malbona.

``Panjo'', Harry diris en la nervoza silento, ``Mi estas testonta la
hipotezon. Laŭ via teorio kiel mi sendas strigon al Herpŭrko ?''

Lia patrino turniĝis de la kuireja lavujo por rigardi lin, aspektante
relative ŝoka. ``Mi—Mi ne scias, mi pensas ke vi nur devas uzi vian
propran strigon.''

Tio devintus soni tre suspektinde, \emph{ho, do, ne estas iu ajn maniero por
  testi vian teorion}, sed la stranga certeco en Harry ŝajnis voli
kroĉiĝi ankoraŭ pli.

``Nu, la letero venis ĉi tien iel'', Harry diris, ``do, mi nur skuos
ĝin ekstere kaj vokos 'letero por Herpŭrko!', kaj mi bone vidos ĉu
strigo prenas ĝin. Paĉjo ĉu vi volas veni kaj rigardi ?''.

Lia patriĉo kapneis detale kaj daŭrigis legi. \emph{Evidente}, Harry
diris al si. Magio estis hontinda aĵo, pri kio nur stultulo kredas; Se
lia patriĉo iris ĝis testi la hipotezon, aŭ simple rigardi ĝin dum ĝi
estis testita, tio estus kiel asocii lin kun ĝi\ldots

Kiam Harry pasis tra la malantaŭa pordo, al la malantaŭa ĝardeno, li
subite ekkonsciis ke se strigo venus por preni la leteron, li havus
problemojn por ekspliki tion al sia patriĉo.

\emph{Sed—Nu—tio ne povas vere okazi, ĉu ne? Ne gravas kion mia cerbo
  ŝajnas kredi. Se strigo malsupreniras por preni ĉi tiun koverton, mi
  havos zorgojn multe pli gravajn ol la opinio de paĉjo.}

Harry profunde enspiris, kaj levis la koverton en la aero.

Li glutis.

Ekkrii 'Letero por Herpŭrko!', tenante koverton alte en la aero en la
mezo de via propra malantaŭa ĝardeno estis, fakte sufiĉe embarasa,
nun kiam li pensis pri tio.

\emph{Ne. Mi estas pli bona ol paĉjo. Mi uzos la sciencan metodon, eĉ
  se tio sentigas min stulta}.

``Letero—'' Harry diris, sed tio fakte eliris kiel flustra kvako.

Harry plifortigis sian volon, kaj kriegis al la malplena ĉielo,
``\emph{Letero por Herpŭrko! Cû mi povas havi strigon?}''

``Harry?'', demandis perpleksa ina voĉo, iu el la najbaroj.

Harry ekmallevis sian manon kvazaŭ ĝi estis en fajro kaj kaŝis la
kovreton malantaŭ sian dorson, kvazaŭ ĝi estis droga mono. Lia tuta
vizaĝo estis varma pro honto.

Maljuna vizaĝo de virino rigardis lin super la barilo, grizaj haroj
eskapante el ŝia hararo. S-ino Figo$^{\textit{n:\ref{nomoj:figo}}}$, la okaza
vartistino. ``Kion vi estas faranta, Harry ?''

``Nenio'', Harry diris per sufokita voĉo. ``Mi—nur testas vere stultan
teorion—''

``Cû vi ricevis akceptan leteron de Herpŭrko ?''

Harry frostiĝis tuj.

``Jes,'' La lipoj de Harry diris, post mallonga momento. ``Mi ricevis
leteron de Herpŭrko. Ili diras ke ili volas mian strigon antaŭ la 31a
de julio, sed—''.

``Sed, vi \emph{ne havas} strigon. Kompatinda kara ! Mi ne povas imagi tion, kion
ili devis pensi por nur sendi la ordinaran leteron.''

Ĉifita brako etendiĝis super la barilo, kaj malfermis atenteman
manon. Preskaŭ sen pensi je tiu ĉi momento, Harry donis al ŝi la koverton.

``Lasu min zorgi pri tio, kara,'' diris S-ino Figo, ``kaj en unu tiktako aŭ du, iu venos.''  

Kaj ŝia vizaĝo malaperis malantaŭ la barilo.

Estis longa silento en la ĝardeno.

Poste infana voĉo diris, trankvile kaj mallaŭte, ``Kio.''
