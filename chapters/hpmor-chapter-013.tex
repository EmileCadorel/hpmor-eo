\chapter{Starigi la malĝustajn demandojn}

\begin{center}\rule{3in}{0.4pt}\end{center}

\emph{``Tio estas unu el la plej evidentaj ŝaradoj kiujn mi iam aŭdis.''}  

\begin{center}\rule{3in}{0.4pt}\end{center}

\lettrine{Ĵ}us kiam Harry malfermis siajn okulojn en la dormejo
de unuaj jaraj knaboj de Korvungo, je la mateno de lia unua plena
tago en Herpŭrko, li sciis ke io estis malbona.

Estis kviete.

\emph{Tro} kvieta.

Ho, pravas\ldots Estis la Kvietus Ĉarmo sur la kapo de lia lito,
kontrolita per eta ŝovilo, kiu estis la sola kialo pro kiu estis ebla
por iu ajn dormi en Korvungo.

Harry sidiĝis kaj rigardis ĉirkaŭe, sin atendante vidi aliajn knabojn vekiĝantaj
por la tago—

La dormejo, malplena.

La litoj, ĉifitaj kaj malfaritaj.

La suno, venante enen je relative alta angulo.

Lia Kvietigilo turnita tute maksimume.

Kaj lia meĥanika vekhorloĝo estis ankoraŭ funkcianta, sed la alarmo
estis elŝaltita.

Li estis permesita dormi ĝis 9:52, verŝajne. Malgraŭ sia plej granda klopodo por
samtempigi sian dorman ciklon de 26-horoj je sia alveno al Herpŭrko, li ne
sukcesis dormi la hieraŭan nokton antaŭ ĉirkaŭ la 1an. Li planis vekiĝi je la 7a
horo kun la aliaj studentoj, li povus elporti senti sin iom dormema dum sia unua
tago se li akirus ian magian helpon antaŭ morgaŭ. Sed nun li mankigis
matenmanĝon, kaj lian vere unuan klason en Herpŭrko, en Herbologio, kiu
komenciĝis autaŭ unu horo kaj dudek minutoj.

La kolero estis malrapide, malrapide vekiĝanta en li. Ho, kiel agrabla
eta bubaĵo. Elŝalti la alarmon. Plialtigi la Kvietigilon. Por ke la
renoma S-ro Harry Potter mankigu sian unuan klason, kaj lin oni riproĉu
ke li estas peza dormanto.

Kiam Harry malkovros kiun faris tion\ldots

Ne, tion oni povis fari nur kun la kunlaboro de ĉiuj dek-du knaboj
de la Korvunga dormejo. Ĉiuj ili verŝajne vidis lian dormantan
formon. Ĉiuj ilin lasis lin dormi ĝis post la matenmanĝo.

La kolero foriris, anstataŭita per konfuzo kaj sento de terura vundo.
Ili \emph{ŝatis} lin. Li pensis. Hieraŭ vespere, li pensis ke ili
ŝatis lin. \emph{Kial}\ldots

Dum Harry stariĝis el sia lito, li vidis pecon da papero frontanta la
kapon de la lito.

La papero diris,
\medskip

\emph{Miaj Korvungaj Kunfratoj}

\emph{Estis tre longa tago. Bonvolu, lasi min dormi kaj ne zorgu pri
  mia mankonta matenmanĝo. Mi ne forgesis mian unuan klason.}

\emph{Amike,\\
  Harry Potter.
}
\medskip

Kaj Harry staris tie frostigata, dum glacia akvo komencis guti tra
liaj vejnoj.

La mesaĝo estis de lia propra skribmaniero, kaj skribita per sia
propra meĥanika skribilo.

Kaj li ne memoris skribi ĝin.

Kaj\ldots Harry rigardis la pecon da papero kun la okuloj preskaŭ fermaj. Kaj
krom se li imagis tion, la vortoj ``Mi ne forgesis'' estis skribitaj malsame,
kvazaŭ li provis diri al si mem ion\ldots?

Ĉu li \emph{sciis} ke li estos Forgesigita? Ĉu li vigilis malfrue, faris
ian krimon aŭ sekretan agadon, kaj tiam\ldots sed li ne \emph{sciis}
la Forgesigantan sorĉon\ldots ĉu iu alia\ldots kio\ldots

Penso alvenis al Harry. Se li \emph{sciis} ke li estos
Forgesigita\ldots

Ankoraŭ en sia piĵamo, Harry kuris ĉirkaŭ sia lito al sia trunko,
premis sian dikfingron kontraŭ la seruro, tiris sian haŭtpoŝon, metis
sian manon en ĝin kaj diris ``Noto por mi mem.''

Kaj alia peco da papero aperis en sia mano.

Harry prenis ĝin, ĝin rigardante. Ĝi ankaŭ estis de sia skribmaniero.

La noto diris:
\medskip

\emph{Kara mi,}

\emph{Bonvolu ludi la ludon? Vi povas ludi la ludon nur unufojon dum
  via vivo. Tio estas neanstataŭebla oportuno.}

\emph{Rekono-kodo 927, mi estas terpomo.}

\emph{Amike,\\
  Vi}

\medskip

Harry kapjesis malrapide. ``Rekono-kodo 927, mi estas terpomo'' estis
efektive la mesaĝo kiun li ellaboris anticipe—kelkajn jarojn pli frue,
rigardante televidon—kiun nur li povis koni. Se li devis identigi
duplikaton de li mem kiel estante vere li, aŭ io ajn. Ĉiaokaze. Estu Preta.

Harry ne povis \emph{fidi} la mesaĝon, povis esti aliaj sorĉoj
implikitaj. Sed tio ekskludis iun ajn simplan bubaĵon. Li definitive
skribis tiun mesaĝon kaj li definitive ne memoris skribi ĝin.

Rigardante la noton, Harry ekvidis inkon sur la alia flanko videbla
trans la papero.

Li renversis ĝin.

La dorsflanko skribis :
\medskip

\begin{center}
  \emph{INSTRUADOJ DE LA LUDO:}

  \emph{vi ne konas la regulojn de la ludo\\
    vi ne konas la vetojn de la ludo\\
    vi ne konas la celojn de la ludo\\
    vi ne scias kiu estras la ludon\\
    vi ne scias kiel finigi la ludon\\
    Vi komencas kun 100 poentoj.\\
    Komencu.}
\end{center}

\medskip

Harry fikse rigardis la ``instruadojn''. Tiu flanko ne estis
manskribita; la skribo estis perfekte regula, tial
artefarita. Aspektis kiel se tiu estis skribita per Citaĵoj-Plumo,
kiel tiu kiun li aĉetis por preni dikton.

Li havis neniun ideon pri kio estis okazanta.

Nu\ldots~etapo unu estis vesti sin kaj manĝi. Eble inversigi la ordon
de tio. Li sentis sian stomakon relative malplena.

Li mankigis matenmanĝon, evidente, sed li estis preta por tiu eventualaĵo, estinte
antaŭvidanta tion. Harry metis sian manon en sian haŭtpoŝon kaj diris
``manĝetstangoj'' atendante akiri la skatolon de cerealaj stangoj, kiun li
aĉetis antaŭ ol li iris al Herpŭrko.

Kiu aperis ne aspektis al la palpo kiel la skatolo de cerealaj stangoj.

Kiam Harry movis sian manon en sian vidkampon, li vidis du etajn
dolĉaĵajn stangojn—ne preskaŭ sufiĉe por matenmanĝo—ligitaj al noto, kaj
la noto estis skribita per la sama skribo ol la instruadoj de la ludo.

La noto diris :
\medskip

\begin{center}
  PROVO MALSUKCESA: -1 POENTO\\
  AKTUALAJ POENTOJ : 99\\
  FIZIKA STATO : ANKORAŬ MALSATA\\
  MENSA STATO : KONFUZITA\\
\end{center}

\medskip

``Arghh'' La buŝo de Harry diris sen ia konscia interveno aŭ decido de
li.

Li staris tie dum preskaŭ unu minuto.

Unu minuton pli malfrue, tio \emph{ankoraŭ} havis neniun senson kaj li
\emph{ankoraŭ} absolute havis neniun ideon pri kio estis okazanta kaj
sia cerbo ne eĉ \emph{komencis} kroĉiĝi al iu ajn hipotezo kvazaŭ
siaj mensaj manoj estis kaptitaj en kaŭĉuka pilko kaj ne povis
kapti ion ajn.

Lia stomako, kiu havis siajn proprajn prioritatojn, sugestis eblan
eksperimentan enketon.

``Ha\ldots'' Harry diris al la malplena ĉambro. ``Mi supozas ke mi ne povas
elspezi poenton por reakiri mian skatolon de cerealaj stangoj?''

Estis nur silento.

Harry metis sian manon en la haŭtpoŝon kaj diris ``Skatolo de cerealaj stangoj.''

Skatolo, kiu ŝajnis laŭ la palpo havi la bonan formon, aperis en sia
mano\ldots Sed tiu estis tro malpeza, kaj ĝi estis malfermita, kaj ĝi
estis malplena, kaj la noto ligita al ĝi diris :

\medskip
\begin{center}
  POENTOJ ELSPEZITAJ: 1\\
  AKTUALAJ POENTOJ: 98\\
  VI GAJNIS: SKATOLON DE CEREALAJ STANGOJ\\
\end{center}

``Mi deziras elspezi unu poenton kaj reale reakiri \emph{la cerealajn
stangojn},'' diris Harry.

Denove silento.

Harry metis sian manon en la haŭtpoŝon kaj diris ``cerealaj stangoj.''

Nenio alvenis.

Harry levis la ŝultrojn malespere kaj iris al la ŝranko, kiu al li estis donita
apud sia lito, por preni siajn sorĉistajn robojn por la tago.

Sur la fundo de la ŝranko, sub liaj roboj, estis la cerealaj stangoj
kaj noto :

\medskip
\begin{center}
  POENTOJ ELSPEZITAJ: 1\\
  AKTUALAJ POENTOJ: 97\\
  VI GAJNIS: 6 CEREALAJ STANGOJ\\
  VI ANKORAŬ PORTAS: PIĴAMO\\
  NE MANĜU KIAM VI ANKORAŬ PORTAS VIAN PIĴAMON, AŬ VI RICEVUS PENALON DE PIĴAMO.
\end{center}
\medskip

\emph{Kaj nun mi scias ke kiu ajn kontrolas la ludon estas freneza.}

``Mia konjekto estas ke la ludo estas kontrolata per Dumbledore,''
Harry diris laŭte. Eble \emph{tiun} fojon li povus starigi novan mondan
rekordon pri ke li estis la plej rapida en la kompreno.

Silento.

Sed Harry komencis kompreni la skemon; la noto estus je la sekva loko
kie li rigardus. Do Harry rigardis sub sia lito.

\medskip
\begin{center}
  HA! HA HA HA HA HA!\\
  HA HA HA HA HA HA!\\
  HA! HA! HA! HA! HA! HA!\\
  DUMBLEDORE NE KONTROLAS LA LUDON\\
  MALBONA KONJEKTO\\
  TRE MALBONA KONJEKTO\\
  -20 POENTOJN\\
  KAJ VI ANKORAŬ PORTAS VIAN PIĴAMON\\
  TIO ESTAS VIA KVARA AGO KAJ VI ANKORAŬ PORTAS VIAN PIĴAMON\\
  PENALO DE PIĴAMO: -2 POENTOJN\\
  AKTUALAJ POENTOJ: 75
\end{center}

Nu, tiu estas enigmo, konsentite. Estis nur lia unua tago je la lernejo kaj
unufoje kiam oni ekskludis Dumbledore'n, li ne konis la nomon de iu ajn alia tie
kiu estis same freneza.

Kun la korpo pli aŭ malpli aŭtomate pilotite, Harry kolektis la aron
da roboj kaj subvestoj, tiris la kavernan etaĝon de sia trunko (li estis
persono tre privata kaj iu povus eniri la dormejon), vestis
sin, kaj reiris supren por formeti sian piĵamon.

Harry paŭzis antaŭ ol tiri la tirkeston de la ŝranko kiu gardis sian piĵamon. Se
la skemo estis vera\ldots

``Kiel mi povas gajni pli da poentoj?''

Tiam li tiris la tirkeston.

\begin{center}
OPORTUNOJ POR FARI LA BONON ESTAS ĈIE SED MALHELO ESTAS TIE KIE LA
LUMO BEZONAS ESTI \\
KOSTO DE LA DEMANDO: 1 POENTON\\
AKTUALAJ POENTOJ: 74\\
ĈARMA SUBVESTO \\
ĈU VIA PATRINO ELEKTIS ĜIN? \\
\end{center}

Harry pistis la noton en sia mano, sia vizaĝo fariĝante skarlata. La
insulto de Drako revenis al li. \emph{Filo de kotsango—}

Je tiu momento li sciis ke prefere li ne diru ĝin laŭtvoĉe. Li
verŝajne akirus Penalon de Insulto.

Harry armis sin per sia haŭtpoŝo kaj bastono. Li forigis la pakaĵon de unu el
siaj cerealaj stangoj kaj ĵetis ĝin en la rubujon de la ĉambro, kie ĝi alteriĝis
sur plejparte nemanĝita rano el ĉokolado, ĉifita koverto kaj kelkaj verdaj kaj
ruĝaj paperoj de envolvaĵo. Li metis la aliajn cerealajn stangojn en sian
haŭtpoŝon.

Li faris finan, senesperan, kaj lastan serĉon vanan de indicoj
ĉirkaŭrigardante la ĉambron.

Kaj poste Harry foriris la dormejon manĝante, por serĉi la karcerojn de
Serpentimo. Ĉiaokaze estis tio, --li \emph{pensis}--, pri kio la noto parolis.

Provi navigi tra la koridoroj de Herpŭrko estis\ldots \emph{ne} tute tiel malbona
kiel vagi en pentraĵo de Escher. Tio estis la speco de aĵo kion vi dirus por
retorika efiko pli ol por diri la veron.

Nelonga tempo poste, Harry estis pensanta ke fakte pentraĵo de Escher
havus ambaŭ avantaĝojn kaj malavantaĝojn kompare al
Herpŭrko. Malavantaĝoj: ne kohera gravita orientiĝo. Avantaĝoj:
Almenaŭ la ŝtupoj ne moviĝus \emph{KIAM VI ESTAS ANKORAŬ SUR ILIN.}

Harry origine grimpis sur kvar ŝtuparojn por iri al sia dormejo. Post ol li
malsuprenrampis ne malpli ol dudek ŝtuparoj sen atingi nenien proksime al la
karceroj, Harry konkludis ke (1) pentraĵo de Escher estus infana ludo kompare,
(2) li estis iel pli alta en la kastelo ol kiam li komencis, kaj (3) li estis
tiel \emph{komplete} perdita ke li ne estus surprizita se li rigardus trans la
fenestro por vidi du lunojn en la ĉielo.

Savoplano (A) estis ekhalti kaj demandi la bonan direkton, sed ŝajnis ke estis
ekstreme granda manko de vagantaj homoj ĉirkaŭ li, kvazaŭ ĉiuj vaguloj ĉeestis
en leciono, kiel ili estis supozitaj fari aŭ io kiel tio.

Savoplano (B)\ldots

``Mi estas perdita,'' Harry diris laŭtvoĉe. ``Ĉu la, hum, spirito de
Herpŭrko povas helpi min aŭ io simila?''

``Mi ne opinias ke la kastelo havas spiriton,'' rimarkigis ŝrumpinta maljuna
virino en unu el la pentraĵoj sur la muroj. ``Vivon eble, sed neniu spirito.''

Estis mallonga paŭzo.

``Ĉu vi estas—'' Harry diris, post ol li ekfermis la buŝon. Post pripensado, ne,
li \emph{NE} demandos al la pentraĵo ĉu ĝi estis tute konscia je la senco ke ĝi
estis konscia pri sia propra konscio.

``Mi estas Harry Potter,'' diris lia buŝo, pli aŭ malpli pilotita aŭtomate. Ankaŭ pli aŭ
malpli aŭtomate, Harry levis sian manon al la pentraĵo.

La virino en la pentraĵo rigardis malsupren al la mano de Harry kaj levis siajn
brovojn.

Malrapide, lia mano falis al sia flanko.

``Mi bedaŭras,'' Harry diris, ``Mi estas relative nova ĉi tie.''

``Mi sentis tion, juna korvo. Kien vi provas iri?''

Harry hezitis. ``Mi ne tute certas,'' li diris.

``Tial, eble vi jam estas tie.''

``Nu, kien ajn mi \emph{provas} iri, mi ne opinias ke \emph{tio} estas tie\ldots''
Harry ekfermis la buŝon, konscia pri kiom multe li aspektis stulta. ``Lasu min
rekomenci. Mi ludas la ludon sed mi ne konas la regulojn—'' tio ne ankaŭ vere
funkciis, ĉu ne. ``Konsentite, tria provo. Mi serĉas oportunojn por fari bonon
tiel ke mi povu gajni poentojn, kaj ĉio kion mi havas estas tiu mistera
indico pri kiel malhelo estas tie kie la lumo bezonas troviĝi, do mi provas iri
malsupren sed mi ŝajnas daŭri iri supren anstataŭe\ldots''

La maljuna virino en la pentraĵo estis rigardanta lin sufiĉe skeptike.

Harry suspiris. ``Mia vivo inklinas fariĝi bizara.''

``Ĉu estus ĝusta diri ke vi ne scias kien vi estas iranta aŭ kial vi
provas iri tien?''

``*Tute* ĝusta.''

La maljuna virino kapjesis. ``Mi ne certas ke esti perdita estas via
plej grava problemo, junulo.''

``Ĝusta, sed malkiel la pli gravaj problemoj, tiu estas problemo kiun mi povas
solvi kaj \emph{ŭaŭ} tiu ĉi konversacio ŝanĝiĝas al metaforo de la ekzistado de
homaro, mi eĉ ne konsciis ke tio okazis ĝis nun.''

La virino rigardis Harry'n taksante. ``Vi \emph{estas} bonega juna
korvo, ĉu ne? Dum momento, mi komencis min demandi. Nu tiel, laŭ
ĝenerala regulo, se vi daŭrigas turni maldekstren, vi daŭrigas
iri malsupren.''

Tio ŝajnis strange kutima sed Harry ne povis rememori tie, kie li aŭdis tion
antaŭe. ``Hum\ldots vi ŝajnas esti tre inteligenta persono. Aŭ bildo de tre
inteligenta persono\ldots ĉiaokaze, ĉu vi iam aŭdis pri mistera ludo kiun vi povas
ludi nur unufojon, kaj neniu diras la regulojn al vi?''

``Vivo','' diris la virino tuj. ``Tio estas la plej evidenta enigmo
kiun mi iam aŭdis.''

Harry palpebrumis. ``Ne,'' li diris malrapide. ``Mi volas diri, mi akiris realan
noton kaj ĉio, dirinta ke mi devas ludi la ludon sed oni ne diris al mi la
regulojn, kaj iu lasis al mi etajn pecojn da papero, dirantaj, kiom da poentoj
mi perdis pro ke mi malrespektis la regulojn, kiel minus du poentoj de penalo
pro ke mi ankoraŭ portis mian piĵamon. Ĉu vi konas iun ajn tie en Herpŭrko kiu
estas sufiĉe freneza kaj sufiĉe potenca por fari ion kiel tio? Krom Dumbledore,
mi volas diri?''

La bildo de la virino suspiris. ``Mi estas nur bildo, junulo. Mi memoras
Herpŭrko'n kiel ĝi estis—ne kiel ĝi estas. Ĉiu kiun mi povas diri al vi estas ke
se tiu estus enigmo, la respondo estus ke la ludo estas la vivo, kaj ke dum ni
ne faras ĉiujn regulojn ni mem, tiu kiu premias aŭ retiras poentojn estas
ĉiam ni mem. Se tiu ne estas enigmo sed la realaĵo—tial mi ne scias.''

Harry kliniĝis tre malalten al la bildo. ``Dankon, Sinjorino.''

La virino riverencis. ``Mi dezirus ke mi povu diri ke mi memoros vin ŝate,'' ŝi
diris, ``sed mi verŝajne tute ne memorus vin. Adiaŭ, Harry Potter.''

Li kliniĝis denove responde, kaj komencis malsupreniri la plej
proksiman ŝtuparon.

Post kvar maldekstraj turnoj, li troviĝis antaŭ koridoro kiu finiĝis, abrupte, al
monteto de larĝaj ŝtonoj—kvazaŭ estis disfalo, nur la ĉirkaŭaj muroj kaj plafono
estis sendifektaj kaj estis faritaj el tute regulaj kastelaj ŝtonoj.

``Konsentite,'' Harry diris al la malplena aero, ``Mi rezignas. Mi
demandas alian konsileton. Kiel mi iras tien, kien mi bezonas iri?''

``Konsileto! Konsileto, vi diris?''

La ekscitita voĉo venis el pentraĵo proksima sur la muro, tiu estis portreto de
mezaĝa viriĉo vestita de la plej brilega rozkolora robo, kiun Harry iam vidis aŭ
imagis. La portreto portis pintan kaj malnovan ĉapelon falanta, kun fiŝo sur ĝi
(ne desegnaĵo de fiŝo, sed fiŝo).

``Jes!'' Harry diris. ``Konsileto! Konsileto, mi diris! Nur ne \emph{iu ajn}
konsileto, mi serĉas \emph{specifan} konsileton, tio estas por ludo kiun mi
ludas—''

``Jes, Jes! Konsileto por la ludo! Vi estas Harry Potter, ĉu ne? Al mi estis
dirita mesaĝo per Erin la Edzino, mesaĝo kiu al ŝi estis dirita per Mastro
Vizelnazo, kaj al kiu tiu ĉi mesaĝo estis dirita per, mi forgesis. Sed tiu estis
mesaĝo por \emph{mi} kaj kiun mi devas doni al vi! Por \emph{mi!} Neniu zorgas
pri mi, mi ne scias kiel longe, eble ĉiam, mi estas enfermata tie en ĉi tiu
komplete senutila kaj malnova koridoro—konsileto! Mi havas vian konsileton! Tiu
kostos al vi tri poetojn! Ĉu vi volas ĝin ?''

``Jes! Mi volas ĝin!'' Harry estis konscia ke li probable devus gardi
sian sarkasmon sub kontrolo sed li simple ŝajnis nekapabla malhelpi
sin.

``La malhelo povas esti trovata inter la verda studĉambro kaj la Klaso de
Transfiguro de McGonagall! Tiu estas la konsileto! Kaj rapidu, vi estas pli
malrapida ol heliko! Minus dek poentojn pro via malrapideco! Nun vi havas 61
poentojn! Tio estis la restaĵo de la mesaĝo!''

``Dankon,'' Harry diris. Li estis vere malfrua en la ludo, ĉi
tie. ``Hum\ldots mi supozas ke vi ne scias de kie la mesaĝo origine
venis, ĉu ne?'' 

``Ĝi estis dirita per kava voĉo kiu sonoris de fore el fendo en la aero, fendo
kiu malfermiĝis sur fajra abismo! Tio estas tio, kion ili diris al mi!''

Harry ne plu certiĝis, je tiu momento, ĉu tia estis la speco de aĵo pri kia li
devus esti skeptika, aŭ la speco de aĵo kiun li devus simple preni tiele. ``Kaj
kiel mi povas trovi la linion inter la studĉambroj kaj la klaso de
Transfiguro?''

``Nur turniĝu kaj iru maldekstren, dekstren, malsupren, malsupren, dekstren,
maldekstren, dekstren, supren, kaj maldekstren denove, vi estos antaŭ la verda
studĉambro kaj se vi iras enen kaj eliras direkte tra la kontraŭa pordo vi
troviĝos en granda kurba koridoro kiu iras al kruciĝo kaj je la dekstra flanko
de tiu kruciĝo estas longa kaj rekta vestiblo kiu iras al la klasĉambro de
Transfiguro!'' La bildo de la mezaĝa viriĉo paŭzis. ``Almenaŭ tio estas kiel tiu
estis kiam \emph{mi} estis en Herpŭrko. Estas lundo hodiaŭ, nepara jaro, ĉu
ne?''

``Krajono kaj meĥanika papero,'' Harry diris al sia haŭtpoŝo. ``Er,
nuligu tion, papero kaj meĥanika krajono.'' Li rigardis supren. ``Ĉu
vi povas ripeti tion?''

Post kiam li estis perdita denove du fojojn, Harry sentis ke li komencis kompreni la bazan
regulon por navigi trans la ĉiam ŝanĝanta labirinto kiu estis Herpŭrko, nomita,
\emph{demandu al pentraĵo la direkton.} Se tio reflektis ian nekredeble profundan
lecionon de vivo, li ne sukcesis malkovri kion ĝi estis.

La verda studĉambro estis loko mirinde agrabla kun sunlumo fluanta tra
la fenestroj el verda vitralo kiuj montris drakojn en trankvilaj, kaj
bukolikaj scenoj. Ĝi havis seĝojn kiuj aspektis treege komfortaj, kaj
tablojn kiuj ŝajnis perfekte taŭgaj por studi kun akompano de unu al
tri amikoj. 

Harry ne povis \emph{efektive} marŝi direkte tra la ĉambro kaj eliri per la
pordo je la kontraŭa flanko. Estis librobretaroj fiksitaj en la muroj, kaj li
devis legi kelkajn el la titoloj, por ne perdi sian pretendon al la nomo de la
familio Verres. Sed li faris tion rapide, atente pri la plendo pri sia
malrapideco, kaj poste li eliris tra la alia flanko.

Li marŝis laŭ la ``granda kurba koridoro'' kiam li aŭdis la voĉon de
juna knabo ekkrii.

Je momento kiel tio, Harry havis ekskuzon por kuregi sen konsideri ŝpari
energion aŭ fari decan ekzercon de varmiĝo aŭ zorgi kraŝi en aĵojn, subita
furioza flugo kiu iris al egale subita halto kiam li preskaŭ koliziis en
grupon de ses Huflopufoj de unua jaro\ldots

\ldots Kiuj estis alpremiĝantaj unu al la aliaj, aspektante relative timigitaj
kaj kiel se ili senespere volis fari ion sed ne povis trovi kion. Probable ke
tio temis pri la grupo de kvin pli aĝaj Serpentimoj kiuj ŝajnis ĉirkaŭi alian
junan knabon.

Harry estis subite relative kolera.

``\emph{Bonvolu!}'' kriis Harry plenpulme.

Tio eble ne estis necesa. Homoj estis jam rigardanta lin. Sed tio certe servis
por frostigi ĉiujn agojn.

Harry marŝis antaŭen preter la grupo de Huflopufoj al la Serpentimoj.

Ili rigardis malsupren al li kun esprimoj kiuj iris de kolero al amuzo pasante
tra ĝojo.

Parto de la cerbo de Harry kriis pro paniko ke tiuj estis multe pli
aĝaj kaj pli grandaj knaboj kiuj povis piedbati lin plate.

Alia parto diris seke ke iu ajn vidita serioze piedbatante la
Knabon-Kiu-Postvivis havus \emph{monton} da problemoj, speciale se ili estis
grupo da pli aĝaj Serpentimoj kaj ke estis sep Huflopufoj por vidi la scenon,
kaj ke la ŝanco ke ili faris iun ajn daŭran damaĝon kun la ĉeesto de atestantoj
proksimiĝis al nulo. La sola vera armilo kiun la pli aĝaj knaboj havis kontraŭ li
estis sia propra timo, se li permesis tion.

Tiam Harry vidis ke la knabo kiun ili kaptis estis Nevilo Longafundo.

Evidente.

Tio decidigis tion. Harry estis decidinta pardonpeti malfiere al Nevilo kaj tio
volis diri ke Nevilo estis lia, kiel ili \emph{aŭdacis}?

Harry etendis la brakon kaj kaptis Nevilo'n per la pojno kaj
\emph{retiris} lin de inter la Serpentimoj, la knabo faletis pro la
ŝoko dum Harry elirigis lin kaj preskaŭ per la sama movo faris
sian propran vojon tra la sama truo.

Kaj Harry staris je la centro de la Serpentimoj kie Nevilo estis
starinta, rigardante supren al la multe pli aĝaj, pli larĝaj kaj
pli fortaj knaboj.

``Bonan tagon,'' Harry diris. ``Mi estas la Knabo-Kiu-Postvivis.''

Estis relative mallerta paŭzo. Neniu ŝajnis scii kien la
konversacio estis supozita iri de tie.

La okuloj de Harry malsupreniris kaj vidis librojn kaj paperoj dismetitaj
ĉirkaŭe sur la planko. Ho, la malnova ludo kie vi lasis la knabon provi kapti
siajn librojn kaj poste vi faligis ilin el riaj manoj denove. Harry ne povis
memori esti la objekto de tia ludo li mem, sed li havis bonan imagon kaj sia
imago farigis lin furioza. Nu, post ol la pli larĝa situacio estos solvita,
estus sufiĉe facila por Nevilo reveni kaj kapti siajn aferojn, kondiĉe ke la
Serpentimoj restus tro absorbita per li por pensi fari ion ajn al la libroj.

Bedaŭrinde la vagantaj okuloj estis rimarkitaj. ``Hoo,'' diris la plej
larĝa el la knaboj, ``ĉ'vi v'las la etaj' librojn—''

``Silentu,'' Harry diris malvarme. \emph{Konfuzigu ilin. Ne faru ion kion ili
atendas. Ne falu en la skemon kiu permesus al ili ĉikani vin.} ``Ĉu tio estas
parto de nekredeble lerta plano kiu farigos vin gajni ian estontan avantaĝon, aŭ
ĉu tio estas tiel sensenca malhonoro al la nomo de Salazaro Serpentimo kiel
tio—''

La plej larĝa knabo ŝovis Harry'n Potter forte, kio faligis lin el la cirklo de
Serpentimo sur la firma planko el ŝtono de Herpŭrko.

Kaj la Serpentimoj ridis.

Harry starigis per movo kiu al li ŝajnis terure malrapida. Li ankoraŭ ne sciis
kiel uzi sian bastonon, sed estis neniu kialo por lasi tion haltigi lin, sub
tiuj ĉi cirkonstancoj.

``Mi volas pagi tiom da poentoj kiom tio prenos por liberigi min de tiu ĉi
persono,'' Harry diris montrante per sia fingro la plej larĝa Serpentimo.

Poste Harry levis sian alian manon, kaj diris ``Abrakadabra,'' kaj
klakis per la fingroj.

Je la vortoj \emph{Abrakadabra} du el la Huflopufoj kriis, inkluzive de Nevilo,
tri aliaj Serpentimoj saltis urĝe el la vojo de la fingro de Harry, kaj la plej
larĝa Serpentimo ŝanceliĝis malantaŭen kun esprimo de ŝoko, kun subita ruĝa
makulo dekorante sian vizaĝon kaj kolon kaj bruston.

Harry \emph{ne} estis atendanta \emph{tion}.

Malrapide, la plej larĝa Serpentimo atingis sian kapon per la mano, kaj deigis
la paton de ĉerizotorton kiu drapiris lin. La plej larĝa Serpentimo tenis la
paton en sia mano dum momento, rigardadante ĝin, kaj poste lasis ĝin fali sur la
plankon.

Tio probable ne estis la plej bona momento en la mondo por ke unu el la
Huflopufoj komencu ridi, sed tio estis ekzakte kio unu el la Huflopufoj estis
faranta.

Tiam Harry ekvidis noton sur la fundo de la pato.

``Atendu,'' Harry diris, kaj li hastis antaŭen por preni la noton. ``Tiu ĉi noto
estas por mi, mi opinias—''

``\emph{Vi,}'' graŭlis la plej larĝa Serpentimo, ``\emph{vi, estos, —}''

``*Rigardu* al tio!'' kriis Harry, svinganta la noton al la pli aĝa Serpentimo.
``Mi volas diri, nur \emph{rigardu} al tio! Ĉu vi povas kredi ke vi devas pagi
30 poentojn por la ekspedo kaj manipulado de unu aĉa torto? 30 poentoj! Mi havas
malprofiton je la afero, eĉ savinte senkulpan knabon el malespera situacio!
Kaj kostoj de konservado? Kostoj de transporto? Kostoj de elsendo? Kiel vi povas
havi \emph{elsendaj kostoj} por iu \emph{torto}?''

Estis iu alia el tiuj mallertaj paŭzoj. Harry havis mortigajn pensojn pri tiu
Huflopufo kiu ŝajnis ne povi sin halti subridi, tiu stultulo estis farigonta lin
dolorigita.

Harry retropaŝis kaj ĵetis, al la Serpentimoj, sian plej mortigan rigardon. ``Nun,
foriru aŭ mi daŭras farigi vian ekziston pli kaj pli superrealisma ĝis vi tion
faras. Lasu min averti vin\ldots ke perturbi \emph{mian} vivon emas farigi
\emph{vian} vivon\ldots \emph{iom neplaĉa.} Ĉu vi komprenis?''

Per iu sola terura movo, la plej larĝa Serpentimo subite eltiris sian bastonon
por direkti ĝin al Harry kaj je la sama tempo li estis frapita je la alia flanko
de sia kapo per alia torto, tiu ĉi el hela mirtelo.

La noto sur la torto estis relative larĝa kaj klare legebla. ``Vi eble
volas legi la noton sur la torto,'' Harry observis. ``Mi pensas ke
tiu ĉi estas por vi, ĉi foje.'' 

La Serpentimo malrapide levis la manon, prenis la tortan panon, turnis ĝin
aŭdigante malsekan bruon dum pli da mirtelo falis sur la plankon, kaj li legis
la noton, kiu diris:

\begin{center}
  AVERTO\\

  NENIU MAGIO RAJTAS ESTI UZITA SUR LA KONKURSANTO DUM LA LUDO DAŬRAS\\
  PLIA INTERFERO EN LA LUDO ESTOS RAPORTITA AL LA ESTRARO DE LA LUDO
\end{center}

La esprimo de kruta konfuzo sur la vizaĝo de la Serpentimo estis vera
ĉefverko. Harry pensis ke li povus komenci ŝati la kontrolanton de la
ludo.

``Aŭskultu,'' Harry diris, ``Ĉu vi volas ke oni haltu tie? Mi opinias ke aĵoj
fariĝas nekontroleblaj. Kial vi ne reirus al Serpentima ejo kaj mi reirus al
Korvunga ejo, kaj kial ni ĉiuj ne nur trankviliĝus iom, Konsentite?''

``Mi havas pli bonan ideon,'' siblis la plej larĝa Serpentimo. ``Kial vi ne
akcidente rompus ĉiujn viajn fingrojn?''

``Kiel, per la nomo de Merlino, vi enscenigos kredeblan akcidenton post
farinta la minacon antaŭ ĉirkaŭ dek du atestantoj, vi \emph{stultulo}—''


La plej larĝa Serpentimo malrapide, intence antaŭeniris al la manoj de
Harry, kaj Harry frostiĝis, la parto de sia cerbo kiu rimarkis la aĝon
kaj forton de la alia knabo, finfine sukcesis igi ĝin mem aŭdi,
kriante, \emph{KIO DIABLE MI ESTAS FARANTA?}


``Atendu!'' diris unu el la aliaj Serpentimoj, sia voĉo subite
panika. ``Haltu, vi ne devus reale fari tion!''

La plej larĝa Serpentimo ignoris lin, prenis la dekstran manon de Harry
firme en sia maldekstra mano, kaj prenis la montran fingron de Harry
en sia dekstra mano.

Harry rigardis la Serpentimon direkte en la okuloj. Parto de Harry
estis krianta, tio ne estis supozita okazi, tio ne estis \emph{permesita}
okazi, plenkreskuloj neniam lasus ion kiel tio \emph{reale} okazi—

Malrapide, la Serpentimo komencis fleksi lian montran fingron returnite.

\emph{Li ne fakte frakturos mian fingron kaj mi ne rajtas stumbli antaŭ
li faras. Ĝis tiam, tio estas nur alia provo por kaŭzi timon. }

``Haltu!'' diris la Serpentimo kiu obĵetis antaŭe. ``Haltu, tio estas
vere malbona ideo!''

``Mi relative konsentas,'' diris iu glacia voĉo. La voĉo de pli aĝa virino.

La plej larĝa Serpentimo lasis la manon de Harry eliri kaj saltis
malantaŭen kvazaŭ ĝi brulis.

``Profesorino Sproso!'' kriis iu el la Huflopufoj, sonante pli ĝoja
ol iu ajn, kiun Harry jam aŭdis dum sia tuta vivo.

En la vidkampo de Harry, kiam li sin turnis, aperis eta virino korpulenta kun
ĥaosaj kaj buklaj haroj kaj vestoj kovritaj per koto. Ŝi direktis akuzantan
fingron al la Serpentimoj. ``Klarigu vin mem,'' ŝi diris. ``Kio vi estas faranta
al miaj Huflopufoj kaj\ldots'' ŝi rigardis lin. ``Mia bona studento, Harry
Potter.''

\emph{Hu, ho. Tio estas vera, tiu estis ŝia kurso kiun mi mankigis tiun ĉi matenon.}

``Li minacis mortigi nin!'' elbabilis iu el la aliaj Serpentimoj, la
sama kiu vokis por halto.

``Kio?'' Harry diris svage. ``Tio \emph{ne} pravas! Se mi estus
mortigonta vin, mi ne farus publikajn minacojn!''

La tria Serpentimo ridis senhelpe kaj poste haltis abrupte kiam la
aliaj knaboj ĵetis al li mortigajn rigardojn.

Profesorino Sproso adoptis relative skeptikan esprimon. ``Kia mortiga
minaco tiu estis, ekzakte?''

``La Mortiga Malbeno! Li ŝajnigis sin uzi la Mortigan Malbenon al
ni!''

Profesorino Sproso turniĝis por rigardi Harry'n. ``Jes, komplete terura minaco
de dek unu jaraĝa knabo. Tamen malgraŭe, ne io kion vi devus \emph{iam} sonĝi
ŝajnigi vin fari, Harry Potter.''

``Mi eĉ ne scias la \emph{vortojn} de la Mortiga Malbeno,'' Harry diris
prompte. ``Kaj mi ne eltiris mian bastonon iam ajn.''

Nun Profesorino Sproso donis al Harry skeptikan esprimon. ``Mi supozas ke tiu
knabo al si ĵetis du tortojn per si mem, tial.''

``Li ne uzis sian bastonon!'' elbabilis iu el la junaj
Huflopufoj. ``Mi ankaŭ ne scias kiel li faris tion, li simple klakis
siajn fingrojn kaj la tortoj aperis!''

``Vere,'' diris Profesorino Sproso post paŭzo. Ŝi eltiris sian propran bastonon.
``Mi ne devigas vin, ĉar vi ŝajnas esti la viktimo tie, sed ĉu ĝenus vin se mi
tion kontrolas?''

Harry prenis sian bastonon. ``Kion mi faras—''

``\emph{Priore Inkantato,}'' diris Sproso. Ŝi malridetis. ``Tio estas
stranga, via bastono ŝajnas esti neniam uzita.''

Harry levis la ŝultrojn. ``Ĝi neniam estas, fakte, mi nur akiris mian
bastonon kaj lernolibrojn antaŭ kelkajn tagojn.''

Sproso kapjesis. ``Tial oni havas klaran kazon de akcidenta magio de knabo kiu
sentis sin minacita. Kaj la reguloj klare deklaras ke vi ne devas esti tenata
responda pri ĝi. Kaj por \emph{vi}\ldots'' Ŝi turnis sin al la Serpentimoj. Ŝiaj
okuloj falis intence al la libroj de Nevilo kuŝantaj sur la planko.

Estis longa silento dum kiu ŝi rigardis la kvin Serpentimojn.

``Tri poentojn deprenitaj de Serpentimo, por ĉiuj,'' ŝi diris finfine. ``Kaj ses
pli pro \emph{li},'' celante la knabon kovrita per torto. ``Ne iam tedu
miajn Huflopufojn denove, aŭ mian studenton Harry Potter ankaŭ ne. Nun
\emph{foriru}.''

Ŝi ne bezonis ripeti sin, la Serpentimoj turniĝis kaj formarŝis tre rapide.

Nevilo komencis rekapti siajn librojn. Li ŝajnis ploranta, sed nur iomete. Tio
eble estis pro prokrastita ŝoko, aŭ tio eble estis ĉar la aliaj knaboj helpis
lin.

``\emph{Multan} dankon, Harry Potter,'' Profesorino Sproso diris al li. ``Sep
poentojn al Korvungo, unu por ĉiuj Huflopufoj, kiujn vi helpis protekti. Kaj mi
ne diros ion ajn plian.''

Harry papelbrumis. Li atendis ion pli laŭ la linio de leciono pri singardemo kaj
evitado de problemoj, kaj relative severa skoldo pro ke li mankigis sian vere
unuan kurson.

Eble li \emph{devintus} iri al Huflopufo. Sproso estis mojosa.

``\emph{Purgigu},'' Sproso diris al la kaĉo de torto sur la planko, kiu tuj
malaperis.

Kaj ŝi foriris, marŝante laŭ la ĉambrego kiu kondukis al la verda studĉambro.


``Kiel vi \emph{faris} tion?'' siblis unu el la Huflopufaj knaboj ĵus
kiam ŝi estis foririnta.

Harry ridetis arogante. ``Mi povas farigi ion ajn kion mi volas, okazi nur per
klaki per la figroj.''

La okuloj de la knaboj plilarĝiĝis. ``\emph{Vere?}''

``Ne,'' diris Harry. ``Sed kiam vi rakontos al ĉiuj la historion, certigu ke vi
parolos pri ĝi kun Hermione Granger en unua jaro en Korvungo, ŝi havas anekdoton
kiun vi eble trovos amuza.'' Li havis absolute neniun ideon pri kio okazis, sed
li ne estis lasonta pasi oportunon por kreskigi sian legendon. ``Ho, kaj kio
estis ĉio tio pri la Mortiga Malbeno?''

La knabo donis al li strangan rigardon. ``Vi vere ne scias?''

``Se mi scius, mi ne demandus.''

``La vortoj de la Mortiga Malbeno estas,'' la knabo glutis, kaj sia voĉo
malkreskis al murmuro, kaj li levis siajn manojn malproksimen de siaj flankoj
kiel por certigi klare ke li ne estis tenanta sian bastonon. ``\emph{Avada
Kedavra.}''

\emph{Nu, evidente ili estas.}

Harry metis tion sur sian kreskantan liston de aĵoj kiujn li devus neniam diri al
sia Paĉjo, Profesoro Michael Verres-Evans. Estis sufiĉe malbona paroli pri la
fakto ke vi estis la sola persono kiu postvivis la timigantan Mortigan Malbenon,
sen devi konfesi ke la Mortiga Malbeno estis ``Abracadabra.''

``Mi vidas,'' Harry diris post paŭzo. ``Nu, tio estas la lasta fojo,
ke mi diras \emph{tion} antaŭ klaki per la fingroj.'' Tamen tio
produktis efikon kiu povas esti taktike utila.

``*Kial* vi diris—''

``Edukita per Mugloj. Mugloj pensas ke tio estas ŝerco kaj ke tio
estas komika. Serioze, tio estas kio okazis. Pardonu, sed ĉu vi povas
memorigi vian nomon al mi?''

``Mi estas Ernie Mcmillan,'' diris la Huflopufo. Li levis sian manon,
kaj Harry premis ĝin. ``Mi estas honorita renkonti vin.''

Harry klinetis sin. ``Mi estas ĝoja renkonti vin, forgesu la
honoritan aĵon.''

Tiam la aliaj knaboj amasis ĉirkaŭ lin, kaj estis subita inundo da prezentadoj.

Kiam li estis fininta, Harry glutis. Tio estis tre
malfacila. ``Hum\ldots se ĉiuj vi bonvolu pardoni min\ldots Mi havas
ion por diri al Nevilo—''

Ĉiuj okuloj turniĝis al Nevilo, kiu retropaŝis, sia vizaĝo aspektante
anksia.

``Mi supozas,'' Nevilo diris per sia eta voĉo, ``vi estas dironta ke
mi devintus esti pli brava—''

``Ho, ne, nenio kiel tio!'' Harry diris haste. ``Nenio kio rilatas kun
\emph{tio}. Tio etas nur, hum, io, kion la Ordiganta Ĉapelo al mi
diris—''

Subite la aliaj knaboj aspektis \emph{tre} interesataj, krom Nevilo,
kiu aspektis ankoraŭ \emph{pli} anksia.

Ŝajnis ke io estis blokanta la gorĝon de Harry. Li sciis ke li devus simple
elbabili tion, kaj tio estis kvazaŭ li glutis larĝan brikon kiu estis nur
blokita sur la vojo.

Tio estis kvazaŭ Harry devis mane preni la kontrolon de siaj lipoj kaj krei
ĉiujn silabojn individue, sed li sukcesis okazigi tion. ``Mi, be,daŭ,ras.'' Li
elspiris kaj prenis profundan respiron. ``Por kio mi faris, hum, la alia tago.
Vi\ldots ne bezonas esti kompata pri tio aŭ io ajn, mi komprenus se vi nur
malamus min. Tio ne estas pri mi provanta esti mojosa per pardonpeti nek pri ke
vi devas akcepti tion. Kion mi faris estis malbona.''

Estis paŭzo.

Nevilo tenis siajn librojn pli firme al sia brusto. ``Kial vi faris
tion?'' li diris per eta voĉo hezitanta. Li palpebrumis, kvazaŭ li
provis reteni larmojn.  ``Kial \emph{ĉiuj} faras tian al mi, eĉ la
Knabo-Kiu-Postvivis?''

Harry subite sentis pli eta ol li iam sentis sin dum sia tuta vivo. ``Mi
bedaŭras,'' Harry diris denove, la voĉo nun raŭka. ``Tio estas nur\ldots ke vi
aspektis tiel timita, estis kiel se signo estis super via kapo dirante
'viktimon', kaj mi volis montri al vi ke aĵoj \emph{ne} ĉiam turniĝas malbone,
ke kelkfoje la monstroj donas al vi ĉokoladon\ldots Mi pensis ke se mi montris
tion al vi, vi eble konscius ke ne estas tiom pri kio esti timita—''

``Sed \emph{estas},'' murmuris Nevilo. ``Vi vidis tion hodiaŭ, \emph{estas!}''

``Ili ne farus ion ajn vere malbonan antaŭ atestantoj. Ilia ĉefa armilo estas
timo. Tio estas kial ili celis \emph{vin}, ĉar ili povas vidi ke vi estas
timita. Mi volis farigi vin malpli timita\ldots montri al vi ke la timo estas
pli malbona ol la aĵo ĝi mem\ldots aŭ estis tio kion mi diris al mi mem, sed la
Ordiganta Ĉapelo diris al mi ke mi mensogis al mi mem kaj ke mi reale faris
tion, ĉar tio estis amuza. Do tio estas kial mi pardonpetas—''

``Vi dolorigis min,'' diris Nevilo. ``kiam vi kaptis min kaj
tiris min for de ili.'' Nevilo levis sian brakon kaj montris kien
Harry kaptis lin. ``Mi eble havos kontuzon tie pli malfrue, pro kiom
forte vi tiris. Vi dolorigis min pli malbone ol ĉiuj Serpentimoj
faris batante min, fakte.''

``\emph{Nevilo!}'' siblis Ernie. ``Li provis \emph{savi} vin!''

``Mi bedaŭras,'' murmuris Harry. ``Kiam mi vidis tion, mi nur fariĝis\ldots tre
kolera\ldots''

Nevilo rigardis lin konstante. ``Do vi retiris min for tre forte, metis vin mem
kien mi estis kaj diris, 'Bonan tagon, mi estas la Knabo-Kiu-Postvivis'.''

Harry kapjesis.

``Mi pensas ke vi estos vere mojosa iam,'' Nevilo diris. ``Sed por la momento,
vi ne estas.''

Harry glutis la subitan nodon kiu aperis en sia gorĝo kaj formarŝis. Li daŭris
marŝi trans la koridoro al la kruciĝo, tiam turniĝis maldekstren en alian
koridoron, kaj plu marŝis, blinde.

Kion li estis \emph{supozita} fari tie? Neniam fariĝi kolera? Li ne estis certa
ĉu li povus fari ion ajn sen esti kolera kaj kiu konas tion, kio okazus al
Nevilo kaj siaj libroj tiam. Krome, Harry legis sufiĉe da fantaziaj libroj por
scii kiel tio funkcias. Li provos forigi sian koleron kaj li malsukcesos kaj ĝi
daŭros eliri denove. Kaj post tiu tute longa vojaĝo de mem-malkovro, li lernos
ke finfine sia kolero estas parto de li mem kaj nur per akcepti ĝin li lernos
kiel uzi ĝin saĝe. \emph{Star Wars} estis la sola universo kies respondo estas
efektive ke vi estis supozita liberigi vin de negativaj emocioj, kaj tio estis
io pri Yoda kio ĉiam farigis Harry'n malami tiun etan verdan stultulon.

Do la evidenta plano por gajni tempon estis preterlasi la vojaĝon de
mem-malkovro kaj iri direkte al la parto, kie li ekkonscios ke nur per akcepti
sian koleron kiel parto de li mem, li povus gardi ĝin sub kontrolo.

La problemo estis ke li ne sentis sin senkontrola kiam li estis kolera. La
malvarma kolerego farigis lin senti sin kiel se li estis kontrolanta sin. Estis
nur kiam li rigardis malantaŭen ke \emph{eventoj kiel tuto} ŝajnis esti\ldots
eksplodintaj ekster kontrolo, iel.

Li sin demandis kiom la Kontrolanto de la Ludo zorgis pri tiaj aĵoj, kaj ĉu li
gajnus aŭ malgajnus poentojn pro tio. Harry per li mem, opiniis ke li perdis
multajn poentojn, kaj li certis ke la maljuna virino en la pentraĵo estus dirinta al
li ke tiu estis la sola opinio kiu gravis.

Kaj Harry estis ankaŭ sin demandanta ĉu la Kontrolanto de la Ludo sendis
Profesorinon Sproso. Tio estis logika penso: la noto minacis voki la Estrarojn
de la Ludo, kaj poste Profesorino Sproso aperis. Eble ke Profesorino Sproso
estis la Kontrolanto de la Ludo—La Estrino de la Domo Huflopufo estus la lasta
persono kiun iu ajn suspektus, kaj tio devus meti ŝian nomon proksima al la
supro de la listo de Harry. Li estis leginta unu aŭ du misterajn novelojn,
ankaŭ.

``Do kiel mi faras en la ludo?'' Harry diris laŭte.

Folio de papero flugis super sia kapo, kiel se iu ĵetis ĝin de antaŭ li—Harry
ĉirkaŭen turniĝis, sed ne estis iu ajn tie—kaj kiam Harry turniĝis denove, la
noto estis kuŝanta sur la planko.

La noto diris:

\begin{center}
  POENTOJ POR STILO : 10 \\
  POENTOJ POR PENSI BONE : -3,000,000 \\
  KORVUNGAJ BONUSAJ POENTOJ : 70 \\
  AKTUALAJ POENTOJ : -2,999,871\\
  RESTANTAJ VICOJ : 2\\
\end{center}

``\emph{Minus tri milionoj da poentoj?}'' Harry diris indigne al la
malplena koridoro. ``Tio ŝajnas ekscesa! Mi volas reklamacii al la
Estroj de la Ludo! Kaj kiel mi estas supozita fari tri milionojn da
poentoj dum la du sekvaj vicoj?''

Alia noto flugis super sia kapo.

\begin{center}
  REKLAMACIO : MALSUKCESO \\
  DEMANDI LA MALĜUSTAJN DEMANDOJN : -1,000,000,000,000 POENTOJ\\
  AKTUALAJ POENTOJ : -1,000,002,999,871\\
  RESTANTA VICO : 1
\end{center}

Harry rezignis. Kun unu vico restanta, ĉio kion li povis fari estis fari kiel
eble plej bone, eĉ se tio ne estis tre bona. ``Mia konjekto estas ke la ludo
reprezentas la vivon.''

Fina folio de papero flugis super sia kapo, skribanta:
\begin{center}
  PROVO MALSUKCESINTA\\
  MALSUKCESO MALSUKCESO MALSUKCESO\\
  AIIIIIIIIIIEEEEEEEEEEEEEE\\
  AKTUALAJ POENTOJ: MINUS SENFINECO\\
  \textbf{VI PERDIS LA LUDON}\\
  FINA INSTRUKCIO: \emph{iru al la oficejo de Profesorino McGonagall}
\end{center}

La lasta frazo estis skribita per lia propra skribmaniero.

Harry rigardadis fikse la lastan frazon dum momento, poste
levis la ŝultrojn. Finfine. La oficejo de Profesorino McGonagall tio
estos. Se \emph{ŝi} estis la Kontrolanto de la Ludo\ldots

Konsentite, honeste, Harry havis absolute neniun ideon pri kiel li sin sentus se
Profesorino McGonagall estis la Kontrolanto de la Ludo. Lia menso estis nur
desegni kompletan blankaĵon. Tio estis laŭlitere, neimagebla.

Kelkajn pentraĵojn pli malfrue—Tiu ne estis longa vojaĝo, la oficejo de Profesorino McGonagall ne estis for de la kursĉambro de Transfiguriĝo, almenaŭ je lundo de nepara jaro—Harry staris ekstere antaŭ la pordo de la oficejo.

Li frapetis.

``Eniru,'' diris la nesonora voĉo de Profesorino McGonagall.

Li eniris.




