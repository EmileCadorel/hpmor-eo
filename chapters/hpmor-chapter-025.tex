\chapter{Reteni sin proponi solvojn}

\section{Akto 2:\protect\footnotemark}

\authorsnotetext{Pro tio ke la scienco en tiu historio estas ĝenerale
  tute korekta, mi aldonas averton pri la ĉapitroj—22—25, Harry
  malatentis multe da ebloj, la plej grava estante ke estas multe da
  magiaj genoj sed ili ĉiuj estas sur la sama kromosomo (kiu ne
  okazintus nature, sed la kromosomo povus esti artefarita). Tial la
  hereda skemo estus Mendela, sed la magia kromosomo povus malgraŭe
  esti denerigita per kromosoma kruciĝo, kun sia ne magia
  homologa. (Harry legis pri Mendel kaj kromosomoj en libroj pri
  historio de scienco, sed li ne studis sufiĉe veran genetikon por
  koni kiel kromosoma kruciĝo fakte funkciis. He, li nur havas dek unu
  jarojn.) Malgraŭe, eĉ se moderna ĵurnalo de scienco trovus multajn
  problemojn, ĉio, kion Harry prezentis kiel fortaj atestoj, estis fakte
  fortaj atestoj—— ĉiuj aliaj ebloj estis \emph{neverŝajnaj}.}

(La suno brilegis en la Granda Ĉambrego de la ensorĉataj ĉielo supre,
iluminante la studentojn kvazaŭ ili sidis sub la vera ĉielo sennuba
kaj glimanta de iliaj teleroj kaj bovloj, dum refreŝigitaj per nokto
da dormo, ili englutis matenmanĝon en preparado de iu ajn plano,
kiun ili havis por ilia dimanĉo.)

\lettrine{D}o. Estis nur unu aĵo, kiu faris ke vi estas
sorĉisto. Tio ne estis surpriza, kiam vi pensis pri tio. Tio, kion DNA
faris, estis diri al ribosomoj kiel ĉeni acidoj kune en
proteinojn. Konvencia fiziko ŝajnis tute kapabla priskribi aminajn
acidojn, kaj ne gravas kiom multe da aminaj acidoj oni ĉenis kune,
konvencia fiziko diris ke oni neniam povu akiri magion de ĝi.

Kaj tamen, magio ŝajnis esti hereda, sekvante DNA. 

Tiam tio verŝajne \emph{ne estis} ĉar la DNA estis ĉeninta kune
nemagiaj aminaj acidojn en magiajn proteinojn.

Pli probable, la DNA sekvencon, per ĝi mem, ne donis al vi magion,
tute ne.

Magio venis de ie alia.

(Je la tablo de Korvungo, estis unu knabiĉo rigardadanta en la aero,
dum sia dekstra mano aŭtomate metis per kulero negravan manĝaĵon en la
buŝon de kio ajn estis antaŭ li. Vi probable povintus substitui per
stako da koto, kaj li ne rimarkintus.)

Kaj por ia kialo, la fonto de magio atentis al partikulara marko de
DNA, inter la homoj, kiuj estis ordinaraj posteuloj de simioj laŭ ciuaj
aliaj flankoj.

(Fakte estis relative multe da knabiĉoj kaj knabinoj rigardadanta en
la aero. Tio ja estis la Korvunga tablo.)

Estis aliaj linioj de logiko kondukantaj al la sama
konkludo. \emph{Kompleksa} maŝinaro estis ĉiam universala en seksue
reproduktanta speco. Se geno B dependis de geno A, tiam A bezonis esti
utila per si mem, kaj iĝos al preskaŭ universala per ĝi mem, antaŭ ol
B fariĝus sufiĉe kutima por doni taŭgeca advantaĝo. Tiam kiam B estas
universala, aparos variaĵo A* kiu dependos de B, kaj tiam C kiu
dependos de A* kaj B, tiam B* kiu dependos de C, ĝis la tuta maŝino
falus se vi deprenis nur unu solan parton. Sed bezonis ke tio okazu
\emph{kremente}— evoluado neniam rigardis anticipe, evoluado neniam
komencis promocii B'n en \emph{preparo} por ke A fariĝu universala pli
malfrue. Evoluo estis la simpla historia fakto ke la genoj de kio ajn
organismo, kies kvanto da infanoj estis la plej granda, estas fakte
pli oftaj en la sekva generacio. Do ĉiuj partoj de kompleksa maŝinaro
devas fariĝi preskaŭ universala antaŭ ol alia parto en la maŝinaro
evoluas por dependi de ĝia ĉeesto.

Do \emph{kompleksa, interdependanta} maŝinaro, la potenca eleganta
maŝinaro de proteino kiu gvidis vivon, estis ĉiam \emph{universala}
inter seksue reproduktanta speco—krom por kelkaj etaj
\emph{ne}interdependantaj \emph{variaĵoj}, kiu estis elektitaj, dum
plua komplekso estis malrapide konstruiĝanta. Tio estis kial homoj
havis la saman fundamentan desegnon de cerbo, la samajn emociojn la
samajn mimikojn por esprimi tiujn emociojn; tiuj adaptoj estis tro
kompleksaj, do ili devis esti universalaj.

Se magio estis kiel tio, granda adapto kompleksa kun multaj genoj
necesaj, tiam sorĉisto kopulanta kun Muglo rezultus en infano kun nur
duono de tiuj partoj kaj duono de la maŝinaro ne farintus multe. Kaj
tiel neniam estus Mugle naskita ulo. Eĉ se ĉiuj la partoj eniris
unuope la genojn de la muglo, ili neniam rekunmetiĝis en la formon
necesa por farigi sorĉiston.

Ne estis ia genetike izolita valo da homoj, kiu sekvis evoluan vojon
gvidanta al komplikaj magiaj sekcioj de la cerbo. La kompleksa
genetika maŝinaro, se sorĉitoj reproduktus kun Mugloj, neniam
rekunmetiĝus en Mugle naskita.

Do kiel ajn viaj genoj farigis vin esti sorĉisto, tio ne \emph{estis}
per enhavi skemo de kompleksa maŝinaro.

Tio estis la kialo, pro kiu Harry divenis ke la Mendela skemo estis
tie. Se magiaj genoj estis komplikaj, kial estus pli ol unu geno?

Kaj magio mem ŝajnis relative komplika. Pordon ŝlosanta sorĉo
preventus la pordon esti malfermita \emph{kaj} preventis vin de
transfiguri la ĉarninoj \emph{kaj} rezistus al \emph{Finite
  Inkantatem} kaj \emph{Alohomora}. Multaj elementoj indikis la saman
direkton: vi povis nomi tion cela orientiĝo, aŭ en pli simpla
lingvaĵo, celeco.

Estis nur du konitaj kaŭzoj de celeca komplekso. Natura selekto, kiu
kreis aĵojn kiel papilioj. Kaj inteligenta inĝenierio, kiu kreis aĵojn
kiel aŭtoj.

La magio ne aspektis kiel io kio estis mem-replikita en ekzistado. Sorĉoj
estis celece komplika, sed ne kiel papilioj, kiuj estis komplikaj por
ke ili faru kopiojn de ili mem. Sorĉoj esis komplikaj por la celo de
servi ilian uzanton, kiel aŭto.

Iu inteligenta inĝeniero kreis la fonton de la magio, kaj diris al ĝi
ke ĝi atentu al partikulara marko de DNA.

La evidenta sekva penso estis ke tio rilatis al `Atlantido'.

Harry estis demandinta al Hermione pri tio pli frue—en la trajno al
Herpŭrko, post ol li aŭdis Drako paroli pri tio— kaj laŭ tio, kion ŝi
sciis, nenio pli ol la vorto estis konita.

Tio eble povintus esti nura legendo. Sed tio estis ankaŭ sufiĉe
kredinda ke civilizo de magiaj uzantoj, speciale \emph{antaŭ} la
Interdikto de Merlino, sukcesis eksplodigi sin mem.

La linio de rezonado daŭris: Atlantido estis izolita civilizo kiu iel
naskigis la fonton de la magio, kaj diris al ĝi ke ĝi nur servu homoj
kun la Atlantida marko genetika, la sango de Atlantido.

Kaj per simila logiko: la vortoj, kiujn sorĉisto diris, la movoj de la
bastono, tiuj mem ne estis sufiĉe komplikaj por krei la efikojn de la
sorĉoj surbaze de nenio—ne kiel la tri miliardoj da bazaj paroj de
homa DNA fakte \emph{estis} sufiĉe komplikaj por krei homa korpo
surbaze de nenio, aŭ ne kiel la komputilaj programoj estis faritaj el
miloj da bitokoj da datumoj.

Do la vortoj kaj la bastonaj movoj estis nur baskuloj, leviloj de ia
kaŝita kaj pli kompleksa maŝino. Butonoj, ne planoj.

Kaj ekzakte kiel komputila program ne kompilus se vi farus nur unu
solan eraron, la fonto de la magio ne respondus al vi krom se vi ĵetas
viajn sorĉojn ekzakte laŭ la bona maniero.

La ĉeno de logiko estis nehaltigebla.

Kaj ĝi fatale gvidis al sola fina konkludo.

La malnovaj prauloj de la sorĉisto, antaŭ miloj da jaroj, diris al la
fonto de la magio ke ĝi ŝvebu aĵojn nur se vi diras\ldots{}

`Flugapika Leviosu.'

Harry lasis sin fali sur la tablon de matenmanĝo, ripozante sian
frunton en sia dekstra mano.

Estis historio de la aŭroro de la artefarita inteligento—kiam ili nur
komencis kaj neniu jam konsciis ke la problemo estos malfacila—pri
profesoro kiu komisis unu el siaj studentoj la solvon de la problemo
de komputila vido.

Harry komencis kompreni kiel tiu studento devis senti sin.

Tio povas preni tempon.

Kial ĵeti la Alohomora'n sorĉo, prenis pli da klopodo, se tio estis nur
kiel premi butonon?

Kiu estintus sufiĉe stulta por krei la sorĉon \emph{Avada Kedavra},
kiu povis esti ĵetita nur per uzi malamon?

Kial senvorta transfiguro postulis ke vi faru kompletan mensan
apartigon inter la koncepto de formo kaj la koncepto de materio?

Harry eble ne estos fininta kun la problemo antaŭ ol li ricevos la
diplomon de Herpŭrko. Li povas ankoraŭ esti laboranta sur la problemo
kiam li havos \emph{tridek jarojn.} Hermione pravis, Harry \emph{ne}
konsciis tion plej profunde. Li nur faris inspiran paroladon pri
determinismo.

La menso de Harry mallonge konsideris ĉu li devus atendi ke li neniam
solvos la problemon, sed tiam li decidis ke tio estus iri multe tro
for.

Plue, se li povas akiri sufiĉe da senmorteco en la kelkaj unuaj
jardekoj, ĉio estos bone.

Kiun metodon la Mastro de la Tenebroj uzis? Tio farigis lin pensi, ke
la fakto ke la Mastro de la Tenebroj iel sukcesis postvivi la morton
de sia unua korpo estis preskaŭ \emph{senlime} pli grava ol la fakto
ke li provis regi la Magian Brition—

``Pardonu min,'' diris neatendita voĉo de malantaŭ li per tre
neatenditaj tonoj. ``Kiam konvenos al vi, S-iĉo Malfojo petas la
favoron de konversacio.''

Harry ne sufokiĝis pro siaj matenmanĝaj cerealoj. Anstataŭ li turnis
sin ĉirkaŭe kaj rigardis S-iĉon Krabe.

``Pardonu \emph{min,}'' diris Harry. ``Ĉu vi ne volis diri `L'ĉefo volas k'vi parolu kun'li?'\ ''

S-iĉo Krabe ne aspektis feliĉa. ``S-iĉo Malfojo instrukciis al mi paroli konvene.''

``Mi ne povas aŭdi vin,'' Harry diris. ``Vi ne parolas konvene.'' Li
returnis sin al sia bovlo da etaj bluaj neĝeroj kristalaj kaj intence
manĝis alian plenan kuleron.

``L'ĉefo volas k'vi parolu kun'li,'' venis de minacan voĉon malantaŭ
li. ``Vi pli b'ne venu vid' lin se v'scias kion 'stas bona por vi.''

Jen. \emph{Nun} ĉion estis iranta laŭ la plano.

\latersection{Akto 1:}

``Iu \emph{kialo?}'' diris la maljuna sorĉisto. Li retenis la furiozon
iri sur sian vizaĝon. La knabo antaŭ li estis la viktimo, kaj certe ne
bezonis esti timigita plie. ``Estas \emph{nenio}, kio povas ekskuzi—''

``Tio, kion mi faris, estis pli malbona.''

La maljuna sorĉisto rigidiĝis pro subita teruro. ``Harry, \emph{Kion vi faris?}''

``Mi trompis Drako'n por ke li kredu ke mi trompis lin por ke li
partoprenu al ritualo kiu oferis sian kredon en purismo de sango. Kaj
tio volas diri ke li ne povos esti Morto-Manĝanto kiam li kreskos. Li
perdis ĉion, Direktoriĉo.''

Estis longa mallaŭta momento en la oficejo, nur rompita per la etaj
pufoj kaj fajfoj de la bagetelemaj objektoj, kiuj post sufiĉe da tempo
aspektis kiel silentaj.

``Ve,'' diris la maljuna sorĉisto. ``Mi sentas min stulta. Kaj
\emph{jen} Mi atendis ke vi eble provus elaĉeti la idon de Malfojo
per, diru ni, \emph{montri al li veran amikecon kaj afablecon}.''

``\emph{Ha!} Jes, kiel se \emph{tio} funkciintus.''

La maljuna sorĉisto suspiris. Tio iris tro for. ``Diru al mi,
Harry. Ĉu eĉ aperis al vi, ke estas io \emph{bizara} pri plani elaĉeti
iun per mensogaĵoj kaj trompaĵoj?''

``Mi faris tion se diri iun ajn realajn mensogaĵojn, kaj pro tio ke ni
estas parolanta pri Drako Malfojo, mi pensas ke la vorto, kiun vi
serĉas estas \emph{kongrua}.'' La knabo aspektis relative memkontenta.

La maljuna sorĉisto skuis la kapon pro malespero. ``Kaj \emph{tiu}
estas la heroo. Ni estas ĉiuj kondamnitaj.''

\latersection{Akto 5:}

La longa, mallarĝa tunelo el krudaj ŝtonoj, mallumigita krom per la
bastono de knabiĉo, ŝajnis streĉiĝi je meljoj.

La kialo de tio estis simpla: ĝi fakte streĉiĝis je meljoj.

Estis la tria matene, kaj Fred kaj George estis komencanta la longan
vojon al la sekreta trairejo, kiu kondukis de statuo de iu sorĉistino
de Herpŭrko kun unu okulo, al la keloj de la butiko de dolĉaĵo
Mieledukoj en Herbejo de pŭrkoj.

``Kiel statas la afero?'' diris Fred per mallaŭta voĉo.

(Ne ke estis iu aŭskultanta, sed estis io bizara pri paroli kun
normala voĉo kiam vi estis trairanta sekretan trairejon.)

``Ankoraŭ blinkanta,'' diris George.

``Ambaŭ, aŭ—'' 

``La intermita riparis sin mem denove. La alia estas kiel ĉiam.''

La mapo estas ekstraordinare potenca artefakto, kiu kapablis spuri
ĉiujn konsciajn estaĵojn en la lernejo, je realtempo, per iliaj
nomoj. Preskaŭ certe, ĝi estis kreita dum la origina kreskigo de
Herpŭrko. Estis \emph{ne bone} ke eraroj komencis aperi. Estis ŝancoj
ke neniu krom Dumbledore povus ripari ĝin se ĝi estus rompita.

Kaj le Wizle ĝemeloj ne donos la mapon al Dumbledore. Tio estus
nepardonebla insulto al la Marodantoj—la kvar nekonatoj kiuj sukcesis
ŝteli parton de la \emph{sekura sistemo de Herpŭrko}, io verŝajne
aldonita per Salazaro Serpentimo li mem, kaj misuzita kiel \emph{ilo
  por studenta petolaĵoj}.

Iuj eble trovus tion malrespekta.

Iuj eble trovus tion krima.

La Wizle ĝemeloj firme kredis ke se Godriko Grifindoro estintus tie
por vidi, li aprobintus.

La fratiĉoj marŝis plu kaj plu kaj plu, plejparte silente. La Wizle
ĝemeloj parolis unu al la alia kiam ili pensis pri novaj petolaĵoj, aŭ
kiam unu el ili sciis ion ke la alia ne sciis. Alie, ne estis multe da
kialo. Se ili jam sciis la samajn informojn, ili emis havi la samajn
pensojn kaj preni la samajn decidojn.

(Antaŭ longe, ĉiam kiam magiaj similaj ĝemeloj naskis, oni kutimis
mortigi unu el ili post naskiĝo.)

Post tempo, Fred kaj George suprenrampis en la polvoplenan kelon, kiu
estis kovrita per bareloj kaj bretaroj de strangaj ingrediencoj.

Fred kaj George atendis. Ne estus bonmaniere fari ion ajn alian.

Baldaŭ, maldika maljuna viriĉo en nigra piĵamo malsupreniris la
ŝtuparon kiu gvidis al la kelo, oscedante. ``Saluton, knabiĉoj,''
diris Ambrozio Kanalo. ``Mi ne atendis vin ĉi-vespere. stoko jam
malplena?''

Fred kaj George decidis ke Fred parolos.

``Ne vere, S-iĉo Kanalo,'' diris Fred. ``Ni esperis ke vi povu helpi
nin kun io konsiderinde pli\ldots{}interesa.''

``Nun, knabiĉoj,'' diris Kanalo sonante severa. ``Mi esperas ke vi ne
vekis min nur por ke mi denove diru al vi, ke mi ne vendos al vi iun
ajn komercaĵon kiu povas akiri problemojn al vi. Ne antaŭ ol vi havas
dekses jarojn, ĉiaokaze—''

George elmetis objekton el sia robo, kaj senvorte pasis ĝin al
Kanalo. ``Ĉu vi vidis tion?'' diris Fred.

Kanalo rigardis la numero de la \emph{Ĉiutaga Profeto} de hieraŭ, kaj
kapjesis, sulkante la fronton. La ĉeftitolo de la ĵurnalo estis ``LA
SEKVA MASTRO DE LA TENEBROJ?'' kaj elmontris junan knabiĉon, kiun
studenta kamerao sukcesis kapti en malkarakteriza malvarma kaj minaca
esprimo.

``Mi ne povas kredi tiun Malfojo'n,'' Kanalo diris klake. ``Komenci
batalon kun la knabo, kiam li nur havas dekunu jarojn! La viriĉo devus
esti disŝirita kaj uzita por fari ĉokoladon!''

Fred kaj George palpebrumis samtempe. \emph{Malfojo} estis malantaŭ
Rita Moskito? Harry Potter ne avertis ilin pri tio\ldots{}kio certe
volis diri ke Harry Potter ne sciis. Li neniam metus ilin en tio se li
scius\ldots{}

Fred kaj George interŝanĝis rigardon. Nu, Harry ne bezonis scii antaŭ
ol la laboro estu finita.

``S-iĉo Kanalo,'' Fred diris kviete, ``la knabo-Piu-Postvivis bezonas
vian helpon.''

Kanalo rigardis ilian ambaŭ.

Tiam li lasis sian spiron eliri per sopiro.

``Bone,'' diris Kanalo, ``Kion vi volas?''

\latersection{Akto 6:}

Kiam Rita Moskito intencis kapti bongustan predon, ŝi ne emis rimarki
la kurantajn formikojn, kiuj konstituis la ceteron de la universo, kaj
tio estis kiel ŝi presakŭ trafis en kalva juna viriĉo, kiu paŝis
barante ŝian voĉon.

``S-ino Moskito,'' diris la viriĉo, sonante relative severa kaj
malvarma por iu kun vizaĝo kiu aspektis tiel juna. ``Estas amuze ke mi
renkontas vin tie.''

``El mia vojo, kanajlo!'' diris klake Rita, kaj ŝi provis paŝi ĉirkaŭ li.

La viriĉo sur ŝia vojo imitis la movon tiel perfekte ke estis kvazaŭ
neniu el ili estis movinta, kaj ke ili nur staris senmove dum la
strato glitis ĉirkaŭ ili.

La okuloj de Rita mallarĝiĝis. ``Kiun vi pensas esti?''

``Kiel malsaĝa,'' la viriĉo diris seke. ``Estintus saĝa memori la
vizaĝo de la maskita Morto-Manĝanto, kiu trejnas Harry'n Potter por ke
li estu la sekva Mastro de la Tenebroj. Finfine,'' mallarĝa rideto,
``\emph{tio} certe aspektas kiel iu, kiun vi ne volas hazarde renkonti
en la strato, speciale post ol vi deklaris militon al li en la
ĵurnalo.''

Rita prenis momenton por trovi la referencon. \emph{Tiu} estis Cirinus
Ciuro? Li aspektis tro juna kaj tro maljuna samtempe; lia vizaĝo, se
ĝi trankviliĝus de sia sevara kaj aronganta pozo, apartenus al iu
havanta malpli ol kvardek jarojn. Kaj liaj haroj estis jam falantaj? Ĉu
li ne povis permesi sin pagi kuraciston?

Ne, tio ne gravis, ŝi devis iri ien senprokraste kaj skarabe. Ŝi ĵus
ricevis anoniman konsileton pri S-ino Osto pasante tempon kun unu el
siaj plej junaj asistantoj. Tio valorus relative grandan bonifikon, se
ŝi povus sukcesi kontroli tion. Osto estis alte en la listo de
personoj por bati. La konsilinto diris ke Osto kaj sia juna asistanto
devis tagmanĝi en la speciala ĉambro de Mari, tre populara ĉambro por
iaj celoj; ĉambro kiu, laŭ tio kion ŝi malkovris, estis sekura kontraŭ
ĉiujn aŭskultantaj iloj, sed ne kontraŭ belega blua skarabo kuŝita
kontraŭ unu el la muroj\ldots{}

``El mia \emph{vojo!}'' Rita diris, kaj puŝi Ciuro'n el sia vojo. La
brako de Ciuro depuŝis la ŝian, flankeiĝante, kaj Rita ŝanceliĝis dum
ŝi puŝis en vakuo.

Ciuro tiris la manikon de sia maldekstra brako, montrante sian
maldekstran brakon. ``Observu,'' diris Ciuro. ``neniu malhela
marko. Mi ŝatus ke via ĵurnalo publikas kontraŭdiron.''

Rita lasis eliri nekredeman ridon. Evidente, la viriĉo ne estis vera
Morto-Manĝanto. La artikolo ne estintus publikigita se tio estus
vera. ``Forgesu tion, kanajlo. Nun foriru.''

Ciuro rigardis ŝin fikse dum momento.

Tiam li ridetis.

``S-ino Moskito,'' diris Ciuro, ``Mi esperis trovi ian levilon kiun
konvinkus vin. Jen mi malkovras ke mi ne povas nei ke mi havos
plezuron per simple frakasi vin.''

``Oni jam provis. Nun el mia vojo, kanajlo, aŭ mi serĉos kelkajn
Aŭrorojn por ke ili arestu vin pro obstrukco al ĵurnalismo.''

Ciuro faris etan riverencon al ŝi, kaj poste marŝis for. ``Adiaŭ, Rita
Moskito,'' diris la voĉo de malantaŭ ŝi.

Dum Rita tiris sin antaŭen, ŝi notis en la fundo de sia menso, ke la
viriĉo estis fajfanta melodion marŝante.

Kiel se \emph{tio} timigus ŝin.

\latersection{Akto 4:}

``Mi pardonpetas, ne kalkulu min,'' diris Lee Jordan. ``Mi preferas
la tipon de granda araneo.''

La Knabo-Kiu-Postvivis estis dirinta ke li havis \emph{gravan} laboron
por la Ordeno de la Ĥaoso, io serioza kaj sekreta, pli signifoplena
kaj malfacila ol iliaj kutimaj petolaĵoj.

Kaj tiam Harry Potter komencis parolado kiu estis inspira kvankam
svaga. Parolado pri ke Fred kaj George kaj Lee havis kolocan
potencialon, kiun ili povus uzi se ili nur lernis kiel esti pli
\emph{stranga.} Por farigi la vivon de homoj \emph{nereala,} anstataŭ
nur surprizi ilin kun la ekvivalento de sitelo de akvo pendigita super
pordo. (Fred kaj George interŝanĝis interistajn rigardojn, ili neniam
pensis pri tiu ĉi.) Harry Potter priparolis pri la petolaĵoj, kiujn
ili faris al Nevilo—kiujn, Harry menciis kun kelkaj bedaŭroj, la
Ordiganta Ĉapelo al li riproĉis—sed kiuj devis farigi Nevilo'n dubi
pri \emph{sia propra sano}. Por Nevilo, tio devis sentigi lin kiel
esti subite transportita en alternativan universon. Samemaniere ol
ĉiuj devis senti sin, kiam ili vidis Skoldo'n pardonpeti. Tio estis la
\emph{vera povo de petolado}.

\emph{Ĉu vi estas kun mi?} Harry Potter kriis, kaj Lee Jordan
respondis ke ne.

``Kalkulu nin \emph{ene},'' diris Fred kaj eble George, pro tio ke ne
estis duboj pri ke Godriko Grifindoro estus dirinta jes.

Lee Jordan donis bedaŭranta rideton, stariĝis kaj eliris la dezertan
kaj Silenciuitan koridoron, en kiu la kvar membroj de la Ordeno de la
Ĥaoso kunvenis kaj sidiĝis en konspira cirklo.

La tri membroj de la Ordeno de la Ĥaoso komencis paroli pri la grava
afero.

(Tio ne estis \emph{tiel} malgaja. Fred kaj George ankoraŭ laboros kun
Lee pri petolaĵoj kun grandaj araneoj, kiel kutime. Ili nur komencis
nomi ĝin la Ordeno de la Ĥaoso por varbi Harry'n Potter, post ol Ron
diris al ili pri Harry estante stranga kaj malbona, tiam Fred kaj
George decidis savi Harry'n per montri al li veran amikecon kaj
afablecon. Danke, tio ne ŝajnis ankoraŭ necesa—tamen ili ne tute
certis pri tio\ldots{})

``Do,'' diris unu el la ĝemeloj, ``Temas pri kio?''

``Rita Moskito,'' diris Harry. ``Ĉu vi scias kiun ŝi estas?''

Fred kaj George kapjesis malridetante.

``Ŝi demandis aĵojn pri mi.''

Tiu ne estis bona novaĵo.

``Ĉu vi povas diveni tion, kion mi volas fari?''

Fred kaj George rigardis unu la alian, iom konfuzita. ``Vi volas
manĝigi al ŝi kelkajn el niaj plej interesaj bombonoj?''

``Ne,'' diris Harry. ``Ne, ne, \emph{ne!} tio estas pensi kiel
petolaĵo kun granda araneo! Nu, kion \emph{vi} farus se vi aŭdus ke
Rita Moskito estis serĉanta famojn pri \emph{vi?}''

Tio farigis ĝin evidenta.

Ridetoj malrapide aperis sur la vizaĝoj de Fred kaj George.

``Komencu famojn per ni mem,'' ili respondis.

``\emph{Ekzakte,}'' diris Harry, ridetante larĝe. ``Sed tio ne povas
esti iu ajn famo. Mi volas instrui al homoj neniam fidi tion, kion
ĵurnaloj diras pri Harry Potter, kiel Mugloj neniam kredas tion, kion
ĵurnaloj diras pri Elvis. Unue mi pensis pri inundis Rita'n Moskito
kun tiel eble plej multe da famoj, por ke ŝi ne sciu kiun kredi, sed
tiel ŝi nur elektos la unu kiu sonas la plej kredinda kaj malbona. Do
tio, kion mi volas fari estas krei falsan historion pri mi, kaj fari
ke Rita Moskito kredu ĝin iel. Sed tio devas esti io, kion poste, ĉiuj
\emph{scios} esti falsa. Ni volas trompi Rita'n Moskito kaj ŝiaj
editoroj, kaj \emph{poste} havi pruvon kiu montros ke ĝi estis
falsa. Kaj evidente—ĉar tiuj estas la postuloj—la historio devas esti
tiel eble plej \emph{ridinda}, kaj malgraŭe esti presita. Ĉu vi
komprenas tion, kion vi volas fari?''


``Ne tute\ldots{}'' Fred aŭ George diris malrapide. ``Vi volas ke ni
\emph{inventu} la historion?''

``Mi volas ke vi faru \emph{ĉion},'' Harry Potter diris. ``Mi estas
relative okupata nuntempe, kaj mi volas kapabli diri sincere ke mi
havas nenion ideon pri kio okazis. Surprizu min.''

Dum momento estis malbona rideto sur la vizaĝo de Fred kaj George.

Poste ili refariĝis serioza. ``Sed Harry, ni ne vere scias kiel fari ion kiel tio—'' 

``Do trovu kiel fari,'' Harry diris. ``Mi havas fidon al vi. Ne
\emph{totala} fido, sed se vi \emph{ne povas} fari ĝin, \emph{diru}
tion al mi, kaj mi provos kun iu alia, aŭ mi faros ĝin per mi meme. Se
vi havas tre bonan ideon—por ambaŭ la ridinda historio kaj kiel
konvinki Rita'n Moskito kaj ŝiaj editoroj por ke ili printu ĝin—tiam
iru kaj faru. Sed ne faru ion mezbona. Se vi ne povas trovi ion
imponegan, nur diru tion.''

Fred kaj George interŝanĝis zorgplenajn rigardojn.

``Mi ne povas pensi al io ajn,'' diris George.

``Mi ankaŭ ne,'' diris Fred. ``Ni pardonpetas.''

Harry fiksrigardis ilin.

Kaj poste li komencis klarigi kiel oni komencas por trovi ideojn.

Tio estis konata por preni pli ol du sekundoj, diris Harry.

Vi \emph{neniam} nomus \emph{iun ajn} demandojn nefarebla, diris
Harry, antaŭ ol vi prenis reala horloĝo kaj pensis pri la problemo dum
kvin minutoj, per realaj minutoj. Ne kvin metaforaj minutoj, sed kvin
fizikaj minutoj sur fizika horloĝo.

Kaj \emph{plie}, Harry diris, kun la voĉo insista kaj la deskstra mano
frapanta forte la plankon, vi \emph{ne} komencas tuj serĉi solvojn.

Harry tiam iniciatis klarigon de testo farita per iu nomita Norman
Maier, kiu estis psikologio de organizacio, kaj kiu demandis al du
aroj da homoj de solvantoj de problemoj, problemon konsideri.


La problemo, Harry diris, koncernis tri laboristoj, kiuj devis plenumi
tri taskojn. La debutanta laboristo nur volis fari la plej facilan
laboron. La emerita laboristo volis alterni inter la taskojn, por
eviti enuon. Fakulo de efikeco rekomendis doni al la debutanta persono
la plej facilan taskon kaj al la emerita persono la plej malfacilan,
tio estintus 20\% pli efika.

Oni donis al \emph{unu} aro de la solvantoj de problemoj la
instrukcion ``Ne proponu solvon antaŭ ol la problemo estas diskutita
kiel eble plej plene sen sugesti iun ajn solvon.''

Oni donis neniun instrukcion al la alia aro. Kaj tiuj homoj faris la
naturan aĵon, kaj reagis al la ĉeesto de problemo per proponi
solvojn. Kaj la homoj sin alligis al tiuj solvoj, kaj komencis batali
pri ili, kaj argumentis pri la relativa graveco de la libereco kontraŭ
la efikeco kaj tiel plu.

La unua aro da solvantoj de problemoj, la unu al kiu oni donis la
instrukcion de diskuti la problemon unue kaj poste solvi ĝin, proponis
probable multe pli efikan solvon per lasi la debutantan laboriston
gardi la plej facilan taskon kaj rotacii la aliajn du homoj inter la
du aliaj taskoj, kaj la datumoj de la fakulo diris ke tiu solvo estis
pli efika de 19\%.

Komenci per serĉi solvojn estis preni la aĵojn \emph{tute laŭ erara
  ordo.} Kiel komenci manĝadon per la deserto, nur \emph{malbona.}

(Harry ankaŭ citis iun nomita Robyn Dawes, kiu diris ke ju pli la
problemo estis malfacila, des pli la homoj probable provis solvi ĝin
senprokraste.)

Do Harry lasos la problemon al Fred kaj George, kaj ili diskutos ĉiujn
la aspektojn de ĝi kaj pripensados al ĉiuj, kiuj laŭ iliaj oponioj estas
gravaj. Kaj ili ne provos trovi iun solvon antaŭ ol ili estos fininta
fari tion, almenaŭ evidente ke ili hazarde pensus pri ion bonegan, en
tiu okazo ili povos skribi ĝin sur paperon antaŭ ol ili daŭros ilian
pripensadon. Kaj li ne volis aŭdi ilin pri iu ajn nomita \emph{malsukceso
  pri pensi al io ajn} dum tuta semajno. Kelkaj homoj prenis
\emph{jardekojn} por pripensi pri aĵoj.

``Demandoj?'' diris Harry.

Fred kaj George fiksrigardis unu la alian.

``Mi ne povas pensi pri iu ajn.''

``Mi ankaŭ ne.''

Harry tusis neforte. ``Vi ne demandis pri buĝeto.''

\emph{Buĝeto?} ili pensis.

``Mi povas min kontenti per diri al vi la kvanto,'' Harry diris. ``Sed mi pensas ke \emph{tio} estos pli \emph{inspira}.''

La manoj de Harry plonĝis en lia robo, kaj levis—  

Fred kaj George preskaŭ falis, eĉ se ili sidis.

``Ne elspezos ĝin por la amo de elspezi,'' Harry diris. Sur la ŝtona
planko antaŭ ili glimis tute ridinda kvanto da moneroj. ``Nur elspezos
ĝin se la imponegeco postulas ke vi elspezu ĝin; kaj tion, kion la
imponegeco postulas, ne hezitu elspezi. Se estos restantaĵo, nur
redonu ĝin poste, mi fidas vin. Ho, kaj vi akiros procenton de tio,
kio estas tie, ne gravas kiom multe vi elspezos—''

``Ni \emph{ne povas!}'' elbabilis unu el la ĝemeloj. ``Ne ne akceptas monon por tiaj aĵoj!''

(La ĝemeloj neniam prenis monon por fari ion kontraŭleĝan. Nekonate de
Ambrozio Kanalo, ili vendis ĉiujn el liaj komercaĵo kun marĝeno de
nul-elcento. Fred kaj George volis kapabli atesti-sub Verec-serumo se
necesas—ke ili ne estis profitantaj krimuloj, kaj ke ili nur provizis
publikan servon.)

Harry malridetis. ``Sed mi demandis al vi fari realan
laboron. Plenkreskulo estus pagito por fari ion kiel tio, kaj tio
ankoraŭ kalkulus kiel favoro por amiko. Vi ne nur povas dungi homojn
por tiuj specoj de aĵoj.''

Fred kaj George skuis la kapon.

``Bone,'' Harry diris. ``Mi nur akiros al vi multekostan donacon de
kristnasko, kaj se vi provos redoni ilin, mi nur brulos ilin. Tiele vi
ne eĉ scios kiom multe mi elspezos por vi, krom evidente, ke tio estos
pli ol se vi nur prenas la monon. Kaj mi aĉetos tiujn donacojn por vi
\emph{ĉiaokaze}, do pensu pri tio antaŭ ol vi diros al mi ke \emph{vi
  ne povas pensi pri io ajn imponega}.''

Harry stariĝis, ridetante, kaj turniĝis por foriri dum Fred kaj George ankoraŭ estis gapante pro ŝoko. Li faris kelkajn paŝojn for, kaj tiam returnis sin.

``Ho, unu lasta aĵo,'' Harry diris. ``Lasu Profesoriĉon Ciuro el tio,
kion ajn vi faros. li ne ŝatas publikeco. Mi scias ke tio estus pli
facila konvinki homojn por ke ili pensu strangajn aĵojn pri la
Profesoro de Defenso ol io ajn alia, kaj mi pardonpetas pro ke mi
malhelpas vin tiel maniere, sed bonvolu, lasu Profesoriĉon Ciuro el tio.''

Kaj Harry returnis sin denove kaj faris kelkajn paŝojn— 

Rigardis malantaŭen unu lasta fojo, kaj diris dolĉe, ``Dankon''.

Kaj foriris.

Estis longa paŭzo post kiam li estis foririnta.

``Do,'' diris iu.

``Do,'' diris la alia.

``La Profesoro de Defenso ne ŝatas publikecon, ĉu ne.''

``Harry ne konas nin tre bone, ĉu ne.''

``Ne, efektive.''

``Sed ni ne uzis lian monon por tio, evidente.''

``Evidente ne, tio ne estos bona. Ni faros ion por la Profesoriĉo de Defenso aparte.''

``Ni demandos al kelkaj Grifindoroj por ke ili skribu al Moskito, kaj diru\ldots{}''

``\ldots{}ke lia maniko leviĝis dum la klaso de Defenso, kaj ili vidis la
Marko de Tenebroj\ldots{}''

``\ldots{}kaj ke ili verŝajne instruas al Harry Potter multajn
terurajn aĵojn\ldots{}''

``\ldots{}kaj ke li estas la plej malbona Profesoro de Defenso, kiun
iu ajn iam konis en Herpŭrko, ke li ne nur malsukcesas instrui al ni,
sed ke ankaŭ ĉio, kion li faras estas malbona, komplete kontraŭa al
tio, kio devus esti\ldots{}''

``\ldots{}kiel kiam li atestis ke oni nur povas ĵeti la Mortigan
Malbenon uzante amon, kio farigas ĝin relative senutila.''

``Mi ŝatas ĉi tiun.''

``Dankon.''

``Mi vetas ke la Profesoriĉo de Defenso ŝatos ĝin ankaŭ.''

``Li havas humoron. Li ne nomintus nin kiel li faris se li ne havus
humoron.''

``Sed ĉu ni vere kapablos fari la taskon de Harry?''

``Harry diris ke ni diskutu la problemon antaŭ ol provi solvi ĝin, do
diskutu ni ĝin.''

La Wizle ĝemeloj decidis ke George estos tiu, kiu estos entuziasma kaj
ke Fred estos tiu, kiu dubos.

``Ĉio ŝajnas konflikta,'' diris Fred. ``Li volas ke tio estas sufiĉe
ridinda por ke ĉiuj moku Moskito'n kaj scias ke tio estas malvera, sed
li volas ke Moskito kredas ĝin. Ni ne povas fari la du aĵojn
samtempe.''


``Ni devos falsi pruvojn por konvinki Moskito,'' diris George.

``Ĉu tio estas solvo?'' diris Fred.

Ili pripensis pri tio.

``Eble,'' diris George, ``sed mi ne pensas ke ni devus esti tiel
strikta pri tio, ĉu vi ankaŭ?''

La ĝemeloj levis la ŝultrojn senpove.

``Do tiam la falso de la pruvojn devas esti sufiĉe bona por konvinki Moskito'n,'' diris Fred. ``Ĉu ne vere povas fari tion per ni mem?''

``Ni ne bezonas fari ĝin per ni mem,'' diris George, montrante la
stakon de moneroj. ``Ni povas dungi iun por helpi nin.''

La ĝemeloj akiris pensema aspekto sur iliaj vizaĝoj.

``Tio povas uzi la buĝeto de Harry relative rapide,'' diris
Fred. ``Tio estas multe da mono por ni, sed tio ne estas multe da mono
por iu kiel Kanalo.''

``Eble homoj donos al ni rabaton, se ili scias ke tio estas por
Harry,'' diris George. ``Sed la plej grava, kion ajn ni faros, ĝi
devos esti \emph{malebla}.''

Fred palpebrumis. ``Kion vi volas diri per \emph{malebla?}''

``Tiel malebla ke ni ne akiros problemojn, ĉar neniu kredos ke ni
povintus fari ĝin. Tiel malebla ke eĉ Harry komencus sin demandi. Tio
devas esti superreala, tio devas farigi homojn dubi pri ilia propra
sano, tio devas esti\ldots{}\emph{pli bona ol tio, kion Harry
  farus.}''

La okuloj de Fred estis larĝege malfermita pro mirego. Tio okazis
kelkfoje inter ili, sed ne ofte. ``Sed kial?''

``Ili estis petolaĵoj. Ĉiuj estis petolaĵoj. La torto estis
petolaĵo. La Memoprilko estis petolaĵo. La kato de Kevin Forkfajfo
estis petolaĵo. \emph{Skoldo} estis petolaĵo. \emph{Ni estis} la plej
bonaj petolantoj de Herpŭrko, ĉu ni nur metos nin flanke kaj rezignos
sen batali?''

``Li estas la Knabo-Kiu-Postvivis,'' diris Fred.

``Kaj \emph{ni estas} la Wizle ĝemeloj! Li \emph{defias} nin. Li diras
ke ni povas fari tion, kion li faras. Sed mi vetas ke li ne pensas ke
ni estos iam tiel bona kiel \emph{li}.''

``Li pravas,'' diris Fred, sin sentante relative nervoza. La Wizle
ĝemeloj \emph{kelkfoje} malkonsentis kiam ili havis la samajn
informojn, sed ĉiam tio ŝajnis malnatura, kvazaŭ almenaŭ unu el ili
faris malbonan aĵon. ``Oni parolas pri \emph{Harry Potter}. Li povas
fari la maleblon. Ni ne povas.''

``Jes, ni povas,'' diris George. ``Kaj ni devos fari tion ankoraŭ pli
malebla ol tio, kion li farus.''

``Sed—'' diris Fred.  

``Estas tio, kion Godriko Grifindoro farus,'' diris George.

Tio decidis ĝin, kaj la ĝemeloj returnis al\ldots{}io, kio estis
normala por ili.

``Konsentite, tial—''

``—pensu ni pri tio.''

%  LocalWords:  wid youse Ya’d Honeydukes Ambrosius Entwhistle’s
