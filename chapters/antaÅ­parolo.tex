\chapter*{Antaŭparolo}

\emph{Harry Potter kaj la Metodoj de la Racieco} estas fervorula fikcio
bazita sur la libroj de J.K. Rowling, kiu prezentas rakonton en kiu
Harry Potter estis edukita per Profesoro de universitato, kaj estis
instruita pri la scienca metodo. Tiu libro lokas la rakonton en distra
kaj amuza kunteksto, por prezenti realajn kaj interesajn sciencajn
faktojn. Per sia gajiga aspekto, ĉi tiu libro sukcesas esti iu tre
bona enkonduko al la metodoj de la racieco, kaj mi tre rekomendas
ĝin.

\begin{center}\rule{3in}{0.4pt}\end{center}

Kiam mi komencis la tradukon de ĉi tiu libro, mi malkovris ke kiel ne
estis oficiala traduko de la universo de J.K. Rowling en Esperanto, mi
povis fari ĉion, kion mi volis. Mi unue legis la unuan libron de Harry
Potter en la franca, en kiu la nomoj de la roluloj estis tradukitaj en
la franca por permesi al la legantojn kompreni la vortludoj kiujn
J.K. Rowling kreis. Mi opinias ke tio estas bona ideo, ĉar la
legantoj, kiuj ne konas la angla povas tiel pli ŝati la riĉecon de la
universo, en kiu la nomoj preskaŭ ĉiam priskribas la karakterojn de la
rolulojn, kiujn ili nomas.

Mi povas kompreni, ke pro tio ke la libro \emph{Harry Potter kaj la
Metodoj de la Racieco} ne ĉiam prenas la tempon por prezenti la
rolulojn, kaj kalkulas ke la legantojn jam legis la originan verkon,
ne estas ĉiam facila scii kiuj ili estas. Kaj aldone, neniu legis
la originan libron Harry Potter en la esperanta. Por solvi tiun
problemon, mi aldonis je la fino de ĉi tiu libro tabelon, kiu listigas
la rolulojn ĉeestantajn en la libro kaj iliajn originajn nomojn en la
angla.