\chapter{Indentigi la Hipotezon}

\begin{center}\rule{3in}{0.4pt}\end{center}

\emph{``Vi komencas vidi la skemon, aŭdi la ritmon de la universo.''}

\begin{center}\rule{3in}{0.4pt}\end{center}

Ĵaŭdo.

Se oni volis esti preciza, ĵaŭdo je la 7:24 antaŭtagmeze.

Harry sidis sur sian liton, kun libro kuŝante malrigide en liaj
sengestaj manoj.

Harry ĵus havis ideon por \emph{vere brila} testo eksperimenta.

Tio volis diri atendi unu horo plu por matenmanĝo, sed tio estis kial
li havis cerealajn stangojn. Ne, la ideo absolute definitive devis
esti testita tuj, senprokraste, nun.

Harry metis la libron aparte, saltis eksteren de sia lito, kuris
ĉirkaŭ la lito, tiris la kavernan nivelon de sia trunko, kuris
malsupren la ŝtuparon, kaj komencis movi la stakolojn kaj librojn. (Li
vere bezonis elpaki kaj akiri libroŝrankojn iam, sed li estis en la
mezo de kontesto de lego kun Hermione kaj li estis perdanta do li ne
havas tempon.)

Harry trovis la libron, kiun li volis kaj reiris supren per la ŝtuparon.

La aliaj knaboj estis preparanta por iri malsupren al matenmanĝo en la
Granda Ĉambrego kaj komenci la tagon.

``Pardonu min, Ĉu vi povas fari ion por mi?'' diris Harry. Li estis
turnanta la paĝojn de la indekso de la libro dum li parolis, trovis la
paĝon kun la unuaj dek mil primaj nombroj, iris al tiu ĉi paĝo, kaj
donis la libron al Antonio Orŝtano. ``Elektu du nombrojn el tri
ciferaj en la listo. Ne diru al mi kiujn. Nur multipliku ilin kune kaj
diru la rezulton al mi. Ho, ĉu vi povas fari la multiplikon dufoje por
rekontroli? Bonvolu certigi ke vi havas la ĝustan rezulton, mi ne
certas kion okazos al mi aŭ al la universo se vi faras multiplikan
eraron.''

Tio diris multe pri kiel la vivo estis en tiu ĉi dormejo dum la
estintaj kelkaj tagoj, kaj Antonio ne ĝenis sin per demandi ion ajn
kiel ``Kial vi subite freneziĝis?'' aŭ ``Tio ŝajnas vere stranga, kiuj
estas viaj motivoj por demandi?'' aŭ ``Kio vi volas diri per vi ne
certas kio okazos al la universo?''

Antonio senvorte akceptis la libron kaj elprenis pergamenon kaj
plumon. Harry turniĝis malantaŭen kaj fermis siajn okulojn, certigante
ke li ne vidas ion ajn, dancante tien kaj reen kaj saltante surloke
pro malpacienco. Li kaptis paperblokon kaj mekanikan krajonon kaj
preparis sin por skribi.

``Okej,'' Antonio diris, ``Unu cent okdek-unu mil, kvar cent
dudek-naŭ.''

Harry skribis 181,429. Li repetis kion li ĵus skribis, kaj Antonio konfirmis ĝin.

Poste Harry kuris malsupren en la kavernan nivelon de sia trunko,
ĵetis rigardon al sia brakhorloĝo (la horloĝo diris 4:28, kio volis
diri 7:28) kaj poste fermis la okulojn.

Ĉirkaŭ tridek sekundoj pli malfrue, Harry aŭdis bruojn en la ŝtuparo
sekvitaj per la sono de la kaverna nivelo de la trunko
fermanta. (Harry ne timis sufoki. Aŭtomata Freŝiganta Ĉarmo de Aero
estis parto de tion, kion vi akiris se vi estis preta aĉeti vere bonan
trunkon. Ĉu la magio ne estis mirinda, oni ne bezonis zorgi pri kalkulon
de elektreco.)

Kaj kiam Harry malfermis siajn okulojn, li vidis ekzakte kion li estis
esperanta vidi, faldita peco de papero lasita sur la planko, la donaco
de la estonta li.

Nomu ni tiun ĉi pecon de papero ``Papero 2''.

Harry ŝiris pecon de papero de sian paperblokon.

Nomu ni tiun ĉi pecon ``Papero 1''. Ĝi estas evidente la sama peco de
papero. Vi povus eĉ vidi, se vi rigardus proksime, ke la ĉifonaj eĝoj
estis similaj.

Harry revidis mense la algoritmo, kiun li estis sekvonta.

Se Harry malfermus la Paperon 2 kaj ĝi estus blanka, tial li skribus
``101 x 101'' sur la Papero 1, faldigus ĝin, studus dum unu horo,
reirus en la tempon, faligus la Paperon 1 (kiu per tio iĝos Papero 2),
kaj elirus el la kaverna nivelo por kuniĝus siajn dormeja kunuloj por
matenmanĝo.

Se Harry malfermus la Paperon 2 kaj ĝi havus du nombrojn skribitaj sur
ĝi, Harry multiplikus ilin kune.

Se ilia multipliko egalas 181,429, Harry skribus tiujn nombrojn sur
Papero 1 kaj sendus Papero 1 malantaŭen en la tempo.

Alikaze, Harry aldonus 2 al la nombro je la desktra, kaj skribus la
novan paron de nombroj sur Papero 1. Almenaŭ ke la dekstra nombro iĝos
pli granda ol 997, en kiu kazo Harry aldonus 2 al la nombro je la
maldesktra kaj skribus 101 je la dekstra.

Kaj se Papero 2 dirus 997 x 997, Harry lasus Paperon 1 blanka.

Kio volis diri ke la sola ebla tempa buklo stabila estas tiu en kiu
Papero 2 enhavas la du primajn faktorojn de 181,429.

Se tio funkcius, Harry povus uzi ĝin por trovi ian ajn respondon kiu
estas facila por kontroli sed malfacila por trovi. Li ne simple
montrus ke P=NP se vi havas Tempa Returnilo, tiu ruzo estis pli
generala ol tio. Harry povus uzi ĝin por trovi la kombinaĵon de
kodseruroj, aŭ ĉia pasvorto. Evle eĉ trovi la enirejo de la Ĉambro de
Sekretoj de Serpentimo, se Harry povus trovi ia sistema maniero por
reprezenti ĉiuj lokoj de Herpŭrko. Ĝi estus grandega fraŭdo eĉ por la
normaj fraŭdaj manieroj de Harry.

Harr prenis Papero 2 en sian tremantan manon, kaj malfaldigis ĝin.

Papero 2 diris per leĝere malfirma manskribo:

\medskip
\begin{center}
  NE LUDU KUN LA TEMPO
\end{center}

Harry skribis ``NE LUDU KUN LA TEMPO'' sur Papero 1 per leĝere
malfirma manskribo, faldigis ĝin nete, kaj decidis ke li ne faros iun
ajn plia vere brila eksperimento sur la Tempo ĝis li havas almenaŭ
dek-kvin jaroj. 


Tiel for kiel Harry sciis, tio estis la plej timiga eksperimenta
rezulto en la tuta historio de la scienco.

Estis kelke malfacila por Harry por koncentri sur la lego de sia libro
dum la sekva horo.

Tiel estis kiel la ĵaŭdo de Harry komencis.

\begin{center}\rule{3in}{0.4pt}\end{center}

Ĵaŭdo.

Se oni volis esti preciza, ĵaŭdo je la 3:32 posttagmeze.

Harry kaj ĉiuj la aliaj knaboj de unua jaro estis ekstere sur herba
kampo kun S-ino Brando, starante apud la provizejo de flugantaj
balailoj. La kanbinoj lernus kiel flugi aparte. Ŝajne, pro iu kialo,
knabino ne volis lerni kiel flugi sur flugantaj balailoj en ĉeesto de
knaboj.

Harry estis iom svaga dum la tuta tago. Li nur ne ŝajnis povanta halti
scivoli kiel tiu \emph{speciala} tempa buklo stabila estis elektita el
kio estis, retrospektive, iu relative pli larĝa spaco de eblecoj.

Ankaŭ, serioze, flugantaj balailoj? Ĉu li estis flugonta sur, esence,
simpla linio? Ĉu tio ne estis la sola plej nestabila formo kiun vi
eble povis trovi, krom provi stari sur marmoro globeto?  Kiu elektis
\emph{tiun} desegno por fluganta ilo, el ĉiuj eblecoj? Harry estis
esperanta ke ĝi estis nur vortfiguro, sed ne, ili estis starantaj
antaŭ kio aspektis okulfrape ekzakte kiel ordinaraj lignaj
balailoj. Ĉu iu blokiĝis sur le ideo de balailoj kaj malsukcesis
konsideri iun ajn alian? Tio devis esti tio. Ne estis iu ajn maniero
ke la \emph{optima} desegno por purigi kuirejo kaj flugi estas simila
se vi kreis ilin el nenio.

Tiu estis klara tago kun hela blua ĉielo kaj brila suno kiu estis
petanta por iri en viajn okulojn kaj farigi ke estis neebla vidi, se
oni provis flugi ĉirkaŭ en la ĉielo. La grundo estis agrabla kaj seka,
odorante bakita, kaj iel ŝajnis vere, vere malmola sub la ŝuoj de
Harry.

Harry daŭris memori ke la malplej lerta el la dek-unua jaruloj estis
supozita lerni tion kaj ĝi ne povis estis tiel malfacila.


``Levu vian dekstran manon super la balailo aŭ maldekstran se vi estas
maldekstrula,'' diris S-ro Brando. ``Kaj diru, STARU!''

``STARU!'', ĉiuj kriis.

La balailo eksaltis supren avide en la manon de Harry.

Kiu metis lin je la kapo de la klaso, por unu fojo. Ŝajnis ke diri
``STARU!'' estis multe pli malfacila ol ĝi ŝajnas, kaj la pliparto de
la balailoj estis rulantaj sur la grundo aŭ provantaj forkuri ilian
estuntan rajdanton.

(Evidente Harry vetintus monoj ke Hermione faris almenaŭ tiel bone
kiam estis sia vico de provi, pli frue en la tago. Ne eblis esti ion,
kio li povis regi ekde la unua provo, kio senkomprenigis Hermione'n,
kaj se estus kaj ke ĝi troviĝus esti la rajdo de balailo anstataŭ iu
ajn intelekta, Harry preferus morti.)

Prenis tempon por ke ĉiuj akiris balailon antaŭ ili. S-ino Brando
montris al ili kiel surbalailiĝi kaj poste marŝi ĉirkaŭ la kampo,
korektante kaptojn kaj sintenojn. Ŝajnis ke eĉ el la kelkaj infanoj
kiu estis permesitaj flugi hejme, ili ne estis instruita fari tion
bone.

S-ino Brando rigardis tra la kampo de knaboj, kaj kapjesis. ``Nun,
kiam mi blovos en mia fajfilo, vi piede demetu desur la tero, forte.''

Harry glutis forte, provante haltigi la naŭza sento en sia stomako.

``Gardu vian balailon firme, leviĝis de kelkaj piedoj, kaj poste reiru
rekte malsupren sur la grundo per kliniĝi antaŭen leĝere. Kiam mi
fajfos—tri—du—''

Unu el la balailoj ĵetis sin en la ĉielon, akompanata de krio de juna
knabo—de teruro, ne ĝojo. La knabo estis ŝpinanta je abomeninda rapido
dum li supreniris, ili nur ekvidis mallonge lian blanka vizaĝo—

Kvazaŭ en malrapida moviĝo, Harry saltis desur sian propran balailon,
kaj gratserĉis sian bastonon, kvankam li ne vere sciis kion li planis
fari kun ĝi, li estis havinta ekzakte du lecionojn de Ĉarmo kaj la
lasta estis pri la Ŝveba Ĉarmo sed Harry nur kapablis ĵeti la sorĉon
sukcese unu fojo el tri kaj li certe ne povis ŝvebigi tutan personon—

\emph{Se estas kaŝanta povo en mi, malkovru ĝi mem NUN!}

``Revenu, knabo!'' kriis S-ro Brando (kiu devis esti la plej nehelpa
instruo imagebla por trakti kun freneza balailo, el \emph{instruistino
de flugo}, kaj tute aŭtomata parto de la cerbo de Harry aldonis S-inon
Brando al sia listo de stultuloj.)

Kaj la knabo estis ĵetita desur la balailon.

Li ŝajnis movi tre malrapide trans la aero, unue.

``Ŭingardium Leviosa!'' kriis Harry.

La sorĉo malsukcesis. Li povis senti ĝin malsukcesi.

Estis bruo de frapego kaj malproksima sono de krako, kaj la knabo
estis kuŝita sur la herbo amase.

Harry remetis sian bastonon en sian poŝon, kaj kuris antaŭe je plena
rapido. Li alvenis ĉe la flanko de knabo je la sama tempo ol S-ino
Brando, kaj Harry atingis sian haŭtpoŝo kaj provis memori, ho dio kiu
estis la nomo, ne gravas, li nur provis ``Resaniganta Pako!'' kaj ĝi
aperis en lian manon kaj—

``Rompita pogno,'' S-ino Brando diris. ``Trankviliĝu, knabo, li nur
rompis sian pognon!''

Estis ia mensa ŝanceliĝo kiam la menso de Harry vekiĝu el Panika
Modo.

La Resaniganta Kriza Ilaro Plus estis kuŝanta malfermita antaŭ li, kaj
estis injektilo de likva fajro en la mano de Harry, kiu gardus la
cerbon de la knabo oksigenata se li rompintus sian kolon.

``Ha\ldots'' Harry diris per relative hezitanta voĉo. Li koro estis
frapadanta tiel laŭte ke li preskaŭ povis aŭdis lin mem
anhelis. ``Rompita osto\ldots konsentite\ldots alĝustigeblaj kordoj?''

``Tio estas nur por urĝaĵo,'' seke diris S-ino Brando. ``Metu ĝin for,
li fartas bone.'' Ŝi klinis super la knabo, proponante manon al
li. ``Ek, knabo, estas en ordo, stariĝu!''

``Vi ne serioze farigos lin rajdi balailon denove?'' Harry diris
terure.

S-ino Brando ĵetis rigardaĉon al Harry. ``Evidente ne!'' Ŝi metis la
knabo sur lian piedoj uzante lian sanan brakon—Harry vidis ŝoke, ke
tiu estis Nevilo Longafundo \emph{ankoraŭ}, kio estis ne la
\emph{problemo} kun li?—kaj ŝi turniĝis al ĉiuj rigardantaj
infanoj. ``Neniu el vi movu dum mi kondukas ĉi tiun knabon al la
kuracejo! Vi lasu tiujn balailoj kie ili estas aŭ vi estos eskter
Herpŭrko antaŭ ke vi povas diri 'Kvidiĉo'. Venu, kara.''

Kaj S-ino Brando foriris kun Nevilo, kiu tenis sian pognon, kaj provis
kontroli siajn plorĝemojn.

Kiam ili estis ektser vida kampo, unu el la Serpentimo komencis subridi.

Tiu farigis la alian ankaŭ komenci.

Hrry turniĝis kaj rigardis ilin. Ŝajnis kiel bona tempo por enmemorigi
kelkajn vizaĝojn.

Kaj Harry vidis ke Drako vagis antaŭen al li, akompanata de S-ro Krabe
kaj S-ro Gojle. S-ro Krabe ne estis ridetanta. S-ro Gojlo ja
estis. Drako li mem estis portanta vere kontrolan vizaĝon kiu tremis
okaze, el kiu Harry konkludis ke Drako pensis ke tio estis ridiga sed
ne vidis iun politikan avantaĝon por gajni per ridi pri ĝi nun anstataŭ
en la Serpentima karceroj poste.

``Nu, Potter,'' Drako diris per mallaŭta voĉo kiu ne portis, ankoraŭ
kun tiu vere kontrola vizaĝo kiu estis tremanta okaze. ``Mi nur volis
diri, kiam vi prenas avantaĝon de urĝaĵo por montri gvidadon, vi volas
aspekti kvazaŭ vi estis en tuta kontrolo de la situatio, anstataŭ ol,
diru ni, iri en kompleta paniko.'' S-ro Gojle subridis, kaj Drako
ĵetis al li sufokigan rigardon. ``Sed vi verŝajne gajnis kelkajn
poentojn ĉiaokaze. Ĉu vi bezonas helpon por enfermi tiun ĉi
resanigantan ilaron.''

Harry turniĝis por rigardi la resanigantan ilaron, kiu farigis lin
turni la kapon for de Drako. ``Mi pensas, ke estos bone,'' Harry
diris. Li metis la injektilon je ĝia loko, refaris la ansojn, kaj
stariĝis.

Ernie Macmillan alvenis ĵus kiam Harry estis remetanta la ilaron en
sian mokean haŭtpoŝon.

``Dankon, Harry Potter, en la nomo de Huflopufo,'' Ernie Macmillan
diris formale. ``Ĝi estis vere bona provo, kaj bona pripenso.''

``Bona pripenso, efektive,'' pafis Drako. ``Kial neniu el Huflopufo ne
havas ilian bastono eltirite? Eble ke se vi \emph{ciuj} helpintus
anstataŭ nur Potter, vi povintus kapti lin. Mi pensis ke Huflopufoj
estis supozitaj kunhelpi unu la alian?''

Ernie aspektis kiel se li estis ŝirita inter iĝi kolera kaj voli
morti pro honto. ``Ni ne pensis pri tio ĝustatempe—''

``Ha,'' diris Drako, ``\ldots ne pensis pri ĝi, mi supozas ke tio estis kial
estas pli prefera havi unu Korvungon kial amiko ol ĉiujn Huflopufojn.''

Ho, infero, kiel Harry estis supozita trakti tiun ĉi\ldots ``Vi ne
helpas,'' Harry diris per milda tono. Esperante ke Drako interpretus
tion kiel \emph{vi interferas kun miaj planoj, bonvolu silentu.}

``He, kio tio estas?'' diris S-ro Gojle. Li kliniĝis al la herbo kaj
kaptis supren ion de ĉirkaŭ le grandeco de groca globeto, vitra pilko
kiu ŝajnis esti plenita kun turniĝanta blanka nebulo.

Ernie palpebrumis. ``La Memopriklo de Nevilo!''

``Kio estas Memopriklo?'' demandis Harry.

``Ĝi turniĝas ruĝan se vi forgesis ion,'' Ernie diris. ``Ĝi ne diras
kion vi forgesis, tamen. Donu ĝin, bonvolu, kaj mi ĝin redonos al
Nevilo pli malfrue.'' Ernie levis sian manon.

Subita ridetaĉo aperis sur la vizaĝo de Gojle, kaj li ŝpinis kaj kuregis for.

Ernis staris senmove dum momento pro surprizo, kaj poste kriis ``He!''
kaj kuris sekvante S-ron Gojle.

Kaj S-ro Gojle kaptis balailon, saltis sur ĝi per unu glata movo kaj
demetis en la aeron.

La makzelo de Harry falis. Ĉu S-ro Brando ne diris ke tio farigis lin
esti \emph{forigita?}

``\emph{Tiu stultulo!}'' Drako siblis. Li malfermis la buŝon por krii—

``\emph{He!}'' kriis Ernie. ``Tio apartenas al Nevilo! \emph{Redonu ĝin!}''

La Serpentimoj komencis hurai kaj hui.

La buŝo de Drako fermis firme. Harry ekvidis la subitan aspekton de hezitemo sur lia vizaĝo.

``Drako,'' diris Harry per malsupra tono, ``Se vi ne ordonas al tiu ĉi
stultulo de reiri sur la grundo, la instrustino revenos kaj—''

``Venu kaj kaptu ĝin, Huflopufaĉo!'' kriis S-ro Gojle, kaj granda hurao venis el la Serpetimoj.

``Mi ne \emph{povas}!'' murmuris Drako. ``Ĉiuj en Serpentimo pensus ke
mi estas \emph{malforta}!''

``Kaj se S-ro Gojle estas forigita,'' siblis Harry, ``via partro
pensos ke vi estas \emph{piĉo}!''

La vizaĝo de Drako tordiĝis agonie.

Tiam—

``He, \emph{Serpentŝlimoj},'' kriis Ernie, ``Ĉu neniu iam diris al vi ke
Huflopufoj kunhelpas unu la alian? \emph{Eltiru bastonojn,
  Huflopufoj}''

Kaj estis subite tute multe da bastonoj celantaj al la direkto de S-ro
Gojle.

Tri sekundoj pli malfrue—

``\emph{Eltiru bastonojn, Serpentimoj!}'' kriis ĉirkaŭ kvin malsamaj Serpentimoj.

Kaj estis tute multe da bastonoj celantaj al la direkto de Huflopufoj.

Du sekundoj pli malfrue—

``\emph{Eltiru bastonoj, Grifindoroj!}''

``Faru ion, Potter!'' murmuris Drako. ``Mi ne povas esti tiu kiu
haltigas tion, tiu devas esti vi! Mi ŝuldos favoron al vi, nur pensu
ion, ĉu ve ne estas supozita esti brila?''

En ĉirkaŭ kvin kaj duono da sekundoj, ekkonsciis Harry, iu estus
ĵetinta la Sumeran Simplan Frapan Sorĉon kaj antaŭ ke ĝi estus finita
la profesoroj estus fininta forigi homoj, kaj la solaj knaboj kiuj
restos en sia jaro estos la Korvungoj.

``\emph{Eltiru bastonoj, Korvungoj!}'' kriis Mikael Angulo, kiu
verŝajne sentis sin ellasita de la katastrofo.

``\emph{Gregori Gojle!}'' kriis Harry. ``\emph{Mi defias vin en
kontesto por la posedo de la Memopriklo de Nevilo!}''


Estis subita paŭzo.

``Ho, vere?'' diris Drako per la plej laŭta pafo, kiun harry iam
aŭdis. ``Tiu ŝajnas interesa. Kia kontesto, Potter?''

Er\ldots

``Kontesto'' estis la plej for, kie la inspiro de Harry alporti
lin. Kia kontesto, li ne povis diri ``ŝako'' ĉar Drako ne akceptus sen
ke ĝi aspektas stranga, li ne povis diri ``pojnduelo'', ĉar S-ro Gojle
frakasus lin—

``Kiel pri tio?'' Harry diris laŭte. ``Gregori Gojle kaj mi staras
for de unu la alia, kaj neniu alia rajtas proksimiĝi al neniu el
ni. Ni ne uzas nian bastonon kaj ankaŭ neniu alia rajtas. Mi ne movas
de kie mi estas staranta, kaj li ne ankaŭ. Kaj se mi povas meti mian
manon sur la Memopriklo de Nevilo, tial Gregori Gojle abandonos ĉiun
pretendon de posedo de la Memopriklo kiun li estas tenanta kaj donus
ĝin al mi.''

Estis alia paŭzo dum la aspekto de kvietigo de la homoj ŝanĝis en konfuzon.

``Ha, Potter!'' diris Drako laŭte. ``Mi ŝatus bonvole vidi vin fari
\emph{tion!} S-ro Gojle akceptas!''


``Tial komencu ni!'' diris Harry.

``Potter, \emph{kio?}'' murmuris Drako, kio li iel faris sen movi siajn lipojn.

Harry ne sciis kiel respondi sen movi siajn.

Homoj estis elmetanta ilian bastonon, kaj S-ro Gojle plonĝis gracie al
la grundo, aspektante relative konfuzita. Iuj Huflopufoj komencis
marŝi al S-ro Gojle, sed Harry ĵetis al ilin senesperan pledantan
rigardon kaj ili retropaŝis.

Harry marŝis al S-ro Gojle, kaj li haltis kiam li estis je kelkaj
paŝoj malproksime, sufiĉe for por ke ili ne povas atingi unu la alian.

Malrapide, intence, Harry enigigis sian bastonon.

Ĉiuj aliaj retropaŝis.

Harry glutis. Li sciis je larĝa resumo kion li \emph{intencis} fari,
sed ĝi devis esti farita per iu maniero, per kiu neniu komprenos kion
li faris—

``Konsentite,'' Harry diris laŭta. ``Kaj nun\ldots'' Li prenis
profundan enspiron, kaj levis unu manon, fingrojn preparitaj por
klaki. Estis anheloj el iu ajn aŭdis pri la tortoj, kiu estis preskaŭ
ĉiu. ``\emph{Mi alvokas la frenezecon de Herpŭrko! Feliĉa Feliĉa bum bum
  marĉo marĉo marĉo!}'' kaj Harry klakis siajn fingrojn.

Multe da homoj eksaltis.

Kaj nenio okazis.

Harry lasis la silento daŭris dum kelka momento, entediĝis, ĝis \ldots

``Hum,'' iu diris, ``Ĉu tio estas ĉio?''

Harry rigardis la knabon kiu parolis. ``Rigardu antaŭ vi. Vi vidas tiu
ĉi amaso de grundo kiu aspekta senfrukta, sen iu ajn herbo sur ĝi?''

``Hum, jes,'' diris la knabo, iu Grifindoro (Dean io?).

``Fosu ĝin''

Nun Harry estis akiranta multe da stranga rigardo.

``Er, kial?'' diris Dean io.

``Nur faru tion,'' diris Terio Boto per laca voĉo. ``Neniu intereso
por demandi kial, fidu min pri tio.''

Dean io surĝenuĝis kaj komencis ĉerpi la teron.

Post unu minutoj aŭ preskaŭ, Dean stariĝis. ``Estas nenio ene tie ĉi,'' Dean diris.

Hu, Harry estis planinta iri malantaŭen en la tempon kaj enterigi
trezora mapo kiu estintus gvidanta al alia trezora mapo kiu estintus
gvidanta al la Memopriklo de Nevilo, kiun li metintus tien post
reakiri ĝin el S-ro Gojle\ldots

Post Harry ekkonsciis ke estis multe pli facila maniero, kiu ne
minacos la sekreton de Tempa Returnilo same.

``Dankon, Dean!'' Harry diris laŭte. ``Ernis ĉu vi povas rigardi
ĉirkaŭ la grundo, kie Nevilo falis kaj vidi se vi povas trovi la
Memopriklon de Nevilo?''

Homoj aspektis ankoraŭ pli konfuza.

``Nur faru ĝin,'' diris Terio Boto. ``Li daŭros provi ĝis io funkcias, kaj la terura aĵo estas ke—''

``\emph{Merlin!}'' spiregis Ernie. Li estis tenanta la Memopriklon de
Nevilo. ``Ĝi estas \emph{tie!} Ekzkte kie li falis!''

``\emph{Kio?}'' kriis S-ro Gojle. Li rigardis malsupren kaj vidis\ldots

\ldots ke li estis ankoraŭ tenanta la Memopriklon de Nevilo.

Estis relative longa paŭzo.

``Er,'' diris Dean io. ``Tio ne estas ebla, ĉu ne?''

``Tio estas malkoheraĵo,'' diris Harry. ``Mi farigis min sufiĉe
stranga por distri la universon dum momento, kaj ĝi forgesis ke Gojle
jam kaptis la Memopriklon.''

``Ne, atendu, Mi volas diri, tio estas \emph{tute} neebla—''

``Pardonu min, ĉiu vi ĉiuj estas staranta ĉirkaŭ tie antendanta por
flugi sur balailoj? Jes vi estas. Do silentu. Ĉiaokaze, unufoje kiam
mi metos mian manon sur la Memopriklon de Nevilo, la kontesto estos
finita kaj Gregori Gojle devos abandoni ĉiu pretendo al la Memopriklon
kiun li estas tenanta, kaj donos ĝin al mi. Tiu estas la kondiĉoj, ĉu
vi memoras?'' Harry tendis manon kaj signodonis al Ernie.  ``Nur rulas
ĝin al tie ĉi, pro tio ke neniu estas supozita proksimiĝi al mi,
okej?''

``Antedu!'' kriis iu Serpentimo—Blaise zabini, Harry neniam forgesos
tiu nomon. ``Kiel ni scias ke tiu ĉi estas la Memoras de Nevilo? Vi
povas nur esti lasinta alian Memopriklon tie—''

``La Serpentimo estas forta en tiu ĉi.'' Harry diris ridetante. ``Sed
vi havas miajn vortojn ke la unu kiun Ernie estas tenanta, estas tiu
de Nevilo, neniu komento pri tiu kiu estas en la mano de S-ro Gojle.''

Zapini turniĝis al Drako. ``*Malfojo!ù Vi ne nur lasos lin foriri tiel—''

``Silentu, vi,'' bruegadis S-ro Krabe, starante malantaŭ Drako. ``S-ro
Malfojo ne bezonas ke vi al li diras kion fari!''

\emph{Bona} sbiro.

``Mia veto estis kun Drako, de la Nobla kaj plej malnova Domo de
Malfojo,'' Harry diris. ``Ne kun vi, Zabini. Mi faris tion, kion Drako
diris voli vidi min fari, kaj kiel por la juĝo de la veto, mi lasas
tion al S-ro Malfojo.'' Harry klinis sian kapon al Drako kaj levis
siajn brovojn leĝere. Tio intencis permesi Drako sufice savi sian
prestiĝon.


Estis paŭzo.

``Vi promesas ke tiu estas la Memopriklo de Nevilo?'' Drako diris.

``Jes,'' Harry diris. ``Tio estas la unu kiu reiros al Nevilo kaj lia origina. Kaj la unu kiu Gregori Gojle estas tenanta iras al mi.''

Drako kapjesis, aspektante decida. ``Mi ne estas dubonta la vortojn de
la Nobla Domo de Potter, tial, ne gravas kiel stranga ĉio tio
estis. Kaj la Nobla kaj Plej malnova Domo de Malfojo tenas sin al siaj
vortoj. S-ro Gojle donas tion al S-ro Potter—''

``He!'' Zabini diris. ``Li ne jam gajnas, li ne metis sian manon sur—''

``Ekkaptu Harry!'' diris Ernie, kaj li ĵetis la Memopriklo.

Harry facile kaptis la Memopriklon en la aero, li ĉiam havis bonajn
refleksojn por tio. ``Jen,'' diris Harry, ``Mi gajnas\ldots''

Harry mallaŭtiĝis. Ĉiuj konversacioj haltis.

La Memopriklo estis brilanta hela ruĝa en lia mano, flameganta kiel
miniaturo suno kiu disĵetis ombrojn sur la grundo en la ĝenerala
taglumo.

\begin{center}\rule{3in}{0.4pt}\end{center}

Ĵaŭdo.

Se oni volis esti preciza, ĵaŭdo je la 5:09 posttagmeze, en la oficejo
de Profesorino McGonagall, post la leciono de flugo. (Kun ekstra horo
por Harry metita inter.)

Profesorino McGonagall sidis sur sia tabureto. Harry, li sidis sur varma seĝo antaŭ ŝia skribtablo.

``Profesorino,'' Harry diri streĉe, ``Serpentimoj estis celigantaj
ilian bastonon al Huflopufoj, Grifindoroj estis celigantaj ilian
bastonon al Serpentimoj, iu \emph{stultulo} vokis Korvungoj por ke ili
eltiris ilian bastonon, kaj mi havis eble kvin sekundojn por malhelpi la
tutan aferon eksplodi! Tio estis ĉio kion mi suckesis trovi!''

La vizaĝo de Profesorino McGonagall estis pinĉita kaj kolera. ``*Vi ne
devas uzi la Tempan Returnilon je tia maniero, S-ro Potter!* Ĉu la
koncepto de sekreteco ne estas io kion vi komprenas?''

``Ili \emph{ne scias} kiel mi faris! Ili nur pensas ke mi povas fari
vere strangaj aĵojn per klaki miajn fingrojn! Mi faris aliajn
strangajn aĵojn, kiujn oni ne eĉ povas fari kun Tempa Returnilo, kaj
mi faros \emph{pli} da aĵoj kiel tio, kaj tiu okazo ne eĉ elstaros!
\emph{Mi devis fari ĝin}, Profesorino!''

``Vi \emph{ne} bezonis fari ĝin!'' respondis seke Profesorino
McGonagall. ``Ĉio vi bezonis fari estis igi tiun \emph{anoniman
seprentimon} reiri sur la grundo, kaj igi la bastonon formetiĝi! Vi
povintus defii lin kun ludo de Eksploda Batalo\footnote{La eksploda
batalo estas kartludo, kie la kartoj subite eksplodas dum la partio}
sed ne, vi devis uzi la Tempan Returnilon, per flagranta kaj nenecesa
maniero!''

``Tio estis ĉio kion mi sukcesis trovi! Mi ne eĉ scias kio estas
Eksploda Batalo, ili ne akceptintus ŝakon, kaj se mi proponintus
pojnduelon, mi perdintus!''

``\emph{Tial vi devintus elekti pojnduelon!}''

Harry palpebrumis. ``Sed tial mi perdintus—''

Harry haltis.

Profesorino McGonagall aspektis tre kolera.

``Mi bedaŭras, Profesorino McGonagall,'' Harry diris per eta
voĉo. ``Mi honeste ne pensis pri tio, kaj vi pravas, mi devintus, ĝi
estintus brila se mi pensintus pri tio, sed mi nur tute ne pensis pri
tio \ldots''

La voĉo de Harry mallaŭtiĝis. Estis subite evidenta al li ke li havis
\emph{multajn} aliajn alternativojn. Li povintus demandi al Drako
sugesti ion, li povintus demandi al la aliaj knaboj\ldots lia uzo de
la Tempa Returnilo \emph{estis} flagranta kaj nenecesa. Estis grandega
spaco de ebloj, kial li elektis tiun?

Ĉar li vidis manieron por gajni. Gajni la posedon de negrava bagatelo
kiun la instruistoj reprenintus el S-ro Gojle ĉiaokaze.

Intenco de gajni. Tio estis kio havis lin.

``Mi bedaŭras,'' Harry diris denove. ``Por mia fiero kaj mia stulteco.''

Profesorino McGonagall pasis manon sur sia fronto. Iom ŝia kolero
ŝajnis dispeli. Sed ŝia voĉo estis ankoraŭ elirante tre severe. ``Unu
plia montro kiel tia, S-ro Potter, kaj vi redonos tiun Tempan
Returnilon. Ĉu mi farigas min vere klara?''

``Jes,'' Harry diris. ``Mi komprenas kaj mi bedaŭras.''

``Tial, S-ro Potter, vi estos permesita gardi la Tempan Returnilon
nunmomente. Kaj konsiderante la grandeco de la fiasko kiun vi, fakte,
preventis, mi ne deprenos iun ajn poentojn el Korvungo.''

\emph{Plus vi ne povus klarigi kial vi deprenus la poentojn.} Sed
Harry ne estis sufiĉe stulta por diri tion laŭte.

``Pli grava, kial la Memopriklo ekbrulis tiel?'' Harry diris. ``Ĉu tio
volas diri ke mi estis Forgesigita?''

``Tio konfuzegas min ankaŭ,'' Profesorino McGonagall diris
malrapide. ``Se tio estus tiel simpla, mi pensus ke la tribunalo uzus
Memopriklon, kaj ili ne faras. Mi devus pririgardi tion, S-ro
Potter.'' Ŝi suspiris. ``Vi povas eliri nun.''

Harry komencis stariĝi desur sia seĝo, kaj poste haltis. ``Hum,
pardonu min, mi havas ion alian kion mi volas diri al vi—''

Vi povis apenaŭ vidi la ektremon. ``Kio estas, S-ro Potter?''

``Tio estas pri la Profesoro Ciuro—''

``Mi certas, S-ro Potter, ke tio estas nenio grava.'' Profesorino
McGonagall diris la vortoj en granda hasto. ``Certe, vi aŭdis la
Direktoron dirante al la studentoj ke ili ne devas ĝeni nin kun iu ajn
negrava plendo pri la Profesoro de Defenco?''

Harry estis relative konfuza. ``Sed tio povas \emph{esti} grava,
hieraŭ mi akiris tiun subitan senso de pereo kiam—''

``S-ro Potter! Mi havas senson de pereo ankaŭ! Kaj mia senso de pereo
estas sugestanta ke \emph{vi ne devas fini tiun frazon!}''

La buŝo de Harry ekmalfermiĝis. Profesorino McGonagall sukcesis; Harry estis muta.

``S-ro Potter,'' diris Profesorino McGonagall, ``se vi havas malkovri
ion kio ŝajnas interesa pri Profesoro Ciuro, bonvolu sentas vin libera
por ne disdoni ĝin kun mi aŭ iu ajn alia. Nun mi pensas ke vi prenis
sufiĉe da mia precioza tempo—''  

``\emph{Tio ne estas kiel vi!}'' Harry ekdiris. ``Mi bedaŭras sed tio
ŝajnas \emph{nekredebla} nerespondeca! De kio mi aŭdis, estas ia
malbonŝanco sur la Defenca ofico, kaj se vi jam \emph{scias} ion kio
estas iranta malbona, mi pensus ke vi ĉiuj devus esti singarda—''

``Iu problemo, S-ro Potter? \emph{Mi certe esperas ke ne}.'' La vizaĝo
de Profesorino McGonagall estis senesprima. ``Post kiam Profesoro
Blake estis kaptita en necesejo kun ne malpli ol tri Serpentimoj de
kvin jaroj je la lasta februaro, kaj la jaro antaŭ tio, Profesorino
Somero malsukcesis tiel komplete kiel instruisto ke ŝiaj studentoj
pensis ke Bogarto\emph{Bogarto estas senmorta fluganta formo, kiu
prenas la formon de la plej malbona timo de tiu kiu rigardas ĝin}
estas iu speco de meblo, estus katastrofa se iu problemo kun la
eksterordinare kompetenta Profesoro Ciuro alvenus al mia atento nun,
mi maltimas diri ke la pliparto de niaj studentoj malsuksesus ilian
Defenca D.U.M.E'j\footnote{Diplomo Universala de Magio Elementa} kaj
L.A.S.T'j\footnote{Laciga kaj Afliktiga Sorĉa Testo}.''

``Mi vidas,'' Harry diris malrapide, akiranta ĉio tio. ``Do, per aliaj
vortoj, kio ajn estas malbona kun Profesoro Ciuro, vi senespereme ne
volas scii ĝin ĝis la fino de la lerna jaro. Kaj tial oni estas en
Septembro, li povas mortigi la ĉefministron en la televido kaj eskapi
pri ĝi ĉiuokaze por vi.''

Profesorino McGonagall rigardis lin fikse senpalpebrumi. ``Mi certas
ke oni neniam aŭdos min diri tiel deklaron, S-ro Potter. En Herpŭrko,
ni strebas esti iniciatema pri io ajn kio minacas la eduka atingo de
niaj studentoj.''

\emph{Kiel unua-jara Korvungo kiu ne povas gardi sian buŝo fermita.}
``Mi kredas ke mi komprenas vin komplete, Profesorino McGonagall.''

``Ho, mi dubas pri tio, S-ro Potter. Mi dubas pri tio vere multe.''
Profesorino McGonagall kliniĝis antaŭen, ŝia vizaĝo firmiĝanta
denove. ``Tial vi kaj mi jam parolis pri aferoj multe pli sentivaj ol
tio, mi povas paroli sincere. Vi, kaj vi sola, estas ĥaosmagneto, kiel
mi neniam vidis. Post nia eta butikumada ekskurso en Diagon Aleo,
kaj \emph{post} la Ordiganta Ĉapelo, kaj post \emph{hodiaŭa} eta
epizodo, mi povas bone antaŭvidi ke mi estas destini sidi en la
oficejo de la Direktoro kaj aŭdi iun ridigan rakonton pri Profesoro
Ciuro en kiu vi kaj vi sola ludas ĉefan rolon, post kiu oni ne havas
alian elekton ol eksigi lin. Mi jam rezignaciis pri tio, S-ro
Potter. Kaj se tiu malĝoja evento okazus, iam ajn pli frue ol la iduoj
de Majo, mi ligus vin al la pordegoj de Herpŭrko kun via propra
intesto kaj verŝi fajraj skaraboj en vian nazon. \emph{Nin} ĉu vi
komprenas min komplete?''

Harry kapjesis, siaj okuloj tre larĝaj. Kaj, post unu sekundo, ``Kion
mi gajnas, se mi sukcesas farigi ĝin okazi je la lasta tago de la
lerna jaro?''

``\emph{Eliru mian oficejon!}''

\begin{center}\rule{3in}{0.4pt}\end{center}

Ĵaŭdo.

Devus esti io pri ĵaŭdoj en Herpŭrko.

Estis ĵaŭdo je la 5:32 posttagmeze, kaj Harry estis staranta ĉe la
Profesoro Flirtiko, fronte de la grandega ŝtona gargojlo ki gardis la
enirejo de la oficejo de la Direktoro.

Apenaŭ li revenis de la oficejo de Profesorino McGonagall al la
studĉambroj de Korvungo, ke unu el la studentoj diris al li prezenti
sin al la oficejo de Profesoro Flirtiko, kaj tie Harry lernis ke
Dumbledore volis paroli kun li.

Harry, sentante sin relative anksia, demandis al Profesoro Flirtiko se
la Direktoro estis dirinta ion pri kio li volis paroli.

Profesoro Flirtiko ŝultrolevis en ia senespera maniero.

Verŝajne Dumbledore estis dirinta ke Harry estis multe tro juna por
elvoki la vortoj de povo kaj frenezeco.

\emph{Feliĉa feliĉa bum bum marĉo marĉo marĉo?} Harry pensis sed sen
diri ĝin laŭte.

``Bonvolu ne zorgi tro multe, S-ro Potter,'' knaris Profesoro Flirtiko
de ie ĉirkaŭ la ŝultra nivelo de Harry. (Harry estis dankema al la
giganta grasa barbo de Profesoro Flirtiko, estis malfacila kutimi al
Profesoro kiu ne nur estis pli malalta ol li sed parolis kun pli akuta
voĉo.) ``Direktoro Dumbledore povas ŝajni iom stranga, aŭ tre stranga,
aŭ eĉ ege stranga, sed li neniam dolorigis studenton eĉ iom, kaj mi ne
kredas ke li iam faros.'' Profesoro Flirtiko donis al Harry kuraĝigan
rideton. ``Nur gardu tion mense tuttempe, kaj vi certas ne paniki!''

Tio ne estis helpema.

``Bonan ŝancon!'' knaris Profesoro Flirtiko, kaj kliniĝis al la
gargojlo kaj diris ion kion Harry tute malsukcesis aŭdi. (Evidente, la
pasvorto ne estus tre bona, se vi povus aŭdi iun diri ĝin.) Kaj la
ŝtona gargojlo marŝis flanken per vere natura kaj ordinara movo kiun
Harry trovis relative ŝoka, pro tio ke la gargojlo ŝajnis ankoraŭ
solida, nemovebla ŝtono la tuta tempo.

Malantaŭ la gargojlo estis helika ŝtuparo turnanta malrapide. Estis
ion maltrankvilige hipnota pri tio, kaj eĉ pli maltrankviliga estis ke
sekvi la turnantan helikan ŝtuparon ŝajnis irigi vin nenie.

``Supreniru!'' knaris Profesoro Flirtiko.

Harry relative nervoza, marŝis sur la helika ŝtuparon, kaj ĝi daŭris
turniĝi kaj Harry daŭris iri supren, kaj post relative kapturna
momento, Harry troviĝis antaŭ kverka pordo kun grifa mamo el latuno.


Harry atingis eksteren, kaj turnis la aknon.

La pordo malfermiĝis svinge.

Kaj Harry vidis la plej interesa ĉambro kiun li iam vidis en sia vivo.

Estis etaj metalaj mekanismoj, kiuj siblis aŭ tiktakis aŭ malrapide
ŝanĝis en alian formon aŭ emisiis etajn pufojn de fumo.  Estis dekduoj
da misteraj fluaĵoj en dekduoj da strange formitaj ujoj, ĉiuj
bobelantaj, bolantaj, likantaj, ŝanĝantaj en alian koloron, aŭ
formantaj en intersan formon kiu neniiĝis je la sekundo post vi vidis
ilin. Estis aĵoj kiuj aspektis kiel horloĝo kun multaj montriloj,
enskribitaj per nombroj aŭ per nerekoneblaj lingvoj. Estis braceleto
eltenanta bikonveksan kristalon kiu brilis per miloj da koloroj, kaj
birdo sidanta sur ora platformo, kaj ligna pokalo plenita per io, kiu
aspektis kiel sango, kaj statuo de falko enkrustita en nigra
emajlo. La muro estis tute kovrita per bildoj de dormantaj homoj, kaj
la Ordiganta Ĉapelo estus indiferente balanciĝanta sur ĉapelbretaro
kiu estis ankaŭ tenanta du pluvombrelojn kaj tri ruĝajn pantoflojn por
maldekstra piedo.

En la mezo de ĉio tiu ĉi kaoso estis pura skribtablo el nigra
kverko. Antaŭ la skribtablo estis kverka tabureto. Kaj malantaŭ la
skribtablo estis bone remburita trono eltenanta Albus Percival Ŭlfric
Brian Dumbledore, kiu estis ornamita per longa arĝenta barbo, per
ĉapelo kiel dispremita giganta ĉampinjono, kaj per kiu aspektis por
Muglaj okuloj, kiel tri tavoloj da brila roza piĵamo.

Dumbledore estis ridetanta, kaj siaj brilaj okuloj palpebrumis je
freneza intenseco.

Kun kelka timeto, Harry sidiĝis antaŭ la skribtablo. La pordo fermiĝis
svinge malantaŭ li kun laŭta bruo obtuza.

``Bonan tagon, Harry,'' diris Dumbledore.

``Bonan tagon, Direktoro,'' Harry respondis. Do ili estis nomanta sin
per ilia antaŭnomo? Ĉu Dumbledore nun diros nomi lin—

``Bonvolu, Harry!'' diris Dumbledore. ``Direktoro sonas tiel
formala. Nur nomu min Dir por fari pli mallonga.''

``Mi certe faros, Dir,'' diris Harry.

`Estis mallonga paŭzo.

``Ĉu vi scias,'' diris Dumbledore, ``vi estas la unua persono kiu iam
faris ĝin reale?''

``Ha\ldots'' Harry diris. Li provis kontroli sian voĉon malgraŭ la
subita sento de drono en sia stomako. ``Mi bedaŭras, mi, ha,
Direktoro, vi diris al mi fari tiel, do mi faris—''

``Dir, bonvolu!'' diris Dumbledore ĝoje. ``Kaj estas neniu kialo pro
kiu zorgi, mi ne ĵetos vin tra la fenestro nur ĉar vi faris unu
eraron. Mi donos abundon de averto al vi antaŭe, se vi faras ion malbona!
Krome, kio gravas ne estas kiel homoj parolas al vi, sed kion ili
pensas pri vi.''


\emph{Li neniam dolorigis studenton, nur gardu tion mense tuttempe, kaj vi certas ne paniki.}

Dumbledore tiris etan metalan keston kaj malfermis ĝin, malkovrante
kelkajn etajn flavajn pecetojn. ``Citrona ŝorbeto?'' diris la
Direktoro.

``Er, ne dankon, Dir,'' diris Harry. \emph{Ĉu doni LDA al studento
kalkulas kiel dolorigi rin, aŭ ĉu tio apartenas al la kategorio de
sendanĝera amuzo?} ``Vi, hum, diris ion pri ke mi estas tro juna por
elvoki la vortojn de povo kaj frenezeco?''

``Jes, vi certe estas!'' Dumbledore diris. ``Danke, la Vortoj de Povo
kaj Frenezeco estis perdita sep centjaroj antaŭe kaj neniu havas la
plej leĝera ideo pri kiuj li povas esti nun. Tiu estis nur rimarko.''

``Ha\ldots'' Harry diris. Li estis konscia ke sia buŝo estis pendanta
malfermita. ``Kial vi vokis min tien, tial?''

``\emph{Kial?}'' Dumbledore ripetis. ``Ha, Harry, se mi pasus ĉiu tago
demandante \emph{kial} mi faras aĵojn, mi neniam havus tempon por fari
ion ajn! Mi estas tre okupita persono, ĉu vi scias.''

Harry kapjesis, ridetante. ``Jes, ĝi estas vere impresa
listo. Direktoro de Herpŭrko, Estra Magiisto de la Magekoncilio, kaj
Supera Dieculo de la Internacia Konfederacio de Sorĉitoj. Pardonu min
pro mia demando, sed mi demandis min, ĉu estas ebla akiri pli da ses
horoj se vi uzas pli da unu Returnilo de Tempo? Ĉar tio estas tre
impresa se vi faras ĉion tio en nur tridek horoj ĉiutage.''

Estis alia mallonga paŭzo, dum kiu Harry daŭris rideti. Li estas iom
anksia, fakte tre anksia, sed unufoje kiam evidentiĝas ke Dumbledore
estis intence mokanta lin, io en li absolute rifuzis sidi kaj preni
ĝin kiel sendefenda objekto.

``Mi timas ke la tempo ne ŝatas estis etendita tro multe.'' diris
Dumbledore post la mallonga paŭzo. ``kaj por la momento, ni ŝajnas
esti iom tro larĝa por ĝi, kaj do tio estas konstenta batalo por
adapti nian vivon en la tempo.''

``Efektive,'' Harry diris kun serioza soleneco. ``Tio estas kial estas
prefere iri al la nia subjekto rapide.''

Dum momento, Harry demandis sin, se li estis irinta tro for.

Poste Dumbledore subridis. ``Direkte al la subjekto, ni iru.'' La
Direktoro kliniĝis antaŭen, inklinana sian ĉampinjonan ĉapelon kaj
brosanta sian barbon kontraŭ la skribtablo. ``Harry, ĉi tiu lundo, vi
faris ion kiu devintus esti neebla eĉ kun Tempa Returnilo. Aŭ pli
ĝuste, neebla kun \emph{nur} Tempa Returnilo. De kien tiuj du tortoj
venis, mi scivolas?''

Impulso de adrenalino ekpasis trans Harry. Li faris tion uzante la
Mantelon de Nevidebleco, la unu kiun estis donita al li en kristanaska
stakolo kun noto, kaj la noto diris: \emph{Se Dumbledore vidas ŝanĉon
de posedi unu el la Sanktaĵoj de la Morto, li neniam lasos eskapi sian
kapton\ldots}

``Natura penso,'' Dumbledore daŭris, ``estas ke pro tio ke neniu el la
ĉeestantaj unua-jaruloj estis kapabla ĵeti tian sorĉon, iu alia estis
ĉeestanta, kaj ne vidata. Kaj se neniu povus vidi lin, estus sufiĉe
facila por ri ĵeti la tortojn. Oni povas daŭre suspekti ke pro tio ke
vi havis Tempan Returnilon, vi estis la nevidata, kaj ke pro tio ke la
sorĉo de Elreviĝo estas multe tro malproksime de viaj aktualaj
kapablecoj, vi havis mantelon de nevidebleco.'' Dumbledore ridetis
konspire. ``Ĉu mi pravas ĝis tie, Harry?''

Harry estis frostigita. Li havis la senton ke kompleta mensoĝo ne
estintus saĝa, kaj eble tute ne helpanta, kaj li ne povis pensi pri io
ajn alia por diri.

Dumbledore svingis amikan manon. ``Ne zorgu, Harry, vi ne faris ion
ajn malbona. Mantelo de Nevidebleco ne estas kontraŭ la reguloj—Mi
supozas ke ili estas sufiĉe malabunda por ke neniu eĉ prenis tempon
por meti ilin sur la listo. Sed vere mi demandis min pri io komplete
malsama.''

``Ho?'' Harry diris per la plej normala voĉo kiun li sukcesis havi.

La okuloj de Dumbledore briliĝis per entuziasmo. ``Vi vidas, Harry,
post kiam oni faris kelkajn aventurojn, oni emas kapti la aĵon de tiuj
aferoj. Oni komencas vidi la skemon, aŭdi la ritmon de la
universo. Oni komencas havi suspekton antaŭ la momento de
revelacio. Vi estas la Knabo-Kiu-Post-Vivis, kaj iel mantelo de
nevidebleco trovis vojon al via mano nur kvar tagoj post vi malkovris
pri la magia britio. Tia mantelo ne estas aĉetebla en la Diagon
Aleo, sed estas unu kiu povintus trovi sian vojon al destina
portanto. Kaj do mi ne povas malhelpi min ne scivoli se per stranga
ŝanco, vi trovis ne nur \emph{iun} mantelon de nevidebleco, sed
\emph{la} Mantelon de Nevidebleco, unu el la tri Sanktaĵoj de la Morto
kaj laŭdire kaŝanta la portanto de la vido de la morto ri mem.'' La
rigardo de Dumbledore estis hela kaj avida. ``Ĉu mi povas vidi ĝin,
Harry?''

Harry glutis. Estis tuta inundo de adrenalino en sia sistemo nun, kaj
ĝi estis tute senutila, tiu estis la plej potenca sorĉisto de la mondo
kaj ne estis maniero por ke li povas sukcesi trairi la pordon, kaj
estis nenie en Herpŭrko kie li povis kaŝi sin de li, li estis perdonta
la Mantelon kiu pasis trans la Potter'j ekde kiu sciis kiam—

Malrapide Dumbledore kliniĝis malantaŭen en sia alta seĝo. La hela
lumo estis foririnta de siajn okulojn, kaj li aspektis konfuza kaj iom
malgaja. ``Harry'' diris Dumbledore, ``se vi ne volas, vi povas nur diri ne.''

``Mi povas?'' Harry kvakis.

``Jes, Harry,'' diris Dumbledore. Lia voĉo sonis malĝoja nun, kaj
zorgplena. ``Ŝajnas ke vi timas min, Harry. Ĉu mi povas demandi kion
mi faris por gajni vian malfidon?''


Harry glutis. ``Ĉu estas ia maniero por ke vi faras magian ĵuron
liganta vin al ne forpreni mian mantelon?''

Dumbledore kapneis malrapide. ``Nerompeblaj Votoj ne devas esti uzita
tiom leĝere. Kaj krome, Harry, se vi ne jam konus le sorĉo, vi havus
nur miajn vortojn pri ke la sorĉo ligas min. Jam, certe, vi konscias
ke mi ne \emph{bezonas} vian permeson por vidi la Mantelon. Mi estas
sufiĉe potenca por eltiris ĝin per mi mem, moke haŭtpoŝo aŭ ne.'' La
vizaĝo de Dumbledore estis tre serioza. ``Sed tion mi ne faros. La
mantelo estas via, Harry. Mi ne konfiskos ĝin el vi. Ne eĉ por rigardi
ĝin dum momento, almenaŭ ke vi decidas montri ĝin al mi. Tio estas
promeso kaj ĵuro. Se mi devus malpermesi vin uzi ĝin en la lernejo, mi
devigus vin iri al via sekurĉambro en Gringoto kaj konservi ĝin
tie.''

``Ha\ldots'' Harry diris. Li glutis forte, provante trankviligi la
inundon de adrenalino kaj pensi racie. Li prenis la haŭtpoŝo de sian
zonon. ``Se vi vere \emph{ne} bezonas mian permeson\ldots tial vi
havas ĝin.'' Harry etendis la haŭtpoŝon al Dumbledore, kaj forte
mordis sian lipon, sendante la signalon al si mem en la okazo li estos
Forgesigita poste.

La maljuna metis sian manon en la haŭtpoŝon, kaj sen diri iun ajn
vorton de repreno, eltiris la Mantelon de Nevidebleco.

``Ha,'' diris Dumbledore spire. ``Mi pravis\ldots'' Li pasis la
lumetadantan nigran veluron trans sia mano. ``Maljuna de jarcentoj,
kaj ankoraŭ perfekta kiel le unua tago. Ni perdis multe da nia arto
trans la jaroj, kaj nun mi ne povas fari tian aĵon per mi mem, neniu
povas. Mi povas senti la povon de ĝi kiel eĥo en mia menso, kiel
kantado ĉiam kantita sen iu ajn por aŭdi ĝin\ldots'' La sorĉisto levis
sian rigardon de la Mantelon. ``Ne vendu ĝin,'' li diris. ``ne donu
ĝin al iu ajn kiel posedo. Pensu du fojo antaŭ montri ĝin al iu ajn,
kaj konsideru tri fojo denove antaŭ vi malkaŝas ke ĝi estas unu el la
Sanktaĵoj de la Morto. Traktu ĝin kun respekto, pro tio ke ĝi estas
efektive iu Aĵo de Povo.''

Dum momento la vizaĝo de Dumbledore daŭris pensanta \ldots
\ldots kaj poste li reetendis la Mantelon de Nevidebleco al Harry.

Harry remetis ĝin en sian haŭtpoŝon.

La vizaĝo de Dumbledore estis serioza unufoje plu. ``Ĉu mi povas
demandi denove, Harry, keil vi iris al malfidi min?''

Subite Harry sentis sin honta.

``Estis noto kun la Mantelo,'' Harry diris per eta voĉo. ``Ĝi diris ke
vi provus preni la Mantelon de Nevidebleco el mi, se vi scius. Mi ne
scias kiu metis la noton, tamen, mi vere ne scias.''

``Mi\ldots vidas,'' Dumbledore diris malrapide. ``Nu, Harry, mi ne
kontestos la motivojn de kiu ajn lasis tiun noton. Kiu scias sed eble
ri havis la plej bonajn intencojn? Ri donis la Mantelon al vi, malgraŭ
ĉio.''

Harry kapjesis, impresita per la bonkoro de Dumbledore, kaj hontiĝi
pro la akra kontrasto kun sia propra sinteno.

La maljuna sorĉisto daŭrigis. ``Sed vi kaj mi ambaŭ estas peonoj el la
sama koloro, mi pensas. La knabo kiu finfine venkis Voldemort'n, kaj
la maljuna viro kiu gardis lin sufiĉe longatempe por ke vi povas savi
la situacion. Mi ne riproĉos vian singardon, Harry, ni devas ĉiuj
faras nian plej bonan por esti saĝa. Mi nur demandos ke vi pensas
dufoje kaj konsideras trifoje denove, la sekva fojo ke iu diras
malfidi min.''

``Mi bedaŭras,'' Harry diris. Li sentis sin mizera tiam ĉi, li nur
insultis Gandalf'n esence, kaj la bonkoreco de Dumbledore farigis lin
senti sin ankoraŭ pli malbona. ``Mi ne devintus malfidi vin.''

``Bedaŭrinde, Harry, en tiu mondo\ldots'' La maljuna sorĉisto skuis
sian kapon. ``Mi eĉ ne povas diri ke vi estis malsaĝa. Vi ne konis
min. Kaj por diri vere estas iuj personoj en Herpŭrko kiun vi farus
bone ne fidi. Eble eĉ iuj kiujn vi nomas amikoj.''

Harry glutis. Tio ŝajnis relative malbonaŭgura. ``Kiel kiu?''

Dumbledore stariĝis el sia seĝo, kaj komencis ekzamini unu el siaj
intrumentoj, ciferplato kun ok montriloj de pluraj longoj.

Post kelkaj momentoj, la maljuna sorĉisto parolis denove. ``Li
probable ŝajnas tute ĉarma al vi,'' diris Dumbledore. ``Agrabla—al vi
almenaŭ. Bone parolanto, eble eĉ admira. Ĉiam preta kun helpanta mano,
favor, konsilo—''


``Ho, \emph{Drako Malfojo!}'' Harry diris, sentante relative kvieta ke
ne temis pri Hermione aŭ io kiel tio. ``Ho ne, ne ne ne ne, vi tute
malpravas, li ne estas ŝanĝata min, mi ŝanĝas lin.''

Dumbledore frostiĝis kie li estis rigardante al la ciferplato. ``Vi
\emph{kio}?''

``Mi estas turnonta lin el la Malhela Flanko,'' Harry diris, ``Vi
scias, farigis lin bona ulo.''

Dumbledore rektiĝis kaj turniĝis al Harry. Li estas portanta unu el la
plej mirigita esprimo, kiun Harry iam vidis sur iu ajn, kaj ankoraŭ
malpli sur iu kun longa arĝenta barbo. ``Ĉu vi certas,'' diris la
maljuna sorĉisto post momento. ``ke li estas preta por esti reaĉetita?
Mi timas ke kiu ajn boneco vi pensas vidi en li estas nur deziranta
penso—aŭ pli malbona, ĉarmo, logaĵo—''

``Er, malmulte verŝajne,'' Harry diris. ``Mi volas diri ke se li
provas maskovesti lin kiel bona ulo, li estas nekredebla malbona. Tio
ne estas pri Drako venanta al mi kaj estanta tuta ĉarmo kaj mi
decidanta ke li devas havi kaŝitan kernon de boneco profunda
malsupre. Mi elektis lin por elaĉeto specife ĉar li estas la heredanto
de la Domo Malfojo kaj se vi devus elekti unu personon por elaĉeti,
tio evidente estus li.''

La maldekstra okulo de Dumbledore ektremis. ``Vi intencas semi semojn
de amo kaj bonkoreco en la koro de Drako Malfojo, ĉar vi atendas ke la
heredanto de Malfojo estos utila al vi?''

``Ne nur al \emph{mi}!'' Harry diris indigne. ``Al la tuta Britio
Magia, se tio sukcesas! \emph{Kaj} li mem havos pli feliĉan kaj mense
pli sanan vivon! Aŭskultu, mi ne havas sufiĉe da tempo por elirigi
ĉiujn el la Malhela Flanko, kaj mi devas trovi kie la Lumo povas
gajni la plej da avantaĝo la plej rapide—''

Dumbledore komencis ridi. Ridegante multe pli forta ol Harry atendus,
preskaŭ huriante. Tio ŝajnis tute \emph{malinda}. Maljuna kaj potenca
sorĉisto devintus subridi en profunda minacanta tono, ne ridegi tiel
forte ke li devis spiregi por aero. Harry unufoje laŭlitere falis
desus sia seĝo kiam li rigardis la filmo de la Fratoj Marx \emph{Anasa
  Supo}, kaj tio estis kiel forte Dumbledore estis rideganta nun.

``Tio ne estas \emph{tiel} komika,'' Harry diris post mallonga
tempo. Li estis komencanta zorgi pri la mense sano de Dumbledore
denove.


Dumbledore reprenis sin sub kontrolo kun videbla eforto. ``Ha, Harry,
unu simptomo de la malsano nomita saĝeco estas ke vi komencas ridi pro
aĵoj kiujn neniu alia pensas esti komikaj, ĉar kiam vi estas saĝa, vi
komencas kompreni la ŝercojn!'' La maljuna sorĉisto forviŝis larmojn
el siaj okuloj. ``Ha, mi. Ha mi. Ofte la malbonvolo difektas la
Malbonon fakte, en vera ago.''

La cerbo de Harry prenis momenton por reloki la konatajn vortojn\ldots
``Hey, tio estas citaĵo de \emph{Tolkien}! \emph{Gandalfo} diris
tion!''

``Teodeno, fakte,'' diris Dumbledore.

``Vi estas \emph{Mugle naskita}?'' Harry diris ŝoke.

``Mi timas ne,'' diris Dumbledore, ridetante denove. ``Mi naskis
sepdek jaroj antaŭ kiam tiu libro eldoniĝis, kara infano. Sed ŝajnas
ke miaj Mugle naskitaj studentoj emas pensi same, je iaj manieroj. Mi
amasigis ne malpli ol dudek kopiojn de la Mastro de l'Ringoj kaj tri
aroj de la kompletaj verkoj de Tolkien, kaj mi zorge konservis ĉiun el
ili.'' Dumbledore levis sian bastonon kaj tenis ĝin rekte kaj prenis
pozon. ``\emph{Vi ne pasu!} Kiel tio aspektas?''

``Ha,'' Harry diris per io kio proksmimiĝis kompleta cerba ĉeso. ``Mi
pensas ke mankas al vi Balrogon.'' Kaj ka roza piĵamo kaj premita
ĉampinjona ĉapelo tute ne estis helpanta.

``Mi vidas.'' Dumbledore suspiris kaj malgaje remetis la bastonon al
sia zono. ``Mi timas ke estis nur malmulte da Balrogoj en mia vivo
lastatempe. Nuntempe, estas sole kunveno de la Magekoncilio kie mi
devas provi senespere malhelpi iun ajn laboro de esti farita, kaj
formalaj vespermanĝoj kie fremdaj politikistoj konkuras por vidi kiu
povas esti la plej obstina stultulo. Kaj esti mistera al homoj, scii
aĵojn, pri kiuj mi havas neniu maniero por scii, fari kriptajn frazojn
kiuj povas nur esti komprenita retrospekte, kaj ĉiuj la aliaj etaj
manieroj per kiuj potencaj sorĉistoj amuzas ilin post kiam ili eliris
la parton de la skemo, kiu permesas al ili esti herooj. Parolante pri
tio, Harry, mi havas ion por doni al vi, io kio apartenis al via
patro.''

``Vere?'' diris Harry. ``Dio, kiu povintus scii.''

``Jes, efektive,'' diris Dumbledore. ``Mi supozas ke tio estas iom
antaŭvidebla, ĉu ne?'' Lia vizaĝo iĝis solena. ``Malgraŭe\ldots''

Dumbledore reiris al sia skribtablo kaj sidiĝis, tirante unu el la
tirkestoj. Li metis siajn du brakojn in la tirkesto, kaj, kun leĝera
peno, levis relative larĝan kaj aspektante pezan objekton el la
tirkesto, kiu li tiam metis sur la kverka skribtablo kun surdiganta
sono.

``Tiu ĉi,'' Dumbledore diris, ``estis la roko de via patro.''

Harry fiksrigardis ĝin. Ĝi estis hele griza, senkoloriĝinta, de
neregula formo, kun akraj randoj, kaj vere aspektante kiel tute
ordinara roko maljuna. Dumbledore estis metinta ĝin tiel maniere ke ĝi
kuŝis sur la plej larĝa sekco disponebla, sed ĝi ankoraŭ ŝanceliĝis
malstabile sur la skribtablo.

Harry levis la rigardon. ``Tio estas serĉo, ĉu?''

``Tio ne estas,'' diris Dumbledore, kapneante, kaj aspektante tre
serioza. ``Mi prenis tiun el la ruinoj de la hejmo de James kaj Lily
en Kavo de Godriko, kie ankaŭ mi trovis vin; kaj mi gardis ĝin ekde
tiam, ĝis nun, ĝis hodiaŭ la tago kiam mi povas doni ĝin al vi.''

En la miskaĵo de hipotezoj kiu servis kiel la modelo de la universo de
Harry, la frenezeco de Dumbledore estis rapide kreskinta en
propableco. Sed \emph{estis} ankoraŭ konsiderinda kvanto da propableco
kvotigita al aliaj aternativoj\ldots ``Hum, ĉu tiu estas \emph{magia}
roko?''

``Tiom, kiom mi scias, ĝi ne estas,'' diris Dumbledore, ``sed mi
konsilas vin kun la plej eble rigoreco de gardi ĝin proksime de via
persono konstante.''

Konsentite. Dumbledore estis \emph{propable} freneza, sed se li
\emph{ne estis}\ldots nu, ĝi estus simple tro \emph{embarasa} akiri
problemojn pro ke vi ignoris la konsilon de la nesondebla maljuna
sorĉisto. Tio devis estis kiel la \#4 sur la listo de la 100 plej
evidentaj manieroj de malsukcesi.

Harry marŝis antaŭen kaj metis siajn manojn sur la roko, provante
trovi angulon de kiu li povis levi ĝin sen tranĉi sin mem. ``Mi metos
ĝin en mia haŭtpoŝo, tial.''

Dumbledore malridetis. ``Tio povas esti ne sufiĉe proksima de via
persono. Kaj kio okazus se via moke haŭtpoŝo estus perdita, aŭ ŝtelita?''

``Vi pensas, ke mi devus porti grandan rokon ĉie kie mi iras?''

Dumbledore donis al Harry seriozan aspekton. ``Tio verŝajne montriĝus saĝa.''

``Ha\ldots'' Harry diris. Ĝi aspektis relative peza. ``Mi opinias ke
la aliaj studentoj povus emi fari demandojn pri tio?''

``Diru al ili ke mi ordonis vin fari tion,'' diris Dumbledore. ``Neniu
dubos tion, pro tio ke ili ĉiu pensas ke mi estas freneza.'' Lia
vizaĝo estis ankoraŭ perfekte serioza. 

``Er, por esti honesta se vi emas ordoni al viaj studentoj porti
larĝan rokon mi relative povas vidi kial homoj povas pensi tion.'' 

``Ha, Harry,'' diris Dumbledore. La majuna sorĉisto gestis, balaito de
unu mano kiu ŝajnis enpreni ĉiujn la misterajn instrumentojn ĉirkaŭ la
ĉambro. ``Kiam oni estas juna, oni kredas ke oni scias ĉion, kaj tial
oni kredas ke se oni vidas neniun klarigon por io, tial nenio klarigo
ekzistas. Kiam oni estas pli maljuna, oni ekkonscias ke la tuta
universo funkcias laŭ ritmo kaj racio, eĉ se oni mem ne konas
ĝin. Estas nur via nescio kiu aperas al ni kiel frenezeco.''

``Realaĵo ĉiam sekvas leĝon,'' diris Harry, ``eĉ se oni ne konas la leĝon.''

``Precize, Harry,'' diris Dumbledore. ``Kompreni tion—kaj mi vidas ke
vi \emph{komprenas} ĝin—estas la esenco de la saĝeco.'' 

``Do\ldots \emph{Kial} mi devas porti tiun rokon ekzakte?''

``Mi ne povas pensi pri kialo, fakte,'' diris Dumbledore.

``\ldots vi ne povas.''

Dumbledore kapjesis. ``Sed nur ĉar mi ne povas pensi pri kialo ne
volas diri ke estas neniu kialo.''

La intrumentoj tiktakadis.

``Okej,'' Harry diris, ``Mi ne eĉ certas se mi devus diri tion, sed
tio estas simple ne la prava maniero de trakti kun nia konsentita
nescio pri kiel la universo funkcias.''

``Tio ne estas?'' diris la maljuna sorĉisto, aspektante surprizita kaj seniluziginta.

Harry havis la senton ke tiu konversacio ne turniĝos al sia favoro,
sed li daŭrigis malgraŭ. ``Ne, mi ne eĉ scias se tiu sofismo havas
oficialan nomon, sed se mi devus krei iun per mi mem, ĝi estus
'privilegi la hipotezon' aŭ io kiel tio. Kiel mi povas diri tion
formale\ldots hum\ldots supozu ke vi havas milionon da skatoloj, kaj
nur unu el tiuj skatoloj enhavas diamanton. Kaj vi havas skatolon
plenplena de detektiloj de diamanto, kaj ĉiu detektilo ĉiam eksonoras
kiam ĝi estas en la ĉeesto de diamanto, kaj eksonoras unu fojo el du kiam
ili estas sur skatolo kiu ne havas diamanton. Se vi pasus dudek
detektilojn sur ĉiuj la skatoloj, vi havus, averaĝe, unu malpravan
kandidaton kaj unu pravan kandidaton. Kaj poste vi bezonos nur unu aŭ
du pliajn detektilojn por ke al vi restas nur la unu veran
kandidaton. Temas pri tio, ke kiam estas multe da eblaj respondoj, la
plejparto de la pruvoj kiujn vi bezonas, servas nur por loki la veran
hipotezon el la milionoj da ebloj—alporti ĝin al via atento unue. La
kvanto de pruvoj, kiujn vi bezonas por juĝi inter du aŭ tri eblaj
kanditatoj estas multe pli malgranda kompare. Do se vi nur komencus
sen pruvo kaj antaŭenigas unu particularan eblon sur kiu koncentri
vian antenton, vi preterpasus la plejparto de la laboro. Kiel, vi
vivas en urbo kie estas miliono da homoj, kaj estas murdo, kaj
detektivoj diras, nu, ni ne havas iun ajn pruvojn, do ĉu ni konsideris
la eblon ke Mortimer Snodgrass faris ĝin?''

``Ĉu li faris?'' diris Dumbledore.

``Ne,'' diris Harry. ``Sed pli malfrue, aperis ke la murdinto havis
nigran hararon, kaj Mortimer havas nigran hararon, do ĉiuj estas kiel,
ha, ŝajnas ke Mortimer faris ĝin finfine. Do tio estas ne justa por
Mortimer ke la polico \emph{antaŭenigis lin al ilia atento} sen jam
havi bonajn kialojn por suspekti lin. Kiam estas multe da ebloj, la
plejparto de la laboro servas nur por \emph{loki} la vera
respondo—komenci antenti ĝin. Vi ne bezonas pruvon, aŭ la specon de
oficiala atestaĵo kiun sciencistoj aŭ la tribunalo demandas, sed vi
bezonas specon de \emph{sugesto}, kaj tiu sugesto devas distingi tiun
partikularan eblon de la milionoj aliaj. Alikaze, vi ne povas nur
trovi la bonan respondon el nenie. Vi ne povas trovi eblon kiu valoras
esti konsiderita el nenie. Kaj estas milionoj da aliaj aĵoj kiujn mi
povas fari antastaŭ porti ĉie la rokon de mia patro. Nur ĉar mi
nescias pri la universo ne volas diri ke mi ne certas pri kiel mi
devus rezoni en la ĉeesto de mia necerto. La leĝo por pensi kun
probabloj ne estas malpli solida ol la leĝoj kiuj estras la bone
maljunan logikon, kaj kion vi ĵus faris estas nur \emph{ne
permesita}.'' Harry paŭzis. ``*Almenaŭ*, evidente, ke vi havas iun
sugeston kiun vi ne mencias.''

``Ha'', diris Dumbledore. Li frapetis sian vangon, aspektante
pripensema. ``Interesa argumento, certe, sed ĉu tio ne difektiĝis je
la momento kiam vi faris la analogion inter miliono da eventuala
murdinto inter kiuj nur unu faris la murdon, kaj elekti unu el multe da
eblaj vojoj de ago, kiam multe da eblaj vojoj de ago estus ĉiuj saĝaj? Mi
ne diras ke porti la rokon de via patro estas la plej bona ebla vojo
de ago, nur ke estas pli saĝa fari tion ol ne fari tion.''

Dumbledore ankoraŭ unu fojo metis siajn manojn en la sama skribtabla
tirkesto, en kiu li atingis pli frue, tiu fojo li ŝajnis enradikiĝi
ene—almenaŭ liaj brakoj ŝajnis ankoraŭ movi. ``Mi komentos,''
Dumbledore diris dum Harry estis ankoraŭ provanta trovi kion respondi
al la komplete neatendita repliko, ``tio estas komuna miskoncepto de
Korvungoj, ke ĉiuj la inteligentaj infanoj estas ordigitaj tie,
lasante neniu por la aliaj Domoj. Tio ne pravas; esti ordigita al
Korvungo indikas ke vi estas motivita per via deziro de scii aĵojn,
kio tute ne estas la sama kvalito ol esti inteligenta.'' La maljuna
sorĉisto estis ridetanta kiam li kliniĝis super la
tirkesto. ``Malgraŭe, vi \emph{ree} ŝajnas relative
inteligenta. Malpli kiel ordinara juna heroo, kaj pli kiel juna
mistera maljuna sorĉisto. Mi pensas ke mi povis preni malbonan
demarŝon kun vi, Harry, kaj ke vi eble kapablas kompreni aĵojn ke
malmulte da aliaj povas percepti. Do mi aŭdacos, kaj oferos al vi iun
\emph{alian} heredaĵo.''

``Vi ne volas diri\ldots'' anhelis Harry. ``Mia patro\ldots posedis
\emph{alian rokon?}''

``Pardonu min,'' diris Dumbledore, ``Mi \emph{estas} ankoraŭ pli
maljuna kaj pli mistera ol vi kaj se estas iu ajn revelacio por fari,
tial \emph{Mi} faros la revelavion, dankon\ldots ho, kie \emph{estas}
tiu aĵo!'' Dumbledore atingis malsupren plu en la skribtabla tirkesto,
kaj ankoraŭ plu. Li kapo kaj ŝultroj kaj tuta torso malaperis en ĝi
ĝis nur sia koskoj kaj kruroj restis ekstere, kvazaŭ la skribtabla
tirkesto estis manĝanta lin.

Harry ne povas malhelpi lin demandi sin kiom da aĵoj estis en tie kaj
kiel la kompleta inventaro aspektus.

Finfine Dumbledore reforiris el la tirkesto, tenante la celaĵon de sia
serĉo, kiun li metis sir la skribtablo apud la roko.

Ĝi estis uzita ĝis la osto, kun ĉifonitaj anguloj libro:
\emph{Meznivela Fabrikado de Pocio} per Libatius Borago. Estis bildo
de fumanta fiolo sur la kovrilo.

``Tiu ĉi,'' Dumbledore ekkantis, ``estis la pocia libro de kvina jaro
de via patrino.''

``Kiun mi devas preni kun mi tiutempe,'' diris Harry

``*Kiu enhavas teruran sekreton*. Sekreto kies revelacio povas
montriĝi tiel katastrofa, ke mi devas demandi vin ĵuri—kaj mi devigis
vin ĵuri serioze, Harry, kion ajn vi povas pensi pri ĉio tio—neniam
diri ĝin al iu ajn aŭ io ajn alia.''

Harry kontemplis la pocian libron de kvina jaro de sia patrino, kiu
verŝajne, enhavis teruran sekreton.

La problemo estis ke Harry reale prenis tian ĵuron vere serioze. Iu ajn
voto estis Nerompebla Voto se farita per la bona speco de persono.

Kaj\ldots

``Mi sentas min soifanta,'' Harry diris, ``kaj tio estas tute ne bona signo.''

Dumbledore komplete ignoris tion, kaj ne faris demandon pri tiu ĉi
kripta deklaro. ``*Ĉu* vi ĵuras, Harry?'' diris Dumbledore. Liaj
okuloj fiksrigardis atente en la okuloj de Harry. ``Alie, mi ne povus
diri ĝin al vi.''

``Jes,'' diris Harry. ``Mi ĵuras.'' Tio estis la problemo kiam vi
estis iu Korvungo. Vi ne povis rifuzi proponon kiel tia, aŭ vi scivolo
manĝus vin vivantan, kaj ĉiuj aliaj sciis tion.

``Kaj mi ĵuras laŭvice,'' diris Dumbledore, ``ke kio mi dirontas al vi
estas la vero.''

Dumbledore malfermis la libron, ŝajne hazarde, kaj Harry kliniĝis por vidi.

``Ĉu vi vidas tiujn ĉi notojn,'' Dumbledore diris per voĉo tiel
mallaŭta, ke ĝi estis preskaŭ murmuro. ``skribitaj sur la marĝenoj de
la libro?''

Harry fermis leĝere la okulojn. La flavaj paĝoj ŝajnis priskribi ion
nomita \emph{pocio de agla belegeco}, multe da la ingrediencoj estante
aĵojn kiujn Harry tute ne rekonis, kaj kies nomoj ne montriĝis deveni
el angla lingvo. Skribaĉite sur la marĝeno estis mane skribita
komentario dirante, \emph{Mi demandas min, kio okazus se mi uzus
sangon de Mallumalo\footnote{Mallumalo estas raso de alohava ĉevalo
kun skeleta korpo nomita \emph{Thestral} en angla lingvo} anstataŭ
mirteloj?} kaj tuj sube estis respondo per malsama manskribo. \emph{Vi
igus malsana dum semajnoj kaj eble mortus}.

``Mi vidas ilin,'' diris Harry. ``Kio estas pri ilin?''

Dumbledore celis al la dua skribaĉo. ``Tiuj kun tiu ĉi manskribo,'' li
diris ankoraŭ per tiu mallaŭta voĉo, ``estis skribitaj per via
patrino. Kaj tiuj kun \emph{tiu} manskribo'', movante sian fingron por
indiki la unuan skribaĉon, ``estis skribitaj per mi. Mi turnis min
mem nevideblan kaj ŝteliris en la dormejon dum ŝi estis dormanta. Lily
pensis ke unu el siaj amikinoj estis skribintaj ilin kaj ili havis la
plej mirigajn batalojn.''

Estis je tiu ekzakta momento, kiam Harry konsciis ke la Direktoro de
Herpŭrko \emph{estis}, fakte, freneza.

Dumbledore estis rigardanta lin kun serioza esprimo. ``Ĉu vi
komprenas la implikacion de tio, kion mi ĵus diris al vi, Harry?''

``Hee\ldots'' Harry diris. Lia voĉo ŝajnis enfermita. ``Pardonu\ldots
mi\ldots ne vere\ldots''

``Ha, nu,'' diris Dumbledore, kaj suspiris. ``Mi supozas ke via
inteligento havas limitojn fine. Ĉu ni povas nur pretendi ke mi ne
diris ion ajn?''

Harry stariĝis el sia seĝo, portante fiksan rideton. ``Evidente,''
Harry diris. ``Vi scias, estas fakte relative malfrue en la tago, kaj
mi estas iom malsata, do mi prefere iras malsupren por vespermanĝi''
kaj Harry marŝis direkte al la pordo.

La pordotenilon komplete malsukcesis turni.

``Vi vundas min, Harry,'' diris la voĉo de Dumbledore en mallaŭtaj
tonoj, kiuj venis de malantaŭ li. ``Ĉu vi almenaŭ konscias ke tio,
kion mi diris al vi estas signo de fido?''

Harry malrapide turniĝis.

Antaŭ li estis vere potenca kaj vere freneza sorĉisto kun longa
arĝenta barbo, ĉapelo kiel dispremita giganta ĉampinjono, kaj portante
tiun kiu aspektis por Muglaj okulok kiel tri tavoloj de hela roza
piĵamo.

Malantaŭ li estis pordo kiu ne ŝajnis funkcianta por la momento.

Dumbledore aspektis relative malgajigita kaj laca, kvazaŭ li volis
kliniĝi sur bastonon de sorĉisto kiun li ne havis. ``Vere'', diris
Dumbledore, ``vi provas ion novan anstataŭ sekvi la saman skemon ekde
cent kaj dek jaroj, kaj homoj komencas forkuri.'' La maljuna sorĉisto
skuis sian kapon malĝoje. ``Mi esperus pli bonan el vi, Harry
Potter. Mi aŭdis ke viaj propraj amikoj ankaŭ pensas ke vi estas
freneza. Mi scias ke ili ne pravas. Ĉu vi ne kredos la saman por mi?''


``Bonvolu, malfermu la pordon,'' Harry diris, sia voĉo tremante. ``Se
vi volas ke me fidas vin iam denove, malfermu la pordon.''

Estis sono de pordo kiu malfermiĝis malantaŭ li.

``Estis pliaj aĵoj pri kiuj mi planis paroli kun vi,'' Dumbledore
diris, ``kaj se vi foriras nun, vi ne scios tiojn, kioj ili estas.''

Kelkefoje Harry absolute \emph{malamis} esti Korvungo.

\emph{Li neniam dolorigis studenton} diris la Grifindora parto de
Harry. \emph{Nur gardu tion mense, kaj vi certos ke vi ne paniku. Vi ne
  forkuros nur ĉar aĵoj komencas iĝi interesaj, ĉu ne?}

\emph{Vi ne povas forlasi la direktoron!} diris la Huflopufa
parto. \emph{Kaj se li elprenas Domajn poentojn? Li povas farigi vian
lernejan vivon tre malfacila se li decidas ke li ne amas vin!}

Kaj peco de li mem kiun Harry ne multe amas, sed ne povis tute sukcese
silentigi estis konsideranta la eblajn advantaĝojn de esti unu el la
malmultaj amikoj de tiu ĉi maljuna sorĉisto freneza kiu ankaŭ troviĝis
esti la Direktoro, la Estra Magiisto, kaj Supera Dieculo. Kaj
bedaŭrinde lia ena Serpentimo ŝajnis multe pli bona ol Drako por turni
homoj al la Malhela Flanko. \emph{Li aspektas kiel iu kiu bezonas iun
por paroli, ĉu ne?} kaj \emph{vi ne volas ke tia potenca viro finas
per fidi iun malpli virta ol vi, ĉu?} kaj \emph{mi scivolas kiajn
nekredeblajn sekretojn Dumbledore povus al vi diri se, vi scias, vi
amikiĝus kun li} kaj eĉ \emph{mi vetas ke li havas veeerre interesan
kolekton de libroj.} 

\emph{Vi estas nur fakso de lunatikuloj}, Harry pensis al la integra
aro, sed li estis unuanime balotvenkita per ĉiuj komponantaj partoj de
li mem.

Harry turniĝis, faris paŝon al la malfermita pordo, levis la brakon,
kaj intence fermis ĝin denove. Ĝi estis nekosta ofero pro tio ke li
restos ĉiaokaze, Dumbledore povis kontroli liajn movojn, sed eble ke
tio impresus Dumbledore'n.

Kiam Harry turniĝis li vidis ke la potenca sorĉisto freneza estis
ankoraŭ unu fojo ridante kaj aspektante amike. Tio estis bona, eble.

``Bonvolu, ne faru tion denove,'' Harry diris. ``Mi ne ŝatas estis kaptita.''

``Mi \emph{bedaŭras} pri tio, Harry'' diris Dumbledore en kiu aspektis
kiel tono de sincera pardonpeto. ``Sed ĝi estus terure nesaĝa lasi vin
foriri sen la roko de via patro.''

``Evidente,'' Harry diris. ``Estis ne racia per mi atendi ke la pordo
malfermiĝas antaŭ kiam mi metas la objektojn de serĉado en mia
inventaro.''

Dumbledore ridetis kaj kapjesis?

Harry reiris ĉe la skribtablo, rotacigis sian moke haŭtpoŝon ĉirkaŭ al
la fronto de sia zono, kaj, kun iu eforto, sukcesis eklevi la rokon en
siaj dekunu jaraĝaj brakoj kaj manĝigis sian haŭtpoŝon.

Li povis fakte senti la pezon malgrandigi dum la Larĝigeblan Lipo
ĉarmo manĝis la rokon, kaj la ruketo kiu sekvis estis relative brua
kaj havis klare plendantan sonon kune. 

La Pocia libro de kvina jaro de lia patrino (kiu enhavis sekreton kiu
estis fakte tre terura) sekvis mallonge pli malfrue.

Kaj poste la interna Serpentimo de Harry faris ruzan sugeston por
farigi sin akceptita per la Direktoro, kiu, bedaŭrinde estis per ia
maniere tiel perfekte prezentita, ke ĝi akiris la apogon de la
plejparto de la fakcio Korvungo.

``Do,'' Harry diris, ``Hum. Dum mi estas ĉirkaŭ tie, mi supozas ke vi
ŝatas vizitigi al mi vian oficejon? Mi estas iom scivolema pri kio
kelkaj de tiuj aĵoj estas,'' kaj tiu estis lia plej granda eŭfemismo
de la monato de septembro.

Dumbledore fiksrigardis lin, kaj poste kapjesis kun leĝera
rideto. ``Mi estas flatita per via intereso,'' diris Dumbledore, ``sed
mi timas ke ne estas tiom por diri.'' Dumbledore proksimiĝis de la
muro per unu paŝo kaj montris pentaĵo de dormanta viro. ``Tiuj estas
portretoj de estintaj Direktoroj de Herpŭrko.'' Li turniĝis kaj
montris sian skribtablon. ``Tiu estas mia skribtablo.'' Li montris
sian seĝon. ``Tiu estas mia seĝo—''

``Pardonu min,'' Harry diris, ``Fakte mi estis scivolema pri tiuj.''
Harry montris etan kubon kiu estis kviete murmuranta ``blorpel\ldots
blorpel\ldots blorpel''.

``Oh, la etaj bagatelaj aĵoj?'' diris Dumbledore. ``Ili venis kun la
Direktora oficejo, kaj mi havas absolute neniun ideon pri kion la
plejparto de tiuj faras. Tamen \emph{tiu} ciferplato kun la ok
montriloj nombras la nombro de, nomu ni ilin ternoj, per maldekstra
sorĉistinoj inter la landlimo de Francujo, vi ne kredus kiom da laboro
ĝi prenis por fari ĝin. Kaj \emph{tiu} ĉi kun la oraj skuiĝantoj estas
mia propra invento kaj Minerva neniam, iam ajn malkonvros
kion ĝi faras.''

Dumbledore faris paŝon al la ĉapela bretaro dum Harry estis ankoraŭ
absorbanta tiujn informojn. ``Tie evidente ni havas la Ordigantan
Ĉapelon, mi kredas ke la ambaŭ de vi jam renkontis. Ĝi diris al mi ke
li devas neniam esti remetita sur via kapo ne gravas la
cirkonstancoj. Vi estas nur la dekkvara studento pri kiu li diris
tion, Baba Yaga estis unu el la aliaj kaj mi diros al vi pri la aliaj
dekdu kiam vi estos pli maljuna. Tio estas pluvombrelo. Tiu estas alia
pluvombrelo.'' Dumbledore faris kelkajn aliajn paŝojn kaj turniĝis,
nun ridetante tute larĝe. ``Kaj evidente, la plejparto da homoj kiuj
venas en mia oficejo volas vidi Faŭkse.''

Dumbledore staris apud la birdo sur la ora platformo.

Harry proksimiĝis, relative konfuzita. ``Tiu estas Faŭkse?''

``Faŭkse estas fenikso,'' diris Dumbledore. ``Tre malabunda, tre
potenca magia estulo.''


``Ha\ldots'' Harry diris. Li mallevis sian kapon kaj rigardis fikse la
etajn, perlajn okulojn nigrajn, kiuj ne montris la plej leĝeran signon
de potenco aŭ inteligento.

``Haa\ldots'' Harry diris denove.

Li estis relative certa ke li rekonis la formo de la birdo. Tiu estis relative malfacila por manki.

``Humm\ldots''

\emph{Diru ion itenligentan!} La menso de Harry roris al si
mem. \emph{Ne nur staru kaj sonu kiel babilada piĉo!}

\emph{Nu, kion diable mi estas supozita diri?} La menso de Harry replikis.

\emph{Io ajn!}

\emph{Vi volas diri, io ajn krom ``Faŭkse estas koko'' —}

\emph{Jes, io ajn krom tio!}

``Do, hu, kian specon de magio fenikso faras, tial?''

``Iliaj larmoj havas la povon de kuraci,'' Dumbledore diris. ``Ili
estas estulo de fajro, kaj movas inter lokoj tiel facile kiel fajro
povas estingi ĝin mem en iu loko kaj ekbruligas en alia. La grandega
premo de ilia natura magio maljunigas ilian korpon rapide, kaj jam ili
estas tiel proksime de senmorta kiel iu ajn estulo kiu ekzistas en la
mondo, ĉiu fojo kiam ilia korpo maltenas ilin, ili sin mem bruligas en
eksplodo de fajro kaj lasas malantaŭ ili kokidon aŭ kelkefoje ovon.''
Dumbledore proksimiĝis kaj inspektis la kokon,
malridetante. ``Hm\ldots aspektanta iom pala, mi dirus.''

Antaŭ ol tiu deklaro tute registriĝis en la menso de Harry, la koko
estis jam brulanta.

La beko de la koko malfermiĝis, sed ĝi ne havis tempon por fari pli ol
nur sola grako antaŭ ĝi komencis velki kaj bruleti. La brulego estis
mallonga, intensa, kaj tute mem-enhavita; estis neniu sento de brulo.

Kaj poste la fajro mortis nur kelkaj sekundoj post ĝi ekkomencis,
lasante malantaŭ ĝi nur etan, mizeran amason de cindroj sur la ora
platformo.

``Ne aspektu tiel konsternita, Harry!'' diris Dumbledore. ``Faŭkse ne
estis dolorigita.'' la mano de Dumbledore trempis e sian poŝon, kaj
poste la sama mano kribis tra la cindroj kaj levis etan flavetan
ovon. ``Rigardu, jen ovo!''

``Ho\ldots ŭaŭ\ldots nekredebla\ldots''

``Sed nun, ni vere devus daŭrigi niajn aferojn,'' Dumbledore
diris. Lasante la ovon malantaŭ en la cindroj de la koko, li reiris al
sia trono kaj sidiĝis. ``Estas preskaŭ la horo de la vespermanĝo,
finfine, kaj ni ne volus bezoni uzi niajn Tempajn Returnilojn.''

Estis violenta lukto de povo en la registaro de Harry. Serpentimo kaj
Huflopufo interŝanĝis flankojn post kiam ili vidis la Direktoro
de Herpŭrko ekbruligi kokon.

``Jes, aferoj,'' diris la lipoj de Harry. ``Kaj poste verspermanĝo''

\emph{Vi sonas kiel babilada piĉo denove} observigis la Interna
Kritikisto de Harry.

``Nu,'' Dumbledore diris.''Mi timas havi konfeson por fari,
Harry. Konfeson kaj pardonpeton.''

``Pardonpetoj estas bonaj'' \emph{tio ne eĉ volas diri ion! Kion mi estas diranta?}

La maljuna sorĉisto suspiris profunde. ``Vi eble ne pensus tiel, post
esti kompreninta tion, kion mi devas diri. Mi timas, Harry, ke mi
manipulis vin dum tuta via vivo. Estis mi kiu konfidis vin al la zorgo
de viaj malvirtaj vicgepatroj—''

``Miaj vicgepatroj ne estas malvirtaj!'' elbabilis Harry. ``Miaj \emph{gepartoj}, mi volas diri!''

``Ili ne estas?'' Dumbledore diris, aspektante surprizita, kaj
seniluziita. ``Ne eĉ iom malvirtaj? Tio ne konformas al la
skemo\ldots''

La interna Serpentimo de Harry kriis je plenaj mensaj pulmoj,
\emph{SILENTU, VI STULTULO, LI FORPRENOS VIN DE ILI!}

``Ne, ne,'' diris Harry, kun la lipoj frostigitaj en abomena
grimaco. ``Mi nur provis ŝpari viajn sentojn , ili fakte estas tre
malvirtaj\ldots''

``Ili estas?'' Dumbledore kliniĝis antaŭen, fiksrigardante lin
atente. ``Kion li faras?'' 

\emph{Parolu rapide}. ``Ili, ha, mi devas lavi la manĝilaron, kaj fari
la lesivon, kaj ili ne lasas min legi multe da libroj kaj—''

``Ha, bona, tio estas bona por aŭdi,'' diris Dumbledore, kliniĝis
malantaŭen denove. Li ridetis iel malgaja. ``Mi pardonpetas pro tio,
tial. Nun, kie mi estis? Ha, jes. Mi bedaŭras diri, Harry, ke mi estas
respondeca por virtuale ĉio malbona kiu iam ajn okazis al vi. Mi scias
ke tio probable farigas vin tre kolera.''


``Jes, mi estas tre kolera!'' diris Harry. ``Grrr!''

La interna kritikisto de Harry prompte premiis lin kun la Ĉiu-Tempo
Premio de la plej Malbona Aktorado en la Historio de Iam ajn.

``Kaj, mi nur volis ke vi scias,'' Dumbledore diris, ``Mi volis diri
al vi kiel eble plej rapide, nur en la okazo se io okazas al unu el ni
pli malfrue, ke mi vere, vere bedaŭras. Por ĉio kio jam okazis al vi,
kaj ĉio kio okazos.''

Humideco brilis sur la okuloj de la maljuna sorĉisto.

``Maj mi estas tre kolera!'' diris Harry. ``Tiel kolera ke mi volas
foriri tuj almenaŭ ke vi havas ion alian por diri al mi!''

\emph{Nur foriry, antaŭ kiam li ekbruligas vin!} kriegis Serpentimo, Huflopufo kaj Grifindoro.

``Mi komprenas,'' diris Dumbledore. ``Unu lasta aĵo tial, Harry. Vi ne
devas provi eniri tra la malpermesita pordo je la koridoro de la tria
nivelo. Estas neniu ebla maniero ke vi povas pasi trans ĉiuj la
kaptiloj, kaj mi ne volas aŭdi ke vi doloris provante. Sed, mi dubas
ke vi povas eĉ pasi la unuan pordon, pro tio ke ĝi estas ŝlosita kaj vi
ne konas la sorĉon \emph{Alohomora}—''

Harry ekturniĝis kaj kuregis al la elirejo kiel eble plej rapide, la
pordotenilo turniĝis agrable en lia mano kaj poste li estis kuranta
malsupren laŭ la helika ŝtuparo dum ĝi turniĝis, liaj piedoj preskaŭ
stumblantaj sur ili mem, ĵus post mallonga momento li estis je la
malsupro kaj la gargojlo estis marŝanta flanke kaj Harry ĵetis sin el
la ŝtuparŝakto kiel kanonkuglo.

\begin{center}\rule{3in}{0.4pt}\end{center}

Harry Potter.

Devus esti io pri Harry Potter.

Estis ĵaŭdo por ĉiuj, finfine, kaj tiu speco de aĵo ne ŝajnas okazi al
iu ajn alia.

Estis ĵaŭdo je la 6:21 posttagmeze, kiam Harry Potter, ĵetis sin el la
ŝtuparŝakto kiel kanonkuglo kaj rapidiĝante kiel eble plej rapide,
kuris direkte en Minerva McGonagall kiam ŝi estis turnanta je la
angulo en ŝia vojo al la oficejo de la Direktoro.

Dankinde neniu el ili estis tre dolorigita. Kiel estis eksplikita al
Harry iomete pli frue en la tago—kiam li estis rifuzanta iri ie ajn
proksime de balailo denove—Kvidiĉo bezonis solidajn
Bastgojn\footnote{Bastgoj estas nigra fera pilko uzita en la sorĉista
sporto Kvidiĉo. Ĝi celas frapi la ludantojn. \emph{N.d.t}} el fero nur
por havi decan ŝancon de vundi la ludantojn, pro tio ke la sorĉistoj
emis esti multe pli rezistaj ol la Mugloj al frapfortoj.

Harry kaj Profesorino McGonagall ambaŭ finis kuŝanta sur la planko,
kaj la pergamenoj kiujn ŝi estis portanta iris ĉie ĉirkaŭ la koridoro.

Estis terura, terura paŭzo.

``Harry Potter,'' elspiris Profesorino McGonagall de kie ŝi kuŝis sur
la planko direkte apud Harry. Lia voĉo krekis al proksime de
kriego. ``\emph{Kion vi estis faranta en la oficejo de la
Direktoro?}''


``Nenio!'' blekis Harry.

``\emph{Ĉu vi estis parolanta pri la Defenca Profesoro?}''

``Ne! Dumbledore vokis min por iri supre tie, kaj li donis al mi tiun
ĉi grandan rokon kaj diris ke ĝi estis tiu de mia patro, kaj ke mi
devus porti ĝin kun mi ĉie!''

Estis alia terura paŭzo.

``Mi vidas,'' diris Profesorino McGonagall, lia voĉo estis iomete pli
kvieta. Ŝi stariĝis, brosis sin mem, kaj ĵetis rigardon al la
pergamenoj, kiuj saltis je neta stako kaj paŝetadis kontraŭ la muro de
la koridoro kvazaŭ por kaŝi sin de ŝia rigardo. ``Mian kompatemon,
S-ro Potter, mi pardonpetas pro esti dubinta vin.'' 

``Profesorino McGonagall,'' Harry diris. Lia voĉo estis tremanta. Li
puŝis sin mem desur la planko, stariĝis, kaj rigardis lian fidinda
\emph{sana} vizaĝo. ``Profesorino McGonagall\ldots''

``Jes, S-ro Potter?''

``Ĉu vi opinias ke mi devus?'' Harry diris per eta voĉo. ``Porti la
rokon de mia patro ĉie kun mi?''

Profesorino McGonagall suspiris. ``Tio estas inter vi kaj la
Direktoro, mi tion timas.'' Ŝi hezitis. ``Mi dirus ke ignori la
Direktoron komplete estas preskaŭ neniam saĝa. Mi \emph{estas}
bedaŭranta aŭdi vian dilemon, S-ro Potter, kaj se estas iu ajn maniero
por ke mi \emph{povas} helpi vin kun io ajn, kion vi decidos fari—''

``Hum,'' Harry diris. ``Fakte, mi pensis ke unu foje mi scios kiel, mi
povus Transfiguri la rokon en ringon kaj porti ĝin sur mia fingro. Se
vi povus intrui al mi kiel daŭrigi Transfiguron—'' 


``Estas bona ke vi demandis al mi unue,'' Profesorino McGonagall
diris, ŝia vizaĝo iĝante iom severa. ``Se vi perdus la kontrolon de la
Transfiguro, la retroirigo tranĉus vian fingron, kaj probable ŝiri
vian manon je duono. Kaj je via aĝo, eĉ ringo estas tro larĝa kiel
celo por ke vi povas daŭrigi la transfiguron senfine, sen ke ĝi estas
serioza dreno de via magio. Sed mi povas ekkapti ringon imititan por
vi kun loko por juvelo, \emph{eta} juvelo, en kontakto kun via haŭto,
kaj vi povas ekzerci vin daŭrigi sekuran subjekton, kiel
marŝmalo. Kiam vi estos daŭriginta ĝin sukcese, eĉ dum via dormo, dum
tuta monato, mi permesos vin transfiguri, ah, la rokon de via
patro\ldots'' La voĉo de Profesorino McGonagall mallaŭtiĝis. ``Ĉu la
direktoro \emph{vere}—''

``Jes. Ha, \ldots hum\ldots''

Profesorino McGonagall suspiris. ``Tio estas iom stranga eĉ por li.''
Ŝi klinis sin kaj kaptis la amason de pergamenoj. ``Mi bedaŭras pri
tio, S-ro Potter. Mi pardonpetas denove pro misfidi vin. Sed nun estas
mia propra vico por vidi la Direktoron.''

``Ha\ldots bonan ŝancon, mi imagas. Er\ldots''

``Dankon, S-ro Potter.''

``Hum\ldots''

Profesorino McGonagall marŝis al la gargojlo, neaŭdible diris la
pasvorton, kaj paŝis tra en la turniĝanta helika ŝtuparo. Ŝi komencis
kreski el vido, kaj la gargojlo komencis reiri al sia loko—

``\emph{Profesorino McGonagall la direktoro ekbruligis kokon!}''

``Li \emph{ki-}''




