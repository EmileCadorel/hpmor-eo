\chapter{Kredo en Kredo}

\lettrine[ante=“]{K}{aj} Janet estis skibo,” diris la portreto de la
juna virino malalta kun ore ornamita ĉapelo.

Drako tion skribis. Tiu estis nur la dudek oka, sed estis tempo reiri kaj kunveni kun Harry.

Li estis bezoninta demandi al aliaj portretoj ke ili helpu lin por
traduki, la angla multe ŝanĝiĝis, sed la plej maljunaj portretoj
priskribis sorĉojn de unua jaro, kiuj aspektis terure similaj al tiuj,
kiujn ili havis hodiaŭ. Drako rekonis duonon el ili, kaj la alia duono
ne aspektis pli potenca.

La malsana sento en lia stomako kreskis post ĉiuj respondoj, kiujn li
akiris, ĝis finfine, nekapabla elporti ĝin plu, li rezignis kaj
anstataŭ demandis al aliaj portretoj la strangan demandon de Harry pri
skibaj geedziĝoj. La kvin unuaj portretoj ne konis iun ajn, kaj fine
li demandis al tiuj portretoj demandi al \emph{iliaj} konatoj por ke
ili demandu al \emph{iliaj} konatoj, kaj li tiel sukcesis trovi
kelkajn homojn, kiuj fakte konfesis esti amikoj kun skiboj.

(La Serpentimo de unua jaro klarigis ke li estis laboranta sur grava
projekto kun iu Korvungo, kaj ke la Korvungulo diris al li ke ili
bezonis tiun informon, kaj poste forkuris sen diri kial. Tio rikoltis
multe da afablaj esprimoj.)

La piedoj de Drako estis pezaj dum li marŝis trans la koridoroj de
Herpŭrko. Li devintus kuri sed li ne ŝajnis kapabla koletigi la
energion. Li estis daŭranta pensi ke li ne volis scii pri tio, li ne
volis esti implikita en io ajn rilata al tio, li ne volis ke tio estu
sia respondeco. Harry Potter faru tion, se la magio estis velkanta,
Harry Potter zorgu pri tio\ldots

Sed Drako sciis ke tio ne estis bona.

Malvarma la karceroj de Serpentimo, verdaj la muroj el ŝtonoj, Drako kutime ŝatis etoso, sed nun ĝi ŝajnis tro multe kiel velkanta.

Lia mano sur la tenilon de la pordo, Harry Potter estis jam ene atendante, kaj portante sian mantelon kun kapuĉo.

``La malnovaj sorĉoj de unua jaro,'' Harry Potter diris. ``Kio vi trovis?''

``Ili ne estis pli potencaj ol la sorĉoj kiujn ni uzas hodiaŭ.''

La pugno de Harry frapis skribtablon, forte. ``Damne. Konsentite. Mia propra eksperimento estis malsukceso, Drako. Estas io nomita la Interdikto de Merlino—''

Drako frapis sin mem sur la fronto, ekkonsciante.

``—kiu malhelpas iun ajn akiri scion de potencaj sorĉoj el libroj, eĉ
se vi trovas kaj legas la notojn de potencaj sorĉistoj, ili ne havus
sencon por vi, ĝi devas iri de vivanta menso al alia. Mi ne povis
trovi iun ajn potencan sorĉon, pri kiu ni havas la instrukciojn sed
kiun ni ne povas ĵeti. Sed se vi ne povas akiri ilin de malnovaj
libroj, kial iun ajn ĝenus sin per transdiri ilin parole post ol ili
haltis funkcii? Ĉu vi akiris la datumojn pri la skibaj paroj?''

Drako komencis levi la pergamenon al Harry—

Sed Harry levis la manon rifuzante. ``Leĝo de scienco, Drako. Unue mi
diras al vi la teorio kaj la prognozo. Poste vi montras al mi la
datumojn. Tiel vi scios ke mi ne faras la teorion, por ke ĝi taŭgu; vi
scias ke la teorio fakte antaŭdiris la datumojn \emph{anticipe.} Mi
devas klarigi tion al vi ĉiaokaze, do mi devas klarigi tion
\emph{antaŭ} ol vi montras la datumojn al mi. Tio estas la regulo. Do
metu vian mantelon, kaj sidiĝu ni.''

Harry Potter sidiĝis je la skribtablo kun ŝiritaj pecoj de papero
disponitaj sur sia surfaco. Drako elirigis sian mantelon el sia sako,
metis ĝin, kaj sidiĝis je la alia flanko fronte al Harry, kaj donante
al la ŝiritaj pecoj de papero perpleksan rigardon. Ili estis
disponitaj en du vicoj kaj la vicoj estis longaj da ĉirkaŭ dudek
pecoj.


``La sekreto de la sango,'' diris Harry Potter, kun intensa esprimo
sur la vizaĝo, ``estas io nomita deoksiribonuklea acido. Vi ne diru
tiun nomon antaŭ iu ajn kiu ne estas sciencisto. Deoksiribonuklea
acido estas la recepto kiu diras al via korpo kiel kreski, du kruroj,
du brakoj, malalta aŭ alta, ĉu vi havas brunaj okuloj aŭ verdaj. Tio
estas materiala aĵo, vi povas \emph{vidi} ĝin se vi havas mikroskopon,
kiuj estas kiel teleskopo, krom ke ĝi rigardas aĵojn kiuj estas tre
etaj anstataŭ tre foraj. Kaj la recepto havas du kopiojn de ĉiuj, ĉiam,
pro la okazo se unu kopio estas rompita. Imagu du longajn vicojn da
pecoj de papero. Je ĉiu loko en la vico, estas du pecoj de papero, kaj
kiam vi havas infanon, via korpo elektas unu pecon de papero hazarde de
ĉiuj lokoj en la vico, kaj la korpo de la patrino faras la saman, kaj
do la infano ankaŭ akiras du pecojn da papero je ĉiuj lokoj en la
vico. Du kopioj de ĉiuj, unu de la patrino kaj unu de la patriĉo, kaj
kiam vi havas infanon, ri akiras unu pecon de papero de vi hazarde je
ĉiuj lokoj.''

Dum Harry parolis, liaj fingroj pasis super la parigitaj pecoj de
papero, montrante unu parton de la paroj kiam li diris ``de la
patrino'', la alian kiam li diris ``de la patriĉo''. Kaj dum Harry
parolis pri elekti pecon de papero hazarde, lia mano ekkaptis knuton
el sia robo kaj ĵetis ĝin, kaj tiam montris la supran pecon de
papero. Ĉio tio sen paŭzo en lia parolado.

``Nun, kiam temas pri io kiel esti malalta aŭ alta, estas multe da
lokoj en la recepto kiuj faras \emph{malgrandajn} diferencojn. Do se
alta patriĉo edziĝas kun malalta patrino, la infano akiras kelkajn
pecojn de papero dirantaj 'alta' kaj kelkajn dirantaj 'malalta', kaj
kutime la infano fariĝas modere alta. Sed ne ĉiam. Kun ŝanco, la
infano eble akiros multe da pecoj dirantaj 'alta', kaj ne tiel multe
da paperoj dirantaj 'malaltaj', kaj kreskos relative granda. Vi povas
havi altan patriĉon kun kvin paperoj dirantaj 'alta' kaj altan
patrinon kun kvin paperoj dirantaj 'alta' kaj kun miriga ŝanco, la
infano akiros ĉiujn el tiuj dek paperoj dirantaj 'alta' kaj fariĝos
pli alta ol ambaŭ sia gepatroj. Ĉu vi vidas? La sango ne estas
perfekta fluaĵo, ĝi ne miskiĝas perfekte. Deoksiribonuklea acido estas
farita el multe da etaj pecoj, kiel glaso da ŝtonetoj anstataŭ glaso
da akvo. Tio estas kial infano ne estas ĉiam ekzakte je la mezo de
siaj gepatroj.''

Drako aŭskultis kun la buŝo malfermita. Kiel, por la nomo de Merlino, la mugloj malkovris tion? Ili povis \emph{vidi} la recepton?

``Nun,'' Harry Potter diris, ``supozu ke, ekzakte kiel la alteco,
estas multe da etaj lokoj en la recepto, kie vi povas havi pecon de
papero kiu diras 'magia' aŭ 'ne-magia'. Se vi havas sufiĉe da pecoj de
papero dirantaj 'magia', vi estas sorĉisto, se vi havas multe da tiuj
pecoj de papero vi estas potenca sorĉisto, se vi havas tro malmulte vi
estas muglo, kaj inter tiuj du, vi estas skibo. Tiam, kiam du skiboj edziĝas,
plejofte la infanoj devus esti ankaŭ skiboj, sed kelkfoje infano estas
bonŝanca kaj akiras la plejparton de la magiaj paperoj de sia patriĉo
\emph{kaj} la plejparton de la magiaj paperoj de sia patrino, kaj
fariĝas sufiĉe forta por esti sorĉisto. Sed probalbe ne tre
potenca. Se vi komencis kun multe da potencaj sorĉistoj kaj ke ili
edziĝis nur inter ili, ili restus potencaj. Sed se ili komencis edziĝi kun
Mugle naskitaj kiuj estis nur preskaŭ magiaj, aŭ skiboj\ldots{}vi
vidas? La sango ne miskiĝus perfekte, tio estus glaso da ŝtonetoj, ne
glaso da akvo, ĉar tio estas kiel la sango funkcias. Estus ankoraŭ
potencaj sorĉistoj nun, kiam ili akiras multe da magiaj paperoj pro
ŝanco. Sed ili ne estus tiel potencaj kiel la plej potencaj sorĉistoj
de la estinteco.''

Drako kapjesis malrapide. Li neniam aŭdis ĝin klarigita laŭ tiu
maniero antaŭe. Estis miriga beleco pri kiel tio taŭgis.

``\emph{Sed,}'' Harry diris. ``Tio estas nur \emph{unu}
hipotezo. Supozu ke anstataŭ estas nur unu \emph{sola} loko en la
recepto kiu farigas vin esti sorĉisto. Nur \emph{unu} loko, kie peco
de papero povas diri 'magia' aŭ 'ne-magia'. Kaj estas du kopioj de
ĉio, ĉiam. Do, tiam estas nur tri ebloj. Ambaŭ la kopioj povas diri
'magia'. Unu kopio povas diri 'magia' kaj unu kopio povas diri
'ne-magia'.  Aŭ ambaŭ la kopioj povas diri 'ne-magia'. Sorĉistoj,
Skiboj, kaj Mubloj. Du kopioj kaj vi povas ĵeti sorĉojn, unu kopio kaj
vi povas ankoraŭ uzi pociojn kaj magiajn ilojn, kaj neniu kopio volas
diri ke vi eĉ havas problemojn por rigardi magion direkte. Mugle
naskita uloj ne vere naskiĝus Mugle, ili naskiĝus de du Skiboj, du
gepatroj ambaŭ havanta unu kopion de magia papero, kiuj kreskis en la
Mugla mondo. Nun imagu ke sorĉistino edziĝas kun skibiĉo. Ĉiuj infanoj
akiros unu paperon diranta 'magia' de la patrino, ĉiam, ne gravas kiu
peco estus elektita hazarde, ambaŭ dirantaj 'magia'. Sed kiel ĵeti
moneron, duono de la tempo la infano akirus paperon diranta 'magia' de
la patriĉo, kaj duono de la tempo la infano akirus de la patriĉo
paperon diranta 'ne-magia'. Kiam sorĉistino edziĝas kun skibiĉo, la
rezulto ne estus multe da malpotencaj sorĉistaj infanoj. Duono de la
infanoj estus sorĉistoj ekzakte same potencaj ol ilia patrino, kaj
duono de la infanoj estus skiboj. Ĉar se estus nur \emph{unu} loko en
la recepto, kiu farigas vin esti sorĉisto, tiam la magio ne estas kiel
glaso da ŝtonetoj kiuj ne povas miskiĝi. Ĝi estas kiel sola magia
ŝtono, iu ŝtono de sorĉisto.''

Harry disponis tri parojn da paperoj flanke al flanke. Sur unu paro li skribis 'magia' kaj 'magia'. Sur alia paro li skribis 'magia' nur sur la supra papero. Kaj la trian paron blanka li lasis.

``En kiu okaza,'' Harry diris, ``ĉu vi havas do ŝtonoj aŭ ne.  Ĉu vi
estas sorĉisto aŭ ne. Potencaj sorĉistoj fariĝus tiele per studi
diligente kaj trejni pli. Kaj se sorĉistoj fariĝas \emph{esence}
malpli potencaj, ne ĉar la sorĉoj estas perditaj sed ĉar homoj ne
povas ĵeti ilin\ldots{}tiam eble ili manĝas la malbonan manĝaĵon aŭ io
kiel tio. Sed se tio fariĝis konstante pli malbona dum ok cent jaroj,
tiam tio povas diri ke la magio si mem estas velkanta.''

Harry disponis du alian paron da folio, flanke al flanke, kaj prenis
plumon. Baldaŭ ĉiuj paroj havis unu peco de papero diranta 'magia' kaj
la aliaj diranta nenion.

``Kaj tio alvenigas min al mia prognozo,'' diris Harry. ``Kio okazas
kiam du skiboj edziĝas. Ĵetu moneron du fojon. Ĝi povas fari
antaŭflankon kaj antaŭflankon, antaŭon kaj dorson, aŭ dorson kaj
dorson. Do kvarono de la tempo, vi akiras du antaŭon, kvarono de la
tempo vi akiras du dorson, kaj duono de la tempo vi akiras unu dorson
kaj unu antaŭon. Same se du skiboj edziĝas. Kvarono de la tempo la
infanoj akiros magian kaj magian, kaj estos sorĉisto. Kvarono de la
tempo ri akiros ne-magian kaj ne-magian, kaj fariĝos muglo. La alia
duono ri estos skibo. Tio estas tre maljuna kaj vere klasika skemo. Ĝi
estas malkovrita per Gregor Mendel, kiu ne estas forgesita, kaj tio
estis la unua aludo iam malkovrita pri kiel la recepto funkcias. Iu
ajn kiu konas ion pri sanga scienco rekonus tiun skemon
senprokraste. Gi ne estus ekzakta, ne pli ol ke se vi ĵetus moneron du
fojon, kvardek fojon, vi ne akirus ekzakte dek parojn de du
antaŭon. Sed se estus sep aŭ tridek sorĉistoj el kvardek infanoj, tio
estus forta indiko. Tio estas la testo, kiun mi demandis al vi por
fari. Nun, rigardu ni la datumojn.''

Kaj antaŭ ol Drako sukcesis eĉ pensi, Harry Potter estis preninta la pergamenon el la mano de Drako.

La gorĝo de Drako estis tre seka.

Du dek ok infanoj.

Li ne tute certis pri la ekzakta nombro sed li preskaŭ certa ke ĉirkaŭ la kvarono estis sorĉistoj.

``Ses sorĉistoj el du dek ok infanoj,'' Harry Potter diris post eta momento. ``Nu, ĝi do estas tio. Kaj infanoj de unua jaro ĵetis la samajn sorĉojn kun la sama potenco ok jarcentoj antaŭe. Via testo kaj mia testo ambaŭ same rezultas.''

Estis longa silento en la klasĉambro.

``Kio nun?'' Drako murmuris.

Li neniam estis tiom terurita.

``Tio ne ankoraŭ estas definitiva,'' diris Harry Potter. ``Mia eksperimento malsukcesis, ĉu mi memoras? Mi bezonas desegni alian teston, Drako.''

``Mi, Mi\ldots'' Drako diris. Lia voĉo estis rompanta. ``Mi ne povas fari tion Harry, tio estas tro multe por mi.''

La rigardo de Harry estis feroca. ``Jes vi povas, ĉar vi devas. Mi
pensis pri tio mi mem, post mi malkovris pri la Interdikto de
Merlino. Drako, ĉi estas maniero por observi la forton de la magio
direkte? Iu maniero kiu ne rilatas al la sango de sorĉistoj aŭ al la
sorĉoj, kiujn ni lernas?''

La menso de Drako nur tute malpleniĝis.

``Io ajn, kio influas la magion, influas sorĉistojn,'' diris
Harry. ``Sed tiam ne ne povas diri ĉu tio venas de la sorĉistoj aŭ de
la magio. Kio estas influata per la magio sed \emph{ne estas}
sorĉisto?''

``Magiaj bestoj, evidente,'' diris Drako sen eĉ pensi pri tio.

Harry Potter malrapide ridetis. ``Drako, tio estas, \emph{brila.}''

\emph{Tio estis la speco de stultaj demandoj, kiujn vi nur farus se vi estus edukita per mugloj.}

Tiam la malsana sento en la stomako de Drako fariĝis ankoraŭ pli
malbona kiam li ekkonsciis tion, kion volus diri ke la magiaj bestoj
\emph{fariĝu} pli malfortaj. Ili scius certmaniere ke la magio estas
velkanta, kaj estis parto de Drako kiu estis jam certa ke tio estos
ekzakte kion ili trovos. Li ne volis vidi ĝin, li ne volis scii\ldots

Harry potter estis jam je la mezo de la vojo al la pordo. ``Venu,
Drako! Estas protreto malfora de tie ĉi, ni nur demandos al li venigi
iun maljunan kaj ni scios senprokraste! Ni havas mantelojn, se iu
vidas nun, ni nur povos forkuri! Nu, iru ni!''

\later

Tio ne prenis longan tempon post tio.

Tiu estas larĝa portreto, sed la tri personoj en ĝi, aspektis relative
premitaj. Estis mezaĝa viriĉo de la dek dua jarcento, vestita per
bendoj de ŝtofo; kiu parolis al juna virino aspektanta malgaja de la
dek kvara jarcento, kun hararo kiu ŝajnis bukli ĉirkaŭ sia kapo kiel
se oni ĵetis al ŝi senmovan sorĉon; kaj ŝi parolis al digna, saĝa
maljuna viriĉo de la dek sepa jarcento kun robusta bantkravato el oro;
kaj lin ili povis kompreni.

Ili demandis pri la Aberentoj.

Ili demandis pri la Feniksoj.

Ili demandis pri la drakoj kaj la troloj kaj la domaj elfoj.

Harry malridetis kaj indikis ke la bestoj kiuj bezonis la plej grandan
kvanton da magio povis esti nur tute mortaj, kaj demandis pri la plej
potenca magia besto konata.

Ne estis io ajn malkutima en la listo, krom speco de malhela estaĵo
nomita diŝiranto de menso, kiun la tradukanto notis kiel finfine
ekstermitaj per Harold Ŝea, kaj tiuj ne sonis duone teruraj kiel
Aberentoj.

Magiaj bestoj estis tiom potencaj hodiaŭ kiom ili ĉiam estis, ŝajne.

La malsana sento en la stomako de Drako malplifortiĝis, kaj nu li nur sentis sin konfuza.

``Harry,'' Drako diris je la mezo de la maljuna viriĉo tradukanta liston de la dek unu povoj de la okuloj de vidatestento, ``kion tio volas diri?''

Harry levis figron kaj la maljuna viriĉo finis la liston.

Tiam Harry dankis ĉiujn la portretojn por ilia helpo—Drako, relative aŭtomate, faris tion ankaŭ kaj pli gracie—kaj ili reiris al la klasĉambro.

Kaj Harry elirigis la origanan pergamenon kun la hipotezoj, kaj komencis skribaĉi.

\vskip 1\baselineskip plus .5\textheight minus 1\baselineskip

\savetrivseps
\setlength{\topsep}{0pt}
\setlength{\partopsep}{0pt}

\begin{centering}
\begin{samepage}
\scshape Observadoj:

\itshape Sorĉado ne estas tiel potenca kiel ĝi estis kiam Herpŭrko estis fondita. \end{samepage}

\vskip 1\baselineskip plus .5\textheight minus 1\baselineskip

\begin{samepage}
\scshape Hipotezoj:

\itshape
        \begin{enumerate}[1.]
                \firmlist
                \setlength{\leftmargin}{\parindent}
                \setlength{\rightmargin}{\parindent}
        \item La magio estas velkanta.
        \item Sorĉistoj interproduktiĝis kun Mugloj kaj Skiboj.
        \item La scio pri ĵeti potencaj sorĉoj estas perdita. 
        \item La sortĉistoj manĝas la malbonan manĝaĵon infane, aŭ io simila krom sango estas faranta ke ili kreskas pli malfortaj.
        \item Mulga teĥnologio interferas kun la magio. (De ok cent jaroj?).
        \item Plej fortaj sorĉistoj havas malpli da infanoj (Drako=solinfano? Kontroli ĉu 3 potencaj sorĉistoj, Ciuro / Dumbledore / La Mastro de la Tenebroj, havas infanojn.)
        \end{enumerate}
\end{samepage}

\vskip 1\baselineskip plus .5\textheight minus 1\baselineskip

\begin{samepage}
\scshape Testoj:
\itshape
        \begin{enumerate}[A.]{
                \firmlist
                \setlength{\leftmargin}{\parindent}
                \setlength{\rightmargin}{1cm}}
        \item Ĉu estas sorĉoj, kiujn ni konas sed ne povas ĵeti (1 aŭ 2) aŭ ĉu la perditaj sorĉoj ne plu estas konataj (3)? {\scshape Rezulto:} Nekonkludiga pro la Interdikto de Merlino. Neniu neĵeteblaj sorĉoj, sed ili simple povus esti ne transmetitaj.

        \item Ĉu praaj studentoj de unua jaro ĵetis la samajn sorĉojn, kun la sama povo, ol hodiaŭ? (Malforta atesto por la 1a kontraŭ la 2a, sed sango povus ankaŭ perdi nur potencan sorĉadon) {\scshape Rezulto:} Sama nivelo de sorĉoj de unua jaro ol hodiaŭ. 

        \item Aldona testo kiu distingas la 1a kaj la 2a uzante scion de sango; mi klarigos pli malfrue. {\scshape Rezulto:} Estas nur unu loko kiu farigas vin esti sorĉisto, kaj aŭ vi havas du foliojn dirantaj 'magia' aŭ ne.

        \item Ĉu magiaj bestoj perdas ilian povon? Distingas inter la 1a kaj la 2a aŭ la 3a. {\scshape Rezulto:} Magiaj bestoj ŝajnas esti tiel forta kiel ili ĉiam estis.
        \end{enumerate}
\end{samepage}
\end{centering}
\vskip 1\baselineskip plus .5\textheight minus 1\baselineskip
\restoretrivseps

``A malsukcesis,'' diris Harry potter. ``B estas malforta atesto por
la 1a kontraŭ la 2a. C falsigas la 2a. D falsigas la 1a. 4a estis
malprobabla kaj B argumentas kontraŭ la 4a ankaŭ. La 5a estas
malprobabla kaj D argumentas kontraŭ ĝi. La 6a estas falsigita kun la
2a. Tio lasas la 3a. Interdikto de Merlino aŭ ne, mi fakte ne trovis
iun ajn sorĉojn, kiuj ne povas esti ĵetita. Do kiam vi aldonas ĉion,
ŝajnas ke la scio estas perdita.''

Kaj le kaptilo ekfermis.

Ekde la paniko foriris, ekde Drako komprenis ke la magio ne estis velkanta, prenis malpli ol kvin sekundoj por ekkonsci.

Drako puŝis sin for de la skribtablo, kaj stariĝis tiel forte ke la seĝo flugis kun grinca sono super la grundo kaj refalis.

``Do ĉio tio estis nur stulta ruzo.''

Harry Potter rigardis lin dum momento, ankoraŭ sidante. Kiam li
parolis, sia voĉo estis kvieta. ``Tio estis justa testo, Drako. Se ĝi
rezultus malsame, mi akceptintus ĝin. Tio ne estas io, pri kiu mi
mensongus. Neniam. Mi ne rigardis la datumojn antaŭ ol mi faris mian
prognozon. Mi diris al vi kiam la Interdikto de Merlino nuligis la
unuan eksperimenton—''

``Ho,'' Drako diris, la kolero komencanta eliri sian voĉo, ``vi ne sciis kiel la tuta aĵo rezultos?''

``Mi ne \emph{sciis} ion ajn, kion vi ne sciis,'' Harry diris ankoraŭ
kviete. ``Mi konfesas ke mi suspektis ĝin. Hermione Granger estis tro
potenca, ŝi devintus esti apenaŭ magia kaj ŝi ne estas. Kiel Mulge
naskita ulo povas esti la plej bona ĵetanto da sorĉoj en Herpŭrko? Kaj
ŝi ankaŭ havas le plej bonajn notojn en ŝiaj testoj, estas tro
multe da koincidoj por ke unu sola infanino estas la plej forta magie
\emph{kaj} la plej forta akademie, almenaŭ ke estas sola kaŭzo. La
ekzisto de Hermione Granger indikas ke estas nur unu aĵo kiu faras vin
esti sorĉisto, io kion vi havas aŭ ne, kaj ke la diferenco de potenco
venas de kiom vi konas kaj kiom vi praktikas. Kaj ne estas malsamaj
klasoj por pursangulo kaj Mulgle naskitaj, kaj tiel plu. Estis tro
multe da kialoj por ke la mondo ne aspektas kiel ĝi aspektus se vi
pravus. Sed Drako, mi ne vidis aĵojn kiujn vi ne povis ankaŭ vidi. Mi
ne faris iun ajn testojn, pri kiuj mi ne al vi parolis. Mi ne
friponis, Drako. Mi volis ke ni laboru kune por trovi la
respondon. Kaj mi neniam pensis ke la magio povus velki antaŭ ol vi
sugestis tion. Tio estis timiga ideo por mi ankaŭ.''

``Ne gravas,'' Drako diris. Li estis laboranta tre malfacile por
kontroli sian voĉon kaj ne nur krii al Harry. ``Vi pretendas ke vi ne
estas forkuronta kaj dironta al iu ajn alia pri tio.''

``Ne sen konsulti vin unue,'' Harry diris. Li malfermis la manojn en
petanta gesto. ``Drako, mi estis tiel ĝentila kiel mi povas sed
\emph{la mondo nur ne estas kiel tio}.''

``Bone. Do vi kaj mi estas finita. Mi nur marŝos for kaj forgesos ke ĉio tio iam okazis.''

Drako turniĝis, sentante brulan senton en sia gorĝo, la sento de
perfido, kaj tio estis kiam li konsciis ke li vere \emph{ŝatis} Harry
Potter, kaj tiu penso ne malrapidigis lin eĉ iomete dum li paŝegis al
la pordo de la klasĉambro.

Kaj la voĉo de Harry Potter venis, nun pli forta, kaj zorganta:

``Drako\ldots{}vi \emph{ne povas} forgesi. Ĉu vi ne komprenas? Tio estis via ofero.''

Drako haltis je la mezo de paŝo kaj turnis sin. ``\emph{Pri kio} vi parolas?''

Sed estis jam malvarmeganta sento en la spino de Drako.

Li sciis antaŭ ol Harry Potter diris ĝin.

``Por fariĝis sciencisto. Vi dubas unu el viaj kredoj, ne nur eta
kredo sed io, kio havas granda signifiko por vi. Vi faras
eksperimentojn, akiras datumojn, kaj la rezulto montras ke le kredo
estis malĝusta. Vi vidis la rezulton kaj komprenis tion, kion ĝi volis
diri.'' La voĉo de Harry Potter estis ŝanceliĝanta. ``Memoru, Drako,
vi ne povas oferi \emph{veran} kredon, ĉar la eksperimentoj konfirmus
ĝin anstataŭ falsi ĝin. Via ofero por fariĝi sciencisto estis via
\emph{malvera} kredo ke la sango de sorĉisto estis miksanta kaj
fariĝanta pli malforta.''

``\emph{Tio ne veras!}'' diris Drako. ``Mi ne oferis tiun kredon. Mi
ankoraŭ kredas tion!'' Lia voĉo estis fariĝis pli laŭta, kaj la
malvarmo pli malbona.

Harry Potter kapneis. Lia voĉo eliris kiel murmuro. ``Drako\ldots{}mi
bedaŭras, Drako, vi \emph{ne} kredas tion, ne plu.'' La voĉo de Harry
laŭtiĝis denove. ``Mi pruvos tion al vi. Imagu ke iu diras al vi ke ri
gardas drakonon en sia domo. Vi diras al ri ke vi volas vidi ĝin. Ri
diras ke ĝi estas nevidebla drako. Vi diras bone, vi aŭskultos ĝin
kiam ĝi movos. Ri diras ke ĝi estas neaŭdebla drako. Vi diras ke vi
ĵetos farunon en la aero tiel ke vi vidu la formon de la drako. Ri
diras ke la drako estas permeabla al faruno. Kaj la rakonto estas ke
ri scias, \emph{antaŭe}, ekzakte kiun rezulton ri devos klarigi. Ri
\emph{scias} ke ĉio rezultos kiel tio rezultus se ne estis drako. Ri
scias antaŭe kiun ekskuzon ili devos fari. Do eble ri \emph{diras} ke
estas drako. Eble ri \emph{kredas} ke ri kredas ke estas drako. Tio
estas nomata kredo-en-kredo. Sed ri ne fakte kredas tion. Vi povas
erari pri tio, kion vi kredas, la plejparto de la homoj neniam
komprenas ke estas diferenco inter kredi ion, kaj pensi ke estas bona
kredi tion.'' Harry Potter estis stariĝinta de sia seĝo, kaj farinta
kelkajn paŝojn al Drako. ``Kaj Drako, vi ne plu kredas en purismo de
la sango, mi tion montros al vi. Se purismo de la sango estas vera,
tiam ne estas logika ke Hermione Granger ekzistas, do kiel vi povas
klarigi tion? Eble ŝi estis sorĉista orfino edukita per mugloj, tiel
kiel mi estis? Mi povas iri al Granger kaj peti al ŝi foton de ŝiaj
gepatroj, por vidi ĉu ŝi aspektas kiel ili. Ĉu vi atendas ke ili
aspektas malsame? Ĉu mi devus fari tiun teston?''

``Ili metintus ŝin kun parencoj,'' Drako diris, la voĉo tremanta. ``Ili aspektos simile.''

``Vi vidas. Vi jam scias kiun rezulton vi devos ekskuzi. Se vi ankoraŭ
kredus en purismo de la sango, vi dirus, certe, iru ni kontroli, mi
vetas ke ŝi ne aspektas kiel siaj gepatroj, ŝi estas tro potenca por
esti vere Mugle naskita—''

``Ili metintus ŝin kun parencoj!''

``Sciencistoj povas fari testojn por certigi ke iu estas la vera
infano de sia patriĉo. Granger probable farus ĝin se mi donus sufiĉe
da mono al sia familio. \emph{Ŝi} ne estus timigita per la rezulto. Do
kion vi atendas ke tiu testo montros? Diru al mi ke mi faru ĝin kaj mi
ĝin faros. Sed vi jam scias tion, kion la rezulto diros. Vi neniam
eblos forgesi. Vi eble \emph{volas} kredi en sanga purismo, sed vi
ĉiam \emph{atendos} vidi ekzakte tion, kio okazus se estus nur unu
aĵo, kiu farigus vin esti sorĉisto. Tio estas la ofero por fariĝi
sciencisto.''

La respiro de Drako estis ĉifonita. ``Ĉu vi konscias tion, kion vi
faris?'' Drako ĵetis sin antaŭen kaj ekkaptis Harry'n per la kolumo de
la robo. Lia voĉo kreskis en krio, ĝi sonis neelteneble laŭda en la
fermita klasĉambro kaj la silento. ``\emph{Ĉu vi konscias tion, kion
  vi faris?}''

La voĉo de Harry estis hezita. ``Vi havis kredon. Tiu kredo estis
malĝusta. Mi helpis vin vidi tion. Kio estas vera estas jam vera, Scii
ne farigas ĝin pli malbona—''

La fingroj de la dekstra mano de Drako fermiĝis en pungon kaj tiu mano
ekiris malsupren kaj impetis nehaltigeble kaj frapis Harry'n Potter je
la makzelo tiel forte ke lia korpo kraŝis sur skribtablon kaj poste
sur la plankon.

``\emph{Stultulo!}'' kriis Drako. ``\emph{Stultulo! Stultulo!}''

``Drako,'' murmuris Harry de la planko, ``Drako, mi bedaŭras, mi ne
pensis ke tio okazus antaŭ monatoj, mi ne atendis ke vi vekiĝas kiel
sciencisto tiel rapide, mi pensis ke mi havus pli da tempo por prepari
vin, lerni al vi la teĥnikojn, kiuj faras ke akcepti vian malpravon
estas malpli dolora—''

``Kio pri mia patriĉo?'' Drako diris. Lia voĉo tremis pro la kolero. ``Ĉu vi preparintus \emph{lin} aŭ ĉu vi nur ne \emph{zorgis} pri tio, kio okazus post tio?''

``Vi ne povas diri tion al \emph{li!}'' Harry diris, lia voĉo kreskanta en alarmon. ``Li ne estas sciencisto! Vi promesis, Drako!''

Dum momento, la penso ke Patriĉo ne sciante venis kiel malpezigo.

Kaj poste la vera kolero komencis kreski.

``Do vi planis ke mi mensogu al li kaj ke mi diru al li ke mi ankoraŭ
kredas,'' Drako diris, la voĉo tremanta. ``Mi ĉiam devos mensongi al
li, kaj nun kiam mi plenkreskos mi ne povos esti Morto-Manĝanto, kaj
mi ne eĉ kapablos klarigi al li kial.''

``Se via patriĉo vere amas vin,'' murmuris Harry de la planko, ``li
ankoraŭ amos vin kvankam vi ne fariĝos Morto-Manĝanto, kaj ŝajnas
kvazaŭ via patriĉo vere amas vin, Drako—''

``\emph{Via} duonpatriĉo estas sciencisto,'' Drako diris. La vortoj
venis kiel pikantaj tranĉiloj. ``Se \emph{vi} ne fariĝos sciencisto,
li ankoraŭ amos vin. Sed vi estos \emph{iom malpli aparta} por li.''

Harry ektremis. La knabo malfermis la buŝon, kvazaŭ li estis dironta
'Mi bedaŭras', kaj poste fermis la buŝon, ŝajnante repensi tion pli
bone, kio estis aŭ tre inteligenta aŭ tre ŝanca, ĉar Drako provintus
mortigi lin.

``Vi devintus averti min,'' Drako diris. Lia voĉo laŭtiĝis. ``\emph{Vi
  devintus averti min!}''

``Mi\ldots{}mi tion faris\ldots{}ĉiujn fojojn kiam mi parolis al vi
pri la povon, mi diris al vi la koston. Mi diris, ke vi devos akcepti
vian malpravon. Mi diris al vi ke tio povus esti la plej malfacila
vojo por vi. Ke tio estis la ofero, kiun ĉiuj devus fari por fariĝi
sciencisto. Mi avertis vin ke la eksperimento povus diri iun aĵon, kaj
via familio kaj amikoj alian aĵon—''


``\emph{Vi nomas tiun averto?}'' Drako kriis nun. ``\emph{Vi nomas tiun averto? Kiam ni faras ritualon kiu necesigas neinversigeblan sinoferon?}''


``Mi\ldots{}Mi\ldots{}'' La knabo sur la planko glutis. ``Mi supozas
ke eble mi ne estis klara. Mi bedaŭras. Sed tio, kion la vero povas
detrui, devas esti destruita.''

Frapi lin ne estis sufiĉa.

``Vi malpravas pri unu aĵo,'' Drako diris, la voĉo mortiga. ``Granger
ne estas la plej forta studento en Herpŭrko. Ŝi nur havas la plej
bonajn notojn en kurso. Vi estas malkovronta la diferencon.''

Subita ŝoko montriĝis sur la vizaĝo de Harry, kaj li provis ruli
rapide por stariĝi—

Estis jam tro malfrue por li.

``\emph{Armfiorpelu!}''

La bastono de Harry flugis trans la ĉambro.

“\emph{Gom ĝabbar!}~\footnote{Skribita \emph{Gom jabbar} en la angla,
  ĝi estas venena krudilo uzita en la libro \emph{Dune}}”

Sento de inka nigreco batis la maldekstran manon de Harry.

``Tio estas tortura sorĉo,'' diris Drako. ``Tio estas por akiri
informon el homoj. Mi nur lasos ĝin sur vi kaj ŝlosos la pordon
malantaŭ mi kaj foriri. Eble mi faros ke la ŝlosa sorĉo eluziĝos post
kelkaj horoj. Eble mi ne eluziĝigos ĝin antaŭ ol vi mortos ĉi
tie. Havu plezuron.''

Drako moviĝis glate malantaŭen, la bastono ankoraŭ sur Harry. La mano
de Drako malsupreniris, prenis lian sakon, sen ke sia rigardo tremis.

La doloro jam montriĝis sur la vizaĝo de Harry Potter kiam li
parolis. ``Malfojoj estas super la leĝo de neplenaĝa magio, mi
supozas? Tio ne estas ĉar via sango estas pli forta. Tio estas ĉar vi
jam praktikis. Je la komenco vi estis tiel malforta kiel ĉiuj el
ni. Ĉu mia prognozo estas malĝusta?''

La mano de Drako blanhiĝis pro la premo sur la bastono, sed lia rigardo restis rekta.

``Nur por ke vi sciu,'' Harry Potter diris tra premitaj dentoj, ``se
vi dirintus al mi ke mi estas malprava, mi aŭskultintus. \emph{Mi} ne
torturintus \emph{vin}, kiam vi montrintus al mi ke mi malpravis. Kaj
vi tion faros. Iam. Vi vekiĝis kiel sciencisto, kaj eĉ se vi neniam
lernis kiel uzi tiun povon, vi ĉiam,'' Harry anhelis, ``serĉos,
manierojn, por testi, viajn kredojn—''

La iro de Drako estis malpli glata nun, iomete pli rapida, kaj li
penis gardi sian bastonon je la direkto de Harry kiam li atingis la
pordon por malfermi ĝin, paŝante dorsen el la klasĉambro.

Tiam Drako refermis la pordon.

Li ĵetis la plej potencan ŝlosa sorĉon, kiun li konis.

Drako antendis ĝis li aŭdis Harry'n krii antaŭ li ĵetis \emph{Silenciu.}

Kaj poste li formarŝis.

\later

“\emph{Aaahhhhh! Finite Inkantatem! Aaaahhh!}”

La maldekstra mano de Harry estis metita en poton de bolanta oleo kaj
lasita tie. Li metis ĉion kion li havis en la \emph{Finite Inkantatem}
kaj tio malgraŭe ne funkciis.

Kelkaj sorĉoj bezonis specifan kontraŭan sorĉon, aŭ vi ne povis rompi
ilin, aŭ eble tio estis pro ke Drako estis tiel pli forta.

“\emph{Aaaaahhhh!}”

La mano de Harry vere komencis dolori lin, nun, kaj tio interferis kun liaj provoj por pensi kreeme.

Sed post kelkaj krioj, Harry konsciis tion, kion li devis fari.

Lia haŭtpoŝo bedaŭrinde estis je la malbona flanko de sia korpo, kaj
necesis kelkajn tordojn por ke li atingu ĝin, speciale kiam sia alia
brako svingiĝis reflekse, nehaltigebla provo por forigi la fonton de
la doloro. Antaŭ ol li sukcesis, sia alia brako denove forjetis sian
bastonon.

``Kuraca \emph{ahhhh} ilaro! Kuraca ilaro!''

Sur la planko, la verda lumo estis tro malhela por ke oni povu vidi.

Harry ne povis stari. Li ne povis rampi. Li rulis sur la planko al
tie, kie li pensis ke sia bastono estis, kaj ĝi ne estis tie, kaj kun
unu mano li sukcesis starigi sin mem sufiĉe alte por vidi sian
bastonon. Li rulis tien, kaj prenis ĝin, kaj denove rulis al tie, kie
la Kuraca ilaro estis malfermita. Estis ankaŭ relative multe da krioj,
kaj iom da vomo.

Necesis ok provoj antaŭ ol li sukcesis ĵeti \emph{Lumos.}

Kaj tiam, nu, la pako ne estis desegnita por esti malfermita kun nur
unu mano, ĉar ĉiuj la sorĉistoj estis stultuloj, tio estis la
kialo. Harry uzis siajn dentojn kaj do necesis tempon antaŭ kiam li
sukcesis envolvi sian maldekstran manon en sensentigan bandaĝo.

Kiam la tuta sento en la maldekstra mano estis fine foririnta, Harry
lasis sian menson rompiĝi, kaj restis senmove sur la planko, kaj
ploris dum momento.

\emph{Bone,} la menso de Harry diris silente, kiam ĝi resaniĝis sufiĉe
por pensi kun vortoj denove. \emph{Ĉu tio valoris tion?}

Malrapide, la funkcia mano de Harry atingis skribtablon.

Harry tiris sin mem por stariĝi.

Prenis longan enspiron.
imunaa 
Elspiris..

Ridetis.

Tio ne estis granda rideto, sed ĝi estis rideto malgraŭe.

\emph{Dankon, Profesoriĉo Ciuro, mi ne povintus perdi sen vi.}

Li ne ankoraŭ elaĉetis Drako'n, ne eĉ preskaŭ. Male al tio, kion
Drako mem nun pensis, Drako estis ankoraŭ la infano de Morto-Manĝanto,
komplete. Li estis ankoraŭ knabiĉo, kiu kreskis pensante ke
``seksatenco'' estis io, kion la mojosaj pli aĝaj knabiĉoj faras. Sed
tio estis tre bona komenco.

Harry ne povis pretendi ke ĉio iris ekzakte kiel planite. Ĉio estis
improvizita je la momento. La \emph{plano} ne intenciis ke tio okazu
antaŭ decembro ĉirkaŭe, post Harry instruis al Drako la teĥnikoj por
ne nei ateston kiam oni vidis ĝin.

Sed li vidis la timigitan rigardon sur la vizaĝo de Drako, kaj
konsciis ke Drako estis \emph{jam} konsideranta alternativan hipotezon
serioze, kaj profitis de tiu okazo. Iu kazo de vera scivolo havis la
saman specon de elaĉeta povo en racieco ol tiu, kiun iu kazo de vera
amo havis en filmoj.

Retrospekte, Harry donis al si horojn por fari la plej gravan
malkovron de la historio de la magio, kaj monatoj por transiri la
nedisvolvitaj mensaj bariloj de dekunu-jaraĝa knabiĉo. Tio povis
indiki ke Harry havis ian gravan kognan nesufiĉon pri la respekto de
la takso de la tempo por plenumi taskojn.

Ĉu Harry iros en la infero de la scienco pro tio, kion li faris? Harry
ne certis. Li sin aranĝis por gardi en la menso de Drako la eblon ke
la magio estis velkanta, kaj certigi ke Drako plenumos la parton de la
eksperimento, kiu ŝajnis unue iri je tiu direkto. Li atendis ĝis post
ol li klarigis genetikon, por antentigi Drako'n pri magiaj bestoj
(tamen Harry pensis pri la malnovaj artefaktoj kiel la Ordiganta
Ĉapelo, kiun neniu ne plu povas dupliki, sed kiu daŭras funkcii). Sed
Harry ne fakte troigis iun ajn ateston, ne distordis la signifon de iu
ajn rezulto. Kiam la Interdikto de Merlino nuligis la teston, kiu
devintus esti definitiva, li tion diris al Drako tuje.

Kaj tiam estis la parto \emph{post} tio\ldots{}

Sed li ne fakte \emph{mensogis} al Drako. Drako kredis tion, kaj
\emph{tio igis ĝin vera.}

La fino, konfese, ne estis amuza.

Harry turniĝis, kaj ŝanceliĝis al la pordo.

Estis tempo por testi la ŝlosan sorĉon de Drako.

La unua etapo estis simple provi turni la tenilon. Drako povintus blufi.

Drako ne blufis.

``\emph{Finite Inkantatem.}'' la voĉo de Harry eliris relative raŭka, kaj li ne povis sento ke la sorĉo funkciis.

Do Harry provis denove, kaj ĉi-foje ĝi ŝajnis bona. Sed alia turno de la tenilo montris ke ĝi ne funkciis. Neniu surprizo tie. 

Estis tempo por eliri la grandan kanonon. Harry prenis profundan enspiron. La sorĉo estis unu el la plej potencaj sorĉoj, kiujn li lernis ĝis nun.

“\emph{Alohomora!}”

Harry ŝanceliĝis iomete post ol li diris tion.

Kaj la pordo de la klasĉambro ankoraŭ ne malfermiĝis.

Tio ŝokis Harry'n. Harry ne planis iri ien ajn proksime de la
malpermesita koridoro de Dumbledore, evidente. Sed la sorĉo, kiu
malfermis magiajn serurojn, ŝajnis esti utila sorĉo ĉiaokaze, kaj do
Harry lernis ĝin. Ĉu la malpermesita koridoro de Dumbledore intencis
trompi homojn tiel stultaj ke ili ne rimarkus ke la sekureco estis pli
malbona ol tio, kion Drako Malfojo povis fari?

Timo denove rampis en la sistemo de Harry. La afiŝo de la kuraca ilaro
diris ke la sensentiga bandaĝo nur povis esti uzita sendanĝere dum
maksimume tri dek minutoj. Post tio ĝi simple malfiksiĝos aŭtomate,
kaj ne estos reuzebla antaŭ 24 horoj. Nuntempe estis la 6a kaj 51
minutoj, li metis la sensentigan bandaĝon antaŭ ĉirkaŭ kvin minutoj.

Do Harry faris paŝon malantaŭen, kaj rigardis la pordon
pripensante. Ĝi estis solida panelo el kverka ligno malhela, nur
rompita per la tenilo el latuno.

Harry ne konis iun ajn eksplodan aŭ tranŝantan aŭ frakasantan sorĉon,
kaj transfiguri eksplodaĵon estus malrespekto de la regulo, kiu
malpermesis transfiguri iun ajn por ke ĝi estu brulita. Acido estis
likvo kaj farus haladzon\ldots{}

Sed tio ne estis obstaklo por \emph{kreema pensanto.}

Harry metis sian bastonon kontraŭ la ĉarninoj el latuno de la pordo,
kaj koncentrigis siajn pensojn sur la formo de kotono kiel pura
abstrakaĵo, apartigita de vera materialo el kotono, kaj ankaŭ sur la
pura materialo krom la skemo kiu faris ke ĝi estis ĉarnino el latuno,
kaj kunmetis la du konceptojn, trudante la formon super la esenco. Kun
unu horo de trejnado al transfiguro ĉiutage dum unu monato, Harry
atingis la punkton kie li povis transfiguri subjekton de kvin kubaj
centimetroj en apenaŭ unu minuto.

Post du minutoj la ĉarnino tute ne estis ŝanĝita.

Kiu ajn, kreis la ŝlosan sorĉon de Drako, pensis pri tio, ankaŭ. Aŭ la
pordo estis parto de Herpŭrko kaj la kastelo imuna.

Rapida rigardo al la muroj montris ke ili estis solidaj ŝtonoj. Tiel
estis la planko. Tiel estis la plafono. Vi ne povis transfiguri parton
de io, kio estis solida tuto; Harry bezonintus transfiguri la tutan
muron, kaj tio prenus horojn aŭ eble tagojn da daŭra klopodo, se li
efektive kapablus fari tion, kaj se la muro ne estis koneksa kun la
resto de la tuta kastelo\ldots{}

La Returnilo de Tempo de Harry ne malfermiĝos antaŭ la 9a horo. Poste
le povos reiri je la 6a, antaŭ ol la pordo estu ŝlosita.

Kiom longe la tortura sorĉo daŭros?

Harry glutis forte. Larmoj denove eliris el liaj okuloj.

Lia brila kaj kreema menso ĵus oferis la sagacan sugeston ke Harry
povus tranŝi sian manon uzante la segilon de la ilaro, kiun li havis
en sia haŭtpoŝo. Tio dolorigus, evidente, sed eble dolorigus malpli ol
la dolora sorĉo de Drako, pro tio ke la nervoj estus forigitaj; kaj li
havis bandaĝojn en la kuraca ilaro.

Kaj tio estis evidente hide stulta ideo, ke Harry bedaŭrus dum la resto de sia vivo.

Sed Harry ne sciis se li povis elporti du horojn de torturo.

Li volis eliri la klasĉambron, li volis eliri la klasĉambron
\emph{nun,} li ne volis atendi tie kriante dum du horoj ĝis li povas
uzi la Returnilon de Tempo, li bezonis eliri kaj trovi iun por forigi
la torturan sorĉon de sia mano\ldots{}

\emph{Pripensu!} Harry kriis en sia cerbo. \emph{Pripensu! Pripensu!}

\later

La studentejo de Serpentimo estis preskaŭ malplena. Homoj estis je la
vespermanĝo. Pro kelkaj kialoj Drako li mem ne sentis sin tre malsata.

Drako fermis la pordon de sia privata ĉambro, ŝlosis ĝin, ĵetis
silentigan sorĉon, sidiĝis sur sia lito, kaj komencis plori.

Tio ne estis justa.

Tio ne estis justa.

Estis la unua fojo ke Drako vere \emph{perdis}, lia patriĉo avertis
lin ke reale perdi dolorintus je la unua fojo ke tio okazas, sed li
perdis \emph{tiel multe}, tio ne estis justa, tio ne estis justa ke li
perdis \emph{ĉion} je la unua fojo kiam li perdis.

Ie en la karceroj, knabiĉo, kiun Drako fakte ŝatis, estis krianta pro
la doloro. Drako neniam dolorigis iun ajn, kiun li ŝatis antaŭe. Puni
homojn, kiuj meritis tion, estis supozita esti amuza, sed tio nur
sentigis lin malsana. Lia patriĉo ne avertis lin pri tio, kaj Drako
demandis al si ĉu tio estis malfacila leciono, kiun ĉiuj devis lerni
kiam ili kreskos, aŭ ĉu Drako estis nur malforta.

Drako preferintus ke Pansy estu krianta. Tio sentigus lin pli bone.

Kaj la plej malbona parto estis scii ke dolorigi Harry'n Potter eble estis eraro.

Kiu alia estis tie por Drako nun? Dumbledore? Post tio, kion li faris? Drako preferintus esti brulita vivante.

Drako devus reiri al Harry Potter, ĉar estis nenie alia, kien li povis
iri. Kaj se Harry Potter diris ke li ne volis lin, tiam Drako estus
nenio, nur mizera eta knabiĉo, kiu neniam povos esti Morto-Manĝanto,
kaj neniam povos aliĝi al la fakcio de Dumbledore, kaj neniam lernos
sciencon.

La kaptilo estis perfekte instalita, perfekte plenumita. La patriĉo de
Drako avertis lin ankoraŭ kaj ankoraŭ ke tio, kion oni oferis dum
malhelaj ritoj ne povis esti reakirita. Sed lia patriĉo ne sciis ke
la malbenitaj mugloj inventis ritojn, kiuj bezonis neniun bastonon,
ritoj kiujn vi povis fari trompe sen eĉ tion scii. Kaj tio estis nur
unu el la teruraj sekretoj, kiujn sciencistoj konis kaj kiujn Harry
Potter prezentis al li.

Drako komencis plori pli forte.

Li ne volis tion, li \emph{ne volis tion} sed estis nenio maniero por iri malantaŭen. Estis tro malfrue. Li jam estis sciencisto.

Drako sciis ke li devus reiri kaj liberigi Harry'n Potter kaj pardonpeti. Tio estus la inteligenta aĵo por fari.

Anstataŭ Drako restis sur sia lito kaj ĝemis.

Li jam dolorigis Harry'n Potter. Eble estis la sola fojo ke Drako iam dolorigos lin, kaj li devos memori tion dum la resto de sia vivo.

Oni lasu lin krii.

\later

Harry faligis la restaĵon de la segilo sur la plakon. La ĉarnino el
latuno montriĝis netrapenetrebla, ne eĉ gratita, kaj Harry komencis
suspekti ke eĉ la malespera ago de provi transfiguri acidon aŭ
eksplodon ankaŭ malsukcesos por malfermi la pordon. Je la pozitiva
flanko, la provo detruis la segilon.

Lia brakhorloĝo diris ke estis 7:02 posttagmeze, kun malpli ol kvindek
minutoj restantaj, kaj Harry provis memroy, ĉu estis aliaj tranŝaj
aĵoj en sia haŭtpoŝo, kiuj bezonis esti detruita, kaj sentis alian
alvenon da larmoj akumuliĝi. Se almenaŭ, kiam sia Returnilo de Tempo
malfermiĝos, li povus reiri kaj \emph{malhelpi}—

Kaj estis je tiu momento ke Harry ekkonsciis ke li agis \emph{stulte}.

Ne estis la unua fojo ke li estis ŝlosita en ĉambro.

Profesorino McGonagall jam diris al li, kiu estis la bona manerio por
fari tion.

\ldots{}ŝi ankaŭ diris ke li ne uzu la Returnilon de Tempo por tiu speco de aĵo.

Ĉu Profesorino McGonagall konscios ke tiu okazo \emph{vere} rajtigis specialan escepton? Aŭ ŝi nur reprenos la Returnilon de Tempo komplete?

Harry kolektis ĉiujn liajn aĵojn, ĉiujn la pruvojn, en sian
haŭtpoŝon. \emph{Purgigu} okupis sin je la vomaĵo sur la planko,
kvankam ne je la ŝvito, kiu trempis sian robon. Li lasis la
renversitajn skribtablojn renversitaj, tio ne estis sufiĉe grava por
ke li faru tion kun nur unu mano.

Kiam li estis fininta, Harry rigardis sian brakhorloĝon. 7:04 posttagmeze.

Kaj tiam Harry atendis. Sekundoj pasis, ŝajnante jaroj.

Je la 7a kaj 7, la pordo malfermiĝis.

La barba vizaĝo de Profesoriĉo Flirtiko aspektis relative zorga. ``Ĉu
vi fartas bone, Harry?'' diris la knara voĉo de la direktoro de la
Domo Korvungo. ``Mi havas noton diranta ke vi estas ŝlosita tie—''

%  LocalWords:  nd Gom jabbar Aaahhhhh Aaaahhh Aaaaahhhh ahhhhh
