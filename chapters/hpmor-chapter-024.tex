\chapter{Hipotezo de Makiavela Inteligento}

\section{Akto 3a:}

\lettrine{D}{rako} atendis en eta alkovo kun fenestro, kiun li trovis proksime de la Granda Ĉambrego, la stomako agitiĝante.

Estos prezo, kaj ĝi ne estos eta. Drako sciis tion, ekde li vekiĝis kaj konsciis ke li ne aŭdacis eniri la Grandan Ĉambregon, ĉar li povus vidi Harry'n Potter tie, kaj Drako ne sciis tion, kio okazus poste.

Paŝoj proksimiĝis.

``Tie vi estas,'' diris la voĉo de Vincent. ``L'ĉef' ne'stas bonhumora hodiaŭ, do vi pli b'ne atentas viajn paŝojn.''

Drako senhaŭtigos tiun stultulon vivante, kaj sendos la senhaŭtan korpon kun demando por pli inteligenta servisto, kiel morta gerbilo.

Aro da paŝoj foriris, kaj la alia aro proksimiĝis.

La agitiĝo en la stomako de Drako pli malboniĝis.

Harry Potter alvenis en vido. Lia vizaĝo estis klare neŭtrala, sed lia blue ornamita robo aspektis strange malrekta, kvazaŭ li ne metis ĝin tute korekte—

``\emph{Via Mano,}'' Drako diris tute sen pensi.

Harry levis sian maldesktran brakon, kvazaŭ por rigardi ĝin.

La mano pendis senforte je ĝia fino, kiel io morta.

``Sinjorino Pimfito diras ke tio ne estas definitiva,'' Harry diris
kviete. ``Ŝi diras ke ĝi plejparte resaniĝos antaŭ ol la kursoj
komencos morgaŭ.''

Dum mallonga momento, la novaĵo kvietigis lin.

Kaj poste Drako konsciis.

``Vi vidis Sinjorino Pimfito,'' murmuris Drako.

``Nature,'' diris Harry Potter, kvazaŭ deklarante ion evidentan. ``Mia mano ne funkciis.''

Malrapide aperis al Drako kia \emph{kompleta} stultulo li estis, multe pli malbona ol la aliaj Serpentimoj, kiujn li dispremis.

Li nur konsideris evidenta ke neniu irus al la aŭtoritatoj kiam iu Malfojo faris ion al ili. Ke neniu volus esti sub la rigardo de Lucius Malfojo, neniam.

Sed Harry Potter ne estis timigita eta Huflopufo provante resti ekster la ludo. Li jam ludis la ludon, kaj la okuloj de lia patriĉo estis jam sur li. 

``Kion alian diris S-ino Pimfito?'' diris Drako, kun koro batanta forte. 

``Profesoro Flirtiko diris ke la sorĉo ĵetita sur mia mano estis
malhela sorĉo, kaj treege grava afero, kaj ke rifuzi diri kiu faris
ĝin, estis tute neakceptinda.''

Estis longa paŭzo.

``Kaj poste?'' Drako diris, per tremanta voĉo.

Harry Potter ridetis leĝere. ``Mi pardonpetis profunde, kaj tiu
aspektigis Profesoron Flirtiko \emph{tre} severa, kaj poste mi diris
al Profesoro Flirtiko ke ĝi estis, efektive, treege grava, sekreta,
kaj \emph{delikata} afero, kaj ke mi jam informis la Direktoron pri la
projekto.''

Drako anhelis. ``Ne! Flirtiko ne emas akcepti nur tion! Li iros demandi al Dumbledore!''

``Efektive,'' diris Harry Potter. ``Mi estis prompte tirita al la oficejo de la Direktoro.''

Drako estis tremanta nun. Se Dumbledore alkondukus Harry'n Potter
antaŭ la Magekoncilio, konsente aŭ ne, kaj farigus la
Knabo'n-Kiu-Postvivis atesti sub Verec-serumo ke Drako torturis
lin\ldots{}tro multe da homoj amis Harry'n Potter, lia patriĉo povus
\emph{perdi} tiun baloton\ldots

Lia patriĉo eble povus konvinki Dumbledore'n por ke li ne faru tion, sed
tio multe \emph{kostus}. Kostus terure multe. La ludo havis regulojn
nun, vi ne plu nur povus minaci iun hazarde. Sed Drako estis irinta en
la manoj de Dumbledore per sia propra volo. Kaj Drako estis tre valora
ostaĝo.

Tamen pro tio ke Drako ne povis fariĝi Morto-Manĝanto, li ne estis
tiel valora kiel sia patriĉo pensis.

La penso tordis lian koron, kiel ŝiranta sorĉo.

``Kaj kio?'' murmuris Drako.

``Dumbledore deduktis tuje ke estis vi. Li sciis ke ni kunlaboris.''

La plej malbona scenaro ebla. Se Dumbledore ne divenintus kiu faris
tion, li eble ne riskus uzi legilimencion nur por eltrovi\ldots{}sed
se Dumbledore \emph{sciis\ldots}

``Kaj?'' Drako devigis sin diri.

``Ni parolis iomete.''

``Kaj?''

Harry Potter ridetaĉis. ``Kaj mi klarigis ke li agus per sia propra
intereso se li ne farus ion ajn.''

La menso de Drako kuris al brika muro kaj disŝprucis. Li nur rigardis
fikse Harry'n Potter kun la buŝo restante malstreĉe malfermita kiel
stultulo.

Prenis longe por ke Drako memoru.

Harry konis la misteran sekreton de Dumbledore, la unu kiun Skoldo uzis por ĉantiĝi lin.

Drako povis vidi tion nun. Dumbledore rigardante lin severe, kasante
sian avidecon kiam li klarigis al Harry, kiel terure serioza afero tio
estis.

Kaj Harry afable dirante al Dumbledore ke li devus gardi sian buŝon
ferma se li scius tion, kio estis bona por li.

Lia patriĉo avertis Drakon kontraŭ homoj kiel tio, homoj kiuj povis
ruinigi vin kaj malgraŭe resti tiel ŝatinda ke estis malfacile malami
ilin konvene.

``Post kio,'' Harry diris, ``la direktoro diris al Profesoriĉo
Flirtiko ke tio estis, efektive, sekreto kaj delikata afero pri kiu li
estis jam informita, kaj ke li ne pensis ke sin okupi je ĝi tiam
helpus nek mi nek iu ajn. Profesoriĉo Flirtiko diris ion, pri la emo
de Dumbledore fari planon irante multe tro for, kaj mi devis haltigi
lin je tiu momento kaj klarigi ke ĝi estis mia \emph{propra} ideo, kaj
ne io, kion la direktoro devigis min fari, do Profesoriĉo Flirtiko
turniĝis kaj komencis lekcii \emph{min}, kaj la direktoro haltigis
\emph{lin}, kaj diris ke la Knabo-Kiu-Postvivis estis kondamnita por
havi strangajn kaj danĝerajn aventurojn, do ke mi estus pli sekura se
mi farus ĝin intence, anstataŭ atendi ke ili okazas hazarde, kaj estis
kiam Profesoriĉo Flirtiko ĵetis siajn etajn manojn kaj komencis kriegi
per sia alta voĉo al \emph{ambaŭ} el ni pri kiel li ne zorgis pri tio,
kion ni estis preparanta, sed ke tio ne okazos denove dum mi estas en
Korvungo Domo, aŭ li ĵetos min ekstere kaj mi povos iri en Grifindoro,
kiu estas la Domo al kiu tiuj \emph{Dumbledoraj} aĵoj apertenas—''

Harry tre malfaciligis ke Drako malamu lin.

``Ĉiaokaze'', Harry diris, ``Mi ne volas esti ĵetita el Korvungo, do
mi promesis al Profesoriĉo Flirtiko ke nenio kiel tio okazos denove,
kaj ke se malgraŭe tio okazus, mi dirus al li kiu faris ĝin.''

La okuloj de Harry devintus esti malvarmaj. Ili ne estis. Lia voĉo
devintus fari mortan minacon. Ĝi ne faris.

Kaj Drako vidis la demandon, kiu devus esti evidenta, kaj tio mortigis la humoron senprokraste.

``Kial\ldots{}vi ne al ili diris?''

Harry marŝis al la fenestro, sub la eta radio de suna lumo, brilanta en la alkovo, kaj turnis sian kapon eksteren, al la verdaj terenoj de Herpŭrko. La helo brilis sur li, sur lia robo kaj sur lia vizaĝo.

``Kial mi ne al ili diris?'' Harry diris. Lia voĉo haltigis. ``Mi
supozas ĉar mi nur ne povis esti kolera kontraŭ vi. Mi sciis ke mi
dolorigis vin unue. Mi ne eĉ nomos tion justa, ĉar tio, kion mi faris
al vi estis pli malbona ol tio, kion vi faris al mi.''

Estis kiel kuri al alia brika muro. Harry povintus paroli per la arkaika greko, ke Drako ne malpli komprenintus.

La menso de Drako serĉis skemon sed trovis absolute nenion. La
deklaro estis cedo, tio ne estis la plej bona intereso de Harry.  Ĝi
ne eĉ estis tio, kion Harry devintus diri por fari ke Drako fariĝu pli
fidela servisto, nun ke Harry havis povon super li. Por tio, Harry
devintus emfazi kiel ĝentila li estis, ne kiel multe li dolorigis
Drako'n.

``Malgraŭe,'' Harry diris, kaj nun lia voĉo estis pli mallaŭta,
preskaŭ murmuro, ``bonvolu, ne faru tion denove, Drako. Tio doloras,
kaj mi ne certas ke mi povos pardoni al vi duan fojon. Mi ne certas ke
mi kapablos voli tion.''

Drako nur ne komprenis.

Ĉu Harry provis esti \emph{amiko} kun li?

Ne estis maniero por ke Harry Potter povu esti sufiĉe stulta por kredi
ke estis ankoraŭ eble post tio, kion li faris.

Vi povis esti la amiko kaj alianculo de iu, kiel Drako provis esti kun
Harry, aŭ vi povis destrui ilian tutan vivon kaj lasi ilin kun neniu
alia eblo. Ne ambaŭ.

Sed Drako ne komprenis kion alian Harry \emph{povis} estis provanta.

Kaj stranga penso aperis al Drako, io pri kio Harry ne haltis paroli
hieraŭ.

Kaj la penso estis: \emph{Testu ĝin.}

\emph{Vi vekiĝis kiel sciencisto nun,} Harry estis dirinta, \emph{kaj
  eĉ se vi neniam lernis kiel uzi la povon, vi ĉiam, serĉos,
  manierojn, por testi, viajn kredojn\ldots{}} la malbonaŭguraj
vortoj, diritaj per anheloj pro agonio, kiuj daŭris kuri en la menso
de Drako.

Se Harry \emph{pretendis} esti la pentanta amiko kiu akcidente dolorigis iun\ldots{}

``Vi \emph{planis} tion, kion vi faris al mi!'' Drako diris,
sukcesante meti noton de akuzo en sia voĉo. ``Vi ne faris tion ĉar vi
koleris, vi faris tion ĉar vi \emph{volis} fari tion!''

\emph{Stultulo,} Harry dirontus, \emph{evidente ke mi planis tion, kaj nun vi estas mia—} 

Harry turnis sin al Drako. ``Tio, kio okazis hieraŭ, \emph{ne estis}
la plano,'' Harry diris, sia voĉo ŝajnante blokita en sia gorĝo. ``La
\emph{plano} estis ke mi instruas al vi kial estas ĉiam pli bona koni
la veron, kaj poste ke ni provas kune malkovri la veron pri la sango,
kaj ne gravas la respondo, ni akceptintus ĝin. Hieraŭ mi\ldots{}urĝis
la aĵojn.''

``Ĉiam pli bona koni la veron,'' Drako diris malvarme. ``Kvazaŭ vi
faris al mi \emph{favoron.}''

Harry kapjesis, komplete halucinante Drako'n, kaj diris, ``Kio se
Lucius havas la saman ideon ol mi, pri ke la problemo estas ke la plej
potencaj sorĉistoj havas malpli da infanoj? Li eble komencus planon
por pagi la plej potencaj pursanguloj por ke ili havu pli da
infanoj. Fakte, se la purismo de sango \emph{pravis}, tio estas kion
Liucus \emph{devus} fari—trakti la problemon per li mem, kiam li povas
okazigi aĵojn senprokraste. Nun, Drako, vi estas la sola amiko de
Lucius, kiu povas provi haltigi lin fari senutilan klopodon, ĉar vi
estas la sola persono kiu konas la \emph{realan} veron, kaj kiu povas
antaŭvidi la realajn rezultojn.''

Penso venis al Drako ke Harry Potter estis edukita en loko tiel
stranga ke li estis efektive magia besto anstataŭ sorĉisto. Drako
simple ne povis diveni tion, kion Harry diros aŭ faros poste.

``\emph{Kial?}'' Drako diris. Metante doloron kaj sento de perfido en
sia voĉo kiel eble plej forte. ``Kial vi \emph{faris} tion al mi? Kio
\emph{estis} la plano?''


``Verdire,'' Harry diris, ``vi estas la ido de Lucius, kaj kredu tion
aŭ ne, Dumbledore pensas ke mi apartenas al li. Do ni povas batali
ilian batalon unu kontraŭ la alia. Aŭ ni povas fari ion alian.''

Malrapide, la menso de Drako procesoris tion. ``Vi volas provoki
batalon por ilin fini, kaj poste preni la povon post ol ili estas
ambaŭ lacegaj.'' Drako sentis malvarman anksiecon en sia brusto. Li
devis provi haltigi tion, ne gravas la kosto por li mem—

Sed Harry kapneis. ``Nome de la steloj superaj, \emph{Ne!}''

``Ne\ldots{}?''

``Vi ne konsentos kun tio, kaj mi ankaŭ ne,'' diris Harry. ``Tio estas
\emph{nia} mondo, ni ne volas rompi ĝin. Sed imagu, diru ni, Lucius
pensis ke la konspiro estas via ilo kaj ke vi estas je lia flanko,
Dumbledore pensis ke la konspiro estas mia ilo kaj ke mi estas je lia
flanko, Lucius pensis ke vi farigis min ŝanĝi flankon, kaj Dumbledore
kredis ke la konspiro estis la mia, Dumbledore pensis ke mi farigis
vin ŝanĝi flankon kaj Lucius pensis ke la konspiro estis la via, kaj
do ili ambaŭ helpis nin, sed per manieroj kiujn neniu rimarkus.''

Drako ne bezonis ŝajnigi ke li estis senparola.

Lia patriĉo foje irigis lin vidi teatraĵon nomita \emph{La Tragedio de
  Lajto}, pri tiu \emph{nekredeble} inteligenta Serpentimo nomita
Lajto, kiu planis purigi la mondon de la malbono uzante malnovan ringon, kiu
mortigis ĉiujn ajn, kies nomoj kaj vizaĝoj estis konitaj de li. Kaj li
alfrontis alian nekredeble inteligentan Serpentimon, malbonulo nomita
Lolito, kiu portis maskoveston por kaŝi sian veran vizaĝon; kaj Drako
kriis kaj hurais je ĉiuj la bonaj partoj, speciale je la mezo; kaj tiam
la teatraĵo finis malĝoje kaj Drako estis terute seniluzigita kaj sia
patriĉo delikate indikis ka la vorto 'Tragedio' estis ĉi tie en la
titolo.

Poste, lia patriĉo demandis al Drako se li komprenis kial ili estis vidinta tiun teatraĵon.

Drako diris ke tio estis por instrui al li ke li estu ruza kiel Lajto kaj Lolito kiam li kreskos.

Lia patriĉo diris ke Drako ne povis esti pli malprava, kaj indikis ke
dum Lolito inteligente kaŝis sian vizaĝon, estis neniu bona kialo por
ke li diru al Lajto sian \emph{nomon}. Lia patriĉo poste detruis
preskaŭ ĉiujn partojn de la teatraĵo, dum Drako aŭskultis kun la
okuloj malfermiĝante pli kaj pli. Kaj lia patriĉo finis per diri ke
teatraĵoj kiel ĉi tiu estis \emph{ĉiam} nerealismaj, ĉar se la
verkanto sciis tion, kion iu \emph{fakte} inteligenta kiel Lajto
\emph{efektive} farus, la verkanto provos konkeri la mondon per li mem
anstataŭ nur skribi teatraĵojn pri tio.

Estis kiam lia patriĉo diris al Drako pri la regulo de tri, kiu estis
ke ĉiu komploto kiu bezonis ke pli ol tri malsamaj aĵoj okazas, neniam
funkcius en la reala vivo.

Lia patriĉo plue klarigis ke pro ke nur stultulo provos fari komploton
kiu estis \emph{kiel eble plej komplika}, la vera limo estis du.

Drako ne eĉ povis trovi vortojn por priskribi la krute giganta
malfarebleco de la ĉefa plano de Harry.

Sed tio estis \emph{nur} la speco de eraro kiun vi farus se vi ne
havus mentorojn kaj pensus esti inteligenta kaj lernus pri komplotoj
per rigardi teatraĵojn.

``Do,'' diris Harry, ``kion vi pensas pri la plano?''

``Ĝi estas lerta\ldots{}'' Drako diris malrapide. Krii
\emph{Brilanta!} kaj spiregi pro admiro aspektintus tro
suspektinda. ``Harry, ĉu mi povas demandi ion?''

``Evidente,'' diris Harry.

``Kial vi aĉetis al Granger multekostan haŭtpoŝon?''

``Por montri neniun malbonan emocion,'' diris Harry
senprokraste. ``Tamen mi antendas ke ŝi ankaŭ sentas sin malkonforta
se ŝi rifuzas iun ajn el miaj etaj petoj, kiujn mi povas fari dum la
kelkaj sekvaj monatoj.''

Kaj estis kiam Drako konsciis ke Harry estis fakte provanta esti lia amiko.

La movo de Harry kontraŭ Granger \emph{estis} inteligenta, eble eĉ
brilanta. Faru ke via malamiko ne suspketu vin, \emph{kaj} farigu rin
havi ŝuldojn al vi per amika maniero tiel ke vi povas manovri rin
\emph{nur per demandi al ri}. Drako ne povintus fari tion, sia celo
estintus tro suspektinda, sed la Knabo-Kiu-Postvivis \emph{povas.} Do
la unua etapo de la komploto de Harry estis doni al sia malamiko
multekostan donacon, Drako ne pensintus al tio, sed tio povus
\emph{funkcii\ldots{}}.

Se vi estus la malamiko de Harry, siaj komplotoj eble estus tro
malfacila por kompreni unue, ili eĉ eble aspektus stultaj, sed sia
rezonado havus sencon unufoje kiam vi komprenus ĝin, vi komprenus ke
li provis dolorigi vin.

La maniero per kio Harry agis al Drako nuntempe havis \emph{neniun} sencon.

Ĉar se vi estus la \emph{amiko} de Harry, tiam li provus esti amiko
kun vi per la ekstertera, nekomprenebla maniero, kiun Mugloj al li
instruis, eĉ se tio volis diri detrui vian tutan vivon.

La silento daŭris.

``Mi scias ke mi mistraktis nian amikecon terure,'' Harry diris
finfine. ``Sed bonvolu kompreni, Drako, ke mi nur volis ke la du el ni
trovas la veron kune. Ĉu tio estas io, kion vi povas pardoni?''

Vojforko kun du ebloj, sed kun nur unu vojo de kiu oni povis reveni
pli malfrue se Drako ŝanĝis sian opinion\ldots{}

``Mi supozas ke mi komprenas tion, kion vi provis fari,'' Drako mensongis, ``do jes.''

La okuloj de Harry lumiĝis. ``Mi feliĉas aŭdi tion, Drako,'' li diris milde.

La du sutdentoj staris en la alkovo, Harry ankoraŭ baniĝanta en la
suna radio, kaj Drako en la ombro.

Kaj Drako ekkonsciis kun noto de hororo kaj malespero, ke kvankam
estis efektive timiga destino esti la amiko de Harry, Harry nun havis
tiom multe da bulvardoj por minaci Drako'n ke esti lia malamiko estus
ankoraŭ pli \emph{malbona}.

Probable.

Eble.

Nu, li povus ĉiam ŝanĝiĝi en malamikon pli malfrue\ldots{}

Li estis kondamnita.

``Do,'' Drako diris. ``Kio nun?''

``Ni studos denove sabaton?''

``Tio prefere ne estu kiel la lasta fojo—''  

``Ne zorgu, tio ne estos,'' diris Harry. ``Kelkaj sabatoj kiel \emph{tio}, kaj vi estos antaŭ \emph{mi.}''

Harry ridis. Drako ne.

``Ho, kaj antaŭ ol vi foriras,'' Harry diris, kaj grimacis
idiote. ``Mi scias ke tio estas malbona momento, sed mi volas demandi
konsilon al vi pri io, fakte.''

``Konsentite,'' Drako diris, ankoraŭ distrita per la lasta deklaro.

La okuloj de Harry pli larĝiĝis intence. ``Aĉeti puŝon al Granger uzis
la plejmulton da la oro, kiun mi sukcesis ŝteli el mia sekurĉambro en
Gringoto—''

Kio.

``—kaj McGonagall havas la ŝlosilon de la sekurĉambro, aŭ Dumbledore
nun eble. Kaj mi estas debutigonta komploton kiu eble kostos monon, do
mi al mi demandis ĉu vi konas manieron por ke mi akiru aliron—''

``Mi pruntos monon al vi,'' diris la buŝo de Drako en kruta ekzisteca reflekso.

Harry aspektis konsternita, sed ĝoje. ``Drako, vi ne estas devigita—'' 

``Kiom?''

Harry diris la kvanton kaj Drako ne tute sukcesis gardi la ŝokon
ekster sia vizaĝo. Tio estis preskaŭ la tuto de la poŝmono kiun lia
patriĉo donis al Drako por daŭri la tutan jaron, Drako havus nur
kelkajn galionoj—

Tiam Drako mense frapis sin. Ĉio kion li bezonis fari estis skribi al
sia patriĉo kaj klarigi ke la mono estis foririnta ĉar li sukcesis
\emph{prunti ĝin al Harry Potter,} kaj sia patriĉo sendos al li
specialan gratulantan noton skribita per ora inko, kun giganta ĉokolada
rano, kiun prenos semajnon por esti manĝita, kaj dekoble pli da
galionoj, nur por la okazo se Harry Potter bezonis alian prunton.

``Tio estas tro multe, ĉu ne,'' diris Harry. ``Mi bedaŭras, mi devintus ne demandi—'' 

``Pardonu min, mi \emph{estas} Malfojo, vi scias,'' Drako diris. ``Mi nu estis surprizita ke vi \emph{volis} tiom multe.''

``Ne zorgu,'' Harry diris ĝoje. ``Tio estas nenio por minaci la intereson de via familio, nur mi estanta malbona.''

Drako kapjesis. ``Neniu problemo tiam. Vi volas iri preni ĝin nun?''

``Kompreneble,'' diris Harry.

Dum ili eliris la alkovon, kaj komencis direkti sin al la karceroj,
Drako ne sukcesis malhelpi sin demandi, ``Do ĉu \emph{vi povas} diri
al mi, por kiu komploto tio estas?''

``Rita Moskito.''

Drako sakregis mense, sed estis multe tro malfrue por diri ne.

\later

Antaŭ ol ili atingis la karceroj, Drako komencis kuntiri siajn pensojn.

Li \emph{havis} malfacilaĵojn por malami Harry'n Potter. Harry provis
esti amika, li nur estis freneza.

Sed tio ne estis haltigonta la venĝon de Drako aŭ eĉ malrapidigonta ĝin.

``Do,'' Drako diris, post ol li estis rigardinta ĉirkaŭe por certigi
ke neniu estis proksime. Iliaj voĉoj estus malklarigitaj, evidente,
sed neniam dolorigas esti pli certa. ``Mi pensis. Kiam ni venigos
novajn rekrutojn en la konspiro, ili devos \emph{pensi} ke ni estas
egalaj. Alie bezonus nur \emph{unu} el ili por sciigi mian patriĉon
pri la komploto. Vi jam pripensis pri tio, ĉu ne?''

``Kompreneble,'' diris Harry.

``\emph{Ĉu} ni estos egalaj?'' diris Drako.

``Mi timas ke ne,'' Harry diris. Estis klare ke li provis soni afable,
kaj ankaŭ klare ke li provis forigi grandan kvanton da disdegno, kaj
ke li ne tute sukcesis. ``Mi bedaŭras, Drako, sed vi ne eĉ scias kion
la vorto \emph{Bajezia} en \emph{Bajezia Konspiro} volas diri por la
momento. Vi devos studi dum monatoj antaŭ ol ni prenos neniun alian,
nur por ke vi povas ŝajni \emph{konvinka}.''

``Ĉar mi ne sufiĉe scias pri scienco,'' Drako diris, zorgeme gardante
sian voĉon neŭtrala.

Harry kapneis. ``La problemo ne estas ke vi estas malscia pri specifaj
sciencaj aĵoj kiel deoksiribonuklea acido. \emph{Tio} ne haltigus vin
por esti mia egalulo. La problem estas ke vi ne estas trejnita al la
metodoj de la racieco, la pli profonda sekreta kono pri kiel ĉiuj tiuj
malkovroj estis faritaj unue. Mi provos instrui tiujn al vi, sed ili
estas multe pli malfacilaj por lerni. Pensu pri tio, kion vi faris
hieraŭ, Drako. Jes, vi faris iom da la laboro. Sed mi estis
kontrolanta. Vi respondis al kelkaj demandoj. Mi demandis ĉiujn el
ili. Vi helpis. Mi faris la stiradon per mi mem. Kaj sen la metodoj de
la racieco, Drako, vi ne povas stiri la konspiron tien, kie ĝi bezonis
iri.''

``Mi vidas,'' diris Drako, lia voĉo sonante seniluziigita.

La voĉo de Harry provis afabliĝi ankoraŭ plu. ``Mi provos respekti
vian ekspertizon, Drako, pri aĵoj kiel la aferoj de la homoj. Sed vi
bezonas respekti mian ekspertizon ankaŭ, kaj estas nur neniu manerio
por ke vi povu esti mia egalulo kiam temas pri gvidi la konspiron. Vi
nur estas sciencisto de \emph{unu tago}, vi scias \emph{unu} sekreton
pri deoksiribonuklea acido, kaj vi ne estas trejnita al iu ajn el la
metodoj de la racieco.''

``Mi komprenas,'' diris Drako.

Kaj tio estis vera.

\emph{Aferoj de la homoj,} Harry diris. Kapti la kontrolon de la
konspiro verŝajne ne estus eĉ iomete malfacila. Kaj poste, li mortigus
Harry'n Potter nur por estis certa—

La memoro, pri kiom malsana li sentis sin lastan nokton sciante ke
Harry estis krianta, reaperis en la menso de Drako.

Drako sakris mense.

Bone. Li \emph{ne mortigos} Harry'n. Harry estis kreskita per Mugloj,
ne estis sia kulpo ke li estis freneza.

Anstataŭ, Harry vivos, nur por ke Drako povu diri al li ke tio estis
por lia propra bono, vere, kaj ke li devus esti dankema—

Kaj kun subita tiko de surprizita plezuro, Drako ekkonsciis ke tio
fakte \emph{estis} por la propra bono de Harry. Se Harry provas
plenumi sian planon pri provi preni Dumbledore kaj lia patriĉo por 
stultuloj, li simple \emph{mortos.}

Tio estis \emph{perfekta.}

Drako forprenos ĉiujn la revojn de Harry, ekzakte kiel Harry faris al li.

Drako diros al Harry ke tio estis por lia propra bono, kaj tio estos perfekte ĝusta.

Drako uzos la konspiron kaj la povo de la scienco por purigi la
sorĉistan mondon, kaj sia patriĉo estos fiera pri li, kiel se li
estintus Morto-Manĝanto.

La malbonaj komplotoj de Harry estos abortigitaj, kaj la forto de la
bono dominos.

La perfekta venĝo.

Krom se\ldots{}

\emph{Nur pretendu ke vi pretendas esti sciencisto,} Harry estis dirinta al li.

Drako ne havis la korektajn vortojn por priskribi tion, kio estis malĝusta kun la menso de Harry— 

(pro tio ke Drako neniam aŭdis la termon \emph{profundo de rikuro})

—sed li povis diveni, kiajn komplotojn tio implikis.

\ldots{}krom se ĉio, estis ekzakte tio, kion Harry \emph{volis} ke
Drako faru, kiel parto de ankoraŭ pli \emph{granda} komploto, al kiu
Drako partoprenus per provi abortigi tiun planon. Harry eble eĉ
\emph{sciis} ke sia plano estis nefarebla, ĝi eble havis neniun alian
celon \emph{krom} trompi Drako'n por ke li malhelpu ĝin—

Ne. Tio estis pura \emph{frenezeco.} Devus esti limo. La Mastro de la
Tenebroj, li mem, ne farintus tion tiel serpente. Tia aĵo ne okazis en
la vera vivo, nur en la stultaj historioj antaŭ enlitiĝo de lia
patriĉo, pri stultaj gargojloj, kiuj finfine plifruigis la planon de
la heroo, ĉiufoje kiam ili provas haltigi lin.

\later

Flanke de Drako, Harry marŝis kun rideto sur la vizaĝo, pensante pri
la originoj de la evoluado de la homa inteligento.

Je la komenco, antaŭ ol homoj tute komprenis kiel evoluado funkciis,
ili havis aberaciajn pensojn kiel \emph{homa inteligento evoluis por
  ke ni povu inventi pli bonajn ilojn.}

La kialo pro kio, tio estis aberacia estis ke nur unu persono en tribo
bezonis inventi ilon, kaj poste ĉiuj aliaj povus uzi ĝin, kaj ĝi
etendos al aliaj triboj, kaj ankoraŭ povos esti uzita per iliaj
posteuloj je kelkaj centoj da jaroj pli malfrue. Tio estis bonega laŭ
la perspektivo de la scienca progreso, sed laŭ evoluada termino, tio
volis diri ke la persono, kiu inventis ion, ne havis tiel multe da
evolua \emph{avantaĝo}, kaj do ne havis pli da infanoj ol la
aliaj. Nur \emph{relativa} evolua avantaĝo povas kreskigi la relativan
oftecon de geno en la populacio, kaj gvidi solan mutacion ĝis ĝi estas
universala kaj ĝi estas posedita de ĉiuj. Kaj brilaj inventoj nur ne
estis sufiĉe komunaj por provizi tian unuforman premon de selektado,
kiu estis necesa por promocii mutacion. Estis nature, ke se oni
rigardus homojn, kun iliaj fusiloj, tankoj kaj nukleaj armiloj, kaj
komparis ilin kun la ĉimpanzoj, oni povus konjekti ke la inteligento
estis tie por fari teĥnologion. Nature, sed malprave.

Antaŭ ol homoj tute komprenis kiel evoluado funkciis, ili havis
aberaciajn pensojn kiel \emph{la klimato ŝanĝis, kaj triboj devis
  migri, kaj homoj estis devigitaj fariĝi pli inteligentaj por ke ili
  povu solvi ĉiujn la novajn problemojn.}

Sed homoj havis cerbon kvaroble pli granda ol tiu de ĉimpanzo. 20\% de
la metabola energio de la homo nutris la cerbon. Homoj estis
\emph{ridinde} pli inteligenta ol aliaj specioj. Tia aĵo ne okazis pro
tio ke la medio farigis la malfacilecon de siaj problemoj iom pli
alta. Tial la organismoj fariĝus nur iomete pli inteligenta. Fini kun
tiu giganta grandega cerbo probable necesis ian \emph{forkurantan}
evoluan procezon, io kio puŝus kaj puŝus sen limoj.

Kaj hodiaŭaj sciencistoj havis relative bonan ideon pri tio, kio estis
tiu forkuranta evolua procezo.

Harry unufoje legis faman libron nomita \emph{Politiko de
  ĉimpanzo}. La libro priskribis kiel plenkreska ĉimpanzo nomita Luit
konfrontis maljuiĝa alfulo, Jeroen, kun la helpo de juna, kaj de
nelonge matura ĉimpanzo nomita Niki. Niki neniam intervenis direkte
dum la bataloj inter Luit kaj Jeroen, sed preventis la aliaj
subtenantoj de Jeroen por ke ili ne helpu lin, distrante ilin ĉiam
kiam konfliktoj aperis inter Luit kaj Jeroen. Kaj kiam Luit venkis,
kaj fariĝis la nova alfa, kun Niki estante la dua plej potenca\ldots{}

\ldots{}tamen ne prenis multe da tempo antaŭ ol Niki formis aliancon
kun la venkita Jeroen, renversis Luit'n, kaj fariĝis la \emph{nova}
nova alfa.

Tio vere farigis vin estimi tion, kion milionoj da jaroj da prahomoj
provantaj superruzi \emph{unu la alian}—evolua batalo sen limoj—gvidis
al la vojo de kreskinta mensa kapablo.

'ĉar, v'scias, homo tute vidintus tion veni.

\later

Kaj apud Harry, Drako marŝis, forigante la rideton kiun li havis pensante pri sia venĝo.

Iam, eble post jaroj sed iam, Harry Potter lernos ekzakte tion, kion volis diri subtaksi Malfojo'n.

Drako vekiĝis kiel sciencisto post nur unu tago. Harry diris ke tio ne
estis supozita okazi antaŭ monatoj.

Sed evidente, se vi estis Malfojo, vi estus pli potencaj sciencistoj
ol iu kiu ne estis Malfojo.

Do Drako lernos ĉiujn la metodojn de la racieco de Harry Potter, kaj
tiam kiam la momento estiĝos—

