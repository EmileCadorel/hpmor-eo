\chapter{Kontrolo de Impulsoj}

\lettrinepara[ante=“]{T}{urpin,} Lisa!”

\hplettrineextrapara
Murmureto murmureto murmureto harry potter murmureto murmureto
serpentimo murmureto murmureto ne serioze damne murmureto murmureto

``KORVUNGO!''

Harry kuniĝis la aplaŭdojn, kiuj hurais la junan knabinon kiu marŝis
timide al la Korvunga tablo, dum la garnituroj de ŝiaj roboj ŝanĝiĝis
en malhelan bluon. Lisa Turpin ŝajnis ŝirita inter sia deziro de
sidiĝi kiel eble malproksime de Harry Potter, kaj sia deziro hasti kaj
inserti sin perforte apud li por komenci elŝiri respondojn el li.

La fakto ke li estis je la centro de eksterordinara kaj kurioza
evento, kaj poste ke li estis Ordigita en la Domo Korvungo, estis
preskaŭ kvazaŭ li estis trempita en rostokradra saŭco kaj ĵetita en
kavon plena da malsategaj katidoj.

``Mi promesis al la Ordiganta Ĉapelo ke mi ne parolos pri tio,''
murmuris Harry por la nenombrebla fojo.

``Jes, vere.''

``Ne, mi vere promesis al la Ordiganta Ĉapelo ke mi ne parolos pri tio.''

``Bonege, mi promesis al la Ordiganta Ĉapelo ke mi ne parolos pri
\emph{la plimulto} kaj la cetero estas \emph{privata} ekzakte kiel
\emph{la via estis}, do \emph{haltu demandi}.''


``Vi volas scii, kio okazis? Bonege! Jen parto de kio okazis! Mi diris
al la Ĉapelo ke la Profesorino McGonagall minacis fajrigi ĝin kaj ĝi
postulis ke mi diru al Profesorino McGonagall ke ŝi estas
impertinenta junulino kaj ke ŝi devus foriri ĝian herbejon.''

``Se vi ne kredas, kion mi diras, tial \emph{Kial vi demandas?''}

``Ne, mi ne ankaŭ scias kiel mi venkis la Mastron de la Tenebroj! Diru
al mi se vi tion malkovras!''

\emph{``Silente!''} kriis Profesorino McGonagall de la podio de la
Ĉefa Tablo. \emph{``Neniu parolado ĝis la fino de la Ceremonio de la
Ordigado''}

Estis mallonga falo de la laŭteco, tial ke ĉiuj atendis por vidi ĉu ŝi
faros specifan kaj kredeblan minacon, kaj poste la murmuroj
rekomencis.

Poste la maljunulo kun arĝenta barbo strariĝis desur sia grandega ora
seĝo, ridetante bonhumore.

Senpera silento. Iu panike donis baton per la kubuto al Harry kiam li
provis daŭrigi murmuron, kaj Harry haltis meze de frazo.

La maljuna viriĉo kun bonhumora aspekto residiĝis.

\emph{Noto por mi mem: Oni ne ŝercas kun Dumbledore}

Harry provis ankoraŭ kompreni ĉion kio okazis dum la incidento kun la
Ordiganta Ĉapelo. Kaj ne malpli grava estis tio kio okazis sammomente
kiam Harry elmetis la Ĉapelon desur sia kapo; je tiu momento, li aŭdis
etan murmuron kvazaŭ el nenie, io kio sonis strange kiel Anglalingvo
kaj siblo je la sama momento, io kio diris, \emph{``Ssalutadon de
  Sserpentimo al Sserpentimo: Sse vi volass sserĉi miajn ssektretojn,
  parolu al mia sserpento.''}

Harry kvazaŭdivenis ke tio ne estis supozita esti parto de la oficiala
Ordigada procezo. Kaj ke tio estis iom da ekstra magio enmetita tie
per Salazaro Serpentimo dum la kreado de la Ĉapelo. Kaj ke la Ĉapelo
ĝi mem ne sciis pri tio. Kaj ke tio estis kaŭzita kiam la Ĉapelo diris
``SERPENTIMO'', kun eble kelkaj aliaj kondiĉoj. Kaj ke iu Korvungo
kiel li \emph{vere, vere ne estis supozata aŭdi tion.}  Kaj ke se li
povus trovi iun fidindan manerion por farigi Drako'n ĵuri gardi la
sekreton, tial li povus demandi pri tio al li. Tio estus belega
momento por havi Komed-Teon ĉemane.

\emph{Knabo, vi decidis ne sekvi malsupren la vojon de Mastro de la
  Tenebroj kaj la universo komencas ŝerci kun vi je la momento kiam la
  Ĉapelo eliras de via kapo. Kelkfoje, tio nur ne pagas batali la
  destinon. Eble mi atendos ĝis morgaŭ por komenci mian propradecidon
  pri ne esti Mastro de la Tenebroj.}

``GRIFINDORO!''

Ron Wizle akiris multe da aplaŭdoj, kaj ne nur el
Grifindoroj. Verŝajne la familio Wizle estis vaste ŝatita ĉirkaŭ
tie. Harry post momento, ridetis kaj komencis aplaŭdi kun la aliaj.

Tamen, ne estis pli bona momento ol hodiaŭ por turni sian dorson al la
Malhela Flanko.

Al la diablo la destino kaj al la diablo la universo. Li tion montros
al la Ĉapelo.

``Zabini, Blaise!''

Paŭzo.

``SERPENTIMO!'' kriis la Ĉapelo.

Harry aplaŭdis Zabini'n ankaŭ, ignorante la strangan rigardon, kiujn li
akiris de ĉiuj inkluzive de Zabini.

Neniu alia nomo estis vokita post tio, kaj Harry ekkonsciis ke
``Zabini, Blaise'' sonis proksime al la fino de la alfabeto. Bonege,
do nun li \emph{nur} aplaŭdis Zabini'n\ldots Ha nu.

Dumbledore stariĝis denove, kaj komencis sin direkti al la
podio. Verŝajne ili aŭdos paroladon—

Kaj Harry estis frapita per la inspiro de \emph{brila} testo
eksperimenta.

Hermione estis dirinta ke Dumbledore estis la plej potenca sorĉisto
ankoraŭ vivanta, ĉu ne?

Harry atingis sian poŝon kaj murmuris, ``Komed-Teo''.

Por ke la Komed-Teo funkcius, ĝi devus farigi ke Dumbledore diru ion
\emph{tiel} ridinda dum lia parolado, ke eĉ en la stato de mensa
prepariteco de Harry, li sufokiĝus. Kiel ke, ĉiuj la
lernantoj de Herpŭrko devus ne porti iun ajn vestojn dum la tuta
lerneja jaro, aŭ ke ĉiuj estos ŝanĝitaj en katojn.

Tamen se \emph{iu ajn en la mondo} povis rezisti la povon de la
Komed-Teo, tiu devus esti Dumbledore. Do se tio funkciis, la Komed-Teo
estis laŭlitere \emph{nevenkebla}.

Harry retiris la ringon de la Komed-Teo sub la tablo, volante fari
tion diskrete. La boteleto faris kvietan siblan bruon. Kelkaj kapoj
turniĝis por rigardi lin, sed baldaŭ returniĝis kiam—

``Bonvenu! Bonvenu al nova jaro en Herpŭrko!'' diris Dumbledore,
radiante al la lernantoj kun la brakoj malfermitaj larĝe, kvazaŭ
nenio povis plaĉi al li pli ol vidi ĉiujn ilin tie.

Harry prenis unuan plenbuŝon da Komed-Teo, kaj mallevis la boteleton
denove. Li provos gluti la sodakvon iom post iom, kaj provos ne
sufokiĝi, ne gravas tio kion Dumbledore dirus—

``Antaŭ ol ni komencas nian bankedon, mi deziras diri kelkajn
vortojn. Jen ili estas: Feliĉa feliĉa bum bum marĉo marĉo marĉo!
Dankon!''

Ĉiuj aplaŭdis kaj hurais, kaj Dumbledore sidiĝis denove.

Harry sidis frostigita dum sodakvo gutis el la anguloj de sia buŝo. Li
sukcesis, almenaŭ, sufokiĝi silente.

Li vere vere \emph{vere} devintus ne fari tion. Mirige kiel \emph{pli
  evidenta} tio estis \emph{unu sekundo} post ol tio \emph{tro
  malfruis}.

Retrospektive, li verŝajne devintus rimarki ke io estis malbona kiam
li pensis pri ĉiuj estante ŝanĝitaj en katojn\ldots aŭ eĉ antaŭ
tio. Li devintus memori lian mentalan noton pri ne ŝerci kun
Dumbledore\ldots aŭ lian freŝan propradecidon pri esti pli atenta al
aliaj\ldots aŭ eble li ne havis \emph{iun ajn peceton} da komuna
saĝo\ldots

Tio estis senespera. Li estis koruptita je la kerno. Saluto al la
Mastro de la Tenebroj Harry. Vi ne povas batali kontraŭ la fatalo.

Iu demandis al Harry ĉu ĉiu estis bonorde. (Aliaj komencis servi al si
manĝaĵojn, kiu magie aperis sur la tablo, nu tio ne gravas.)

``Ĉiu estas bonorde,'' Harry diris. ``Pardonu min. Hm. Ĉu tiu
estis\ldots normala parolado por la Lernejestro? Vi ĉiuj\ldots ne
ŝajnas\ldots tre surprizitaj\ldots''

``Ho, Dumbledore estas freneza, evidente,'' diris Korvungo kun pli aĝa
aspekto, sidante apud Harry, kaj kiu prezentis sin kun nomo, kiun
Harry tute ne memoris. ``Tre amuza, nekredeble potenca sorĉisto, sed
tute freneza.'' Li paŭzis. ``Pli malfrue mi dezirus ankaŭ demandi kial
viaj lipoj likis verdan fluaĵon kiu poste malaperis, tamen mi
anticipas ke vi promesis al la Ordiganta Ĉapelo ke vi ne parolu pri
tio ankaŭ.''

Kun grandega peno, Harry malhelpis sin antaŭ ol li rigardis malsupren
al la kulpiganta boteleto de Komed-Teo en sia mano.

Finfine, la Komed-Teo ne simple arbitre \emph{materiigis} ĉeftitolon
de la Ĉikananto pri li kaj Drako. Drako klarigis tion per maniero kiu
farigis ke tio ŝajnis kvazaŭ ĝi okazis\ldots nature? Kvazaŭ ĝi
\emph{ŝanĝis la historion por taŭgi?}

Harry estis mentale imagante sin frapante sian fronton kontraŭ la
tablo. \emph{Pum pum pum} iris sia kapo en sia menso.

Alia lernanto malaltigis sian voĉon al murmuro. ``Mi aŭdis ke
Dumbledore estas sekrete genia elpensinto kiu kontrolas multe da aĵoj
kaj ke li uzas la frenezon kiel kovrilo tiel ke neniu suspektas lin.''

``Mi aŭdis tion ankaŭ,'' murmuris tria lernanto, kaj estis kaŝaj
kapjesoj ĉirkaŭ la tablo.

Tio ne povis malhelpi kapti la atenton de Harry.

``Mi vidas,'' murmuris Harry, malaltigante sian propran voĉon. ``Do
ĉiuj scias ke Dumbledore estas sekrete elpensinto.''

Multe el la lernantoj kapjesis. Unu aŭ du aspektis subite pripensemaj,
inkluzive de la pli aĝa lernanto sidanta apud Harry.

\emph{Ĉu vi estas certa ke tiu estas la tablo de Korvungo?} Harry
sukcesis ne demandi tion laŭte.

``Brila!'' Harry murmuris. ``Se ĉiuj scias, neniu suspektos ke tiu
estas sekreto!''

``Ekzakte,'' murmuris lernanto, kaj poste li malridetis. ``Atendu, tio
ne ŝajnas tute prava—''

\emph{Noto por mi mem: La 4-a kvarilo de Herpŭrko lernantoj, alinome
Korvunga Domo, ne estas la plej ekskluziva programo de la Mondo por
talentaj infanoj.}

Sed almenaŭ li lernis gravan fakton hodiaŭ. La Komed-Teo estis
ĉiopova. Kaj \emph{tio} volis diri\ldots

Harry palpebrumis mirinde kiam lia menso finfine faris la evidentan
kunligon.

\ldots \emph{tio} volis diri ke ĵus kiam li malkovros sorĉon por
portempe falsi sian senton de humuro, li povus farigi ke io ajn okazu,
nur per fari ke ĝi estu la sola aĵo, kiu sufiĉe surprizus lin por
sufokiĝi, kaj tiam trinki la boteleton de Komed-Teo.

\emph{Nu, tio estis mallongeta vojaĝo al dieco? Êc mi pensis ke tio
  prenus pli da tempo ol mia unua tago de lernejo.}

Kiam oni pripensas pri tio, li ankaŭ tute detruis Herpŭrko'n ĵus dek
minutoj post kiam li estis ordigita.

Harry sentis certan kvanton da bedaŭro pri tio—nur Merlino sciis kion
freneza Lernejestro faros al liaj sep sekvaj jaroj de lernejo—sed li
ne povis sin \emph{malhelpi} ne senti pikon da fiero, ankaŭ.

Morgaŭ. Ne pli malfrue ol morgaŭ, morgaŭ plej malfrue, li haltigos
sian marŝon laŭ la vojo, kiu kondukis al la Mastro de la Tenebroj
Harry. Perpektivo kiu ŝajnis pli timega minutoj post minutoj.

Kaj nun ankaŭ, iel, pli kaj pli alloga. Parto de lia menso estis jam
bildiganta la uniformojn de la servistoj.

``Manĝu,'' la pli aĝa lernanto sidante apud li graŭlis, pikante
Harry'n je la ripoj. ``Ne pensu. Manĝu.''

Harry aŭtomate komencis plenigi sian teleron kun io ajn estis antaŭ
li, bluaj kolbasoj kun etaj brilantaj partoj, tio ne gravis.

``Pri kio vi estis pensanta, la Ordigado—'' komencis diri Padma Patil,
unu el la aliaj Korvungoj de unua jaro.

``Neniu ĝenado dum manĝhoroj!'' diris almenaŭ tri personoj
ĥore. ``Regulo de la Domo!'' aldonis alio. ``Se ne, ni ĉiuj malsatos
ĉirkaŭ tie.''

Harry trovis sin vere, vere esperante ke sia lerta nova ideo
\emph{fakte} ne funkcius. Kaj esperante ke la Komed-Teo funkciis per
iu alia manerio, kaj ne \emph{reale} havis la ĉiapovon de ŝanĝi la
realecon. Tio estis nur ke li ne povis elporti la penson ke li vivis
en universo kiu vere funkciis tiele. Estis io \emph{malinda} pri
kreksi per la ruza uzo de ŝaŭma trinkaĵo.

Sed li \emph{estis} testonta tion eksperimente.

``Vi scias,'' diris la pli aĝa lernanto apud li per kvieta agrabla
tono, ``Ni havas sistemon por trudi ke personoj kiel vi manĝu, ĉu vi
volas malkovri kion ĝi estas?''

Harry rezignis kaj komencis manĝi sian bluan kolbason. Tiu estis
sufiĉe bona, speciale la brilantaj partoj.

Vespermanĝo pasis je surpriza rapideco. Harry provis gustumi almenaŭ
etan parton de la bizaraj novaj manĝaĵoj, kiujn li vidis. Sia scivolo
ne povis elporti la penson \emph{ne scii} kiel io gustis. Dank' al
Dio, tiu ne estis restoracio en kiu vi devis peti nur unu aĵon kaj kie
vi neniam malkovrus kiel ĉiuj la aliaj aĵoj sur la menuo gustis. Harry
\emph{malamis} tion. Tio estis kvazaŭ tortura ĉambro por iu ajn kun
eĉ iomete da scivolo: \emph{Malkovru nur unu el la misteroj de la listo,
  ha ha ha!}

Tiam, estis la momento de la deserto, por kiu Harry tute forgesis
gardi spacon. Li rezignis post gustumi etan pecon da
melastorto. Certe, ĉiuj tiuj aĵoj repasus almenaŭ unufojon plie dum la
daŭro de la lerneja jaro.

Do, kio estis sur lia listo de aĵoj por fari, krom la ordinaraj
lernejaj aĵoj?

\emph{Aĵo por fari 1. Serĉi sorcôjn por alteracii menson por ke vi
  povu testi la Komed-Teon kaj vidi ĉu vi vere malkovris vojon al
  ĉiapovo. Fakte, simple serĉi ian ajn mensan magion, kiun vi povas
  trovi. Menso estas la fondaĵo de nia povo kiel homo, ia ajn magio
  kiu afekcias ĝin estas de la plej grava speco de magio ekzistanta.}

\emph{Aĵo por fari 2. Fakte tiu estas Âjo 1 kaj la alia estas Aĵo
  2. Iri tra la librobretoj de la bibliotekoj de Herpŭrko kaj
  Korvungo, kutimiĝi pri la sistemo kaj certigi ke vi almenaŭ legis
  ĉiujn titolojn de la libroj. Dua paso: legi ĉiujn
  enhavotabelojn. Kunordigi kun Hermione kiu havas multe pli bona
  memoron ol vi. Malkovri ĉu estas sistemo de prunto inter bibliotekoj
  en Herpŭrko kaj vidi ĉu vi ambaŭ, speciale Hermione, povas viziti
  tiujn bibliotekojn ankaŭ. Se alia Domo havas privatajn bibliotekojn,
  malkovri kiel eniri laŭleĝe aŭ ŝtelenirante.}

\emph{Opcio 3a: Fari ke Hermione ĵuru gardi la sekreton kaj provi komenci
serĉi pri 'De Serpentimo al Serpentimo: se vi volas serĉi miajn
sekretojn, parolu al mia serpento.' Problemoj: Tio ŝajnas tre
konfidenca kaj tio povus preni multe da tempo hazarde trairi librojn
enhavantaj indikaĵojn.}

\emph{Aĵo por fari 0. Kontroli kiaj sorĉoj por serĉi kaj ekstrakti
  informojn ekzistas, se tiaj ekzistas. Biblioteka magio ne estas
  finfine tiel grava kiel mensa magio, sed tia havas multe pli alta
  prioritato.}

\emph{Opcio 3b: Serĉi sorĉon por magie ligi Drako'n al sekreto, aŭ
  magie kontroli la sincerecon de la promeso de Drako pri gardi la
  sekreton (Veritaserumo?) kaj poste demandi} al li \emph{pri la
  mesaĝo de Serpentimo}

Fakte\ldots Harry havis relative malbonan senton pri la opcio 3b.

Nun ke Harry pensis pri tio, li ankaŭ ne sentis sin tiel tute bone pri
la Opcio 3a.

La pensoj de Harry rememoris la verŝajne plej malbonan momenton de sia
tuta vivo, tiuj longaj sekundoj de hororo el frostigita sango sub la
Ĉapelo, kiam li pensis ke li jam malsukcesis. Li deziris tiam reiri
ĵus kelkajn minutojn antaŭen por ŝanĝi ion, io ajn antaŭ ol estu tro
malfrue\ldots

Kaj poste rezultis ke ne estis tro malfrue finfine.

Lia deziro estis koncedita.

Vi ne povis ŝanĝi historion. Sed vi povis fari ke ĝi iru bone
komence. Faru ion malsimilan \emph{la unuan} fojon.

Tiu tuta afero pri serĉi la sekretojn de Serpentimo\ldots ŝajnis esti
horore simila al kiel iaj aferoj estas, multe da jaroj pli malfrue,
kiam vi rigardas malantaŭen por diri, 'Kaj \emph{tiam} estis kiam ĉio
komencis iri malbone'.

Lia deziro estis koncedita. Nun kio?.

Harry malrapide ridetis.

Tio estis relative kontraŭ intuicia penso\ldots sed\ldots

Sed li \emph{povis}, estis neniu regulo dirante ke li ne povis, li
\emph{povis} simple pretendi ke li neniam aŭdis ĉi tiun etan
murmuron. Lasi la universon iri laŭ vojo ekzakte simila al la vojo laŭ
kiu ĝi irintus se tiu kritika momento neniam okazus. Je dudek jaroj
pli malfrue, estos kiel li senespere dezirus ke tio okazu dudek jaroj
pli frue, kaj dudek jaroj antaŭ dudek jaroj pli malfrue estis ĵus
nun. Ŝanĝi la foran estintecon estis facila, vi nur devis pensi pri
tio je la bona momento.

Aŭ\ldots tio estis ankoraŭ \emph{pli} kontraŭ intuicia\ldots li povis
eĉ informi, oh, oni diru, \emph{Profesorinon McGonagall}, anstataŭ
Drako \emph{aŭ} Hermione. Kaj ŝi povus trovi kelkajn bonajn personojn por
fari ke tiu eta ekstra sorĉo malaperu de la Ĉapelo.

Jes, evidente. Tio ŝajnis \emph{rimarkinde} bona ideo, unufoje kiam
Harry fakte \emph{pensis} pri ĝi.

Tiel evidenta retrospektive, sed strange iel, tiuj opcioj 3c kaj 3d ne
envenis lian menson.

Harry premiis sin mem per +1 poento por sia
kontraŭ-Mastro-de-la-Tenebroj-Harry programo.

Tiu estis kruela petolo kiun la Ĉapelo ludis al li, sed vi ne povas
argumenti la rezultojn pro konsekvencaj kialoj. Tamen, tio certe donis
al li vidon pli ĝustan de la perspektivo de la viktimo.

\emph{Aĵo por fari 4: Pardonon peti al Nevilo Longafundo.}

Konsentite, li estis laŭ la bona vojo tie, nun li nur devis resti sur
ĝi. \emph{Ĉiuj tagoj, per ĉiuj manieroj, mi iĝas pli hela kaj pli
  hela\ldots}

La plejparto de homoj ĉirkaŭ Harry ankaŭ haltis manĝi tiun momenton,
kaj la manĝilaroj servante la desertojn komencis neniiĝi, kaj estis
same por la uzitaj teleroj.

Kiam ĉiuj la teleroj estis neniiĝintaj, Dumbledore unu fojon plian
stariĝis de sia seĝo.

Harry ne povis malhelpi sin, sed sentis urĝon trinki alian
Komed-Teo'n.

\emph{Vi \emph{volas} ŝerci,} Harry direktis la penson al tiu parto de li mem.

Sed la eksperimento ne havis valoron se ĝi ne estas reproduktita, ĉu
ne?  Kaj la damaĝo estis jam farita, ĉu ne? Ĉu li ne volis vidi tion,
kio okazus \emph{ĉi tiun} fojon? Ĉu li ne estis \emph{scivola}? Kio se
li akiris malsaman rezulton?

\emph{He, mi vetas ke vi estas la sama parto de mia cerbo kiu puŝis
  min por fari la petolon al Nevilo Longafundo.}

Er, eble?

\emph{Kaj ĉu tio ne estas} eksterordinare \emph{evidenta ke se mi
  farus tion, mi bedaŭrus unu sekundo post kiam estos tro malfrue?}

Hum\ldots

\emph{Jes. Do, NE.}

``Ahem,'' diris Dumbledore de la podio, karesante sian longan arĝentan
barbon. ``Nur kelkaj pliaj vortoj nun ke ni ĉiuj manĝis kaj
trinkis. Mi havas kelkajn komencajn regulojn, kiujn mi volus doni al
vi.''

``Lernantoj de unua jaro devas noti ke la arbaro sur la tereno estas
malpermesata al ĉiuj. Tio estas kial ĝi estas nomata la Malpermesata
Arbaro. Se ĝi estus permesita ĝi estus nomita la Permesata Arbaro.''

Simplega. \emph{Noto por mi mem: Malpermesata Arbaro estas
  malpermesata.}

``Estis ankaŭ demandita al mi per S-ro Ŝtello$^{\textit{n:\ref{nomoj:ŝtello}}}$,
la kortisto, ke mi rememorigu ĉiujn el vi ke nenia magio devas esti
uzita inter instruadoj en la koridoroj. Bedaŭrinde, ni ĉiuj scias ke
kio \emph{devus esti}, kaj kio \emph{estas}, estas du malsamaj
aĵoj. Dankon por gardi tion enmense.''

Er\ldots

``Juĝadoj de Kvidiĉo okazos je la dua semajno de la trimestro. Iu ajn
interesita pri ludi por la teamo de sia Domo devus kontakti S-ino
Alto. Iu ajn interesita pri komplete redifini
la Kvidiĉo'n devus kontakti Harry'n Potter.''

Harry enspiris sian propran salivon kaj komencis tusi ĵus kiam ĉiuj
okuloj turniĝis al li. Kiel \emph{diable}! Li ne renkontis la okulojn
de Dumbledore iam ajn\ldots li tion ne \emph{opiniis}. Li certe ne
pensis pri Kvidiĉo iam ajn! Li ne parolis pri tio al iu ajn krom Ron
Wizle kaj li ne opiniis ke Ron dirintus tion al iu ajn alia\ldots aŭ
Ron kuris al profesoro por plendi? \emph{Kiel damne\ldots}

``Aldone, mi devas diri al vi ke ĉi tiun jaron, la koridoro de la tria
etaĝo je la dekstra flanko estas ekster la atingeblo de iu ajn kiu ne
deziras morti per tre dolora morto. Tiu estas gardata per komplikaj
serioj da danĝeraj kaj potenciale mortigaj kaptiloj, kaj vi ne povas
superi ĉiujn el ili, speciale se vi estas nur en unua jaro.''

Harry estis sensenta tiam.

``Kaj finfine, mi prezentas miajn plej grandajn dankojn al Cirinus
Ciuro por heroe konsenti preni la pozicion de Profesoro de Defendo
kontraŭ la Malhelaj Artoj en Herpŭrko.'' La rigardo de Dumbledore
movis serĉante trans la lernantoj. ``Mi esperas ke ĉiuj lernantoj
kondutos plej eble ĝentile kaj \emph{tolereme} al la Profesoro Ciuro,
pro la eksterordinara servo kiun li faras al vi kaj al la lernejo, kaj
ke vi \emph{ne tedos nin} kun iu ajn \emph{ĝema plendo} pri li, krom
se \emph{vi} volas provi fari sian laboron.''

Pri kio \emph{tio} temis?

``Mi nun cedas la lokon al nia nova instruisto, Profesoriĉo Ciuro, kiu
deziras diri kelkajn vortojn.''

La juna, maldika, nervoza viriĉo, kiun Harry unue renkontis en la
Likanta Kaldrono malrapide faris sian vojon al la podio, rigardante
time je ĉiuj direktoj. Harry ekvidis la malantaŭon de lia kapo, kaj
ŝajnis ke la Profesoriĉo Ciuro jam iĝis kalva, malgraŭ sia ŝajna
juneco.

``Mi al mi demandas, kio estas malbona kun \emph{li,}'' murmuris la
lernanto, kiu aspektis pli aĝa kaj sidis apud Harry. Similaj kvietaj
komentoj estis interŝanĝitaj aliloke laŭ la tablo.

Profesoro Ciuro faris sian vojon al la podio, kaj staris tie,
palpebrumante. ``Ha\ldots'' li diris. ``Ha\ldots'' Tiam lia kuraĝo
ŝajnis foriĝinta komplete, kaj li staris tie silente, kelkfoje
ektremante.

``Ho, bone,'' murmuris la pli aĝa lernanto, ``ŝajnas kiel alia
\emph{longa} jaro en kurso de Defendo—''

``Saluton, miaj junaj lernantoj,'' Profesoro Ciuro diris kun seka,
memfida tono. ``Ni ĉiuj scias ke Herpŭrko emas suferi pro certa
malfeliĉo je ĝia elekto de tiu ĉi pozicio, kaj sendube, multe el vi
jam sin demandas kiu pereo devos okazi al mi tiun jaron. Mi certigas
vin, ke tiu pereo ne estos mia nekompetenteco.'' Li ridetis
mallarĝe. ``Kredu tion aŭ ne, mi delonge deziris iam provi labori kiel
la Profesoro de Defendo kontraŭ la Malhelaj Artoj tie en Herpŭrko, la
Lernejo de Magio. La unua persono kiu instruis tiun kurson estis
Salazaro Serpentimo li mem, kaj ĝis la dek-kvara jarcento, estis
tradicio ke la plej grandaj sorĉistoj de batalo de ĉiu inklino
instruis tie. Inter la estintaj Profesoroj de Defendo estis ne nur la
legenda vaganta heroo Harold Shea sed ankaŭ la, mi citas, senmorta,
fino de citaĵo, Baba Yaga, jes, mi vidas kelkajn el vi estante ankoraŭ
ektremantaj je la sono de ŝia nomo kvankam ŝi mortis sescent jaroj
antaŭe. Tiu tempo devis esti interesa por ĉeesti en Herpŭrko, ĉu vi ne
opinias?''

Harry glutis forte, provante forigi la subitan ondegon da emocioj kiu
transiris lin kiam Profesoriĉo Ciuro komencis paroli. La preciza tono
multe memorigis lin pri iu docento de Oksfordo, kaj Harry komprenetis
ke li ne vidos sian hejmon aŭ siajn Panjo kaj Paĉjo ĝis Kristnasko.

``Vi kutimas pri ke la pozicio de Defendo estas okupata per
nekompetentuloj, kanajloj kaj malbonŝanculoj. Por iu ajn kiu iomete
konas de historion, ĝi portas komplete alian reputacion. Ne ĉiuj kiuj
instruis tie estis la plej bonaj, sed la plej bonaj ĉiuj instruis en
Herpŭrko. Kun konsiderinda akompanantaro, kaj post tiom da tempo
anticipante tiun ĉi tagon, mi estus hontinda se mi fiksus al mi mem
normon pli malalta ol la perfekto. Kaj do mi intencas ke ĉiuj el vi
ĉiam memoros tiun ĉi jaron kiel la \emph{plej bona} kurso de Defendo
kiun vi iam ajn havis. Kion vi lernos tiun jaron servos eterne kiel
via firma fundamento de la artoj de Defendo, ne gravas kiu estis kaj
kiu estos viaj instruistoj.''

La esprimo de la Profesoriĉo Ciuro iĝis serioza. ``Ni havas
\emph{multe da} terenon por regajni kaj ne multe da tempo por tion
fari. Tial mi intencas foririgi la instruajn konvenciojn de Herpŭrko
je multe da aspektoj, kaj ankaŭ aldoni fakultativajn kursojn post
lerneja tempo.'' Li faris paŭzon. ``Se tio ne estas sufiĉa, eble mi
povas trovi novajn manierojn por motivi vin. Vi estas miaj longe
atenditaj lernantoj, kaj vi \emph{faros} laŭeble viaj \emph{plej} bone
en mia longe atendita kurso de Defendo. Mi aldonus ian teruran
minacon, kiel 'Alie vi suferos horore', sed tio estus kliŝo, ĉu vi ne
opinias?  Mi fieras pri esti pli imagema ol tio. Dankon.''

Tiam la vigleco kaj memfido ŝajnis drenitaj el la Profesoriĉo
Ciuro. Lia buŝo malfermiĝis gape kvazaŭ li subite troviĝis sin fronte
al neatendita aŭdienco, kaj li turniĝis per spasma ekskuo kaj reiris
al sia seĝo, fleksita kvazaŭ li estis kolapsonta sur li mem kaj
implodonta.

``Li ŝajnas iom bizara,'' murmuris Harry.

``Meh,'' diris la lernanto kiu aspektis pli aĝa. ``Vi vidis nenion.''

Dumbledore reiris al la podio.

``Kaj nun,'' diris Dumbledore, ``antaŭ ol ni enlitiĝos, kantu ni la
kanton de la lernejo! Ĉiuj prenu sian plej ŝatan melodion kaj sian
plej ŝatajn parolojn kaj tiele ni iru!''

































