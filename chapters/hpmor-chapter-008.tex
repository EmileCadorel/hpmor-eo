\chapter {Pozitiva biao}

\lettrine{N}eniu demandis helpon, tio estis la problemo. Ili nur estis
vaganta, parolanta, manĝanta, aŭ rigardadanta la aeron dum iliaj
gepatroj estis klaĉantaj. Pro iu stranga kialo, neniu estis sidanta
kaj leganta libron. Kaj eĉ kiam ŝi kuraĝe prenis la iniciativon por
sidiĝi kaj daŭrigi sian trian legon de \emph{Herpŭrko: unu historio,}
neniu ŝajnis voli sidiĝi apud ŝi.

Krom helpi homojn pri iliaj hejmtaskoj, aŭ pri io ajn alia, kion ili
bezonis, ŝi vere ne sciis kiel renkonti homojn. Ŝi ne sin sentis kiel
timida persono. Ŝi opiniis pri si mem ke ŝi estis el tiuj, kiuj solvas
problemojn. Kaj por la momento, se ne estis ia demando kiel ekzemple
``mi ne povas memori kiel fari longan dividon'', estis nur tro
malkomforte iri al iu kaj diri\ldots kion? Ŝi neniam kapablis malkovri
kion diri? Kaj ne ŝajnis ke estis libro de informoj pri tio, kaj tio
estis ridinda. La tuta afero pri renkonti homojn neniam ŝajnis
sentebla por ŝi. Kial devus \emph{ŝi} preni la tutan respondecon si
mem, kiam estis du personoj koncernataj? Kial plenkreskuloj neniam
helpis?  Ŝi esperis ke kelkaj aliaj knabinoj simple marŝos al
\emph{ŝi} kaj diros, ``Hermione, la instruisto petis de mi ke mi
fariĝu via amikino.''

Sed ke tio estu tute klara, Hermione Granger, sidanta sole je la unua
tago de la lernejo en unu el la malmultaj kupeoj kiuj estis malplenaj,
en la lasta vagono de la trajno, kun la pordo de la kupeo lasita
malfermite, por la okazo se iu por iu ajn kialo volis paroli al ŝi,
\emph{ne} estis malgaja, soleca, malĝoja, melankolia, senespera, aŭ
turmentita pro siaj problemoj. Ŝi volonte estis releganta
\emph{Herpŭrko: unu historio}'n por la tria fojo kaj ŝi tute ĝuis
tion, kun nur iu malforta nuanco de ĉagreno en la fundo de sia menso
pro la ĝenerala malracio de la mondo.

Estis sono de malfermo de inter trajna pordo, kaj poste sonoj de paŝoj
kune kun stranga kaj serpenta sono venanta de la koridoro de la
trajno. Hermione metis \emph{Herpŭrko: unu historio}'n flanke,
stariĝis kaj eligis sian kapon eksteren—por la okazo se iu bezonis
helpon—kaj vidis junan knabiĉon en sorĉista robo, verŝajne de unua aŭ
dua jaro laŭ sia alteco, kaj aspektante tute stulta pro la koltuko
ĉirkaŭ sia kapo. Eta trunko staris sur la grundo proksime de li. Dum
ŝi vidis lin, li frapetis al la pordo de alia, fermita kupeo, kaj li
diris per voĉo leĝere nesonora pro la koltuko, ``Pardonu min, ĉu mi
povas fari rapidan demandon?''

Ŝi ne aŭdis la respondon de la interno de la kupeo, sed post ol la
knabiĉo malfermis la pordon, ŝi pensis ke ŝi aŭdis lin diri—kondiĉe ke
ŝi ne miskomprenis—``Ĉu iu ĉi tie konas la ses kvarkojn aŭ kie mi povas
trovi knabinon de unua jaro nomata Hermione Granger?''

Post ol la knabiĉo estis ferminta la pordon de la kupeo, Hermione
diris. ``Ĉu mi povas helpi vin kun io?''

La vizaĝo en la koltuko turnis sin por rigardi al ŝi, kaj la voĉo
diris, ``Ne, krom se vi povas nomi la ses kvarkojn aŭ diri al mi kie
trovi Hermione Granger.''

``Supra, suba, stranga, ĉarma, vera, bela, kaj kial vi estas serĉanta
ŝin?''

Estis malfacile por diri pro la distanco, sed ŝi pensis ke ŝi vidis la
knabiĉon ridetaĉi larĝe sub la koltuko. ``Ha, do \emph{vi estas} la
knabino de unua jaro nomata Hermione Granger,'' diris la juna,
malmulte sonora voĉo. ``En la trajno por Herpŭrko.'' La knabiĉo
komencis marŝi al ŝi kaj al ŝia kupeo, kaj lia trunko serpentumis
malantaŭ li. ``Teĥnike, ĉio, kion mi bezonis fari, estis \emph{serĉi}
vin, sed ŝajnas ke mi devas paroli kun vi aŭ inviti vin en mia partio,
aŭ akiri ŝlosilan objekton magian de vi aŭ malkovri ke Herpŭrko estis
konstruita sur la ruinoj de malnova templo aŭ io. LK aŭ
NLK\footnote{LK estas la akronimo de Lundanta Karaktero, kaj NLK de Ne
  Ludanta Karaktero. Tiuj estas referencoj de Rolludo, kie NLKj estas
  fiktivaj karakteroj ludataj per la ludestro, kaj ne per lundatoj},
tiu estas la demando.''

Hermione malfermis la buŝon por respondi al tio, sed poste ŝi ne povis
pensi pri iu ajn ebla respondo al\ldots \emph{kion ajn} ŝi ĵus aŭdis,
dum la knabiĉo marŝis al ŝi, rigardis en la kupeo kaj kapjesis
kontentige, kaj sidiĝis sur la benko vidalvide de la ŝia. Lia trunko
paŝetadis enen malantaŭ li, kreskis ĝis trioble pli ol sia antaŭa
diametro kaj serpetumis apud ŝia propra trunko per stranga
maltrankviliga maniero.

``Bonvolu sidiĝi,'' diris la knabiĉo, ``kaj bonvolu fermi la pordon
malantaŭ vi, se vi volas. Ne zorgu, mi ne mordos iun ajn, kiu ne
mordis min unue.'' Li estis jam malbobenanta la koltukon deĉirkaŭ sia
kapo.

La imputo ke la knabiĉo opiniis ke ŝi estis \emph{timita} de li farigis
ŝian manon fermi la pordon violente, enferminte ĝin en la muro per
nenecesa forto. Ŝi turniĝis kaj vidis la junan vizaĝon kun hela,
ridantaj verdaj okuloj, kaj kolera ruĝa-malhela cikatro sur la frunto,
kiu memoris ion al ŝi en la malantaŭ de siaj mensoj, sed por la
momento ŝi havis pli gravajn aĵojn pri kioj pensi. ``Mi ne diris ke mi
estis Hermione Granger!''

``*Mi* ne diris ke vi \emph{diris} ke vi estis Hermione Granger, mi
nur diris ke vi estas Hermione Granger. Se vi estas demandanta kiel mi
scias tion, tio estas ĉar mi scias ĉion. Bonan vesperon, gesijnoroj
mia nomo estas Harry James Potter-Evans-Verres aŭ Harry Potter
mallonge dirite, mi scias ke probable tio ne volas diri ion ajn por
\emph{vi} por unu fojo—''

La menso de Hermione finfine faris la konekton. La cikatro sur lia
frunto, la formo de fulmo. ``Harry Potter! Vi esta en \emph{Moderna
  Magia Historio} kaj \emph{La Kresko kaj Falo de la Malhelaj Artoj}
kaj \emph{La Grandaj Sorĉaj Eventoj de la Dudeka Jarcento.}'' Tio
estis fakte la vere unua fojo en ŝia tuta vivo ke ŝi \emph{renkontis}
iun kiu estis en \emph{libro}, kaj tio estis relative stranga sento.

La knabiĉo palpebrumis tri foje. ``Mi estas en \emph{libroj}? Atendu,
evidente ke mi estas en libroj\ldots kiel stranga penso.''

``Ve, ĉu vi ne sciis?'' diris Hermione. ``Mi estus trovinta ĉion, kion
mi povas se temis pri mi.''

La knabiĉo parolis relative seke. ``F-ino Granger, Estis malpli ol 72
horoj post ol mi estis en la Diagon Aleo kaj ol mi malkovris mian
renomon. Mi pasis la lastaj du tagoj aĉetante sciencajn
librojn. \emph{Kredu min}, mi intencas trovi ĉion, kion mi povas.'' La
knabiĉo hezitis. ``Kion la libroj diris pri mi?''

La menso de Hermione Grange retroiris, ŝi ne konsciis ke ŝi estos
testita pri \emph{tiuj} libroj, do ŝi legis ilin nur unu fojon, sed
tio estis nur unu monato antaŭe, do la materialo estis ankoraŭ freŝa
en siaj mensoj. ``Vi estas la sola persono kiu postvivis je la Mortiga
Malbeno, do vi estas nomita la Knabiĉon-Kiu-Postvivis. Vi estis naskita
de James Potter kaj Lily Potter antaŭe Lily Evans, la 31a de Julio
1980. La 31a de Oktobro 1981 la Mastro de la Tenebro,
Tiu-Kiun-Oni-Ne-Nomu, tamen mi ne scias kial ne, atakis vian
hejmon. Vi estis trovita vivanta kun cikatro sur via frunto en la
ruinoj de la hejmo de viaj gepatroj apud la brulitaj restoj de la
korpo de Vi-Scias-Kiun. La Ĉefa Sorĉisto Albus Percival Wulfric Brian
Dumbledore sendis vin ien, neniu scias kien. \emph{La Kresko kaj Falo
  de la Malhelaj Artoj} pretendas ke vi postvivis danke al la amo de
via patrino kaj ke via cikatro enhavas ĉiun la povon de la Mastro de
la Tenebroj, kaj ke la centaŭroj timas vin, sed \emph{La Grandaj
  Sorĉaj Eventoj de la Dudeka Jarcento} ne mencias ion ajn kiel tio,
kaj \emph{Moderna Magia Historio} avertas ke estas multe de
ekstravagaj teorioj pri vi''.

La buŝo de la knabiĉo estis pendanta malfermite. ``Ĉu oni petis vin
atendi Harry'n Potter en la trajno por Herpŭrko, aŭ io kiel tio?''

``Ne,'' Hermione diris. ``Kiu paroli pri mi al vi?''

``Profesorino McGonagall kaj mi opinias ke mi vidas kial. Ĉu vi havas
eidetikan\footnote{fotografa} memorion, Hermione?''

Hermione kapneis. ``Ĝi ne estas fotografa, mi ĉiam esperis ke ĝi
estas, sed mi devas legi mian lernejajn librojn kvin fojojn por memori
ilin entute.''

``Vere,'' la knabiĉo diris per leĝere strangolita voĉo. ``Mi esperas ke
tio ne ĝenus vin se mi testus tion—ne ke mi ne kredas vin, sed kiel la
proverbo diras, 'kredu, sed kontrolu'. Konjekti ne havas sencon se oni
simple povas fari eksperimenton.''

Hermione ridetis, relative arogante. Ŝi tiel amis
testojn. ``Bonvolu, Faru.''

La knabiĉo metis la manon en la haŭtpoŝon je sia flanko kaj diris
``Magia Preparato kaj Pocio per Arsenio Jigo$^{\textit{n:\ref{nomoj:jigo}}}$.''
Kiam li retiris sian manon, ĝi estis tenanta la libron, kiun li nomis.

Tuj Hermione volis unu el tiuj haŭtpoŝoj pli ol ŝi neniam volis ion.

La knabiĉo malfermis la libron ie je la mezo kaj rigardis
malsupren. ``Se vi farus \emph{Oleon de Akreco}—''

``Mi povas \emph{vidi} la paĝon de tie, vi scias !''

La knabiĉo klinis la libron por ke ŝi ne povu vidi ĝin denove, kaj
returnis la paĝojn. ``Se vi preparus \emph{pocion de grimpo de
  araneo}, kiu estus la sekva ingredienco kiun vi aldonus post la
Akromentula$^{\textit{a:\ref{nomoj:akromentulo}}}$ silko?''

``Post esti faliginta la silkon, atendi ĝis la pocio estas ŝanĝinta
ekzakte al la sama koloro ol la ombro de la sennebula ĉielo de la
aŭroro, 8 gradoj de la horizonto kaj 8 minutoj antaŭ ol la pinto de la
suno unue fariĝas videbla. Kirli ok fojojn maldesktrume kaj unu fojon
dekstrume, kaj poste aldoni ok gutojn de mukbuleto de unukorno.''

La knabiĉo fermis la libron kun akuta klako kaj remetis la libron en
sian haŭtpoŝon, kiu glutis ĝin kun eta rukta sono. ``Nu nu nu
\emph{nu} nu nu. Mi devus ŝati fari proponon al vi, F-ino Granger.''

``Propono?'' Hermione diris malfide. Knabinoj ne estis supozita
aŭskulti tion.

Tio estis je tiu momento ke Hermione ekkonsciis la alian aĵon—nu, unu
el la aĵoj—kiu estis stranga pri la knabiĉo. Ŝajne, homoj kiuj estas
\emph{en} libroj fakte \emph{sonis} kiel libro kiam ili parolis. Tio
estis tute miriga malkovro.

La knabiĉo metis sian manon en sian haŭtpoŝon kaj diris, ``boteleto de
sodakvo'', ekstraktis helan verdan cilindron. Li levis ĝin al ŝi kaj
diris, ``Ĉu mi povas al vi donaci ion por trinki?''

Hermione ĝentile akceptis la eferveskan trinkaĵon. Fakte ŝi
\emph{estis} sentanta sin iom soifa je ĉi tiu momento. ``Multan
dankon,'' Hermione diris dum ŝi malfermigis la boteleton. ``Ĉu tio
estas via propono?''

La knabiĉo tusis. ``Ne,'' li diris. Ekzakte kiam Hermione komencis
trinki, li diris, ``Mi deziras ke vi helpas min por konkeri la
universon.''

Hermione finis sian gluton kaj mallevis la boteleton. ``Ne dankon, mi
ne estas diabla.''

La knabiĉo rigardis ŝin mirigite, kvazaŭ li estis atendinta alian
respondon. ``Nu, mi estis parolanta iom retorike,'' li diris. ``En la
senco de la Bakonia\footnote{Francis Bacon - angla filozofo kaj unu el
  la fondintoj de la moderna scienco} projekto, vi scias, ne pri
politika povo. 'La realigo de ĉiuj eblaj aĵoj' kaj tiel plu. Mi volas
konduki eksperimentajn esplorojn pri sorĉo, malkovri la fundamentajn
regulojn, venigi la magion en la fako de scienco, kunigi la sorĉan kaj
la Muglan mondojn, kreski la vivnivelon de la tuta planedo, venigi la
homecon jarcentoj antaŭen, malkovri la sekreton de la senmorteco,
kolonii la sunsitemon, esplori la galaksion, kaj plej grave, malkovri
kio diable estas vere okazanta tie ĉar ĉio tio estas senhonte
neebla.''

Tio sonis iom pli interesa. ``Kaj?''

La knabiĉo fiksrigardis ŝin nekredema. ``*Kaj?* Tio ne \emph{sufiĉas}?''

``Kaj kion vi petos de mi?'' diris Hermione.

``Mi volas ke vi helpu min por fari miajn esplorojn, evidente. Kun via
enciklopedia memorio aldonita al mia inteligenteco kaj racio, ni finos
la Bakonian projekton post minimuma tempo, kaj per 'minimuma tempo' mi
volas diri probable almenaŭ tridek-kvin jarojn.''

Hermione komencis juĝi tiun knabiĉon ĝene. ``Mi ne vidis vin fari ion
ajn inteligenta. Eble ke mi lasos \emph{vin} helpi min kun \emph{miaj}
esploroj''

Estis certa silento en la kupeo.

``Do vi petas ke mi demonstru mian inteligentecon, tial,'' diris la
knabiĉo post longa paŭzo.

Hermione kapjesis.

``Mi avertas vin ke defii mian lertecon estas danĝera projekto, kaj
povas inklini al farigi vian vivon multe pli superreala.''

``Mi ne ankoraŭ estas impresita,'' Hermione diris. Nerimarkite, la
verda drinkaĵo unu fojon plian leviĝis al ŝiaj lipoj.

``Nu, eble ke \emph{tio} impresos vin,'' la knabiĉo diris. Li kliniĝis
antaŭen, kaj rigardis al ŝi intense. ``Mi jam faris kelkajn
eksperimentojn kaj mi malkovris ke mi ne bezonas la bastonon, mi povas
farigi ion ajn okazi nur per fingroklako.''

Tio alvenis ĵus kiam ŝi estis meze de gluto, ŝi sufokiĝis, tusis kaj
forigis la helan verdan fluaĵon\ldots

\ldots{}sur ŝia tute nova, neniam uzita robo de sorĉistino, dum la vere unua
tago de lernejo.

Hermione reale kriegis. Tiu estis akuta sono, kiu sonis kiel sireno de
aera atako en la fermita kupeo. ``\emph{Aĉ! Miaj vestoj!''}

``Ne paniku!'' diris la knabiĉo. ``Mi povas ripari ĝin por vi. Nur
rigardu!'' Li levis manon, kaj klakis siajn fingrojn.

``Vi—'' Kaj ŝi rigardis malsupren al si mem.

La verda fluaĵo estis ankoraŭ tie, sed dum ŝi rigardis, ĝi komencis
neniiĝi kaj velki kaj nur post kelkaj momentoj, tio estis kvazaŭ ŝi
neniam ekverŝis ion ajn sur si mem.

Hermione rigardis fikse la knabiĉon, kiu portis relative memkontentan
rideton.

Magio senbastona, senvorta! Je \emph{lia} aĝo? Kiam li akiris la
librojn de lernejo nur \emph{tri tagoj} antaŭe.

Tiam, ŝi memoris tion, kion ŝi legis, kaj ŝi anhelis kaj eksaltis
malantaŭen for de li. \emph{La tuta magia povo de la Mastro de la
  Tenebroj!  En lia cikatro!}

Ŝi leviĝis haste sur siajn piedojn. ``Mi, Mi, Mi devas iri al la
necesejo, atendu tie konsentite—'' Ŝi devis trovi plenkreskulojn, ŝi
devis diri al ili—

La rideto de la knabiĉo velkis. ``Tio estis simpla ruzo, Hermione, mi
bedaŭras, mi ne volis timigi vin.''

Ŝia mano haltis sur la anso de la pordo. ``\emph{Ruzo?}''

``Jes,'' diris la knabiĉo. ``Vi petis min ke mi demonstru mian
inteligentecon. Do mi faris ion, kio ŝajnas neebla. Tio estas ĉiam
bona maniero por fanfaroni. Mi ne povas \emph{reale} fari ion ajn
simple per klaki miajn fingrojn.'' Poste la knabiĉo haltis. ``Almenaŭ,
mi ne \emph{pensas} ke mi povas, mi neniam fakte testis tion
eksperimante.'' La knabiĉo levis sian manon, kaj klakis siajn fingrojn
denove. ``Ne, neniu banano.''

Hermione estis pli konfuzita ol ŝi neniam estis dum sia tuta vivo.

La knabiĉo estis tiam ridetanta denove pro la aspekto sur ŝia
vizaĝo. ``Mi \emph{avertis} vin ke defii mian inĝenion povas inklini
al farigi vian vivon multe pli superreala. Memoru tion la venontan
fojon, kiam mi avertos vin pri io.''

``Sed, sed,'' Hermione balbutis. ``Kion vi \emph{faris}, tial?''

La rigardo de la knabiĉo prenis mezuranta, pezantan kvaliton, kiun ŝi
neniam vidis antaŭe sur iu de la sama aĝo ol ŝi. ``Vi opinias ke vi
havas tion, kio necesas por esti sciencisto per vi mem, kun aŭ sen mia
helpo? Tial, vidu oni kiel \emph{vi} esploras konfuzantan fenomenon.''

``Mi\ldots'' La mensoj de Hermione fariĝis blankaj dum kelkaj
momentoj. Ŝi amis testojn sed ŝi neniam estis testita kiel tio antaŭe.
Panike, ŝi provis pensi pri io ajn, kion ŝi legis pri tio, kion
sciencistoj estas supozitaj fari. Ŝia spirito transsaltis dentradojn,
grundis kontraŭ ĝi mem, kaj rekraĉis la instruadojn por fari projekton
de scienca esploro :

\emph{Etapo 1: Formi hipotezon. Etapo 2: Kondukti eksperimenton por
testi vian hipotezon. Etapo 3: Mezuri la rezultojn Etapo 4: Konstrui
kartonan afiŝon.}


Etapo 1 estis formi hipotezon. Tio volis diri, provi pensi pri io kio
\emph{povis} esti okazinta. ``Nu. Mia hipotezo estas ke vi ĵetis
ĉarmon sur miajn robojn por fari ke ĉio ekverŝita sur ĝi velkas.''

``Konsentite,'' diris la knabiĉo, ``ĉu tio estas via respondo?''

La ŝoko estis malaperanta, kaj la menso de Hermione komencis funkcii
konvene. ``Atendu, tio ne povas esti prava. Mi ne vidis vin tuŝi vian
bastonon aŭ diri iun ajn sorĉon, do kiel vi povintus ĵeti la ĉarmon?''

La knabiĉo atendis, lia vizaĝo neŭtrale.

``Sed supozu ke ĉiuj miaj roboj venis el butiko \emph{jam} ĉarmitaj,
por gardi ilin purajn. Tio estus speco de ĉarmo utila por ili. Vi
malkovris tion per ekverŝi ion sur \emph{vi} pli frue''

Nun la brovoj de la knabiĉo leviĝis. ``Ĉu \emph{tio} estas via
respondo?''

``Ne, mi ne faris etapon 2, 'fari eksperimenton por testi vian
hipotezon'.''

La knabiĉo fermis la buŝon kaj komencis rideti.

Hermione rigardis la boteleton de trinkaĵo, kiun ŝi aŭtomate metis en
la tastenilo je la fenestro. Ŝi prenis ĝin kaj rigardis enen, kaj
rimarkis ke ĝi estis ankoraŭ trione plena.

``Nu,'' diris Hermione, ``la eksperimento, kiun mi volas fari, estas
verŝi ĝin sur mia robo kaj vidi kio okazas, kaj mia prognozo estas
ke la makulo malaperos. Sed se tio ne \emph{funkcias}, mia robo estos
makulita, kaj mi ne volas tion.''

``Faru tion sur la mia,'' diris la knabiĉo, ``tiel maniere vi ne
bezonas zorgi pri ke via robo fariĝu makulita.''

``Sed—'' Hermione diris. Estis io malprava kun tiu pensmaniero sed ŝi
ne sciis kiel diri tion ekzakte.

``Mi havas ekstrajn robojn en mia trunko,'' diris la knabiĉo.

``Sed estas nenie por ke vi ŝanĝu,'' Hermione obĵetis. Poste, ŝi
pripensis pli bone pri tio. ``Tamen mi supozas ke mi povas foriri kaj
fermi la pordon—''

``Mi havas ien por ŝanĝi en mia trunko, ankaŭe.''

Hermione rigardis la trunkon, kiu, ŝi komencis suspekti, estis multe
pli speciala ol la sia.

``Konsentite,'' Hermione diris. ``Ĉar vi tion diras,'' kaj ŝi iomete
delikate verŝis iom da la verda sodakvo sur la angulo de la roboj de
la knabiĉo. Poste ŝi rigardis fikse al ĝi, provante memori kiom da tempo
la originala fluaĵo prenis por malaperi\ldots

Kaj la verda makulo velkis!

Hermione lasis eliri suspiron de kvietigo, ne malpli ĉar tio volis
diri ke ŝi ne estis traktanta kun la tuto de la magia povo de la
Mastro de la Tenebroj.

Nu, Etapo tria estis mezuri la rezultojn, sed en tiu okazo tio estis
nur vidi ke la makulo malaperis. Kaj ŝi supozis ke ŝi verŝajne povis
preterlasi la Etapon 4'n, pri la kartona afiŝo. ``Mia respondo estas
ke la roboj estas ĉarmitaj por gardi ilin pura.''

``Ne tute,'' diris la knabiĉo.

Hermione sentis pikon de elreviĝo. Ŝi vere esperis ke ŝi ne
\emph{sentis} sin tiel, la knabiĉo ne estis instruisto, sed tiu estis
tamen teston kaj ŝi malprave respondis al demando kaj tio ĉiam
sentigis al ŝi kiel eta punĉo en la stomako.

(Tio diris preskaŭ ĉion, kion vi devis scii pri Hermione Granger, la
fakto ke ŝi neniam lasis tion aresti ŝin, aŭ lasis tion interferi kun
sia amo pri esti testita.)

``La malĝoja aĵo estas,'' diris la knabiĉo, ``ke vi probable faris
ĉion, kion la libro diris al vi pri kiel fari. Vi faris prognozon, kiu
devis distingi inter la roboj estante ĉarmitaj kaj ne ĉarmitaj kaj vi
testis tion, kaj forĵetis la nulan hipotezon kiu estis ke la roboj ne
estis ĉarmitaj. Sed krom se vi legis la vere, vere plej bonajn specojn
da libroj, ili ne tute al vi instruas kiel fari sciencon
\emph{konvene}. Sufiĉe bone por \emph{vere} akiri la bonan respondon,
mi volas diri, kaj ne nur elbuŝi alian publikigon kiel tiuj, pri kiuj
Paĉjo ĉiam plendas. Do lasu min provi klarigi—sen malkaŝi la
respondon—tion, kion vi faris malbone tiun fojon, kaj mi donos al vi
alian ŝancon.''

Ŝi komencis senti rankoron kontraŭ la tiel supera tono de la knabiĉo
dum li estis nur alia dek-unujaraĝulo kiel ŝi, sed tio estis akcesora
kompare al malkovri tion, kion ŝi faris malbone. ``Konsentite.''

La esprimo de la knabiĉo iĝis intense. ``Tiu estas ludo bazita sur
fama eksperimento nomita la 2-4-6 tasko, kaj tio estas kiel ĝi
funkcias. Mi havas \emph{regulon}—konita de mi, sed ne de vi—kiu
taŭgas por kelkaj triopoj da numeroj, sed ne aliaj. 2-4-6 estas
ekzemplo de triopo por kiu la regulo taŭgas. Fakte\ldots lasu min
skribi la regulon, nur por ke vi sciu ke ĝi estas fiksa regulo, kaj
faldi tiun kaj doni ĝin al vi. Bonvolu ne rigardu, pro tio ke mi
deduktis pli frue ke vi povas legi inverse.''

La knabiĉo diris ``papero'' kaj ``meĥanika krajono'' al sia haŭtpoŝo,
kaj ŝi fermis la okulojn firme, dum li skribis.

``Jen,'' diris la knabiĉo, kaj li estis tenanta firme falditan pecon
da papero. ``Metu ĝin en via poŝo,'' kaj ŝi tion faris.

``Nun la maniero laŭ kiu la ludo funkcias,'' diris la knabiĉo, ``estas
ke vi donos al mi triopon da numeroj, kaj mi diras al vi 'Jes' se la
tri numeroj estas ekzemplo de la regulo, kaj 'Ne' se ili ne estas. Mi
estas la Naturo, kaj la regulo estas unu el miaj leĝoj, kaj vi estas
esploranta min. Vi jam scias ke 2-4-6 akiras 'Jes'. Kiam vi estos
plenuminta ĉiujn la pliaj eksperimentaj testoj, kiujn vi volas—demandu
tiom da triopoj kiom vi pensas necesaj—vi haltos kaj divenos la
regulon, kaj poste vi povos malfaldi la pecon da papero kaj vidi ĉu vi
sukcesis. Ĉu vi komprenas la ludon?''

``Evidente ke mi komprenas,'' diris Hermione.

``Komencu.''

``4-6-8'' diris Hermione.

``Jes,'' diris la knabiĉo.

``10-12-14,'' diris Hermione.

``Jes,'' diris la knabiĉo.

Hermione provis igi sian menson iom pli malproksima, pro tio ke ŝajnis
kvazaŭ ŝi jam faris ĉiujn la testojn, kiujn ŝi bezonis, sed tio ne
povis esti tiel facila, ĉu ne?

``1-3-5.''

``Jes.''

``Minus 3, minus 1, plus 1.''

``Jes.''

Hermione ne povis pensi pri io ajn alia por fari. ``La regulo estas ke
la numeroj devas kreskiĝi je du, ĉiun fojon.''

``Nun supozu ke mi al vi diras,'' diris la knabiĉo, ``ke tiu testo
estas pli malfacila ol ĝi aspektas, kaj ke nur 20\% de plenkreskuloj
sukcesas.''

Hermione malridetis. Kion ŝi mankis? Tiam subite ŝi ekpensis pri testo
kiun ŝi bezonis fari.

``2-5-8!'' ŝi diris triumfe.

``Jes.''

``10-20-30!''

``Jes.''

``La prava respondo estas ke la numeroj devas kreski je la sama kvanto
ĉiun fojon. Tiu ne bezonas esti 2.''

``Nu,'' diris la knabiĉo, ``prenu la paperon kaj vidu ĉu vi
sukcesis.''

Hermione prenis la paperon el sia poŝo kaj malfaldis ĝin.

\emph{Tri realaj numeroj en kreskanta ordo, de plej malalta al plej
  alta.}

La makzelo de Hermione falis. Ŝi havis la klaran senton ke io terure
maljusta estis farita al ŝi, ke la knabiĉo estis malpura aĉa
friponanta mensogulo, sed kiam ŝi pri tio repensis, ŝi ne povis pensi
pri iu ajn malprava respondo, kiun li estus doninta.

``Kion vi ĵus malkovris estas nomata 'pozitiva biao','' diris la
knabiĉo. ``Vi havis regulon en via menso, kaj vi daŭris pensi pri
triopoj, kiuj farigus la regulon diri 'Jes'. Sed vi ne provis testi
iun ajn triopon, kiu farus ke la regulo diras 'Ne'. Fakte vi havis
neniun 'Ne', do 'iu ajn tri numeroj' povintus esti prava kaj povintus
facile esti la regulo. Tio estas kiel kiam homoj imagas
eksperimentojn, kiuj povas konfirmi iliajn hipotezojn anstataŭ provi
imagi eksperimentojn, kiuj povas falsigi ilin—tiu ne tute estas
ekzakte la sama eraro, sed tiu estas proksima. Vi devas lerni kiel
rigardi la negativan flankon de aĵoj, rigardaĉi en la mallumo. Kiam
tiu eksperimento estis plenumita, nur 20\% de plenkreskuloj akiris la
korektan respondon. Kaj multe el la aliaj inventis nekredeblan
komplikan hipotezon kaj metis grandan fidon en ilian malpravan
respondon, tiel ke ili ne faris tiom da eksperimentoj kaj ke ĉio okazis
tiel, kiel ili atendis.''

``Nun,'' diris la knabiĉo, ``ĉu vi volas provi respondi al la
originala problemo denove?''

Liaj okuloj estis tute absorbitaj nun, kvazaŭ tiu estis la \emph{vera}
testo.

Hermione fermis la okulojn kaj provis koncentriĝi. Ŝi estis ŝvitanta
sub siaj roboj. Ŝi havis la strangan senton ke tio estis la plej
malfacila testo, pri kiu oni petis ŝin pripensi aŭ eĉ la \emph{unuan}
fojon ke oni petis ŝin pripensi pri testo.

Kiun alian eksperimenton povis ŝi fari? Ŝi havis Ĉokoladan Ranon, ĉu
ŝi povis provi froti iom da tio sur la roboj kaj vidi ĉu \emph{ĝi}
velkas? Sed tio ankoraŭ ne ŝajnis kiel tia distorta negativa penso,
kiun la knabiĉo demandis. Kvazaŭ, ŝi ankoraŭ demandis iu 'Jes' pri ĉu
la makulo de ĉokolada rano malaperos, anstataŭ demandi iu 'Ne'.

Do\ldots laŭ ŝia hipotezo\ldots kiam devintus la sodakvo\ldots
\emph{ne} malaperi?

``Mi havas eksperimenton por fari,'' Hermione diris. ``Mi volas verŝi
iom da sodakvo sur la grundo, kaj vidi ĉu ĝi \emph{ne} malaperas. Ĉu
vi havas iom da papera mantuko en via haŭtpoŝo, por ke mi povu viŝi la
verŝon se tio ne funkcias?''

``Mi havas buŝtukojn,'' diris la knabiĉo. Li vizaĝo ankoraŭ neŭtrala.

Hermione prenis la boteleton, kaj verŝis iometon da la sodakvo sur la
grundo.

Kelkaj sekundoj pli malfrue, ĝi malaperis.

Tiam la kompreno frapis ŝin kaj ŝi sentis sin kiel piedbatanta sin
mem. ``Evidente! \emph{Vi} donis la boteleton al mi! Tio ne estas la
roboj, kiuj estis ĉarmitaj, tio estis la sodakvo ekde la komenco!''

La knabiĉo stariĝis kaj kliniĝis al ŝi solene. Li estis rikananta
nun. ``Do\ldots ĉu mi povas helpi vin kun viaj elsporoj, Hermione
Granger?''

``Mi, ha\ldots'' Hermione estis ankoraŭ sentanta la svagon de eŭforio,
sed ŝi ne estis tute certa pri kiel respondi al \emph{tio}.

Tio estis interompita per malforta, nekonfirmita, febla, preskaŭ
\emph{kontraŭvola} frapeto al la pordo.

La knabiĉo turnis sin kaj rigardis al la fenestro, kaj diris, ``Mi ne
portas mian koltukon, do ĉu vi povas okupi vin pri tio?''

Tio estis je tiu momento ke Hermione ekkonsciis ke la knabiĉo—ne, la
Knabiĉo-Kiu-Postvivis, Harry Potter—estis vestinta la koltukon ĉirkaŭ
la kapo unue, kaj sentis sin iom stulta pro ke ŝi ne ekkonsciis tion
pli frue. Tio estis sufiĉe stranga, pro tio ke ŝi pensintus ke Harry
estintus fiere montranta sin mem al la mondo; kaj la penso aperis al
ŝi ke li estis fakte pli timida ol li ŝajnis.

Kiam Hermione tiris la pordon malfermiten, ŝi estis salutita per
tremanta juna knabiĉo kiu aspektis ekzakte kiel li frapetis.

``Pardonu min,'' diris la knabiĉo kun ege malforta voĉo, ``Mi estas
Nevilo Longafundo$^{\textit{n:\ref{nomoj:longafundo}}}$. Mi serĉas mian dombeston, iu
bufo, mi, mi ne povas trovi rin ie ajn en la vagono\ldots ĉu vi vidis
mian bufon?''

``Ne,'' Hermione diris, kaj tiam ŝia afableco akcelis ĝis plej
rapido. ``Ĉu vi kontrolis la aliajn kupeojn?''

``Jes,'' flustris la knabiĉo.

``Tial ni nur devos kontroli ĉiujn aliajn kupeojn,'' Hermione diris
rapide. ``Mi helpos vin. Mia nomo estas Hermione Granger, parenteze.''

La knabiĉo aspektis kvazaŭ li povis sveni pro dankemo.

``Atendu,'' venis la voĉo de la \emph{alia} knabiĉo—Harry Potter. ``Mi
ne estas certa ke tiu estas la plej bona maniero por fari tion.''

Post aŭdi tion, Nevilo aspektis kiel li povis plori, kaj Hermione
ekreturnis sin, kolere. Se Harry Potter estus la speco de persono, kiu
forlasus etan knabiĉon, simple ĉar li ne volas esti interompata\ldots
``Kio? Kial \emph{ne?}''

``Nu,'' diris Harry Potter, ``Tio necesigos momenton por kontroli la
tutan trajnon per mano, kaj ni povas manki la bufon ĉiuokaze, kaj se
ni ne trovas ĝin antaŭ ol ni alvenas je Herpŭrko, li havos
problemojn. Do, kio estus multe pli racia estas ke ni iru direkte al
la unua vagono, kie la prefektoj estas, kaj demandas al prefekto
helpon. Tiu estis la unua aĵo kion mi faris kiam mi serĉis vin,
Hermione, tamen ili ne fakte sciis. Sed ili eble havas sorĉojn aŭ
magiajn objektojn, kiuj farus ke trovi bufon estas multe pli
facila. Ni havas nur dek-unu jarojn.''


Tio\ldots \emph{estis} multe pli racia.

``Ĉu vi pensas ke vi povas iri al la prefekta vagono per vi mem?''
demandis Harry Potter. ``Mi havas kialojn por ne voli tro montri mian
vizaĝon.''

Subite Nevilo anhelis kaj retropaŝis. ``Mi memoras tiun voĉon! Vi
estas unu el la Mastroj de la Ĥaoso! \emph{Vi estas tiu kiu donis al
  mi ĉokoladon!}''

Kio? Kio kio \emph{kio?}

Harry Potter turnis sian kapon el la fenestro kaj sin levis
draste. ``Mi \emph{neniam}!'' li diris, lia voĉo plena da
indigno. ``Ĉu mi aspektas kiel la speco de kanajlo, kiu donus
bombonojn al infano?''

La okuloj de Nevilo pli larĝiĝis. ``*Vi estas* Harry Potter? \emph{La}
Harry Potter? \emph{Vi?}''

``Ne, nur \emph{unu} Harry Potter, estas tri mi en ĉi tiu trajno—''

Nevilo faris etan ekkrion kaj forkuris. Estis mallonga tamburo de
panikaj paŝoj kaj poste la sono de la pordo de la vagono sin
malfermanta kaj sin fermanta.

Hermione sidiĝis severe sur la benko. Harry Potter fermis la pordon
kaj tiam sidiĝis apud ŝi.

``Ĉu vi povas klarigi tion, kio okazis?'' Hermione diris per sia
malforta voĉo. Ŝi scivolis ĉu esti kune kun Harry Potter volis diri
esti ĉiam tiel konfuza.

``Ho, nu, kio okazis estas ke Fred kaj George kaj mi vidis ĉi tiun
kompatindan etan knabiĉon je la trajna stacio—la virino apud li
foriris ekde kelka tempo, kaj li aspektis vere timema, kvazaŭ li estis
certa ke li estis atakota per Morto-Manĝantoj aŭ io.  Nun, estas
diraĵo kiu diras ke la timo estas ofte pli malbona ol la objekto de la
timo ĝi mem, do aperis al mi ke tiu estis knabiĉo kiu fakte povis
profiti per vidi sian plej malbonan koŝmaron fariĝanta reala kaj vidi
ke tiu ne estis tiel malbona kiel li timis—''

Hermione sidis tie kun sia buŝo larĝe malfermita.

``- Kaj Fred kaj George proponis la sorĉon, kiu faras cikatrojn sur la
vizaĝo pli malhelaj kaj malklaraj, kvazaŭ ni estis malmortaj reĝoj kaj
ke tiuj estis niaj tombaj vualoj—''

Ŝi tute ne ŝatis tien, kien tio estis iranta.

``- kaj post ol ni estis fininta doni al li ĉiujn bombonojn, kiuj mi
aĉetis, ni estis kiel, `Donu al li iom da mono! Ha ha ha! Havu kelkajn
Knutojn, knabiĉo!  Havu arĝentan Siklon!' kaj dancantaj ĉirkaŭ li kaj
ridantaj malbone kaj tiel plu. Mi pensas ke estis kelkaj homoj en la
amaso kiuj volis interveni unue, sed la apatio de spektatoj retenis
ilin almenaŭ ĝis ili vidis kion ni estis farantaj, kaj tiam mi pensas
ke ili estis tro konfuzitaj por fari ion ajn. Finfine li diris per ege
malforta flustro 'foriru' do la tri el ni, kriis kaj forkuris,
ekkriante ion pri la lumo brulanta nin. Espereble li ne timas esti
ĉikanita en la estonteco. Tio estas nomata la terapio per desensigo,
parenteze.''

Konsentite, ŝi \emph{ne} divenis ĝuste kien tio estis iranta.

La brulanta fajro el indigno, kiu estis unu el la ĉefa motoro de
Hermione, ekelreviĝis, eĉ se parto de ŝi iamaniere vidis tion, kion
ili provis fari. ``Tio estas abomena! \emph{Vi estas} abomeninda! Ĉi
tiu kompatinda knabiĉo! Kion vi faris estis \emph{malafabla}!''

``Mi opinias ke la vorto, kiun vi serĉas, estas \emph{agrabla}, kaj
ĉiuokaze, vi faras la malpravan demandon. La demando estas, ĉu tio
faris pli da bonoj ol da damaĝoj, aŭ pli da damaĝoj ol da bonoj? Se vi
havas iujn ajn argumentojn por kontribui pri \emph{tiu} demando mi
estos ĝoja pri aŭdi ilin, sed mi ne konsideros iun aliajn kritikojn
antaŭ ol tiu estas solvita. Mi tute konsentas ke tiu, kiun mi faris,
\emph{aspektas} aĉa kaj brutala kaj malafabla, pro tio ke tiu implikas
timigitan etan knabiĉon kaj cetere, sed tio estas apenaŭ la ĉefa
problemo nun, ĉu ne? Tio estas nomata la \emph{konsekvencismon},
parenteze, tio volas diri ke scii ĉu ago estas ĝusta aŭ malĝusta oni
ne decidas laŭ ĉu ĝi aspektas malbona, aŭ malafabla, aŭ io ajn kiel
tio, la sola demando estas kiel tio fariĝos je la fino—kio estas la
konsekvenco.''

Hermione malfermis sian buŝon por diri ion komplete \emph{brulanta}
sed bedaŭrinde ŝi ŝajnis neglektanta la parton, pri pripensi pri io
por diri antaŭ malfermi la buŝon. Ĉio, kion ŝi povis trovi, estis
``Kaj kio se li havas \emph{koŝmarojn?}''

``Honeste, mi ne pensas ke li bezonas nian helpon por havi koŝmarojn,
kaj se li havas koŝmarojn pri \emph{tio}, tiam tiuj estos koŝmarojn
pri teruraj monstroj kiuj al vi donas ĉokoladon, kaj tio tute estis la
\emph{celo}.''

La cerbo de Hermione daŭris singulti pro konfuzo ĉiujn fojojn kiam ŝi
provis fariĝi konvene kolera. ``Ĉu via vivo estas ĉiam tiel bizara?''
ŝi diris finfine.

La vizaĝo de Harry Potter glimis pro fiero. ``Mi \emph{farigas} ĝin
tiel bizara. Vi estas rigardanta la produktaĵon de multe da pezaj
laboroj kaj streboj.''

``Do\ldots'' Hermione diris, kaj mallaŭtiĝis malgracie.

``Do,'' Harry Potter diris, ``Kiom pri scienco vi konas ekzakte? Mi
povas fari kalkulon kaj mi konas kelkajn teoriojn pri Bajesia
probableco kaj kelkajn teoriojn pri decideco kaj multe pri kognitiva
scienco, kaj mi legis \emph{La lektoroj de Feyman} (aŭ volumo unua
ĉiuokaze) kaj \emph{Juĝo sub necerteco: heŭristikoj kaj biaoj} kaj
\emph{Lingvoj en Penso kaj Ago} kaj \emph{Influo: Scienco kaj
  Ekzerco}, kaj \emph{Racia Elekto en Necerta Mondo} kaj \emph{Gödel,
  Escher, Bach} kaj \emph{Paŝo pli malproksimen} kaj—''

La sekvanta kvizo kaj kontraŭ-kvizo daŭriĝis dum pluraj minutoj antaŭ
ol ĝi estis interompita per alia tima frapeto al la pordo. ``Eniru'',
ŝi kaj Harry diris je preskaŭ la sama momento, kaj la pordo
malfermiĝis por malkovri Nevilo'n Longafundo.

Nevilo \emph{estis} vere ploranta nun. ``Mi iris al la unua vagono kaj
trovis p-prefekton sed li d-diris al mi ke prefekto ne devas esti
enuita por etaj aĵoj kiel forestanta bufo.''

La vizaĝo de la Knabiĉo-Kiu-Postvivis ŝanĝiĝis. Liaj lipoj formis
fortikan linion. Lia voĉo, kiam li parolis, estis malvarma kaj
darka. ``En kiu koloro li estis ? Verdo kaj arĝento?''

``N-ne, lia insigno estis r-ruĝa kaj ora.''

``\emph{Ruĝa kaj ora!}'' eksplodis Hermione. ``Sed tiuj estas la
koloroj de \emph{Grifindoro!}''

Harry Potter siblis je tio, per timiga sono, kiu povintus veni el
vivanta serpento kaj farigis ambaŭ ŝin kaj Nevilo'n eksalti. ``Mi
\emph{supozas,}'' Harry Potter kraĉis, ``ke trovi la bufon de unua
jarulo ne estas sufiĉe \emph{heroa} kaj ne indas prefekton de
\emph{Grifindoro}. Venu, Nevilo, mi \emph{iros} kun vi ĉi tiun fojon,
ni vidos ĉu la Knabiĉo-Kiu-Postvivis akiros pli da atento. Unue, ni
trovos prefekton, kiu devos koni sorĉon, kaj se tio ne funkcias, ni
trovos prefekton kiu ne timas malpurigi siajn manojn, kaj se
\emph{tio} ne funkcias, mi komencos varbi miajn fervorulojn kaj se ni
bezonas, ni diserigos la tutan trajnon ŝraŭbo post ŝraŭbo.''

La Knabiĉo-Kiu-Postvivis strariĝis kaj kaptis la manon de Nevilo en la
sian, kaj Hermione ekkonsciis kun subita mensa singulto, ke ili havis
preskaŭ la saman altecon, kvankam iu parto en ŝi insistis pri ke
Harry Potter estis pli alta de unu piedo ol li vere estis, kaj ke
Nevilo almenaŭ estis malpli alta de ses coloj.

``\emph{Restu!}'' diris malafable Harry Potter al ŝi—ne.. atendu, al lia
\emph{trunko}—kaj li fermis la pordon firme malantaŭ li kiam li
foriris.

Ŝi probable devintus iri kun ili, sed dum nur mallonga momento Harry
Potter fariĝis tiel timiganta ke ŝi estis fakte relative ĝoja pri ke
ŝi ne pensis pri tion sugesti.

% La misliterumo "Historio unu Herpŭrko" estas intencita
La menso de Hermione estis nun tiel konfuza ke ŝi ne êc pensis ke ŝi
povis konvene legi ``Historio: unu Herpŭrko'''n. Ŝi sin sentis kvazaŭ
ŝi estis ĵus surveturita per vaporcilindro kaj kvazaŭ ŝi ŝanĝiĝis en
patkukon. Ŝi ne estis certa pri kio ŝi pensis aŭ kiel ŝi sentis sin aŭ
eĉ kial. Ŝi nur sidis ĉe la fenestro kaj rigardadis la movantan
pejzaĝon.

Nu, ŝi almenaŭ sciis kial ŝi sin sentis iomete malĝoja ene.

Eble Grifindoro ne estis tiel mirinda kiel ŝi pensis.
