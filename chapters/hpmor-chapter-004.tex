\chapter{La Hipotezo de la efika merkato}

\lettrine{A}masoj da oraj Galionoj. Stakoj da arĝentaj Sikloj. Stakoj da
bronzaj Knutoj.

Harry staris ĉi tie, fiksrigardante la sekurĉambron de sia familio, la
buŝo malfermita. Li havis tiom multajn demandojn ke li ne sciis
\emph{per kiu} komenci.

Ĝuste malantaŭ la pordo de la sekurĉambro, Profesorino McGonagall
rigardis lin, ŝajnante senzorge sin apogi kontraŭ la muro, sed siaj
okuloj estis atentaj. Nu, tio havis sencon. La fakto, ke oni metis
lin antaŭ granda amaso da oraj moneroj, estis testo de karaktero tiel
pura, ke ĝi estis arketipa.

``Ĉu tiuj moneroj estas faritaj el pura metalo?'' Harry diris fine.

``Kio?'' siblis la gnomo Griphoko$^{\textit{n:\ref{nomoj:griphoko}}}$, kiu estis atendanta apud la
pordo. ``Ĉu vi dubas la integrecon de Gringoto, S-iĉo
Potter-Evans-Verres?''

``Ne,'' respondis Harry distrite, ``tute ne, pardonu min se mi min
malbone esprimis, sinjoriĉo. Mi simple havas neniun ideon pri kiel via
financa sistemo funkcias. Mi nur demandas ĉu Galionoj ĝenerale estas
faritaj el pura oro.''

``Evidente,'' diris Griphoko.

``Kaj ĉu iu ajn povas stampi ilin, aŭ ĉu ili estas produktitaj per iu
monopolo, kiu tiel kolektas takson?''

``Kio?'' diris Profesorino McGonagall.

Griphoko ridetis, prezentante akrajn dentojn. ``Nur stultulo fidus
ion alian krom gnoma monero!''

``Alivorte,'' Harry diris, ``Monero ne estas supozita valori
pli ol la metalo, kiu komponas ĝin?''

Griphoko fiksrigardis Harry'n. Profesorino McGonagall rigardis perplekse.

``Mi volas diri, supozu ke mi venas ĉi tien kun tuno da arĝento. Ĉu mi
povas akiri tunon da Sikloj faritaj el ĝi?''

``Kontraŭ imposto, S-iĉo Potter-Evans-Verres.'' La gnomo rigardis lin
kun brilantaj okuloj. ``Kontraŭ certa imposto. Kie vi trovus tunon da
arĝento, mi scivolas?''  

``Mi estis parolanta hipoteze,'' Harry diris. \emph{Nuntempe,
  ĉiaokaze}. ``kiom vi taksus en kotizo, kiel frakcio de la tuta pezo?''

La okuloj de Griphoko estis atentaj. ``Mi devas konsulti miajn
superulojn\ldots''

``Donu al mi proksimuman rezulton, mi ne postulos, ke Gringoto sekvu
ĝin.''

``Dudekono de la metalo devus bone pagi la stampon''

Harry kapjesis. ``Dankegon, S-iĉo Griphoko.''

\emph{Do, ne nur la sorĉista ekonomio estas preskaŭ tute apartigita de
  la Mugla ekonomio, sed ankaŭ neniu ĉi tie iam aŭdis pri arbitracio.}
La Mugla ekonomio pli larĝa havis fluktuan kurzon de la valuto de oro
al arĝento, do ĉiufoje kiam la rilata valoro inter oro kaj arĝento en
la Mugla ekonomio atingis pli ol 5\% super la proporcio de la pezo de
dek sep Sikloj al unu Galiono, tiam ambaŭ oro kaj arĝento devus
malaperi el la magia ekonomio ĝis estas neeble subteni la valutan
kurzon. Aportu tunon da arĝento, interŝanĝu ĝin kun Sikloj (kaj pagu
5\%), interŝanĝu la Siklojn kun Galionoj, alportu la oron al la Mugla
mondo, interŝanĝu ĝin kun pli da arĝento ol la komenco, kaj ripetu.

Ne estis la valuta kurzo inter Mugla oro kaj arĝento ĉirkaŭ kvindek
por unu? Harry ne opiniis ke tio estis dek sep, ĉiuokaze. Kaj ŝajnis
ke arĝentaj moneroj estis fakte pli \emph{maldikaj} ol la oraj
moneroj.

Tamen, Harry staris en banko kiu \emph{laŭlitere} konservis vian monon
en sekurĉambro plena da oraj moneroj kaj gardata per drakoj, kaj kien
vi devis iri por preni monerojn el via sekurĉambro ĉiufoje kiam vi
volis elspezi monon. Pli subtilaj aĵoj, kiel la kapablo de la
arbitracio redukti la senefikecon de la merkato, verŝajne estis
nekonataj al ili. Li estis tentata fari riproĉajn rimarkojn pri la
krudeco de ilia financa sistemo\ldots

\emph{Sed la malgaja aĵo estis, ke ilia sistemo estis verŝajne pli
bona.}

Aliflanke, sperta mastro de spekulaĵoj verŝajne povintus estri la tutan
sorĉan mondon post unu semajno. Harry notis tiun nocion por la okazo ke
li iam ne plu havus monon, aŭ havus disponeblan semajnon.

Dume, la gigantaj amasoj da oraj moneroj en la sekurĉambro de
Potter'j devus taŭgi por liaj mallongetempaj bezonoj.

Harry kliniĝis antaŭen, kaj komencis kolekti orajn monerojn kun
unu mano kaj faligis ilin en la alian.

Kiam li atingis dudek, Profesorino McGonagall tusis. ``Mi opinias, ke
tio sufiĉos por pagi viajn studmaterielojn, S-iĉo Potter.''

``Hm?'' Harry diris, sia menso estante ie alia. ``Atendu, mi estas
faranta kalkulon de Fermi.''

``Kio?'' diris Profesorino McGonagall, sonante iom alarmita.

``Tio estas matematika aĵo. Nomita laŭ Enrico Fermi. Iu maniero por rapide kalkuli
asprajn nombrojn en via kapo\ldots''

Dudek galionoj pezis unu dekonon da kilogramo, eble? Kaj oro estis,
kio, dek mil Britaj pundoj la kilogramo? Do, Galiono devus valori
proksimume kvindek pundojn\ldots La amaso da oro aspektis alta de
sesdek moneroj kaj larĝa de dudek moneroj en ĉiuj direktoj de la bazo,
kaj la amaso estis piramida, do devus formi ĉirkaŭ trionon da kubo. Ok
mil Galionoj en tiu amaso, proksimume, kaj estis ĉirkaŭ kvin amasoj
kun sama grando, do kvardek mil Galionoj aŭ du milionoj da sterlingaj
pundoj.

Ne malbone. Harry ridetis kun certa kruda feliĉo. Estis tre
bedaŭrinde, ke li estis meze de la malkovro de mirinda nova mondo de
magio, kaj ke li ne povis preni tempon por esplori la mirindan novan
mondon de la riĉeco, kiun rapida kalkulo de Fermi taksis kiel
miliardoj da fojoj malpli interesa.

\emph {Tamen, tio estis la lasta fojo ke mi falĉas la herbon por unu
  aĉa pundo}

Harry deturnis sin de la giganta amaso da moneroj. ``Pardonu al mi la
demandon, Profesorino McGonagall, sed mi kredis kompreni ke miaj
gepatroj havis ĉirkaŭ dudek jarojn, kiam ili mortis. Ĉu tio estas la
normala kvanto da mono, kiun juna paro havas en sia sekurĉambro, en la
sorĉa mondo?'' Se tio estis vera, taso da teo verŝajne kostis kvin mil
pundojn. Regulo unu de la ekonomio, vi ne povas manĝi monon.

Profesorino McGonagall kapneis. ``Via patriĉo estis la lasta heredanto
de iu malnova familio. Estas ankaŭ eble ke\ldots'' La sorĉistino
hezitis. ``Iom da tiu mono povas veni de premioj sur Vi-Scias-Kiun,
pagotaj al lia mur- ah, al kiu ajn povus venki lin. Aŭ tiuj premioj
eble ankoraŭ ne estas kolektikaj. Mi ne certas.''

``Interesa\ldots'' Harry diris malrapide. ``Do iom da tiu vere estas,
iusence, mia. Mi volas diri, gajnita per mi. Iel. Eble. Eĉ se mi ne
memoras tiun okazon.'' La fingroj de Harry frapetis kontraŭ la kruro
de sia pantalono. ``Tio sentigas min malpli kulpa pri elspezi
\emph{vere etan frakcion de ĝi! Ne paniku Profesorino McGonagall!}''

``S-iĉo Potter! Vi estas neplenaĝulo, kaj tiel, vi nur estos permesita
fari raciajn deprenojn el—''

``Mi estas \emph{tute} racia! Mi tute konsentas kun financa singardeco
kaj impulsa kontrolo! Sed mi \emph{vidis} kelkajn aĵojn dum la vojo al ĉi
tie, kiuj konsistigus saĝajn, plenkreskajn aĉetojn\ldots''

Harry fiksrigardis Profesorinon McGonagall, komecante silentan fiksan
konkurson.

``Kiel kio?'' Profesorino McGonagall finfine diris.

``Trunko, kies interno estas pli granda ol la ekstero?''

La vizaĝo de Profesorino McGonagall severiĝis. ``Tiuj estas vere
multekostaj, S-iĉo Potter!''

``Jes, sed—'' Harry pledis. ``Mi certas ke kiam mi estos plenkreskulo, mi
volos iun tiel. Kaj mi kapablas pagi iun tiel. Laŭrezone, estus tute
sence aĉeti ĝin nun anstataŭ poste, kaj akiri la uzon de ĝi
tuj. Tio estas la sama mono, en la du okazoj, ĉu ne?  Mi volas diri, mi volus
iun bonan, kun multaj ĉambroj interne, sufiĉe bona por ke mi ne devu
simple âceti iun pli bonan poste\ldots'' Harry haltis, espere.

La rigardo de Profesorino McGonagall ne ŝanceliĝis. ``Kaj, kion vi
\emph{gardus} en tia trunko, S-iĉo Potter—''

``Librojn''

``Evidente,'' suspiris Profesorino McGonagall.

``Vi devintus diri al mi multe pli frue pri la fakto ke tia magia
artikolo ekzistas! Kaj ke mi kapablas aĉeti ĝin! Nun mia patriĉo kaj mi
devos pasi la du sekvajn tagojn pasi \emph{rapide} tra ĉiuj brokantaj
librejoj por akiri malnovajn lernolibrojn, por ke mi havu decan
sciencan bibliotekon kun mi en Herpŭrko—kaj eble etan sciencfikcian
kolekton, se mi povas kunveni ion decan el brokanta ujo. Aŭ plej bona,
mi faros la interkonsenton iom pli dolĉa por vi, konsentite? Nur lasu
min aĉeti—''

``\textbf{S-iĉo Potter!} Vi opinias ke vi povas \emph{subaĉeti} min?''

``Kio? \emph{Ne!} Ne tiel! Mi diras ke Herpŭrko povas gardi kelkajn el
la libroj, kiujn mi aportos, se vi opinias ke iuj el tiuj estos bonaj
aldonaĵoj al la biblioteko. Mi akiros malmultekostajn librojn, kaj mi
nur volas havi ilin ie proksima. Estas akceptebla subaĉeti iun kun
\emph{libroj}, ĉu ne? Tio estas—''

``Familia tradicio.''

``Jes, ekzakte''

La korpo de Profesorino McGonagall ŝajnis falonta, siaj ŝultroj
malleviĝante inter sia nigra robo. ``Mi ne povas nei la sencon de viaj
vortoj, malgraŭ ke mi multe deziras ke mi povu. Mi permesos vin
retiri cent pliajn Galionojn, S-iĉo Potter.'' Ŝi suspiris denove. ``Mi
scias ke mi povas bedaŭri tion, kvankam mi ĝin faras.''

``Jen la spirito! Kaj ĉu `poŝo el moke-haŭto' faras tion, kion mi pensas?''

``Ĝi ne povas fari tiom kiom trunkoj,'' la sorĉistino diris kun
videbla malbonvolo, ``sed\ldots poŝo el moke-haŭto kun Ekstrakta
Ĉarmo kaj Nedetektebla Etenda Ĉarmo povas konteni multajn artikolojn
ĝis ili estas vokitaj per la persono, kiu metis ilin en ĝi—''

``Jes! Mi certege bezonas iun el tiuj ankaŭ! Ĝi estus kiel la supera
zono de definitiva mojoseco. La utila zono de Batman! Ne gravas mia
svisa armea tranĉilo, mi povus porti tutan ilaron en ĝi! Aŭ
\emph{libroj}! Mi povus havi la tri ĉefajn librojn, kiujn mi estas
leganta, kun mi ĉiutempe, kaj simple eltiri iun el ili ie ajn! Mi ne
plu devos malŝpari alian minuton de mia vivo! Kion vi diras pri tio
Profesorino McGonagall? Tio estas por la bonfarto de la legado de
infano, la plej bona ebla kialo.''

``\ldots Mi suposas ke vi povas aldoni aliajn dek Galionojn.''
Griphoko estis favoranta Harry'n kun rigardo de sincera respekto, eble
eĉ rekta admiro.

``Kaj eta kvanto da plia mono, kiel vi menciis pli frue. Mi pensas ke
mi memoras vidi unu aŭ du aliajn aĵojn, kiujn mi povus voli stoki en
tiu haŭtpoŝo.''

``\emph{Ne puŝu S-iĉo Potter.}''

``Sed, ho, Profesorino McGonagall, kial fuŝi mian paradon? Certe, tiu
ĉi estas \emph{feliĉa} tago, dum kiu mi malkovras ĉiujn aferojn pri
sorĉado unuafoje! Kial aktori la rolon de la grumblema plenkreskulo,
kiam anstataŭ tio, vi povus rideti kaj memori vian propran infanecon,
rigardante la ravan aspekton de mia juna vizaĝo, kiam mi aĉetas
kelkajn ludilojn uzante bagatelan frakcion de la riĉeco kiun mi gajnis
per venki la plej teruran sorĉiston, kiun la Britanio iam konis, ne ke
mi akuzas vin pri ne esti dankema aŭ ia ajn, sed tamen, kio estas
kelkaj ludiloj kompare al tio?''

``\textbf{Vi,}'' grumblis Profesorino McGonagall. Estis rigardo sur
ŝia vizaĝo, tiel timinda kaj terura ke Harry grincetis kaj retropaŝis,
stumblante sur stako da oraj moneroj kun granda tinta bruo kaj li kuŝiĝis
malantaŭen sur amaso da mono. Griphoko suspiris kaj metis manplaton
sur la vizaĝon. ``Mi farus bonegan servon al la sorĉa Britujo, S-iĉo
Potter, se mi enŝlosus vin en ĉi tiu sekurĉambro kaj lasus vin ĉi
tie.''

Kaj ili foriris sen pli da problemoj.

