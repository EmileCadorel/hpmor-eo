\chapter{Reciprokeco}

\lettrine{L}a lipoj de Petunia Evans-Verres estis tremantaj kaj siaj okuloj estis
plorantaj dum Harry brakumis ŝin je la mezo de la Platformo naŭ de la
stacidomo King's Cross. ``Ĉu vi certas ke vi ne volas ke mi venu kun
vi, Harry?'' 

Harry ekrigardis sian patriĉon Mikael Verres-Evans, kiu aspektis
stereotipe severa sed fiera, kaj poste li ree rigardis sian patrinon,
kiu vere aspektis relative\ldots malkomponita. ``Panjo, mi scias ke vi
ne tro amas la magian mondon. Vi ne bezonas veni kun mi. Mi vere
pensas tion.''

Petunia tremetis. ``Harry, vi ne devus zorgi pri mi, mi estas via
patrino kaj se vi bezonas ke iu venu kun vi—''

``Panjo, mi estos sola en Herpŭrko dum \emph{monatoj} kaj
\emph{monatoj}. Se mi ne povas elturniĝi en tranja platformo sole,
estas pli bone malkovri frue anstataŭ pli malfrue, por ke oni povu
haltigi ĉion.'' Li mallaŭtigis sian voĉon al flustro. ``Aldone, Panjo,
ili ĉiuj amas min ĉi tie. Se mi havas iun ajn problemon, ĉio, kion mi
bezonas fari, estas eltiri mian bendon.'' Harry frapetis la ekzercobendon
kovranta sian cikatron, ``Kaj mi havos multe pli da helpo ol mi povas
elporti.''

``Ho, Harry,'' Petunias flustris. Ŝi genuiĝis malsupren kaj brakumis lin
forte, vizaĝo kontraŭ vizaĝo, iliaj vangoj ripozantaj unu kontraŭ la
alia. Harry povis senti la ĉifonitan spiron, kaj poste li aŭdis
nesonoran ploron eskapi. ``Ho, Harry, mi amas vin, ĉiam memoru tion.''

\emph{Estas kiel ŝi timas ke ŝi neniam vidu min denove,} la penso
ekaperis en la kapon de Harry. Li sciis ke sia penso estis prava, sed
li ne sciis kial Panjo estis tiel timanta.

Do, li faris konjekton. ``Panjo, vi scias ke mi ne ŝanĝiĝos en vian
fratinon simple ĉar mi lernas magion, ĉu ne? mi faros iun ajn magion,
kiun vi petos—se mi povas, mi volas diri—aŭ se vi volas ke mi ne uzu
iun ajn magion ĉirkaŭ la hejmo, mi faros tion ankaŭ, mi promesas ke mi
neniam lasos magion veni inter ni—''

Firma brakumo haltigis liajn vortojn. ``Vi estas bonkora,'' lia
patrino murmuris al sia orelo. ``Tre bonkora, mia filiĉo.''

Harry sufokiĝis iomete, tiam pro aŭdi tion.

Lia patrino liberigis lin, kaj stariĝis. Ŝi prenis poŝtukon el sia
mansaketo, kaj kun tremanta mano, spongis la fluan ŝminkon ĉirkaŭ siaj
okuloj.

Estis neniu demando pri ĉu sia patriĉo akompanus lin al la magia parto
de la stacio King's Cross. Lia patriĉo havis problemojn simple por
rigardi la trunkon de Harry direkte. Magio kuris en familioj, kaj
Mikael Verres-Evans eĉ ne povis marŝi.

Do anstataŭ sia patriĉo nur skrape purigis sian gorĝon.  ``Bonan
ŝancon al la lernejo, Harry'', li diris. ``Ĉu vi opinias ke mi aĉetis
al vi sufiĉe da libroj?''

Harry eksplikis al sia patriĉo kial li opiniis ke povis esti granda
ŝanco por fari ion vere revolucian kaj gravan. Profesoriĉo
Verres-Evans kapjesis, kaj liberigis du kompletajn tagojn de sia ege
plenigita agendo, por estri la plej genian aventuron trans brokantaj
librobutikoj iam ajn, kiu koveris kvar urbojn kaj realigis
\emph{tridek} kestojn da sciencaj libroj, kiuj nun loĝis en la kaverna
etaĝo de la trunko de Harry. La pliparto de la libroj kostis unu
pundon aŭ du, sed kelkaj el ili tute ne kostis tiom malmulte, kiel la
tre freŝa \emph{Manlibro de Ĥemio kaj Fiziko} aŭ la kompleto de la
\emph{Encyclopaedia Britannica} de 1972. Lia patriĉo provis malhelpi
Harry'n vidi la ekranojn de la kasoj, sed Harry komprenis ke sia
patriĉo estis spendinta \emph{almenaŭ} mil pundojn. Harry diris al sia
patriĉo, ke li repagos lin, ekde kiam li malkovros kiel konverti
magian oron en Muglan monon, kaj sia patriĉo diris al li ke li saltu
en lagon.

Kaj poste lia patriĉo demandis al li: \emph{Ĉu vi opinias ke mi aĉetis
sufiĉe da libroj?} Estis tute klara, kiun respondon lia patriĉo volis
aŭdi.

La gorĝo de Harry estis raŭka pro iu kalio. ``Vi neniam povas havi
sufiĉe da libroj,'' Li recitis la devizon de la familio Verres, kaj
lia patriĉo surgeniuiĝis kaj donis al li rapidan, firman
brakumon. ``Sed vi certe provis,'' Harry diris, kaj li sentis sin
sufokiĝanta denove. ``Tio estis tre, tre, \emph{tre} bona provo.''

Lia paĉjo rektiĝis. ``Do\ldots'' li diris. ``Ĉu \emph{vi} vidas la kajon
Naŭ kaj Tri-Kvaronoj?''

La stacio King's Cross estis giganta kaj plena, kun muroj kaj plankoj
pavimitaj el ordinaraj teraj makulaj kaheloj. Ĝi estis plena da
ordinaraj homoj, kiuj rapidis pro iliaj ordinaraj aferoj, kaj kiuj
havis ordinarajn konversaciojn, kiuj faris ege multe da ordinaraj
bruoj. La stacio King's Cross havis kajon Naŭa (sur kiu ili estis
staranta) kaj kajon Deka (tuj apud) sed estis nenio inter la kajo Naŭa
kaj la kajo Deka krom maldika, senpromesa barilo. Granda luko supre
lasis multajn lumojn lumigi la tutan mankon de iu platformo Naŭ kaj
Tri-Kvaronoj.

Harry insiste rigardis ĉirkaŭ ĝis siaj okuloj malsekiĝis, pripensante,
\emph{ek}, \emph{vidpovo de magisto}, \emph{ek}, \emph{vidpovo de
  magisto}, sed absolute nenio aperis al li. Li pensis pri elpreni
sian bastonon kaj skui ĝin, sed Profesorino McGonagall avertis lin
kontraŭ la uzado de bastono. Kaj cetere, se estis alia duŝo da
multkoloraj fajreroj, tio povus konduki lin al esti arestita pro kreo
de fajraĵo en fervoja stacio. Kaj ĉio tio se oni supozis ke lia
bastono ne decidus fari ion alia, kiel eksplodi la tutan stacion
King's Cross.  Harry nur iom foliumis siajn lernolibrojn (tamen tiu
foliumo estis sufiĉe jam tute stranga) en tre rapida klopodo por
determini kiajn sciencajn librojn aĉeti dum la sekvaj 48 horoj.

Nu, li havis—Harry ekrigardis sian brakhorloĝon—unu tutan horon por
kompreni tion, ĉar li estis supozita esti en la trajno je la dek unua
horo. Eble tio estis la ekvivalento de la testo de Intelekta Kvociento
kaj la stultaj knaboj ne povis fariĝi magistoj. (Kaj la ekstra tempo,
kion vi donas al vi, determinis vian skrupulon, kiu estas la dua plej
grava faktoro de la lerneja sukceso.)

``Mi sukcesos kompreni,'' Harry diris al siaj atendantaj
gepatroj. ``Tio estas verŝajne ia testo.''

Lia patriĉo malridetis, ``Hm, eble serĉu vojeton el signoj de piedo,
kondukantaj al iu loko, kiu ne ŝajnas havi senson—''

``\emph{Paĉjo!}'' Harry diris. ``Haltigu tion! Mi ne eĉ provis
kompreni tion per mi mem!'' Tio estis tre bona sugesto, kaj tio estas
ankoraŭ pli malbona.

``Pardonu,'' lia patriĉo pardonpetis.

``Ha\ldots'' La patrino de Harry diris. ``Mi ne opinias ke ili farus
tion al lernantoj, ĉu ne? Ĉu vi certas ke Profesorino McGonagall ne diris
ion ajn al vi?''

``Eble ŝi estis distrita,'' Harry diris sen pripensi.

``\emph{Harry!}'' siblis lia patriĉo kaj patrino unisone. ``\emph{Kion
  vi faris?}''

``Mi, um—”'' Harry glutis. ``Rigardu, ni ne havas tempon por tio nun—''

``\emph{Harry!}''

``Mi opiniis tion! Ni ne havas tempon por tio nun! Ĉar tio estas vere
longa historio kaj mi devas kompreni kiel iri al la lernejo!''

Lia patrino havis manon sur sia vizaĝo. ``Kiom malbone tio estas?''

``Mi, ha,'' \emph{Mi ne povas paroli pri tio pro kialo de Nacia
Sekureco}, ``proksimume duone malbona ol la incidento kun la
Projekto de Scienco?''

``\emph{Harry!}''

  ``Mi, er, ho rigardu, estas kelkaj homoj kun strigo, mi iras demandi
al ili kiel fari!'' kaj Harry forkuris for de siaj gepatroj al la
familio kun fajraj ruĝharuloj, lia trunko aŭtomate serpentumis
malantaŭ li.

La dika virino rigardis lin dum li estis alvenanta. ``Bonan Tagon,
kara. Unua tago en Herpŭrko? Ron estas nova, ankaŭ—'' kaj poste, ŝi
rigardis lin fikse kaj atente. ``\emph{Harry Potter?}''

Kvar knabiĉoj kaj ruĝhara knabino kaj strigo, ĉiuj turniĝis samtempe,
kaj poste frostiĝis surloke.

``Ho, \emph{sed!}'', Harry protestis. Li planis iri kiel Harry Verres,
almenaŭ ĝis li alvenos al Herpŭrko. ``Mi aĉetis bandon kaj ĉio! Kial
vi scias kiu mi estas?''

``Jes,'' La patrino de Harry diris, alvenante malantaŭ li per longaj
leĝeraj paŝoj, ``Kial vi \emph{scias} kiu li estas?'' Ŝia voĉo indikis
certan timegon.

``Via bildo estis en la ĵurnaloj.'' diris unu el la du ĝemeloj tute
identaj.

``\emph{HARRY!}''

``*Paĉjo!* tio ne estas tio, kion vi kredas! Tio estas ĉar mi venkis la
Mastron de la Tenebroj Vi-Scias-Kiu, kiam mi havis unu jaron!''

``\emph{KIO?}''

``Panjo povas tion klarigi.''

``\emph{KIO?}''

``Ha\ldots Mikael kara, estas kelkaj aferoj pri kiuj, mi opiniis ke mi
farus pli bone per ne enui vin kun ili, ĝis nun—''

``Pardonu min,'' Harry diris al la ruĝhara familio kiu estis
rigardadanta lin, ``sed tio estus ege helpa se vi povus diri al mi
kiel iri al la kajo Naŭ kaj Tri-Kvaronoj \emph{tuj}.''

``Ha\ldots'' diris la virino. Ŝi levis manon kaj montris la muron
inter la kajoj. ``Simple marŝu direkte al la barilo inter la kajo naŭa
kaj la deka. Ne haltu kaj ne timu ke vi kraŝu sur ĝi, tio estas tre
grava. La plej bona estus ke vi kuru se vi estas nervoza.''

``Kaj ĉiaokaze, ne pensu pri elefanto.''

``*George!* Ignoru lin, Harry kara, estas neniu kialo pro kiu vi ne
devas pensi pri elefanto.''

``Mi estas Fred, Panjo, ne George—''

``Dankon!'' Harry diris kaj li komencis iri al la barilo—

Atendu unu minuton, tio ne funkcius \emph {se li ne kredus pri tio?}

Tio estis dum momento kiel tiu ke Harry malamis sian menson por fakte
funkcii sufiĉe rapide por kompreni ke ĉi tiu estis okazo pri kiu la
``racia dubo'' aplikiĝis. Se li komencus pensi ke li transiros tra la
barilo li estus bone, nur nun li estis zorganta pri ĉu li sufiĉe
kredis por ke li povu transiri tra la barilo, kio volis diri ke li
fakte estis zorganta pri kraŝi sur ĝi—

``\emph{Harry! revenu ĉi tien, vi havas kelkajn klarigojn por fari!}'' Tiu estis lia patriĉo.

Harry fermis siajn okulojn kaj ignoris ĉion, kion li konis pri
pravigita kredebleco, kaj simple provis kredi \emph{vere forte} ke li
transiru tra la barilo kaj—

—la sonoj ĉirkaŭ li ŝanĝiĝis.

Harry malfermis siajn okulojn kaj faletis haltante, sentante relative
malpura pro ke li faris intencan klopodon por kredi ion.

Li estis staranta sur luma, subĉiela kajo apud iu giganta trajno sola,
longa da dek kvar vagonoj, kondukitaj per masiva vapora maŝino el
skarlata metalo kun alta kamentubo, kiu promesis morton al la kvalito
de la aero. La kajo estis jam leĝere homplena (eĉ se Harry fruis
unu tutan horon); dekduoj da infanoj kaj iliaj gepatroj svarmis
ĉirkaŭ benkoj, tabloj, kaj diversaj kolportistoj kaj budoj.

Tute ne bezonis diri ke estis neniu loko kiel tiu en la stacio King's
Cross kaj neniu ĉambro por kaŝi ĝin.

\emph{Konsentite, do aŭ (a) Mi simple teleportis ie tute alia aŭ (b)
ili povas faldi spacon kiel ne eble aŭ (c) ili simple ignoras ĉiujn
regulojn.}

Estis serpenta sono malantaŭ li, kaj Harry turnis sin por observi ke
sia trunko efektive sekvis lin kun siaj etaj ungegaj tentakloj. Ŝajne,
por magia celo, lia pakaĵo ankaŭ sukcesis kredi kun sufiĉa forto
por transiri tra la barilo. Tio estis iom maltrankviliga kiam Harry
komencis pripensi pri tio.

Iom pli malfrue, la ruĝhara knabiĉo kiu aspektis la plej juna alvenis
el la fera arkado (fera arkado?) kurante, kaj tirante sian trunkon
malantaŭ li, kaj preskaŭ kraŝis sur Harry. Harry, sin sentante stulta
pro ke li estis staranta meze de la vojo, rapide komencis formovi de
la lokon, kaj la ruĝhara knabiĉo sekvis lin, tirante forte la tenilon
de sia trunko por ke li povu sekvi la paŝojn. Post kelkaj momentoj,
blanka strigo flirtis tra la arkado kaj iris apogi sin sur la ŝultro
de la knabiĉo.

``Kor,'' diris la ruĝhara knabiĉo, ``ĉu vi \emph{vere} estas Harry Potter?''

\emph{Ne ankoraŭ tio}. ``Mi ne havas logikan manieron por scii tion
certe. Miaj gepatroj edukis min por kredi ke mia nomo estas Harry
James Potter-Evans-Verres, kaj multaj homoj diris al mi ke mi aspektas
kiel miaj gepatroj, mi volas diri miaj aliaj gepatroj, sed,'' Harry
malridetis, konsciante, ``laŭ ĉio kion mi scias, povas facile esti ĉarmo
por polimorfigi infanon en specifan aspekton—''

``Er, kio, amiko?''

\emph{Ne celas Korvungo, mi notas tion.} ``Jes, mi estas Harry Potter.''

``Mi estas Ron Wizle,'' diris la alta maldika knabiĉo kun lentugoj
kaj longa nazo, kaj li levis manon, kiun Harry ĝentile tremis dum ili
marŝis. La strigo donis al Harry strangan ceremonian kaj ĝentilan
ululon (fakte pli io kiel iu ``eehhh'' sono, kiu surprizis Harry).

Je tiu momento, Harry konsciis la eblon de tuja katastrofo. ``Nur unu
sekundo,'' li diris al Ron, kaj li malfermis unu el la tirkestoj de sia
trunko, la unu kiu, se li memoris ĝuste, estis por vintraj vestoj—ĝi
estis—kaj poste li trovis la plej malpezan koltukon, kiun li posedis, sub
sia vintra palto. Harry metis sian bandon, kaj nur rapide malfaldis
la koltukon kaj ligis ĝin ĉirkaŭ sia vizaĝo. Estis iom varme, precipe
dum la somero, sed Harry povis vivi kun tio.

Poste li fermis tiun tirkeston, kaj malfermis alian por eltiri nigran
robon de magisto, kiun li metis sur siaj ŝultroj, nun ke li estis
ekster la Mugla teritorio.

``Jen,'' Harry diris. La sono elvenis leĝere ŝutkovrita tra la koltuko
sur sia vizaĝo. Li turnis sin al Ron. ``Kiel mi aspektas? Stulta, mi
scias, sed ĉu mi estas identigebla kiel Harry Potter?''

``Er,'' Ron diris. Li fermis sian buŝon, kiu estis malfermita. ``Ne
vere, Harry.''

``Tre bone,'' Harry diris. ``Tamen, por ne malhelpi la celon de la
tuta plenumo, vi de nun nomos min kiel,'' Verres povintus ne plu
funkcii. ``S-iĉo Spu.''

``Konsentite, Harry,'' Ron diris malcerte.

\emph{La Forto ne estas speciale granda en ĉi tiu.} ``Nomu\ldots min\ldots
S-iĉo\ldots Spu.''

``Konsentite, S-iĉo Spu—'' Ron haltis. ``Mi ne povas fari tion, tio igas min
senti min stulta.''

\emph{Tio ne estas nur sento.} ``Bone. \emph{Vi} elektas nomon.''

``S-iĉo Pafilego,'' Ron diris tuj. ``Por la Pafilegoj de Ŝudle.''

``Ha\ldots'' Harry sciis ke li estis malbonege bedaŭronta pro demandi
tion. ``Kiu aŭ kio estas la Pafilegoj de Ŝudle?''

``*Kiuj estas la Pafilegoj de Ŝudle? Nur la plej grandioza teamo en la
tuta historio de la Kvidiĉo! Evidente, ili finis je la fundo de la
ligo je la pasinta jaro, sed—''

``Kio estas Kvidiĉo?''

Demandi tion estis ankaŭ eraro.

``Do lasu min kompreni tion ĝuste,'' Harry diris kiam ŝajnis ke la
ekspliko de Ron (kun rilataj manaj gestoj) trankviliĝis. ``Kapti la
Sniĉon valoras \emph{unu cento kaj kvindek poentojn}''

``Jes—''

``Kiom da dek poentoj por golejoj, unu flanko ĝenerale faras \emph{sen} konti la
Sniĉon?''

``Hm, eble dek kvin aŭ dudek dum profesiaj matĉoj—''

``Tio estas nur malbona. Tio malrespektas ĉiujn eblajn regulojn de
fasonado de ludo. Rigardu, la resto de la ludo ŝajnas kvazaŭ ĝi havas
ian sencon, kvazaŭ, por sporto mi volas diri, sed vi estas esence
diranta ke kapti la Sniĉon preskaŭ superŝutas iun ajn ordinaran
interspacon da poentoj. La du serĉantoj estas supre flugantaj cirkaŭ
serĉante la Sniĉon kaj kutime ne interagas kun iu ajn alia, kaj ekvidi la
Sniĉon unue estas plimulte pro ŝanco—''

``Tio ne estas ŝanco!'' protestis Ron. ``Vi devas gardi viajn okulojn
movantaj laŭ la bona skemo—''

``Tio ne estas \emph{interaktiva}, ne estas tien kaj reen kun la aliaj
ludantoj kaj kiom amuza estas rigardi iun nekredeble bona por
movi siajn okulojn? Kaj poste kiu ajn serĉanto havas ŝancan
plonĝon por ekkapti la Sniĉon igis la tutan laboron de ĉiuj aliaj ludantoj
senutila. Estas kiel iu prenis realan ludon kaj greftis tiun sensencan
ekstran postenon por ke vi povu esti la plej Grava Ludanto sen bezoni
vere esti implikita aŭ lerni la reston de ĝi. Kiu estis la unua
serĉanto, la stulta filo de la reĝo, kiu volis ludi Kvidiĉon sed ne
povis kompreni la regulojn?'' Fakte, nun ke Harry pripensis pri tio,
tio ŝajnis kiel mirinde bona hipotezo. Metu lin sur balailo kaj diru
al li ke li kaptu la brilan aĵon\ldots

La vizaĝo de Ron prenis sulkiĝintan askpeton. ``Se vi ne amas Kvidiĉon,
vi ne bezonas moki ĝin!''

``Se vi ne povas kritiki, vi ne povas optimigi. Mi sugestas kiel
\emph{plibonigi la ludon}. Kaj tio estas vere simpla. Forigu la
Sniĉon.''

``Ili ne ŝanĝos la ludon pro ke \emph{vi} diris tion!''

``Mi \emph{estas} la Knabo-Kiu-Postvivis, vi scias. Homoj aŭskultos
min. Kaj eble ke se mi povas konvinki ilin por ŝanĝi la ludon en
Herpŭrko, la nova maniero etendiĝos.''

Aspekto de tuta teruro aperis sur la vizaĝo de Ron. ``Sed, sed, se
vi forigus la Sniĉon, kiel oni scius kiam la ludo finas?''

``*Aĉetu\ldots horloĝon.* Tio estus multe pli justa ol havi matĉon,
kiu kelkfoje finas post dek minutoj kaj kelkfoje ne finas post pluraj
horoj, kaj aldone la programo estus multe pli antaŭvidebla por la
spektantoj'' Harry suspiris. ``Ho, haltu porti tiun aspekton de
absoluta abomeno, mi verŝajne ne fakte prenos tempon por detrui tiun
mizeran ekskuzon por nacia sporto, kaj refari ĝin pli forta kaj pli
inteligenta laŭ mia propra figuro. Mi havas multe, multe, \emph{multe}
pli gravajn aferojn pri kiuj zorgi.'' Harry aspektis
pripensante. ``Kaj ree, tio ne \emph{prenus} tiom da tempo por skribi
la naŭdek-kvin Tezojn de la Reformo de SenSniĉo kaj najle fiksi ĝin
sur pordo de kirko—''


``Potter,'' trene diris la voĉo de juna knabo. ``*kio* estas sur via vizaĝo
kaj \emph{kio} estas staranta apud vi?''

La abomena aspekto de Ron estis anstataŭigita per kompleta malamo. ``\emph{Vi!}''

Harry turnis sian kapon; kaj efektive tiu estis Drako Malfojo, kiu
eble estis devigita porti la standarda roboj de la lernejo; sed
kompensis per trunko almenaŭ tiel magia kaj multe pli eleganta ol tiu
de Harry, ornamita per arĝento kaj smeraldo kaj elportanta tiun, kiu -
Harry supozis - estis la blazono de la familio Malfojo, bela dentega
serpento sur krucitaj eburaj bastonoj.

``Drako!'' Harry diris. ``Er, aŭ Malfojo se vi preferas, tamen tio
sonas kiel Lucius por mi. Mi estas feliĉa ke vi fartas tiel bone post,
hm, nia lasta kunveno. Tiu estas Ron Wizle. Kaj mi provis esti
inkognito, do nomu min, he,'' Harry rigardis malsupren al sia roboj,
``Sijnoriĉo Nigro.''

``\emph{Harry!}'' sbilis Ron. ``Vi ne povas uzi \emph{tiun} nomon!''

Harry palpebrumis. ``Kial ne?'' Ĝi \emph{sonis} agrable malhela, kiel
internacia mano de mistero—

``Mi dirus ke ĝi estas \emph{bela} nomo,'' diris Drako, ``sed ĝi
apartenas al la Nobla kaj la plej Maljuna Domo de Nigro. Mi nomos vin
S-iĉo Arĝento.''

``*Vi* foriru de\ldots de S-iĉo Oro,'' Ron diris fride, kaj li faris paŝon
antaŭe. ``Li ne bezonas paroli al tiuj kiel vi!''

Harry levis pacigan manon. ``Mi uzos S-iĉon Bronzo, dankon por la noma
skemo. Kaj, Ron, hm,'' Harry luktis por trovi manieron por diri tion,
``Mi estas feliĉa ke vi estas\ldots tiel entuziasma pri defendi min,
sed paroli kun Drako ne speciale ĝenas min—''


Tio ŝajne estis tro multe por Ron, kiu turniĝis al Harry kun okuloj
kiuj estis nun flamantaj pro ofendo. ``*Kio?* Ĉu vi \emph{scias} kiu
li estas?''

``Jes, Ron,'' Harry diris, ``Vi probable memoras ke mi nomis lin Drako
sen ke li bezonis sin prezenti.''

Drako subridis. Poste liaj okuloj lumiĝis rigardante la blankan
strigon sur la ŝultro de Ron. ``Ho, kio estas \emph{tio}?'' Drako
diris per tono plena da malico. ``Kie estas la fama rato de la familio
Wizle?''

``Enterigita en la ĝardeno,'' Ron diris fride.

``Oh, tiel malgaja. Pot\ldots ha, S-iĉo Bronzo, mi devus mencii ke
estas komune konsentite ke la familio Wizle havas \emph{la plej
bonan anekdoton pri dombesto neniam rakontita}. Ĉu vi volas rakonti
ĝin, Wizle?''

La vizaĝo de Ron tordiĝis. ``Vi ne pensus ke tio estas komika, se tio
okazus al \emph{via} familio!''

``Ho'', Drako ronronis, ``sed tio neniam okazus al la Malfojoj.''

Ron pugnigis siajn manojn—

``Sufiĉas,'' Harry diris, metante tiom da kvieta aŭtoritato en sia
voĉo, kiom li povis. Estis klara ke kio ajn tio estis, ĝi estis dolora
memoro por la ruĝhara knabiĉo. ``Se Ron ne volas paroli pri tio, li ne
estas obligata paroli pri tio, kaj mi demandos al vi ke vi ankaŭ ne
parolu pri tio.''

Drako turniĝis kun surprizita aspekto al Harry, kaj Ron
kapjesis. ``Tio estas prava, Harry! Mi volas diri S-iĉo Bronzo! Vi
vidas kian personon li estas? Nun diru al li ke li foriru!''

Harry nombris ĝis dek en sia kapo, kio por li estis tre rapida
\emph{12345678910}—stranga kutimo lasita de kiam li havis kvin jarojn
kaj kiam sia patrino unua foje instruis lin fari tion, kaj Harry
opiniis ke sia maniero estis pli rapida kaj devus esti same
efika. ``Mi ne diros al li ke li foriru,'' Harry diris trankvile, ``Li
estas bonvena paroli kun mi se li volas.''

``Nu, mi ne intencas havi interrilatojn kun iu ajn, kiu havas
interrilatojn kun Drako Malfojo,'' Ron anoncis malvarme.

Harry levis siajn ŝultrojn. ``Tio dependas de vi, mi ne intencas lasi
iun ajn diri kun kiu mi povas paroli kaj kun kiu mi ne povas paroli.''
Silente kantante, \emph{bonvolu foriri, bonvolu foriri\ldots}

La vizaĝo de Ron iĝis blanka pro surprizo, kvazaŭ li fakte atendis ke
tiu frazo funkciu. Poste Ron turniĝis, kaj denove tiris la
kondukŝurnon de sia pakaĵo kaj kolere foriris laŭ la kajo.

``Se vi ne ŝatas lin,'' Drako diris scivole, ``kial vi ne simple
foriris?''

``Hm\ldots lia patrino helpis min por kompreni kiel alveni sur ĉi tiu
kajo de la Stacidomo King's Cross, do estis iom malfacile diri al li,
ke li foriru. Kaj tio ne estas ke mi \emph{malamas} tiun Ron'n ulo,''
Harry diris, ``Mi nur, nur\ldots'' Harry serĉis siajn vortojn.

``Vi ne vidas iun ajn kialon ke li ekzistas?'' proponis Drako.

``Pli aŭ malpli.''

``Ĉiaokaze, Potter\ldots se vi vere estis edukita per Mugloj—'' Drako
paŭzis tie, kvazaŭ li atendis neon, sed Harry ne diris ion
ajn. ``-tial, vi eble ne scias, kiel estas esti fama. Homoj volas
tute preni \emph{vian tempon}. Vi \emph{devas} lerni kiel diri ne.''

Harry kapjesis, metante pripensantan aspekton sur sia vizaĝo. ``Tio
ŝajnas bona konsilo.''

``Se vi provas esti agrabla, vi nur finos pasiganta la plejparton de
via tempo kun la plej insistantaj. Decidu kun kiu vi \emph{volas}
pasigi tempon kaj farigu la ceterajn foriri. Vi nur alvenis ĉi tien,
Potter, do ĉiuj juĝos vin laŭ tiuj, kiujn ili vidos kun vi, kaj vi ne
volas esti vidita kun iu kiel Ron Wizle.''

Harry kapjesis denove. ``Se tio ne ĝenas vin ke mi demandas, kiel vi
rekonis min?''

``\emph{Sinjoriĉo Bronzo,}'' Drako diris trene, ``Mi \emph{renkontis}
vin, memoru. Mi vidis iun iranta kun koltuko ĉirkaŭ la kapo,
aspektante tute ridinda. Do mi faris \emph{supozon}.''

Harry klinis sian kapon, akceptante la komplimenton. ``Mi bedaŭras
terure pri tio,'' Harry diris . ``Pri nia unua renkonto, mi volas
diri. Mi ne intencis embarasi vin antaŭ Lucius.''

Drako mansignis dum li donis strangan rigardon al Harry. ``Mi nur
dezirintus ke Patriĉo envenis kiam \emph{vi} estis flatanta
\emph{min}—'' Drako ridis. ``Sed, mi dankas \emph{vin} por tio, kion
vi diris al Patriĉo. Sen tio, mi havintus multe pli da problemojn por
ekspliki.''

Harry klinis sin ankoraŭ pli. ``Kaj dankon al \emph{vi} por tio, kion
vi reciproke diris al Profesorino McGonagall.''

``Nedankinde. Krome, unu el la asistantinoj devis ĵuri per la nomo de
sia plej proksima amiko absolutan sekreton, ĉar Patriĉo diris ke estas
bizaraj onidiroj, kiuj cirkulas, kiel ke vi kaj mi havis batalon aŭ
io kiel tio.''

``Aj,'' Harry diris, tremante. ``Mi \emph{vere} bedaŭras—''

``Ne, ni tion kutimas, Merlino scias ke estas jam multe da onidiroj
pri la familio de Malfojo.''


Harry kapjesis. ``Mi estas feliĉa aŭdi ke vi ne havas problemon.''

Drako ridaĉis. ``Patriĉo havas, hm, \emph{delikatan} senson de humuro,
sed li komprenas tion, kio volas diri trovi amikon. Li komprenas tion
tre bone. Li igis min ripeti tion antaŭ ol mi iris en liton ĉiujn
noktojn dum la lasta monato, 'Mi trovos amikojn en Herpŭrko'. Kiam mi
klarigis ĉion al li kaj ke li vidis tion, kion mi estis faranta, li
aĉetis glaciaĵon por mi.''

La makzelo de Harry falis. ``\emph{Vi sukcesis ŝanĝigi tion en glaciaĵon?}''

Drako kapjesis, askpektante tiel fiera kiel tia faritaĵo
meritis. ``Nu, Patriĉo \emph{sciis} tion, kion mi estis faranta,
evidente, sed li estas tiu, kiu instruis al mi \emph{kiel} fari, kaj
se mi ridetas laŭ la bona maniero \emph{dum} mi faras tion, tio iĝas
patriĉo-filiĉo aĵo kaj poste li \emph{devas} aĉeti glaciaĵon al mi aŭ
mi al li donus tian malgajan aspekton, kvazaŭ mi opinias ke mi
seniluziigis lin.''

Harry rigardis Drako'n kalkulante, perceptante la ĉeeston de alia
mastro. ``Vi havis \emph{lecionojn} pri kiel manipuli homojn?''

``Evidente,'' Drako diris fiere. ``Mi estas Malfojo. Patriĉo pagis tutorojn por
mi.''

``Ŭaŭ,'' Harry diris. Esti leginta \emph{Influo: Scienco kaj Praktiko}
de Robert Cialdini verŝajne ne estas multe kompare al tio. (tamen tio
estas rimarkinda libro). ``Via patriĉo estas preskaŭ tiel genia kiel mia
patriĉo.''

La brovoj de Drako leviĝis leĝere. ``Ho? Kaj kion faris \emph{via} patriĉo?''

``Li aĉetis por mi librojn.''

Drako konsideris tion. ``Tio ne ŝajnas tre impresa.''

``Vi devas esti tie. Ĉiaokaze, mi estas feliĉa aŭdi ĉion tion ĉi. Pro la
maniero laŭ kiu Lucius rigardis vin, mi pensis ke li krucumos
vin.''

``Mia patriĉo vere amas min,'' Drako diris firme. ``Li neniam faros tion.''

``Hm\ldots'' Harry diris. Li memoris la nigran robon, la elegentan
vizaĝon kun blanka hararo, kiu entrudiĝis la butikon de Sinjorino
Malkin, kaptante en la mano ĉi tiun belan, mortigan kanon kun arĝenta
tenilo. Ne estis facile vidi lin kiel amega patriĉo. ``Ne prenu
tion malbone, sed kiel vi \emph{scias} tion?''

``Kio?'' Estis klara ke tio estis demando, kiun Drako ne kutime
demandis al si mem.

``Mi starigas la fundamentan demandon de la racieco: Kial vi kredas
tion, kion vi kredas? Kion vi opinias scii, kaj kiel vi opinias ke vi
scias tion? Kio igas vin opinii ke Lucius ne oferos vin same kiel li
oferus ion ajn por potenco.''

Drako ĵetis alian strangan rigardon. ``Simple, kion vi scias pri
Patriĉo?''

``Hm\ldots li havas seĝon en la Magekoncilio$^{\textit{a:\ref{nomoj:magenkoncilio}}}$,
seĝon en la asembleo de gubernatoroj de Herpŭrko, li estas nekredeble
riĉa, kaj li havas la orelon de la Ministro
Falsio$^{\textit{n:\ref{nomoj:falsio}}}$, kaj lian fidon, kaj li estas la plej
eminenta puristo de sango ekde la foriro de la Mastro de la
Tenebroj, li estas antaŭa Morto-Manĝanto, rekonita por havi la Markon
de la Tenebroj sed li elirigis sin per pretendi ke li estis sub la
influo de la Impera Malbeno, kio estas ridinde neverŝajna kaj preskaŭ
ĉiuj scias tion\ldots malbono kun majuskla 'M' kaj naskita murdisto\ldots
Mi pensas ke tio estas ĉio.''

La okuloj de Drako mallarĝiĝis al fendoj. ``McGonagall diris tion al vi,
ĉu ne?''

``Ne, ŝi ne dirintus ion ajn pri Lucius al mi, krom ke mi devas resti
malproksime de li. Do dum la incidento en la butiko de Pocioj, kiam
Profesorino McGonagall estis okupata per krii sur la butikisto kaj
provis farigi ĉion sub sia kontrolo, mi kaptis unu el la klientoj, kaj
demandis al \emph{ri} pri Lucius.''

La okuloj de Drako estis larĝe malfermitaj denove. ``Ĉu vi vere?''

Harry donis al Drako konfuzan rigardon. ``Se mi mentis je la unua
fojo, mi ne estas dironta la veron nur pro ke vi demandis.''

Estis mallonga paŭzo dum Drako absorbis tion.

``Vi komplete iros en Serpentimo'n.''

``Mi komplete iros en Korvungo'n, multajn dankojn. Mi nur volas
potencon por akiri librojn.''

Drako subridis. ``Jes, bone. Ĉiaokaze\ldots por respondi al tio, kion
vi demandis\ldots'' Drako prenis profundan respiron, kaj sia vizaĝo
iĝis serioze. ``Patriĉo unu foje mankis baloton de Magekoncilio por
mi. Mi estis sur balailo kaj mi falis kaj rompis multe da miaj
ripoj. Tio vere doloris min. Mi neniam estis tiel dolorigita antaŭe kaj
mi pensis ke mi estis mortonta. Do Patriĉo mankis ĉi tiun vere gravan
baloton, ĉar li estis tie apud mia lito en Sankta Mungo, tenante miajn
manojn kaj promesante ke mi saniĝos.''

Harry ekrigardis for malkomforte, poste, kun peno, li devigis sin mem
rigardi denove Drako'n. ``Kial vi diras al mi \emph{tion}? Tio ŝajnas
\ldots privata\ldots''

Drako donis seriozan rigardon al Harry. ``Unu el miaj tutoroj unufoje
diris ke homoj kreas proksiman amikecon per scii privatajn aĵojn pri
unu la alia, kaj ke la kaŭzo pro kio homoj ne havas proksimajn amikojn
estas ke ili estas tro sinĝena por interŝanĝi ion vere gravan pri
ili.'' Drako turnis siajn manojn invintante. ``Via vico?''

Scii ke la esperama vizaĝo de Drako verŝajne estis instruita al li per
monatoj da praktiko, ne igis tion êc malpli efika, Harry
observis. Fakte, tio \emph{igis} tion \emph{malpli} efika, sed
bedaŭrinde ne \emph{senefika}. Oni povis diri la saman de la lerta uzo
de Drako de la reciprokeca premo per nepetita donaco. Iu tekniko pri kiu
Harry legis en siaj libroj pri psikologio (eksperimento montris ke
senkondiĉa donaco de \$5 estis dufoje pli efika ol kondiĉa donaco de
\$50 por igi ke homoj plenigu enketon). Drako faris nepetitan donacon
de konfidenco, kaj nun invitis Harry'n por fari konfidencon
reciproke\ldots kaj la aĵo estis ke Harry sentis sin premita. Rifuzo,
Harry estis certa, estus ricevita per aspekto de malgaja seniluziĝo,
kaj eble eta kvanto da malestimo indikanta ke Harry perdis poentojn.

``Drako,'' Harry diris, ``nur por ke vi sciu, mi rekonis ekzakte
tion, kion vi estas faranta tuj. Miaj propraj libroj nomas tion
\emph{reciprokeco}, kaj ili parolas pri kiel doni al iu donacon da du
Sikloj estis montrita dufoje pli efika ol oferti al li dudek Siklojn
por igi ke li faru tion, kion vi volas\ldots''  Harry mallaŭtiĝis.

Drako aspektis vere malgaja kaj seniluziĝa. ``Tio ne intencis esti
ruzo, Harry. Tio estas reala maniero por iĝi amikoj.''

Harry levis manon. ``Mi ne diris ke mi ne respondos. Mi nur bezonas
tempon por selekti ion, kio estas privata sed ne estas noca. Ni
diru\ldots Mi volas ke vi scias ke mi ne povas esti rapidigita por
fari aĵon.'' Paŭzo por pripensi povus grande malarmi le potencon de
multaj teknikoj de konformigo, post kiam vi lernis kiel rekoni tion,
kio ili estas.

``Konsentite,'' Drako diris. ``Mi atendos ke vi trovas ion. Ho, kaj
bonvolu retiri tiun koltukon dum vi ĝin diros.''

\emph{Simpla sed efika.}

Kaj Harry ne povis malhelpi sin rimarki kiel mallerta, konsterna,
sengracia sia provo por rezisti manipuladon aperis kompare al tiu de
Drako. \emph{Mi bezonas tiujn tutorojn.}

``Konsentite,'' Harry diris post momento. ``Jen estas la mia.'' Li
rigardis ĉirkaŭ kaj poste rulis la koltukon sur sia vizaĝo,
malkovrante ĉion krom sia cikatro. ``Hm\ldots Ŝajnas ke vi vere povas
fidi vian patriĉon. Mi volas diri\ldots se vi parolus al li serioze,
li ĉiam aŭskultus vin kaj prenus vin serioze.''

Drako kapjesis.

``Kelkfoje,'' Harry diris, kaj glutis. Tio estis surprize malfacila,
sed tio estis intencita por esti tiel. ``Kelkfoje, mi deziras ke mia
propra patriĉo estu kiel la via.'' La okuloj de Harry eksaltis for de
la vizaĝo de Drako, pli aŭ malpli aŭtomate, kaj poste Harry devigis
sin rigardi Drako'n denove.

Poste, tio frapis Harry \emph{Kion diable li ĵus diris}, kaj Harry
haste aldonis, ``Ne ke mi deziras ke mia patriĉo estu perfekta ilo
de morto kiel Lucius, mi nur volis diri ke li prenu min serioze—''

``Mi komprenas,'' Drako diris kun rideto. ``Jen\ldots nun ĉu ne ŝajnas
kvazaŭ ni estas iom pli proksima al fariĝi amikoj?''

Harry kapjesis. ``Jes. Ŝajnas, fakte. Hm\ldots sen ofendo, sed mi
remetas mian maskoveston, mi \emph{vere} ne volas trakti kun—''

``Mi komprenas.''

Harry rerulis la koltukon sur sia vizaĝo.

``Mia patriĉo prenas ĉiujn el siaj amikoj serioze,'' Drako
diris. ``Pro tio li havas multe da amikoj. Vi devus renkonti lin.''

``Mi pripensos pri tio,'' Harry diris kun neŭtrala voĉo. Li skuis sian
kapon scivole. ``Tial vi vere estas sian malfortan punkton. Ĉu ne?''

Nun Drako estis donanta al Harry \emph{vere} strangan rigardon. ``Vi
volas iri akiri ion por trinki kaj trovi lokon por sidi?''

Harry ekkonsciis ke li estis staranta je la sama loko dum tro longe,
kaj streĉis sin, provante kraki sian dorson. ``Kompreneble.''

La kajo komencis pleniĝi nun, sed estis ankoraŭ iu loko pli kvieta
fora de la ruĝa vapora maŝino. Laŭ la vojo ili pasis antaŭ budo
enhavanta kalvan barban viron vendanta ĵurnalojn kaj komiksojn kaj
stakigitajn boteletojn fluoreske verdajn.

La butikisto estis, fakte, apogita sur si mem kaj trinkanta unu el la
fluoreske verdaj boteletoj ĵus kiam li ekvidis la delikatan kaj
elegantan Drako'n Malfojo proksimiĝante kun mistera knabo aspektante
nekredeble stulta kun koltuko ligita ĉirkaŭ la vizaĝo, kio farigis la
butikiston ekhavi subitan tusadon dum li trinkis, kaj salivumi amason
da fluorekse verda likvo sur sian barbon.

``'ardonu min,'' Harry diris. ``sed kio \emph{estas} tiu aĵo, ekzakte?''

``Komed-Teo,'' diris la butikisto. ``Se vi trinkas ĝin, io surpriza
okazos, kaj tiu igos vin tuskraĉi sur vi aŭ sur iu alia. Sed tiu estas
ĉarmita por malaperi post kelkaj sekundoj—''

Efektive la makulo sur lia barbo estis jam malaperita.

``Kiom komika,'' diris Drako. ``Kiom vere, vere komika. Venu, S-iĉo
Bronzo, iru ni trovi alian—''

``Atendu,'' Harry diris.

``*Ho, estu serioza!* Tio estas tiel, tiel \emph{junula}!''

``Ne, mi bedaŭras Drako, mi \emph{devas} enketi pri tio. Kio okazas se
mi trinkas Komed-Teo'n dum mi faras kiel eble plej bone por gardi la
konversacion komplete serioza?''

La butikisto ridetis misterie. ``Kiu scias? Amiko marŝas en kostumo de
rano? io neatendata estas destinita okazi—''

``Ne. Mi bedaŭras. Mi simple ne kredas tion. Tio malrespektas mian
forte mistraktitan suspension de nekredemo laŭ tiel multe da niveloj ke
mi ne eĉ havas la vortojn por priskribi ĝin. Estas, estas simple
neniu \emph{maniero} por ke ridinda \emph{trinkaĵo} povas manipuli la
realaĵon por krei \emph{komedian aranĝadon}, aŭ mi rezignos
kaj retiriĝos al Bahamoj—''

Drako ĝemis. ``Ĉu vi \emph{vere} faros tion?''

``Vi ne estas devigata trinki tiun sed mi \emph{devas}
esplori. \emph{Mi devas}. Kiom kostas tiu ĉi?''

``Kvin Knutojn la boteleto,'' La butikisto diris.

``*Kvin Knutojn?'' Vi povas vendi eferveskajn trinkaĵojn kiuj manipulis la
realaĵon por \emph{kvin Knutoj la boteleto?}'' Harry atingis sian
haŭtpoŝon kaj diris ``Kvar Sikloj kaj kvar Knutoj'' kaj metis ilin sur la
giĉeton. ``Du dekduoj da boteletoj, bonvolu.'' 

``Mi ankaŭ prenos unu,'' Drako suspiris, kaj komencis atenigi siajn
poŝojn.

Harry kapneis rapide. ``Ne, mi pagos, ne prenu tion kiel favoro, mi
volas vidi se tio funkcias sur vi ankaŭ.''  Li prenis boteleton el la
stako nun metita sur la giĉeto kaj ĵetis ĝin al Drako, poste li
komencis plenigi sian haŭtpoŝon. La Larĝigeblan Lipon de la haŭtpoŝo
manĝis la boteletojn akompanate de etaj bruoj de ruktoj, kiuj ne
ekzakte helpis restaŭri la fidon de Harry pri ke li iam malkovros
racian klarigon pri cîo tio ĉi.

Du dek du ruktoj pli malfrue, Harry havis la lastan aĉetitan boteleton
en la mano, Drako estis rigardanta lin atendante, kaj la du el ili
retiris la ringon je la sama tempo.

Harry rulis sian koltukon supren por ekspozi sian buŝon, kaj ili klinis
ilian kapon malantaŭen kaj trinkis la Komed-Teo'n.

Tiu iel \emph{gustis} hele verde—eferveskege kaj pli citronguste ol
citrono.

Krom tio, nenio alia okazis.

Harry rigardis al la butikisto, kiu estis rigardanta ilin bonvole.

\emph{Kompreneble, se tiu ulo ĵus prenis la avantaĝon de natura
akcidento por vendi dudek-kvar boteletojn de nenio al mi, mi
aplaŭdos lian kreeman entrepreneman viglecon kaj poste mortigos lin.}

``Tio ne ĉiam okazas tuj,'' la butikisto diris. ``Sed tio estas
garantia por okazi unu fojon por unu boteleto, aŭ repagita.''

Harry prenis alian longan gluton.

Ankoraŭ unufoje, nenio okazis.

\emph{Eble, mi devas nur trinki la tutan aĵon kiel eble plej rapide\ldots
kaj esperi ke mia stomako ne eksplodas pro la karbona dioksido, aŭ
ke mi ne ruktas dum mi trinkas\ldots}

Ne, li povis elporti esti \emph{iom} pacienca. Sed, honeste, Harry ne
vidis kiel tio povis funkcii. Vi ne povas iri al iu kaj diri ``Nun mi
suprizigos vin'' aŭ ``kaj nun, mi diros la pintumon de
la ŝerco, kaj tiu estos vere komika.'' Tio runigis la valoron de la
ŝoko. En la stato de mensa prepariteco de Harry, Lucius Malfojo
povintus marŝi en kostumo de baletistino, ke tio ne igus lin havi
ĝustan ekkraĉon.

``Ĉiuokaze, ni sidiĝu,'' Harry diris. Li preparis sin por trinki alian
gluton kaj komencis iri al la foraj seĝoj, tio metis lin en la ĝustan
angulon por rigardi malantaŭen kaj vidi la stativon de ĵurnaloj de la
budo kiu estis dediĉa al la ĵurnalo nomita \emph{La Ĉikananto}, kiu
estis montranta la sekvantan ĉeftitolon:

\paragraph{\emph{KNABO-KIU-POSTVIVIS IGIS}}

\paragraph{\emph{DRAKO MALFOJO'n GRAVEDA}} ``\emph{He!} kriis Drako dum
hela verda likvo ekverŝis ĉie sur lin el la direkto de Harry. Drako
turniĝis al Harry kun fajro en la okuloj kaj kaptis sian propran
boteleton. ``Vi, filo de Kotsango! vidu ni kiel \emph{vi} ŝatas ke oni
kraŝas sur vi!'' Drako prenis intence guston el sia boteleto ĵus antaŭ
ol siaj okuloj ekvidis la ĉeftitolon.

Per kruta reflekso, Harry provis protekti sian vizaĝon dum la ŝpruco
de likvaĵo flugis al sia direkto. Bedaŭrinde, li uzis la manon kiu
tenis la Komed-Teo'n, sendante la reston de la verda linkvaĵo sur
siajn ŝultrojn.

Harry fiksrigardis la boteleton kiu estis en sia mano, dum li estis
ankoraŭ sufokanta kaj ŝpruciganta, kiam la verda koloro komencis
malaperi el la roboj de Drako.

Poste li rigardis supren kaj rigardis fikse la ĉeftitolon de la ĵurnalo.

\paragraph{\emph{KNABO-KIU-POSTVIVIS IGIS}}

\paragraph{\emph{DRAKO MALFOJO'n GRAVEDA}} La lipoj de Harry malfermiĝis
kaj diris, ``buh-bluh-buh-buh\ldots''

Tro multe da rivalaj protestoj, tio estis la problemo. Ĉiufoje kiam
Harry provis diri ``Sed, ni havas nur dek-unu jarojn!'' La protesto
``Sed, viriĉo ne povas esti graveda!'' petis por la unua prioritato
kaj estis poste dispremita per ``Sed estas nenio inter ni, vere!''


Poste Harry denove rigardis malsupren la boteleton en sia mano.

Li sentis profundan deziron forkuri kriante plenpulme ĝis li falas
pro manko de oksigeno, kaj la sola aĵo kiu haltigis lin estis ke li
unufoje legis ke tiu entjera paniko estas la signo de \emph{reale} grava
scienca problemo.

Harry graŭlis, ĵetis la boteleton violente en la proksiman rubujon, kaj
paŝegis ree al la budo. ``Unu ekzemplero de \emph{La Ĉikananto},
bonvolu.'' Harry pagis kvin pliajn Knutojn, ekstraktis alian boteleton de
Komed-Teo el sia haŭtpoŝo, kaj paŝegis al la areo de pikniko kun
la blond-hara knabiĉo, kiu estis rigardadanta sian propran boteleton kun
aspekto de malkaŝa admiro.

``Mi reprenas tion, kion mi diris,'' Drako diris, ``Tio estis sufiĉe bona.''

``He, Drako, ĉu vi scias tion, kion mi vetas esti pli bona por fariĝi
amiko ol interŝanĝi sektretojn? Plenumi murdon.''

``Mi havas tutoron, kiu diras tion,'' Drako konfesis. Li metis manon
en sian robon kaj gratis sin per facila, natura movo. ``Kiun vi havas
en menso?''

Harry brute ĵetis malsupren \emph{La Ĉikananto}'n sur la tablon de
pikiniko. ``La ulo kiu elpensis tiun ĉeftitolon.''

Drako ĝemis . ``Ulino. \emph{dek jaraĝaj} ulino, ĉu vi povas kredi
tion? Ŝi iĝis freneza post kiam sia patrino mortis kaj sia patriĉo,
kiu posedas la ĵurnalon, estas \emph{konvikita} ke ŝi estas profetino,
do kiam li ne scias li demandas al Luno Bonamo$^{\textit{n:\ref{nomoj:bonamo}}}$,
kaj kredas \emph{ion ajn}, kion ŝi diras.''

Ne vere pripensante pri tio, Harry tiris la ringon de sia sekva
boteleto de Komed-Teo kaj preparis sin por trinki. ``Ĉu vi mokas
min?  Tio estas eĉ pli malbona ol la ĵurnalismo de Mugloj, kaj tion mi
imagis esti fizike neebla.''

Drako graŭlis. ``Ŝi havas ian obsedon pri la Malfojoj, kaj sia patriĉo
estas politike kontraŭ ni, do li presas ĉiujn vortojn. Tuj kiam mi
estos sufiĉe aĝa, mis seksperfortos ŝin.''

Verda linkvaĵo ekŝprucis el la naztruoj de Harry, trempante la
koltukon kiu ankoraŭ kovris tiun areon.  Komed-Teo kaj pulmo ne bone
miskiĝis, kaj Harry pasis la sekvantajn sekundojn furioze tusi.

Drako rigardis lin akre. ``Io malbona?''

Estis je tiu momento ke Harry subite konsciis ke (a) la sono venanta
el la resto de la kajo de la trajno transformiĝis al io pli kiel
nebula blanka bruo, proksimume je la sama momento kiam Drako metis la
manon en sian robon, kaj (b) kiam li parolis pri plenumi murdon kiel
maniero por krei ligojn, estis ekzakte unu persono en la konversacio
kiu pensis ke ili estis ŝercanta.

\emph{Nu, Ĉar li} aspektis \emph{kiel normala infano. Kaj li estis
  normala infano, li nur estis tiu, al kiu oni povus sin} atendi
\emph{ se etalona viriĉo havus Darth Vader'n kiel amega patriĉo}.

``Jes, Nu,'' Harry tusis, ho dio, kiel li faros por eliri ĉi tiun
glitan konversacion, ``Mi nur estas surpriza pri kiel vi estas
kompleza pri diskuti tion tiel malkaŝe, vi ne ŝajnas zorgi pri esti
kaptita aŭ io ajn.''

Drako ronkis. ``Ĉu vi ŝercas? La vortoj de \emph{Luno Bonamo}
kontraŭ la miaj?''

Sankta fekaĵo sur sankta bastono. ``Ne ekzistas aĵo kiel magia
detektilo, mi supozas?'' \emph{Aŭ DNA testo\ldots jam.}


Drako rigardis ĉirkaŭe. Liaj okuloj mallarĝiĝis. ``Tio estas prava, vi
scias nenion. Rigardu, mi klarigos aĵojn por vi, mi volas diri la
manieron laŭ kiu tio vere funkcias, kiel se vi estus jam en Serpentimo
kaj farus la saman demandon al mi. Sed vi devas ĵuri ke vi ne diros ion ajn
pri tio.''

``Mi ĵuras,'' Harry diris.

``La tribunalo uzas Verec-serumon, sed tio estas ŝerco vere, vi nur
bezonas farigi vin Forgesigita antaŭ atesti, kaj poste pretendi ke la
alia persono estas ĉarmita kun falsa memoro. Evidente se vi estas nur
normala persono, la tribunalo konjektos pri Forgeso prefere al ĉarmo
de falsa memoro. Sed la tribunalo havas singardemon, kaj se mi estas
koncernata, do tio entrudiĝas en la honoron de Nobela Domo, do tio
iros al la Magekoncilio, kie Patriĉo havas la voĉojn. Poste mi estos
rekonita senkulpa, kaj la Bonamo familio devos pagi kompensaĵojn pro
ke ili provis malheligi mian honoron. Kaj tion ili scias, de la
komenco, do ili simple gardos iliajn buŝojn fermitaj.''

Frosta tremo venis al Harry, tremo kiu venis kun la instrukcioj de
gardi sian voĉon kaj vizaĝon normalaj. \emph{Noto por mi mem: Renversi
  la registrejo de la magia britio ekde la plej frua oportuno.}

Harry tusis denove por klarigi sian gorĝon. ``Drako, bonvolu, bonvolu,
\emph{bonvolu} ne prenu tion malbone, mi havas nur unu parolon, sed
kiel vi diris mi povos esti en Serpentimo kaj mi vere volas demandi
por informaj celoj, do kio okazos se \emph{teorie dirante} mi
\emph{atestas} ke mi aŭdis vin plani tion?''

``Do, se mi estus iu ajn krom Malfojo, mi havos problemojn,'' Drako
respondis arongante. ``Pro tio ke mi \emph{estas} Malfojo\ldots
Patriĉo havas la voĉojn, kaj poste li dispremos vin\ldots nu, mi
supozas malfacile, pro tio ke vi \emph{estas} la Knabo-Kiu-Postvivis,
sed patriĉo estas sufiĉe bona kun tiaj aĵoj.'' Drako
malridetis. ``Cetere, \emph{vi} parolis pri mortigi ŝin, kial vi ne
zorgis pri \emph{mi} atestante post ol ŝi mortas?''

\emph{Kiel, ho, kiel mia tago igos tiel malbona?} La buŝo de Harry
estis jam movanta pli rapide ol li povis pensi. ``Tio estis kiam mi
pensis ke ŝi estis \emph{pli aĝa}! Mi ne scias kiel tio funkcias tie,
sed en la Mugla Britio la tribunalo estus multe pli konsternita pri iu
mortiganta infanon—''

``Tio havas sencon,'' Drako diris, ankoraŭ aspektante iom
suspekta. ``Sed, ĉiuokaze, tio estas ĉiam pli inteligenta, se tio
komplete ne iras al la oreloj de la Aŭroroj. Se ni estas atenta pri
fari nur aĵojn, kiujn la Kuraca Ĉarmo povas repari, ni povas simple
Forgesigi ŝin poste, kaj refari la tutan aĵon la sekvan semajnon.''
Poste la blond-hara knabo subridis, juna akra sono. ``Krome, imagu
ŝin diranta ke ŝi estis seksperfortita per Drako Malfojo \emph{kaj} la
Knabo-Kiu-Postvivis, eĉ \emph{Dumbledore} ne kredos ŝin.''


*Mi disŝiros vian mizeran magian restaĵon de la malhelaj aĝoj en
pecetoj pli malgrandaj ol la atomoj, kiuj konsistigas ĝin.* ``Fakte,
ĉu oni povas meti tion apuda? Post ol mi malkovris ke tiu ĉeftitolo
estis elpensita de knabino, kiu estas malpli aĝa ol mi, mi havas
malsamajn pensojn por mia venĝo.''

``Vere? Do, diru,'' Drako diris, kaj li komencis preni alian guston de
sia Komed-Teo.

Harry ne sciis ĉu la sorĉo funkciis pli ol unu fojo por unu boteleto,
sed li sciis ke li povis eviti la kulpigon, do li zorge altempigis
tion ekzakte kiel necese :

``Mi estis pensanta, \emph{iun tagon mi geedziĝos kun ĉi tiu
  virino}.''

Drako faris aĉan sonon kaj likis verdan fluaĵon el la anguloj de sia
buŝo kiel difektita radiatoro de aŭto. ``\emph{Ĉu vi freneziĝis?}''

``Tute kontraŭe, mi estas tiel sana, ke tio brulas kiel glacio.''

``Vi havas pli bizaran guston ol Lestrangoj$^{\textit{n:\ref{nomoj:lestrango}}}$,''
Drako diris, sonante dunone admira pri tio. ``Kaj mi supozas ke vi
volas ŝin tute por vi mem, ĉu ne?''

``Jes. Mi povas ŝuldi favoron al vi por tio—''

Drako svingis sian manon. ``Ne, tiu ĉi estas senpaga.''

Harry fiksrigardis malsupren al la boteleto en sia mano, la malvarmo
enloĝiĝinte en sia sango. Ĉarma, feliĉa, oferema kun siaj favoroj al
siaj amikoj, Drako ne estis psikopato. Tio estis la malgaja kaj
abomena parto, koni homan psikilogion sufiĉe bone por \emph{scii} ke
Drako \emph{ne estis} monstro. Estis dek mil societoj trans la mondo
en kiu, cî tiu konversacio povus okazi. Ne, la mondo estus tute
malsama loko se bezonis \emph{diabla mutaciulo} por diri tion, kion
Drako diris. Tio estis vere simpla, vere homa, tio estis la normaleco
kiam nenio ekstera intervenis. Por Drako, siaj malamikoj ne estis homoj.

Kaj en ĉi tiu malrapida tempo de ĉi tiu malrapida lando, tie kaj nun,
kiel en la malhela momento antaŭ la aŭroro, antaŭ la aĝo de la
racieco, la filiĉo de sufiĉe potenca nobeliĉo simple prenus por donita
ke li estis super la leĝoj, almenaŭ kiam temis pri kamparana
knabino. Estis lokoj en la Mugla mondo, kie tio estis ankoraŭ la sama
afero, landoj kie tia nobelaro ankoraŭ ekzistis, kaj estis ankoraŭ
pensita simile, aŭ eĉ pli malhelaj landoj kie ne temis nur pri
nobelaro. Estis simple kiel ĉiuj lokoj kaj tempoj, kiuj ne originis de
la klerigado. Idaro, kiu ŝajne ne inkluzivis la tutan magian Brition,
tamen, estis transkultura poluado de aferoj kiel boteletoj kun ringo.

\emph{Kaj se Drako ne ŝanĝas siajn mensojn pri sia envio de venĝo, kaj
  se mi ne forĵetas mian propran ŝancon de feliĉo en la vivo per
  geedziĝi kun iu freneza virino, do ĉio kion mi ĵus aĉetis estas
  tempo, kaj ne tiel multe\ldots}

Por unu knabino, ne por la aliaj.

\emph{Mi min demandas kiel malfacila estus fari liston de ĉiuj el la
  supro de la puristoj de sango kaj mortigi ilin.}

Ili provis ekzakte tion dum la Franca revolucio, pli aŭ malpli—fari
liston de ĉiuj la malamikoj de la progreso kaj forigi ĉion super la
kolo—kaj tio ne fukciis bone laŭ tio, kion Harry memoris. Eble, li
bezonis senpolvigi la kelkajn historiajn librojn, kiujn sia patriĉo
aĉetis por li, kaj vidi ĉu tio, kio misiris dum la Franca revolucio,
estus facile riparebla.

Harry rigardis supren al la ĉielo, kaj al la pala formo de la la luno,
videbla tiun matenon en la sennebula ĉielo.

\emph{Do la mondo estas difektita kaj misa kaj freneza, kaj kruela kaj
  sanga kaj malhela. Ĉu tio estas nova? Vi ĉiam sciis tion,
  ĉiuokaze\ldots}

``Vi aspektas serioza,'' Drako diris. ``Lasu min diveni, via muglaj
gepatroj diris al vi ke tiaj aferoj estas malbonaj.''

Harru kapjesis, ne tute fidante sian voĉon.

``Nu, kiel Patriĉo diris, eble estas kvar domoj, sed finfine ĉiuj
apartenas al Serpentimo aŭ al Huflopufo. Kaj sincere, vi ne apartenas
al la flanko de Huflopufo. Se vi decidus turni vin al la flanko de la
Malfojoj sekrete\ldots nia povo kaj nia reputacio\ldots vi povus
foriĝi de aferoj, el kiuj \emph{Mi} mem ne povus. Ĉu vi volas
\emph{provi} dum momento? Por vidi kiel tio estas?''

\emph{kia lerta eta serpento. dek-unu jaraĝa kaj jam konvinkanta
  vian predon ke ri ne devas kaŝi sin\ldots}

Harry pripensis, konsideris, elektis sian armilon. ``Drako, ĉu vi
volas klarigi la tutan aferon de pureco de sango al mi? Mi estas iel
nova.''

Larĝa rideto aperis sur la vizaĝo de Drako. ``Vi vere devus renkonti
mian patriĉon kaj demandi al \emph{li}, vi scias, li estas nia
gvidanto.''

``Donu al mi la version de tridek sekundoj.''

``Konsentite,'' Drako diris. Li prenis profundan enspiron, kaj lia
voĉo fariĝis legere pli basa, kaj li prenis kadencon. ``Nia potenco
fariĝis pli malforta, generacio post generacio, samtempe la malpureco
de kotsango pligrandiĝis. Kiam Salazario kaj Godriko kaj Koŭena kaj
Helga$^{\textit{n:\ref{nomoj:huflopufo}}}$ unufoje kreskigis Herpŭrko'n per ilia
povo, kreis la Medalionon kaj la Glavon kaj la Diademon kaj la
Pokalon, neniu sorĉisto de niaj velkantaj tagoj altiĝis sufiĉe por
rivali kun ili. Ni estas velkantaj, ĉiuj velkantaj al Mugloj, ĉar oni
brediĝas kun iliaj gentoj kaj permesas ke niaj
Skiboj$^{\textit{a:\ref{nomoj:skibo}}}$ vivas. Se la malpuro ne estas regita,
baldaŭ niaj bastonoj rompos kaj la tuto de nia arto ĉesos, la gento de
Merlino finos kaj la sango de la Atlantido malsukcesos. Niaj infanoj
estos lasitaj gratantaj la koton por postvivi kiel la ordinaraj
Mugloj, kaj la malhelo kovros la mondon eterne.'' Drako trinkis alian
guston el lia boteleto, aspektante kontenta; tio ŝajnis esti la tuta
argumento laŭ lia opinio.


``Konvinka,'' Harry diris, volante diri tion priskribe anstataŭ
norme. Tio estis standarda skemo: La Falo el Graco, la bezono de gardi
tion, kio restas de la pureco kontraŭ la malpuro, la estinteco irante
supren kaj la estonteco irante nur malsupren. Kaj tiu skemo ankaŭ
havas sian \emph{replikon}\ldots ``Mi devas korekti vin pri unu
punkto, tamen. Via informo pri la Mugloj iomete ne plu estas
aktuala. Ni ne plu ekzakte gratas la koton.''

La kapo de Drako ekmoviĝis. ``*Kio?* Kion vi volas diri per \emph{ni}?''

``Ni. La sciencistoj. La gento de Francis Bacon kaj la sango de la
Klerigado. Mugloj ne nur sidas ĉirkaŭe plorante pri ke ili ne havas
bastonojn, ni havas niajn proprajn povojn nun, kun aŭ sen magio. Se
ĉiuj el viaj povoj malaperus, tiam ni ĉiuj perdus ion preciozan, ĉar
la magio estas la sola indiko por scii kiel le universo vere
funkcias—sed oni ne estos lasita grantanta la koton. Onia domo ankoraŭ
estos malvarma dum somero kaj varma dum vintro, estos ankoraŭ
kuracisto kaj medikamentoj. Scienco povas gardi onin vivanta se la
magio malaperas. Tio estos katastrofo, sed ne laŭlitere la fino de ĉiu
la lumo de la mondo. Nur por diri.''

Drako estis retropaŝinta de pluraj paŝoj kaj sia vizaĝo aspektis kiel
miksaĵo de timo kaj nekredemo. ``\emph{Pri kio, per la nomo de Merlino,
  vi estas parolanta, Potter?}''

``He, mi aŭskultis \emph{vian} rakonton, ĉu vi ne aŭskultos la mian?''
\emph{Mallerta}, Harry riproĉis sin mem, sed Drako fakte haltis
retropaŝi kaj ŝajnis aŭskulti.

``Ĉiaokaze,'' Harry diris, ``Mi diras ke vi ne ŝajnas tre atentinta
pri tio, kio okazis en la Mugla mondo.'' Verŝajne ĉar la tuta sorĉa
mondo ŝajnis rigardi la reston de la tero kiel ladurbo, meritante tiom
ĵurnalaj informadoj kiom la \emph{Financial
  Times}\footnote{\emph{N.d.t}: angla gazeto, kiu fokusiĝas pri
  ekonomiaj temoj} aljuĝas la rutinajn agoniojn de la
Burundo. ``Bonege. Rapidan kontrolon. Ĉu la sorĉistoj jam iris sur la
lunon? Vi scias ĉi tiu aĵo?'' Harry montris supren al la giganta kaj
fora globo.

``\emph{Kio?}'' Drako diris. Estis relative klara ke tio penso neniam
alvenis al ĉi tiu knabo. ``*Iri* al la—tiu nur estas—'' Lia fingro
montris la etan palan aĵon en la ĉielo. ``Vi ne povas
Aperei$^{\textit{a:\ref{nomoj:apereco}}}$ ie, kie vi neniam iris kaj kiel iu ajn
  povus iri al la Luno unue?''

``Atendu,'' Harry diris al Drako, ``Mi deziras montri al vi libron,
kiun mi alportis kun mi, mi pensas ke mi memoras en kiu skatolo ĝi
estas.'' Harry ekstaris, genuis kaj tiris la ŝtuparon kiu iris al
la kaverna etaĝo de sia trunko, poste li malsupreniris kaj levis
skatolon desur alia skatolo, estante danĝere proksima al trakti
siajn librojn kun malrespekto, kaj deprenis la kovrilon de la
skatolo kaj rapide sed zorge puŝis stakon da libroj—

(Harry heredis la preskaŭ magian povon de Verres de memori kie ĉiuj
siaj libroj estis, eĉ se li vidis ilin nur unufoje, kio estis relative
mistera pro la manko de iu ajn genetika ligo.)

Kaj Harry kuris reen sur la ŝtupoj kaj reŝovis la ŝtuparon en la
trunkon per sia kalkano, kaj, anhelante, turnis la paĝojn de la libro
ĝis li trovis la bildon, kiun li volis montri al Drako.

La unu kun la blanka, seka, kraterita tero, kaj la homoj en kostumo,
kaj la blua kaj blanka globo pendanta supre ĉio tio.

Tiu bildo.

\emph{La} bildo, se nur unu bildo en la tuta mondo devontus postvivi.

``\emph{Tio},'' Harry diris, sia voĉo tremanta ĉar li ne tute povis
gardi la fieron distance, ``estas, kiel la Tero aspektas de la Luno.''

Drako malrapide klinis sin antaŭen. Estis stranga esprimo sur sia juna
vizaĝo. ``Se tiu estas \emph{vera} bildo, kial ĝi ne movas?''

\emph{Movas?} Ho. ``Muglo povas movigi bildojn sed ili bezonas pli
granda skatolo por vidigi ilin, ili ne povas taŭgi ilin sur sola paĝo
de libro por la momento.''

La fingro de Drako moviĝis al unu el la kostumoj. ``Kio estas tiuj
aĵoj?'' Lia voĉo komencante ŝanceliĝi.

``Tiuj estas homoj. Ili vestas kostumoj, kiuj kovras iliajn tutajn
korpojn por doni al ili aeron, ĉar ne estas aero sur la Luno.''

``Tio estas neebla,'' Drako flustris. Estis teruro en siaj okuloj, kaj
kompleta konfuzo. ``Ne, Mugloj ne povas fari tion,
neniam. \emph{Kiel}\ldots''

Harry reprenis la libron, turnis la paĝojn ĝis li trovis tion, kion li
vidis. ``Tiu estas raketo irante supren. La fajro puŝas ĝin pli alta
kaj pli alta, ĝis ĝi atingas la Lunon.'' Li turnis paĝojn denove,
``Tiu estas raketo sur la grundo. Tiu eta grajno apud ĝi estas
persono?''  Drako anhelis. ``Iri al la Luno kostas la ekvivalenton
de\ldots probable ĉirkaŭ mil milionoj da Galionoj.'' Drako
sufokiĝis. ``Kaj tio prenis la efortojn de\ldots probable pli da homoj
ol vivas en la tuta magia Britio.'' \emph{Kaj kiam ili alvenis, ili
  lasis plakon kiu diris, 'Ni venis en paco, por la tuta homaro.'
  Tamen vi ne estas preta por aŭdi ĉi tiujn vortojn, Drako
  Malfojo\ldots}

``Vi diras la veron,'' Drako diris malrapide. ``Vi ne falsus
tutan libron nur por tio—kaj mi povas aŭdi tion en via
voĉo. Sed\ldots, sed\ldots''

``Kiel, sen bastono kaj sen magio? Tio estas longa historio,
Drako. Scienco ne funkcias per skui bastonojn kaj kanti sorĉojn, tiu
funkcias per scii kiel la universo funkcias laŭ tiel profunda maniero
ke vi scias ekzakte tion, kion vi devas fari por farigi la universon
fari tion, kion vi volas. Se la magio estas kiel ĵeti la \emph{imperan
  malbenon} sur iu por ke ri faru tion, kion vi volas, tiam scienco
estas kiel koni rin tiel bone ke vi povas konviki rin ke tio estis ria
propra ideo ekde la komenco. Tio estas multe pli malfacila ol skui
bastonon, sed tio funkcias kiam bastonoj ne, ekzakte kiel kiam la
\emph{Imperiosu$^{\textit{a:\ref{nomoj:imperiosu}}}$} malsukcesas, vi ankoraŭ povas
provi persvadi la personon. Kaj Scienco kreas sin de generacio al
generacio. Vi devas vere \emph{scii} tion, kion vi faras por plenumi
sciencon—kaj kiam vi vere komprenas ion, vi povas ekspliki ĝin al iu
alia. La plej bonaj sciencistoj de la lasta jarcento, la plej helaj
nomoj pri kiuj oni ankoraŭ parolas kun respekto, iliaj povoj estas
nenio por la plej bonaj sciencistoj de hodiaŭ. Ne estas ekvivalento en
scienco de via perdita arto kiu kreis Herpŭrko'n. En scienco, niaj
povoj pliiĝas post la jaroj. Kaj ni komencas kompreni kaj malkonfuzi
la sekretojn de la vivo kaj heredado. Ni kapablos ekzameni la sangon,
pri kiu vi parolas, kaj vidi tion, kio farigas vin sorĉisto, kaj post
unu aŭ du pli generacioj, ni kapablos konvinki tian sangon, por ke li
farigu de ĉiuj infanoj tre potencajn sorĉistojn. Do vi vidas, via
problemo ne estas tiel malbona kiel ĝi aspektas, ĉar post nemulte pli
ol jardekoj, scienco kapablos solvi ĝin por vi.''

``Sed\ldots'' Drako diris. Lia voĉo estis tremanta. ``Se Mugloj havas
tian povon\ldots do\ldots kio \emph{ni} estas?''

``Ne, Drako, ne estas tio, ĉu vi ne vidas? Scienco uzas la
komprenpovon de homoj de rigardi la mondon kaj kompreni kiel ĝi
funkcias. Tio ne povas malsukcesi sen ke la homaro malsukcesas
ankaŭ. Via magio povas halti, kaj vi povas malami tion, sed vi ankoraŭ
estos \emph{vi}. Vi ankoraŭ vivos por plori pri tio. Sed ĉar scienco
ripozas sur mia huma intelekto, ĝi estas la povo kiu oni ne povas
forigi el mi, sen forigi \emph{min}. Eĉ se la leĝoj de la universo
ŝanĝiĝas por mi, kaj ke mia tuta scio estas nuligita, mi nur devos
kompreni la novajn leĝojn, kiel tio estis farita unue. Tio ne estas
aĵo de Muglo, tio estas aĵo de Homo, tio simple rafinas kaj trejnas la
povon, kiun vi uzas ĉiujn fojojn kiam vi rigardas ion, kaj vi ne
komprenas kaj vin demandas 'Kial?' Vi estas Serpentimo, Drako, ĉu vi ne
komprenas la implicon?''

Drako levis la rigardon el la libro al Harry. Lia vizaĝo montris
tagiĝan komprenon. ``Sorĉisto povas lerni kiel uzi tiun povon.''

Tre zorge, nun\ldots la kaptilo estis metita, nun la hoko\ldots ``Se vi
volas lerni pensi kiel \emph{homo} anstataŭ kiel \emph{sorĉisto}, do
vi povas trejni kaj rafini viajn povojn kiel homo.''

Kaj se \emph{ĉi tiu} instruo ne estis en \emph{ĉiuj} instruadoj de
scienco, Drako ne bezonis tion scii, ĉu ne?

La okuloj de Drako estis nun enpensaj. ``Vi\emph jam faris tion?''

``Ĝis iu grado,'' Harry konsentis. ``Mia trejno ne estas kompleta. Ne
je dek unu jaroj. Sed—mia patriĉo \emph{ankaŭ} pagis por mi tutorojn,
vi vidas.'' Certe, ili estis malsataj studentoj, kaj tio okazis nur
pro ke Harry dormis laŭ ciklo de 26-horoj, sed oni lasu tion apud por
la momento\ldots

Malrapide, Drako kapjesis. ``Vi opinias ke vi povas ellerni
\emph{ambaŭ} artojn, kunmeti la du artojn, kaj\ldots'' Drako
fiksrigardis Harry'n. ``fari de vi mem la Mastron de la du mondoj?''

Harry faris diablan ridon, tio ŝajnis simple veni nature je ĉi tiu
momento. ``Vi devas kompreni, Drako, ke la tuta mondo kiun vi konas,
la tuta magia Britio, estas nur unu kvadrato sur multe pli larĝa
tabulo. La tabulo kiu enhavas lokojn kiel la Luno, kaj la steloj en la
nokta ĉielo, kiuj estas tiel lumaj kiel la Suno sed simple neimageble
malproksimaj, kaj aĵoj kiel galaksioj, kiuj estas ege pli gigantaj ol
la Tero aŭ la Suno, aĵoj tiel larĝaj ke nur sciencistoj povas vidi
ilin kaj vi eĉ ne scias ke ili ekzistas. Sed mi vere \emph{estas}
Korvungo, vi scias, ne Serpentimo. Mi ne volas regi la universon. Mi
nur pensas ke ĝi povas esti pli saĝe organizata.''

Estis timego sur la vizaĝo de Drako. ``Kial vi diras al \emph{mi} tion?''

``Ho\ldots ne estas multe da homoj, kiuj scias kiel fari \emph{veran}
sciencon—kompreni ion por la unua fojo, eĉ se tio frenezigas onin. Helpo
estus bonvena.''

Drako fikse rigardis Harry'n kun la buŝo malferma.

``Sed ne faru eraron, Drako, vera scienco tute ne estas kiel magio, vi
ne povas simple fari sciencon kaj resti senŝanĝa kiel lerni kiel diri la
vortojn de nova ĉarmo. La povoj venas kun kosto, kosto tiel alta ke la
plejparto de la homoj rifuzas pagi ĝin.''

Drako kapjesis al tio kvazaŭ, finfine, li aŭdis ion kion li povis
kompreni. ``Kaj kion ĉi tiu kostas?''

``Lerni akcepti ke vi eraris.''

``Hm,'' Drako diris post ol drama paŭzo streĉis dum momento. ``Ĉu vi
klarigos tion?''

``Kiam vi provas kompreni kiel io funkcias je tiu profunda nivelo, la unuaj
naŭdek eksplikoj kiujn vi havos, estos malpravaj. La centa estos
prava. Do vi devas lerni kiel akcepti ke vi eraras, ankoraŭ kaj
ankoraŭ kaj ankoraŭ denove. Tio ne ŝajnas multe, sed tio estas tiel
malfacila ke la plejparto de la homoj ne povas fari sciencon. Ĉiam
demandi al vi mem, ĉiam preni alian rigardon al aĵoj, kiujn vi
konsideris kiel evidentaj,'' kiel havi Sniĉon en Kvidiĉo, ``Kaj
ĉiufoje kiam vi ŝanĝas viajn pensojn, vi ŝanĝas vin mem. Sed mi iras
multe tro antaŭen tie. Multe tro antaŭen. Mi nur volas ke vi
sciu\ldots Mi proponas profitigi vin de iom da miaj konatoj. Se vi
volas. Estas nur unu kondiĉo.''

``Hu, nu,'' Drako diris. ``Vi scias, Patriĉo diras ke kiam iu diras
tion al vi, tio estas neniam bona signo, neniam.''

Harry kapjesis. ``Nun, ne miskomprenu min, kaj ne pensu ke mi provas
stiri kojnon inter vi kaj via patriĉo. Ne temas pri tio. Nur estas ke
mi volas trakti kun iu kiu havas mian aĝon, anstataŭ trakti kun Lucius
trans vi. Mi opinias ke via patriĉo konsentus kun tio ankaŭ, li scias ke
vi devas kreski iam. Sed viaj movoj en nia ludo devos esti viaj
propraj. Tiu estas mia kondiĉo—mi traktos kun vi, Drako, ne kun via
patriĉo.''


``Mi devas foriri,'' Drako diris. Li stariĝis. ``Mi devas foriri kaj
pripensi pri tio.''

``Prenu vian tempon,'' Harry diris.

La sonoj de la kajo de la trajno ŝanĝiĝis de nebuloj al bruo dum Drako
vagis for.

Harry malrapide elspiris la aeron kiun li detenis ene sen tute
konscii, kaj poste rigardis la brakhorloĝon sur sia pojno, simpla
meĥanika modelo kiun lia patriĉo pagis por li, esperante ke ĝi
funkcius en ĉeesto de magio. La montrilo de sekundoj ankoraŭ estis
tiktakanta, kaj se la montrilo de minutoj estis prava, do ne estis
tute la dekunua horo ankoraŭ. Li probable devus iri en la trajnon
baldaŭ kaj komenci serĉi \emph{kiu-estis-ŝia-nomo}'n, sed ŝajnis
valori kelkajn minutojn, fari kelkajn ekzercojn de spirado por vidi ĉu
sia sango varmiĝos denove.

Sed kiam Harry rigardis supren el sia brakhorloĝo, li vidis du
siluetojn proksimiĝante kaj aspektante komplete ridindaj kun iliaj
vizaĝoj kovritaj per vintraj koltukoj.

``Bonan tagon, S-iĉo Bronzo,'' diris unu el la maskitaj siluetoj. ``Ĉu
ni povas interesi vin pri kolektiĝi en la Ordeno de la Ĥaoso?''


\begin{center}\rule{3in}{0.4pt}\end{center}

\emph{Post la fakto:}

Malmulte da tempo poste, kiam la tuta agitado de ĉi tiu tago finfine
trankviliĝis, Drako estis klinita sur skribtablo kun plumo en la
mano. Li havis privatan ĉambron en la loĝturo de Serpentimo, kun lia
propra skribtablo kaj lia propra fajro—bedaŭrinde ne eĉ \emph{li}
meritis konekton al la Kamenreto$^{\textit{a:\ref{nomoj:kamenreto}}}$, sed almenaŭ
Serpentimo ne kredis la kompletan nesencan aĵon pri fari ke
\emph{ĉiuj} dormas en dormejo. Ne estis multe da privataj ĉambroj, vi
devis esti el la \emph{plej} bonaj en la Domo de la plej bona
\emph{speco}, sed tion oni povis konsideri evidenta por la Malfojoj.

\emph{Kara Patriĉo,} Drako skribis.

Kaj poste, li haltis.

Inko malrapide gutis de sia plumo, makulante la pergamenon apud la
vortoj.

Drako ne estis stulta. Li estis juna, sed liaj tutorojn trejnis lin
bone. Drako sciis ke Potter verŝajne sentis multe pli da simpatio por
la fakcio de Dumbledore ol li bonvole rivelis\ldots Tamen Drako pensis
ke Potter povis esti tentita. Sed estis tute klare ke Potter provis
tenti Drako'n ekzakte kiel Drako provis tenti lin.

Kaj estis klare ke Potter estis brila, kaj multe pli ol nur leĝere
freneza, kaj ke li ludis vastan ludon, kiun Potter li mem plimulte ne
komprenis, improvizante je la plej rapideco kun la subtileco de
tumulta nundo\footnote{Traduko de \emph{nundu}, kiu estas magia besto
  en la originala universo de Harry Potter, aspektante kiel
  leopardo}. Sed Potter sukcesis elekti taktikon kiun Drako ne povis
simple forlasi. Li proponis al Drako parton de lia propra povo,
vetanta ke Drako ne uzos tiun sen iĝi pli kiel li. Lia patriĉo nomis
tiun altnivela teĥniko, kaj avertis Drako'n ke tiu ofte ne funkcias.

Drako sciis ke li ne komprenis ĉion, kio okazis\ldots sed Potter
proponis al \emph{li} la ŝancon de ludi kaj nun estis \emph{lia}
vico. Kaj se li rakontos la tutan aĵon, tio iĝos la vicon de Patriĉo. 

Finfine, tio estis tiel simple. La plej malaltaj teĥnikoj bezonis la
nescion de la celo, aŭ almenaŭ lian dubon. Flatado devas esti verŝajne
kaŝita kiel admiro. (``Vi devintus esti en Serpentimo'' estas malnova
klasikaĵo, tre efika sur kelkaj tipoj de personoj, kiuj ne atendis
tion, kaj se tio ne funkcias vi povas ripeti ĝin.) Sed kiam vi trovas
la definitivan levilon de iu, ne gravas se li scias ke vi
scias. Potter, en sia freneza hasto, malkovris ŝlosilon por la animo
de Drako. Kaj se Drako sciis ke Potter sciis tion—eĉ se tio estis
evidenta speco de diveno—tio ne ŝanĝis ion ajn.

Do nun, por la unua fojo en sia vivo, li havis veran sekreton
por gardi. Li ludis sian propran ludon. Estis malhela doloro kun
tio, sed li sciis ke Patriĉo estos fiera, kaj tio farigis ĉion bone.

Lasi la inkon makuli—estis mesaĝo tie, kaj iu kiun lia patriĉo
komprenos, ĉar ili ludis la ludon de subtelico pli ol unu fojo—Drako
skribis la veran demandon, kiu vere ronĝis lin pri la tuta afero, la
parto kiun ŝajne li devus kompreni, sed li ne komprenis ĝin, tute ne.

\emph{Kara Patriĉo:}

\emph{Supozu ke mi diras al vi ke mi renkontis kunlernanton en Herpŭrko,
  kiu ne ankoraŭ estas en nia cirklo de konatoj, kiu nomis vin
  'perfekta ilo de morto' kaj kiu diris ke mi estas vian 'malfortan
  punkton'. Kion vi dirus pri li?}

Ne prenis multe da tempo post tio por ke la familia strigo alportas la
respondon.

\emph{Mia amata filo:}

\emph{Mi dirus ke vi feliĉe renkontis iun kiu ĝuas la intiman
  fidon de nia amiko kaj valora aliancano, Severus Skoldo.}

Drako fikse rigardis la leteron dum momento, kaj fine ĵetis ĝin en la
fajron.

