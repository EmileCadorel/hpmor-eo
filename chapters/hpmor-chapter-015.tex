\chapter{Skrupuleco}

\begin{center}\rule{3in}{0.4pt}\end{center}

\emph{``Mi estas certa ke mi trovos tempon ie.''}

\begin{center}\rule{3in}{0.4pt}\end{center}

``\emph{Fridigeru!}''

Harry plonĝigis fingron en la glason de akvo kiu estis sur sia
skribtablo. Ĝi devis esti malvarmigita. Sed varmeta ĝi estis, kaj
varmeta ĝi restis. Ankoraŭ.

Harry sin sentis vere, vere trompita.

Estis centoj de fantazia romano disĵetitaj en la hejmo de
Verres'j. Harry legis grandan kvanton de tiuj. Kaj komencis aspekti
kvazaŭ li havis malhelan flankon. Do post kiam la glaso de akvo rifuzis
kunlabori dum la unaj kelkefoje, Harry ĵetis rigardon ĉirkaŭ la
klasĉambro de Ĉarmo por certigi ke neniu estis rigardanta, kaj poste
prenis longan enspiron, koncentris, kaj igis lin mem kolera. Li pensis
pri la Serpentimoj ĉikanantaj Nevilo'n, kaj la ludo kie iu frapas la
librojn ĉiufoje kiam vi provas rekapti ilin. Li pensis pri kion Drako
Malfojo diris pri la dek jaraĝa Bonamo knabino kaj kiel la
Magekoncilio reale funkciis\ldots

Kaj la furioza eniris sian sangon, li levis sian bastonon en mano kiu
tremis pro malamo kaj diris per malvarma tono ``\emph{Fridigeru!''}
kaj absolute nenio okazis.

Harry estis \emph{fraŭdita}. Li volis skribi al iu kaj demandi
\emph{repagi} de sia malhela flanko kiu klare devintus havi
nerezisteblan povon magian, sed aperis esti \emph{difektita}.

``\emph{Fridigeru!}'' diris Hermione denove ĉe la skribtablo apud
lia. Ŝia akvo estis solida glacio kaj estis blankaj kristaloj formante
sur la rando de la glaso. Ŝi aspektis tute absorbita per sia propra
laboro kaj tute ne konscia de la aliaj studentoj fikse rigardante ŝin
kun malamindaj okuloj, kio estis aŭ (a) dangere forgesema per ŝi aŭ
(b) perfekte ruzega spektaklo leviĝante al la nivelo de delikata arto.

``Ho vere bone, S-ino Granger!'' blekis Filius Flirtiko, ilia
Profesoro de Ĉarmoj kaj Direktoro de la Domo Korvungo, tute eta viro
kun neniu signo ke li estis pasinta ĉampiono de duelo. ``Bonega!
Admirinda!''

Harry estis atendanta esti, je la plej malbona okazo, dua malantaŭ
Hermione. Harry preferintus ke \emph{ŝi} rivalas \emph{sin} evidente,
sed li povintus akcepti tion inverse.

Ekde lundo, Harry iris al la fundo de la klaso, pozicio por kiu li
rivalis komforte la aliajn Mugle kreskitajn studentojn krom
Hermione. Kiu estis tute sola kaj sen rivalo je la antaŭ de la klaso,
kompatindulino.

Profesoro Flirtiko staris sur la skribtablo de unu el la aliaj
Mugle kreskitaj kaj kviete alĝustigis la maniero per kiu ŝi tenis sian
bastonon.

Harry rigardis supren Hermione'n. Li glutis forte. Tio estis ŝia
evidenta rolo en la skemo de aĵoj\ldots ``Hermione?'' Harry diris
nekonfirmite. ``Ĉu vi havas iun ajn ideo pri kio mi povas misfari?''

La okuloj de Hermione lumiĝis per terura lumo de komplezemo kaj io en
la malantaŭ de la cerbo de Harry kriis senespere humilige.

Kvin minutoj pli malfrue, la akvo de Harry ŝajnis rimarkinde pli
malvarma ol la temperaturo de la ĉambro, kaj Hermione estis doninta al
li keklajn verbajn frapetojn sur la kapo kaj al li diris prononci ĝin
pli zorgeme la sekva fojo kaj eliris por helpi iun alian.

Profesoro Flirtiko donis al ŝi Doman poenton por helpi lin.

Harry estis kunpremanta siajn dentojn tiel forte ke sia makzelo
doloris kaj tio ne helpis sian prononcon.

\emph{Tio al mi egalas se tio estas maljusta konkurso. Mi scias ekzakte kion
mi faros kun mia du ekstra horo ĉiutage. Mi sidiĝos en mia trunko kaj
studos ĝis kiam mi sukcesos atingi la nivelon de Hermione Granger.}

\begin{center}\rule{3in}{0.4pt}\end{center}

``Transfiguro estas unu el la plej kompleksa kaj danĝera magio kiun
vi lernos en Herpŭrko,'' diris Profesorino McGonagall. Estis neniu
spuro de iu ajn malseriozo sur la vizaĝo de la maljuna sorĉistino
severa. ``Iu ajn stultumanta en mia klaso eliros kaj neniam eniros
denove. Vi estas avertitaj.''

Ŝia bastono malsupren iris kaj frapetis ŝian skribtablon, kiu glate
ŝanĝis ĝin mem en porkon. Iuj Mugle naskitaj studentoj faris kelkaj
etaj jelpoj. La porko ĉirkaŭrigardis kaj ronkis, aspektante konfuzita,
kaj poste reŝanĝis en skribtablon.

La Profesorino de Transfiguro rigardis ĉirkaŭ la klasĉambro, kaj
poste liaj okuloj fiksis sur unu studento.

``S-ro Potter,'' diris Profesorino McGonagall. ``Vi nur ricevis viajn
lernolibroj de kelkaj tagoj antaŭe. Ĉu vi komencis legi vian
Transfiguriĝa libro?''

``Ne, pardonu min Profesorino,'' Harry diris.

``Vi ne bezonas pardonpeti, S-ro Potter, se vi estus postulata legi
antaŭe, oni dirintus tion al vi.'' La fingroj de McGonagall tuŝis la
skribtablon kiu estis antaŭ ŝi. ``S-ro Potter, ĉu vi povas provi
diveni se tiu estas skribtablo kiun mi transfiguris en porkon, aŭ se
tiu estas porko kaj mi mallonge deprenis la Transfiguron? Se vi
estus leginta la unuan ĉapitron de via libro, vi scius.''

Harry leĝere sulkigis la brovojn. ``Mi supozas ke tio estas pli facila
komenci kun porko, pro la fakto ke se vi komencus kun skribtablo, ĝi
povus ne scii kiel stari.''

Profesorino McGonagall kapneis. ``Tio ne estas via kulpo, S-ro Potter,
sed la ĝusta respondo estas ke en Transfiguro vi ne provas
diveni. Malĝusta respondo estos markita per ekstrema severeco,
demandoj sen respondo estos markitaj per granda indulgemo. Vi devas
lerni tion, kion vi ne scias. Se mi faras demandon al vi, ne gravas
kiel evidenta aŭ elementa, kaj ke via respondo estas 'Mi ne certas',
mi ne imputos ĝin al vi kaj kiu ajn ridos, perdos Domajn poentojn. Ĉu
vi povas diri al mi kial tiu regulo ekzistas, S-ro Potter?''

\emph{Ĉar sola eraro en Transfiguro povas esti nekredebla danĝera.}
``No.''

``Ĝusta. Transfiguro estas pli danĝera ol Apereco, kiu ne estos al
vi instruita antaŭ via sesa jaro. Bedaŭrinde, Transfiguro devas esti
instruita kaj praktikita je juna aĝo por maksimumigi vian adoltan
kapablecon. Do tio estas danĝera temo, kaj vi devus esti tute timigita
pri fari erarojn, ĉar neniu el miaj studentoj iam estis permanente
vundita kaj mi \emph{ege ĉagrenos} se vi estas la unua klaso kiu
\emph{fuŝas mian rekordon}.''

Pluraj studentoj ekglutis.

Profesorino McGonagall stariĝis kaj moviĝis al la muro malantaŭ sia
skribtablo, kiu tenis lignan tabulon blankan. ``Estas multaj kialoj
pro kiuj Transfiguro estas danĝera, sed unu kialo staras super la
aliaj.'' Ŝi kreis marklion verŝajne el nenio, kaj skizis literojn el
hela ruĝa; kiun ŝi post substrekis, uzante la saman markilon, en blua:

TRANSFIGURIĜO NE ESTAS PERMANENTA!

``Transfiguro ne estas permanenta!'' diris Profesorino
McGonagall. ``Transfiguro ne estas permanenta! Transfiguro ne
estas permanenta! S-ro, supozu ke studento transfigurus briketo de
ligno en tason de akvo, kaj vi trinkus ĝin. Kion vi imagas al vi
okazos kiam la Transfiguro eluziĝos?'' Estis paŭzo. ``Pardonu min,
mi ne devintus demandi tion al vi, S-ro Potter, mi forgesis ke vi emas
havi imagon nekutime pesimisma—''

``Mi bonfartas,'' Harry diris glutante forte. ``Do la unua respondo
estas ke mi ne \emph{scias},'' Profesorino kapjesis aprobe, ``sed mi
\emph{imagas} ke povus esti\ldots ligno en mia stomako, kaj en mia
sangocirkulado, kaj se iu de tiu akvo estis absorbita en la histo de
mia korpo—ĉu ĝi estis pulpo aŭ solida ligno aŭ\ldots'' La kompreno de
magio de Harry lasis lin fali. Li ne povis kompreni kiel ligno ŝanĝis
en akvon unue, do li ne povis kompreni tion, kio okazos post kiam la
akvaj molekuloj estis miksitaj per ordinaraj termikaj movoj kaj ke la
magio eluziĝis kaj la ŝanĝo inversis.


La vizaĝo de McGonagall estis rigida. ``Kiel S-ro Potter ĝuste
rezonis, li iĝus ege malsana kaj bezonus senpere kameniri al Hospitalo
S-ta Mungo, se li havus iun ajn ŝanĉo de postvivi. Bonvolu malfermu
vian libron al paĝo 5.''

Eĉ sen sono en la movanta bildo, vi povis diri ke la virino kun terura
senkolorigita haŭto estis krianta.

``La krimulo kiu origine transfiguris oron en vinon kaj donis ĝin al
la virino por trinki, 'por repago de ŝuldo' kiel li diris, ricevis
kondamnon de dek jaroj en Azkabano. Bonvolu turnu al paĝo 6. Tiu estas
Aberento. Ili estas la gardistoj de Azkabano. Ili elsuĉas vian magion,
vian vivon, kaj iun ajn ĝojan penson kiun vi provas havi. La bildo sur
la paĝo 7 estas de la krimulo dek jaroj pli malfrue, je sia
liberigo. Vi rimarkos ke li estas morta—jes, S-ro Potter?''

``Profesorino,'' Harry diris, ``Se la plej malbona okazas, en kialo
kiel tio, ĉu estas iu maniero por \emph{daŭrigi} la Transfiguron?''

``Ne,'' Profesorino McGonagall diris kategorie. ``Subteni
Transfiguron estas konstanta dreno de via magio, kiu grimpas kun la
dimensio de la cela formo. Kaj vi bezonus esti kontakte kun la celo en
la daŭro de kelkaj horoj, kiu etas, en tia okazo, neebla. Katastrofo
kiel tiu estas \emph{neriparebla}!''

Profesorino McGonagall kliniĝis antaŭen, lia vizaĝo tre severa. ``Vi
absolute neniam, sub ia ajn cirkonstanco, transfiguros iun ajn en
likvaĵon aŭ gason. Neniu akvon, neniu aeron. Nenia kiel akvon, nenia
kiel aeron. Eĉ se ĝi ne intencas esti trinkita. Likvaĵo
\emph{vaporiĝas}, etaj pecoj kaj fragmentoj de ĝi atingas en la
aeron. Vi ne transfiguros iun ajn por esti brulita. Tiu faros fumon,
kaj iu povus enspiri tiun fumon! Vi neniam transfiguros iun ajn kiu
povas imageble iri en la korpon de iu per ia maniero. Neniu
manĝaĵo. Nenia kiu \emph{aspektas} kiel manĝaĵo. Ne eĉ kiel komika eta
bubaĵo kie vi intencas diri al ili pri la kota torto antaŭ kiam ili
fakte manĝas ĝin. Vi neniam faros tion. Punkto. En tiu klasĉambro aŭ
ekstere de ĝi aŭ \emph{ie ajn}. Ĉu tio estas bone komprenita per
\emph{ĉiuj el vi}?''

``Jes,'' diris Harry, Hermione, kaj kelkaj aliaj. La resto ŝajnis esti
muta.

``\emph{Ĉu tio estas bone komprenita per ĉiuj el vi?}''

``Jes,'' ili diris aŭ grumblis aŭ murmuris.

``Se vi rompas iun ajn el tiuj reguloj vi ne plu studos Transfiguron
dum la resto de via ĉeesto en Herpŭrko. Ripetu post mi. Mi neniam
transfiguros iun ajn en likvaĵon aŭ gason.''

``Mi neniam transfiguros iun ajn en likvaĵon aŭ gason.'' diris la studentoj en ĉifona ĥoro.

``Denove! Pli laŭte! Mi neniam transfiguros iun aj en likvaĵon aŭ gason.''

``Mi neniam transfiguros iun ajn en likvaĵon aŭ gason.''

``Mi neniam transfiguros ian ajn kiu aspektas kiel manĝaĵo aŭ ian ajn
alian kiu iras en homa korpo.''

``Mi neniam transfiguros iun ajn por esti brulita ĉar ĝi povas fari fumon.''

``Vi neniam transfiguros iun ajn kiu aspektas kiel mono, Mugla mono
inkluzivita,'' diris Profesorino McGonagall. ``La koboldoj havas
manieroj por trovi kiun faris tion. En agnoskita leĝo, la kobolda nacio
estas en konstanta stato de milito kun ĉiuj magiaj falsantoj. Ili ne
sendos Aŭrorojn. Ili sendos armeon.''

``Mi neniam transfiguros ian ajn kiu aspektas kiel mono,'' ripetis la studentoj.

``Kaj \emph{preter ĉio},'' diris Profesorino McGonagall, ``vi neniam
transfiguros iun ajn vivantan subjekton, \emph{speciale vi mem}. Tio
igos vin vere malsana kaj eble morta, dependanta de kiel vi
transfiguris vin kaj kiom longe vi daŭrigis la ŝanĝon.'' Profesorino
McGonagall paŭzis. ``S-ro Potter estas aktuale levanta sian manon ĉar
li vidis Animagus transformo—specife, homa transformo en katon kaj
retransformo. Sed Animagus transformo ne estas \emph{libera}
Transfiguro.''

Profesorino McGonagall prenis etan pecon de ligno el sia poŝo. Kun
tapo de ŝia bastono, ĝi ŝanĝi en vitra pilko. Ŝi tiam diris
``\emph{Kristferigu!}'' kaj la vitra pilko ŝanĝis en ŝtalan pilkon. Ŝi
frapetis ĝin kun sia bastono iu lasta fojo kaj la ŝtalan pilkon iĝis
pecon de lignon denove. ``*Kristferigu* transformas subjekton el
solida vitro en saman forman celon el solida ŝtalo. Ĝi ne povas fari
la inverson, nek povas ŝanĝi skribtablon en porkon. La plej ĝenerala
formo de Transfiguro—libera Transfiguro, kiun vi lernos
tie—kapablas transformi iun ajn subjekton en iun ajn celon, almenaŭ
koncerne la fizika formo. Pro tiu kialo, libera Transfiguro devas
esti fari senvorte. Uzi Ĉarmojn bezonus malsamajn vortojn por ĉiuj
malsamaj transformoj inter subjekto kaj celo.''

Profesorino McGonagall donis akran rigardon al siaj studentoj. ``*Iuj*
instruistoj komencas kun Transfiguriĝaj Ĉarmoj kaj movas al libera
Transfiguro poste. Jes, tio estus multe pli facila je la
komenco. Sed ĝi povas fiksi malbonajn kutimojn kiuj difktos vian
kapablecon pli malfrue. Tie vi lernos liberan Transfiguron ekde la
\emph{vere komenco}, kiu bezonas ke vi ĵetas sorĉon senvorte, per teni
la subjekta formo, la cela fomo, kaj la transfiguriĝon en via menso.''

``Kaj por respondi al la demando de S-ro Potter,'' Profesorino
McGonagall daŭris, ``tio estas la \emph{libera} transfiguriĝo kiun vi
devas neniam fari sur vivanta subjekto. Estas Ĉarmoj kaj pocioj kiuj
povas sekure, returneble transformi vivantan subjetkon en
\emph{limigitajn} manierojn. Animagus kun manka membro ankoraŭ ne
havos tiun membron post esti transformita, ekzemple. Libera
Transfiguro \emph{ne} estas sekura. Via korpo ŝanĝos kiam ĝi estos
transfigurita—spiri, ekzemple, rezultas en konstanta perdo de
korpmaterio en la ĉirkaŭan aeron. Kiam la Transfiguro eluziĝas kaj
via korpo provas inversiĝi al sia origina formo, ĝi ne tute kapablos
fari tiel. Se vi premas vian bastonon sur via korpo kaj imagas vin kun
oraj haroj, poste viaj haroj falos. Se vi imagas vin kiel iu kun pli
klara haŭto, vi restos longa momento en S-ta Mungo. Kaj se vi
transfiguras vin en korpa formo adolta, kiam la Transfiguro eluziĝos,
vi mortos.''

Tio klarigis kial li vidis tiajn aferojn kiel dika knaboj, aŭ knabinoj mapli ol
perfekte bela. Aŭ maljuna homoj, ĝustadire. Tio ne okazus se ili
simple povus transfiguri ilin ĉiumatene\ldots Harry levis sian manon
kaj provis signi kun siajn okulojn al Profesorino McGonagall.

``*Jes*, S-ro Potter?''

``Ĉu tio eblas transfiguri vivantan subjekton en celon kiu estas
statika, kiel monero—ne, pardonu min, mi terure bedaŭras, nur diru ni
en ŝtalan pilkon.''

Profesorino McGonagall kapneis. ``S-ro Potter, eĉ senviva objekto
suferas etajn ŝanĝojn dum tempo. Ne estos videblaj ŝanĝoj sur via
korpo poste, kaj dum la unua minuto, vi rimarkos nenion malbonan. Sed
post horo vi malsanos, kaj post tago vi estos morto.''

``Erm, pardonu min, do se mi estus leginta la unua ĉapitro mi povus
\emph{diveni} ke la skribtablo estis origine skribtablo kaj ne
porko.'' Harry diris, ``sed nur se mi faras \emph{pluan} supozon ke vi
ne volas mortigi porkon, tiu povas \emph{ŝajni} grande probable
sed—''

``Mi povas antaŭvidi ke korekti viajn testojn estos senfina deveno de
ĝojo por mi, S-ro Potter. Sed se vi havas aliajn demandojn ĉu mi povas
bonvolu demandi ke vi atendas ĝis la fino de la klaso?''

``Neniu plua demando, profesorino.''

``Nu ripetu post mi,'' diris Profesorino McGonagall. ``Mi neniam
provos transfiguri iun ajn vivantan subjekton, speciale mi mem, krom
se mi estas instruita fari tiel uzante specialecan Ĉarmon aŭ pocion.''

``Se mi ne certas se Transfiguro estas sekura, mi ne provas ĝin antaŭ
kiam mi demandis al Profesorino McGonagall aŭ Profesoro Flirtiko aŭ
Profesoro Skoldo aŭ la Direktoro, kiuj estas la solaj aŭtoritatoj
agnoskitaj pri Transfiguro en Herpŭrko. Demandi al alia studento
\emph{ne} estas akceptebla, eĉ se ili diras ke ili memoras fari la
saman demandon.''

``Eĉ se la aktuala Profesoro de Defendo en Herpŭrko diras al mi ke
Transfiguro estas sekura, kaj eĉ se mi vidas la Profesoro de Defendo
fari ĝin kaj nenio malbona ŝajnas okazi, mi ne provos ĝin per mi
mem.''


``Mi havas la absolutan rajton de rifuzi plenumi Transfiguron pri kiu
mi sentas min la plej leĝere maltrankvila. Tial ke ne eĉ la Direktoro
de Herpŭrko povas ordoni al vi fari ĝin, mi certe ne akceptos ian ajn
ordon de la Defenca Profesoro, eĉ se la Defenca Profesoro minacas
depreni cent Domaj poentojn kaj igi mi forpelita.''

``Se mi rompas iun ajn el tiuj reguloj mi ne plu studos Transfiguron
dum mia tempo en Herpŭrko.''

``Ni ripetos tiuj reguloj je la komenco de ĉiu klaso dum la unua
monato,'' diris Profesorino McGonagall. ``Kaj nun, ni estas komanconta
kun alumeto kiel subjekto kaj kudrilo al celo\ldots formetu viajn
bastonojn, dankon, per 'komenci' mi volis diri ke vi komencos preni
notojn.''

Duonhoro antaŭ la fino de la klaso, Profesorino McGonagall eltiris la alumetojn.

Je la fino de la klaso Hermione havis arĝente aspektantan alumeton kaj
la tuta resto de la klaso, Mugle naskitaj aŭ aliaj, havis ekzakte
tiun, kun kiu ili komencis.

Profesorino McGonagall al ŝi premiis alian poento por Korvungo.

\begin{center}\rule{3in}{0.4pt}\end{center}

Post kiam la Transfigura klaso estis finita, Hermione venis al la
skribtablo de Harry dum Harry estis metanta siajn librojn en sian
haŭtpoŝon.

``Vi scias,'' Hermione diris kun senkulpa esprimo sur sia vizaĝo, ``Mi
gajnis du poentoj por Korvungo hodiaŭ.''

``Efektive,'' Harry diris post nelonge.

``Sed tio ne estis tiel bona ol viaj \emph{sep} poentoj,'' ŝi
diris. ``Mi supozas ke mi ne estas tiel inteligenta kiel vi.''

Harry finis meti siajn hejmtaskojn en sian poŝon kaj turniĝis al
Hermione kun siaj okuloj malfermetitaj. Li fakte forgesis pri tio.

Ŝi papelbrumis. ``Tamen, ni havas lecionojn ĉiutage. Mi al si demandas
kiel longe tio al vi prenos por trovi pliajn Huflopufojn por savi?
Hodiaŭ estas lundo. Do tio al vi donas ĝis ĵaŭdo.''

Ambaŭ de ili fikse rigardis la alian en la okuloj, nepapelbrumante.

Harry parolis unue. ``Evidente vi konscias ke tio volas diri militon.''

``Mi ne sciis ke ni estis en paco.''

Ĉiuj la aliaj studentoj estis nun rigardantaj kun fascinitaj
okuloj. Ĉiuj la aliaj studentoj, kaj, bedaŭrinde, Profesorino
McGonagall.

``Ho, S-ro Potter,'' kantis Profesorino McGonagall de la alia flanko
de la ĉambro, ``Mi havas iun bonan novinformon por vi. S-ino Pimfito
aprobis vian sugeston por preventi rompon de Turniĝaj pordetoj, kaj la
plano estas fini la laboro antaŭ la fino de la semajno. Mi diras ke
tio meritas\ldots diru ni, dek poentoj por Korvungo.''

La vizaĝo de Hermione gapis pro la perfido kaj ŝoko. Harry imagis ke
sia propra vizaĝo ne aspektis tro malsama.

``Profesorino\ldots'' Harry siblis.

``Tiuj dek poentoj estas \emph{nediskutable} merititaj, S-ro
Potter. Mi ne donus Doma poentoj pro kaprico. Por vi ĝi eble estis
simpla afero de vidi iun fragila kaj sugesti maniero de protekti ĝin,
sed Turniĝaj Pordetoj estas multekostaj, lak la Direktoro ne estis
feliĉa la lasta fojo kiam unu rompis.'' Profesorino McGonagall
aspektis pripenseme. ``Ve, mi al si demandas se iu ajn studento iam
ajn gajnis deksep Domajn poentojn dum lia unua tago de lernejo. Mi
devos rigardi, sed mi suspektas ke tio estas nova rekordo. Eble ni
devus havi anoncon je la horo de vespermanĝo?''

``PROFESORINO!'' Harry kriegis. ``Tiu estas nia milito! Haltu
enmiksiĝi!''

``Nun vi havas ĝis ĵaŭdo de la sekva semajno, S-ro Potter. Almenaŭ
evidente ke vi engaĝiĝos en iu speco de petolo kaj \emph{perdi}
Domajn poentojn antaŭ tiam. Paroli al profesorino malrespekte,
ekzemple.'' Profesorino McGonagall metis sian fingron sur sian vangon
kaj aspektis pensada. ``Mi anticipas ke vi atingos negativajn nombrojn
antaŭ la fino de vendredo.''

La buŝo de Harry ekmalfermis. Li sendis sian plej bona mortigan
rigardon al McGonagall sed ŝi ŝajnis nur trovi ĝin amuza.

``Jes, definitive anonco je la horo de vespermanĝo,'' Profesorino
McGonagall pensis. ``Sed ĝi povus ofendi la Serpentimojn, do la anonco
devus esti mallonga. Nur la nombroj kaj la fakto pri la rekordo\ldots
kaj se iu ajn venas al vi por helpo kun iliaj lernejaj taskoj kaj
seniluziiĝas pro ke vi ne eĉ komencis legi viajn librojn, vi povas
ĉiam turni ilin al S-ino Granger.''

``\emph{Profesorino!}'' diris Hermione en relative akuta voĉo.

Profesorino McGonagall ignoris ŝin. ``Ve, Mi al mi demandas kiom da
tempo tio prenos antaŭ kiam S-ino Granger faros ion meritante anoncon
je la horo de vespermanĝo? Mi antaŭĝojas pri vidi tion, kio ajn ĝi
estos.''

Harry kaj Hermione, per senparola reciproka konsento, turniĝis kaj
rapidiĝis el la klasĉambro. Ili estis sekvitaj per vico de
hipnotigitaj Korvungoj.

``Hum,'' Harry diris. ``Ĉu ni ankoraŭ konsentas por post la vespermanĝo?''

``Evidente,'' diris Hermione. ``Mi ne volas ke vi prenas ankoraŭ pli
da malfruo sur via studado.''

``Nu, dankon. Kaj lasu min diri ke tiel brila kiel vi jam estas, mi ne
povas malhelpi min scivoli kiel vi estos foje kiam vi havos iom
elementan trejnon en racieco.''

``Ĉu tio estas vere utila? Tio ne ŝajnis helpi vin kun Ĉarmo aŭ Transfiguro.''

Estis mallonga paŭzo.

``Nu, mi nur akiris miajn lerno-librojn kvar tagoj antaŭe. Tio estas
kial mi devis gajni tiujn deksep Domajn poentojn sen uzi mian
bastonon.''

``Kvar tago antaŭe? Eble ke vi ne povas legi ok librojn en kvar tagoj,
sed vi povintus almenaŭ legi \emph{unu}. Kiom da tagoj tio prenos por
fini laŭ tiu ritmo? Vi konas ĉiu tiu matematiko, do ĉu vi povas diri
al mi kio faras ok oble kvar, dividite per nul?''

``Mi havas klason nun, kio ne estis tiel por vi, sed semajnfinoj estas
liberaj, do\ldots limeso de ok oble kvar dividite per epsilono kieam
epsilono proksimiĝas nul plus\ldots 10:47 antaŭtagmeze je dimanĉo.''

``Mi faris tion en \emph{tri} tagoj fakte.''

``2:47 posttagmeze tial. Mi certas ke mi trovos la tempon ie.''

Kaj jen la vespero, kaj jen la mateno, la unua tago.





