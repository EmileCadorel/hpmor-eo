\chapter{Rimarki konfuzon}

\lettrine{L}{a} konsulthoroj de la oficejo de Profesoriĉo Ciuro estis inter
11:40 kaj 11:55 antaŭtagmeze je ĵaŭdo. Tio estis por ĉiuj la studentoj de ĉiuj
ĵaroj. Eĉ frapi la pordon kostis unu Ciuran poenton, kaj se li ne pensis ke la
kialo, kiu alportis vin, valoris lian tempon, vi perdus pliagn kvindek poentojn.

Harry frapis la pordon.

Estis paŭzo. Tiam mordanta voĉo diris, ``Mi supozas, ke vi povus ankaŭ
eniri, S-iĉo Potter.''

Kaj antaŭ ol Harry povis tuŝi la klinkon, la pordo klake malfermiĝis frapante la
muron kun akra krako, kiu sonis kvazaŭ io rompiĝis en la ligno, aŭ en la ŝtono,
aŭ eĉ en ambaŭ.

Profesoriĉo Ciuro estis apoganta sin malantaŭen en sia seĝo kaj leganta libron
de suspekta maljuna aspekto, binditan per malhele blua ledo kun arĝentaj runoj
sur la dorso. Liaj okuloj ne forlasis la paĝojn. ``Mi ne estas bonhumora, S-iĉo
Potter. Kaj kiam mi ne estas bonhumora, mi ne estas afabla persono. Por via
propra bonfarto, plenumu vian aferon rapide kaj foriru.''

Malvarma aerfluo trairis la ĉambron, kvazaŭ ĝi enhavus ion, kio elĵetis
malhelecon kiel lampoj elĵetas lumon, sen esti tute ombrata.

Harry estis iom surprizita. \emph{Ne bonhumora} ne ŝajnis eĉ komenci priskribi
tion. Kio povis ĝeni Profesoriĉon Ciuro tiel multe\ldots{}?

Nu, oni ne nur forlasas siajn amikojn, kiam ili sin sentas malbone. Harry
zorgeme antaŭeniris en la ĉambron. ``Ĉu estas io, kion mi povas fari por
helpi—''

``Ne,'' diris Profesoriĉo Ciuro, ankoraŭ ne rigardante super la libro.

``Mi volas diri, se vi rilatis kun stultuloj, kaj bezonas iun saĝan, kun kiu
paroli\ldots{}''

Estis nekutime longa paŭzo.

Profesoriĉo Ciuro klakfermis la libron, kaj ĝi malaperis kun eta sono de
murmuro. Li rigardis supren, kaj Harry eksaltis.

``Mi supozas, ke inteligenta konversacio estus agrabla por \emph{mi},'' diris
Profesoriĉo Ciuro per la sama mordanta tono, kiu invitis Harry'n eniri.
``\emph{Vi} verŝajne ne trovos tion same, estu avertita.''

Harry prenis longan enspiron. ``Mi promesas, ke mi ne ofendiĝos, se vi ekkoleros
kontraŭ mi. Kio okazis?''

La malvarma aerfluo en la ĉambro ŝajnis pliiĝi. ``Grifindorano de la sesa jaro
ĵetis malbenon al unu el miaj plej promesplenaj studentoj, Serpentimano de la
sesa jaro.''

Harry glutis. ``Kia\ldots{} malbeno?''

La furiozo sur la vizaĝo de Profesoriĉo Ciuro jam ne estis bridita. ``Kial vi
ĝenus vin demandi tiel negravan aferon, S-iĉo Potter? Nia Grifindorano de la
sesa jaro ne opiniis ke tio estis grava!''

``Ĉu vi \emph{seriozas?}'' diris Harry, antaŭ ol li povis haltigi sin.

``Ne, mi estas ege malbonhumora hodiaŭ pro neniu speciala kialo. \emph{Jes, mi
seriozas, vi stultulo!} Li ne sciis. Li \emph{fakte ne sciis.} Mi ne povis kredi
tion ĝis la aŭroroj konfirmis ĝin per Verec-serumo. Li estas en sia \emph{sesa
jaro en Herpŭrko} kaj li ĵetis altnivelan malhelan malbenon \emph{sen scii, kion
ĝi faros.}''

``Vi ne volas diri'', diris Harry, ``ke li eraris pri tio, kion ĝi faras -- ke
li iel legis malĝustan priskribon de la sorĉo—''

``Ĉio, kion li sciis, estis ke ĝi devis esti direktita al malamiko. Li
\emph{sciis} ke tio estis ĉio, kion li sciis.''

Kaj tio sufiĉis por ĵeti la sorĉon. ``Mi ne komprenas, kiel io kun tiel
malgranda cerbo, povas marŝi rekte.''

``Efektive, S-iĉo Potter,'' diris Profesoriĉo Ciuro.

Estis paŭzo. Profesoriĉo Ciuro kliniĝis antaŭen, prenis la arĝentan inkujon de
sia skribtablo, kaj turnis ĝin inter la manoj, rigardante ĝin kvazaŭ li demandus
al si, kiel oni povus torturi inkujon ĝis morto.

``Ĉu la Serpentimano de la sesa jaro estis grave vundita?'' diris Harry.

``Jes.''

``Ĉu la Grifindorano de la sesa jaro estis mugle-kreskigita ?''

``\emph{Jes.}''

``Ĉu Dumbledore rifuzas forpeli lin, ĉar la kompatinda knabo ne sciis?''

La manoj de Profesoriĉo Ciuro blankiĝis sur la inkujo. ``\emph{Ĉu vi volas diri
ion, S-iĉo Potter, aŭ ĉu vi nur diras la evidentecon?}''

``Profesoriĉo Ciuro'', diris Harry grave, ``ĉiuj la mugle-kreskigitaj studentoj
de Herpŭrko bezonas lecionojn pri sekureco -- tiajn, kie oni klarigas al ili
aferojn tiel ridinde evidentajn, ke neniu sorĉe-naskito iam pensus mencii ilin.
Ne ĵetu sorĉon se vi ne scias, kion ĝi faras. Se vi malkovras ion danĝeran, ne
diskonigu ĝin al mondo. Ne faru altnivelan pocion sen superrigardo en la
banĉambro. Jen la kialo de la leĝo pri neplenaĝa magio. Ĉiuj la bazoj.''

``Kial?'', diris Profesoriĉo Ciuro. ``Lasu la stultulojn morti antaŭ ol ili
brediĝas.''

``Se al vi ne ĝenas vin perdi kelkajn Serpentimanojn de la sesa jaro laŭvoje.''

La inkujo ekflamiĝis en la manoj de Profesoriĉo Ciuro, kaj brulis per terura
malrapideco, dum malbelega nigro-aranĝa flamoj ŝiris la metalon kaj ŝajnis
forpreni kelkajn pecojn de ĝi. La arĝento tordiĝis dum ĝi fandiĝis, kvazaŭ ĝi
penis -- sed vane -- eskapi. Oni aŭdis etajn muĝojn, kvazaŭ la metalo mem krius.

``Mi supozas, ke vi pravas'', diris Profesoriĉo Ciuro kun rezignita rideto. ``Mi
devus krei lecionon por certigi, ke mulge-naskitaj uloj, kiuj estas tro stultaj
por vivi, ne kunportu ion ajn valoran, kiam ili foriras.''

La inkujo daŭre kriis kaj brulis en la manoj de Profesoriĉo Ciuro, dum etaj
gutetoj da ankoraŭ flamanta metalo gutis sur la skribtablon, kvazaŭ ĝi mem
plorus.

``Vi ne provas forkuri,'' rimarkis Profesoriĉo Ciuro.

Harry malfermis la buŝon,—

``Se vi intencas diri, ke vi ne timas min,'' diris Profesoriĉo Ciuro, ``\emph{ne
diru tion.}''

``Vi estas la plej timiga persono, kiun mi konas,'' diris Harry, ``kaj unu el
kialoj por tio estas via memregado. Mi simple ne povas imagi vin doloriganta
iun, se vi ne estus konscie decidinta tion.''

La fajro en la manoj de Profesoriĉo Ciuro mortis, kaj li zorge metis la
ruiniĝintan inkujon sur sian skribtablon. ``Vi diras tiel afablajn aferojn,
S-iĉo Potter. Ĉu vi prenis lecionojn pri flatado? De, eble, S-iĉo Malfojo?''

Harry konservis senespriman vizaĝon, kaj nur unu sekundo tro malfrue li
ekkomprenis, ke tio povus aspekti kvazaŭ subskriba konfeso. Profesoriĉo Ciuro ne
zorgis pri la ekstera aspekto de via mieno, sed pri la mensstato, kiu faris ĝin
verŝajna.

``Mi vidas,'' diris Profesoriĉo Ciuro. ``S-iĉo Malfojo estas utila amiko, S-iĉo
Potter, kaj li povas instrui al vi multajn aferojn; sed mi esperas ke vi ne
faris la eraron fidi lin per tro multaj konfidencoj.''

``Li scias nenion, kion mi timus malkaŝi,'' diris Harry.

``Bone,'' diris Profesoriĉo Ciuro, ridetante iomete. ``Do kio estis la origina
afero, kiu kondukis vin ĉi tien?''

``Mi pensas ke mi finis, la antaŭajn ekzercojn de Oklumencio, kaj ke mi nun
estas preta por ricevi tutoron.''

Profesoriĉo Ciuro kapjesis. ``Mi devus konduki vin al Gringoto ĉi-sabatan.'' Li
paŭzis, rigardis Harry'n kaj ridetis. ``Kaj ni eble povos fari el tio etan
ekskurson, se vi volas. Mi ĵus havis bonan ideon.''

Harry kapjesis, kaj ridetis responde.

Dum Harry foriris el la oficejo, li aŭdis Profesoriĉon Ciuron zumi etan
melodion.

Harry estis feliĉa, ke li sukcesis rekuraĝigi lin.

\later

Tiu dimanĉo, ŝajnis esti relative granda nombro da homoj murmuregantaj en la
koridoroj, almenaŭ kiam Harry Potter pretermarŝis ilin.

Kaj multaj fingroj estis direktitaj al li.

Kaj grava kvanto da inaj subridoj.

Tio komencis je la matenmanĝo, kiam iu demandis al Harry, ĉu li aŭdis la
novaĵojn. Harry rapide interrompis lin kaj diris, ke se la novaĵoj estis
skribitaj per Rita Moskito, li ne volis \emph{aŭdi} ilin, li preferis legi ilin
mem en la ĵurnalo.

Tiam okazis, ke ne multaj studentoj en Herpŭrko havis kopion de la \emph{Ĉiutaga
Profeto,} kaj la kopioj, kiuj ne jam estis aĉetitaj de iliaj posedantaj, estis
pasigitaj laŭ iu komplika ordo, tiel ke neniu vere sciis, kiu havis kopion
tiutempe\ldots{}

Do Harry uzis la Silencian sorĉon kaj komencis matenmanĝi, fidante siajn sidajn
najbarojn por forpuŝi la multajn, multajn demandantojn. Li provis laŭeble plej
bone, ignori la malkonfidon, la ridojn, la gratulajn ridetojn, la kompatajn
aspektojn, la timigitajn rigardojn, kaj la faligitajn telerojn kiam novaj homoj
venis por matenmanĝi kaj aŭdis la novaĵon.

Harry sentis sin \emph{sufiĉe} scivola, sed li \emph{vere} ne volis ruinigi la
virtuozecon per aŭdi ĝin rakontita de iu alia.

Li faris siajn hejmtaskojn en la sekureco de sia trunko dum kelkaj horoj, post
kiam li diris al siaj samdormejanoj, ke ili informu lin, se iu ajn trovus
originalan ĵurnalon.

Harry estis ankoraŭ senscia je la 10a antaŭtagmeze, kiam li forlasis
Herpŭrko'n en kaleŝo kun Profesoriĉo Ciuro, kiu estis je la dekstra
antaŭo, kaj kiu aktuale falis en zombian modon. Harry estis sidanta
diagonale, tiel eble plej fore, kiel la kaleŝo permesis, je la
maldekstra malantaŭo. Eĉ tiel, Harry havis la konstantan senton de
pereo dum la kaleŝo tremis sur la eta vojo trans parto de
ne-malpermesita arbaro. Tio farigis la legadon malfacila, speciale ĉar
la enhavo estis malfacila, kaj Harry subite deziris ke li estu leganta
unu el siaj infanaj scienfikciaj libroj anstataŭe—

``Ni estas ekster la kvartalo—S-iĉo Potter,'' diris la voĉo de
Profesoriĉo Ciuro de la antaŭo. ``Estas tempo por foriri.''

Profesoriĉo Ciuro zorge elkaleŝiĝis, tenante sin dum la malsupreniris. Harry, je sia flanko, saltis el la kaleŝo.

Harry demandis al si ekzakte, kiel ili troviĝis en tiu loko, kiam
Profesoriĉo Ciuro diris ``Kaptu!'' kaj ĵetis bronzan Knuton al li, kaj
Harry kaptis ĝin sen pripensi.

Giganta nemateria kroĉo kaptis la abdomenon de Harry kaj ektiris lin
malantaŭen, forte, sed sen iu ajn senso de akcelo, kaj la sekva
momento Harry staris en la mezo de la Diagon Aleo.

(\emph{Pardonu min, sed kio?} diris lia cerbo.)

(\emph{Ni ĵus teleportis,} klarigis Harry.)

(\emph{Tio ne kutimis okazi en la praula medio,} la cerbo de Harry plendis, kaj konfuzigis lin.)

Harry staggered as his feet adjusted to the brick of the street instead of the dirt of the forest corridor they had been traversing. He straightened, still dizzy, with the bustling witches and wizards seeming to sway slightly, and the cries of the shopkeepers seeming to move around in his hearing, as his brain tried to place a world to be located in.

Moments later, there was a sort of sucking-popping sound from a few paces behind Harry, and when Harry turned to look Professor Quirrell was there.

“Do you mind—” said Harry, at the same time as Professor Quirrell said, “I’m afraid I—”

Harry stopped, Professor Quirrell didn’t.

“—need to go off and set something in motion, Mr~Potter. As it has been thoroughly explained to me that I am responsible for anything whatsoever that happens to you, I’ll be leaving you with—”

“News-stand,” Harry said.

“Pardon?”

“Or anywhere I can buy a copy of the \emph{Daily Prophet.} Put me there and I’ll be happy.”

Shortly after, Harry had been delivered into a bookshop, accompanied by several quietly spoken, ambiguous threats. And the shopkeeper had received \emph{less} ambiguous threats, judging by the way he had cringed, and how his eyes now kept darting between Harry and the entrance.

If the bookshop burned down, Harry was going to stick around in the middle of the fire until Professor Quirrell got back.

Meanwhile—

Harry took a quick glance around.

The bookshop seemed rather small and shoddy, with only four rows of bookcases visible, and the nearest shelf Harry’s eyes had jumped to seemed to deal with narrow, cheaply bound books with grim titles like \emph{The Massacre of Albania in the Fifteenth Century.}

First things first. Harry stepped over to the seller’s counter.

“Pardon me,” said Harry, “One copy of the \emph{Daily Prophet,} please.”

“Five Sickles,” said the shopkeeper. “Sorry, kid, I’ve only got three left.”

Five Sickles dropped onto the counter. Harry had the feeling he could have bargained him down a couple of points, but at this point he didn’t really care.

The shopkeeper’s eyes widened and he seemed to really notice Harry for the first time. “\emph{You!}”

“\emph{Me!}”

“Is it \emph{true?} Are you \emph{really}—”

“\emph{Shut up!} Sorry, I’ve been waiting \emph{all day} to read this in the original newspaper instead of hearing about it second hand, so please just \emph{hand it over}, all right?”

The shopkeeper stared at Harry for a moment, then wordlessly reached under the counter and passed over one folded copy of the \emph{Daily Prophet}.

The headline read:

\headline{Harry Potter\\
Secretly Betrothed\\
To Ginevra Weasley}

Harry stared.

He lifted the newspaper off the counter, softly, reverently, like he was handling an original Escher artwork, and unbent it to read…

…about the evidence that had convinced Rita Skeeter.

…and some interesting further details.

…and yet more evidence.

Fred and George had cleared it with their sister first, surely? Yes, of course they had. There was a picture of Ginevra Weasley sighing longingly over what Harry could see, looking closely, was a photo of himself. That had to have been staged.

But \emph{how} on \emph{Earth…?}

Harry was sitting in a cheap folding chair, rereading the newspaper for the fourth time, when the door whispered softly and Professor Quirrell came back into the shop.

“My apologies for—\emph{what} in Merlin’s name are you reading?”

“It would seem,” said Harry, awe in his voice, “that one Mr~Arthur Weasley was placed under the Imperius Curse by a Death Eater whom my father killed, thus creating a debt to House Potter, which my father demanded be repaid by the hand in marriage of the recently born Ginevra Weasley. Do people actually do that sort of thing around here?”

“How could Miss~Skeeter \emph{possibly} be fool enough to believe—”

And Professor Quirrell’s voice cut off.

Harry had been reading the newspaper held vertically and unfolded, which meant that Professor Quirrell, from where he was standing, could see the text underneath the headline.

The look of shock on Professor Quirrell’s face was a work of art almost on par with the newspaper itself.

“Don’t worry,” said Harry cheerfully, “it’s all fake.”

From elsewhere in the store, he heard the shopkeeper gasp. There was the sound of a stack of books falling over.

“Mr~Potter…” Professor Quirrell said slowly, “are you \emph{sure} of that?”

“Quite sure. Shall we go?”

Professor Quirrell nodded, looking rather abstracted, and Harry folded the newspaper back up, and followed him out of the door.

For some reason Harry didn’t seem to be hearing any street noises now.

They walked in silence for thirty seconds before Professor Quirrell spoke. “Miss~Skeeter viewed the original proceedings of the restricted Wizengamot session.”

“Yes.”

“The \emph{original proceedings of the Wizengamot.}”

“Yes.”

“\emph{I} would have trouble doing that.”

“Really?” said Harry. “Because if my suspicions are correct, this was done by a Hogwarts student.”

“That is beyond impossible,” Professor Quirrell said flatly.
“Mr~Potter…I regret to say that this young lady expects to marry you.”

“But \emph{that} is improbable,” said Harry. “To quote Douglas Adams, the impossible often has a kind of integrity which the merely improbable lacks.”

“I see your point,” Professor Quirrell said slowly. “But…no, Mr~Potter. It may be impossible, but I can \emph{imagine} tampering with the Wizengamot proceedings. It is \emph{unimaginable} that the Grand Manager of Gringotts should affix the seal of his office in witness to a false betrothal contract, and Miss~Skeeter personally verified that seal.”

“Indeed,” said Harry, “you would expect the Grand Manager of Gringotts to get involved with that much money changing hands. It seems Mr~Weasley was greatly in debt, and so demanded an additional payment of ten thousand Galleons—”

“\emph{Ten thousand} Galleons for a \emph{Weasley?} You could buy the daughter of a Noble House for that!”

“Excuse me,” Harry said. “I really have to ask at this point, do people actually do that sort of thing around here—”

“Rarely,” said Professor Quirrell, with a frown on his face. “And not at all, I suspect, since the Dark Lord departed. I suppose that according to the newspaper, your father just paid it?”

“He didn’t have any choice,” said Harry. “Not if he wanted to fulfil the conditions of the prophecy.”

“\emph{Give me that,}” said Professor Quirrell, and the newspaper leaped out of Harry’s hand so fast that he got a paper cut.

Harry automatically put the finger in his mouth to suck on, feeling rather shocked, and turned to remonstrate with Professor Quirrell—

Professor Quirrell had stopped short in the middle of the street, and his eyes were flickering rapidly back and forth as an invisible force held the newspaper suspended before him.

Harry watched, gaping in open awe, as the newspaper opened to reveal pages two and three. And not much long after, four and five. It was like the man had cast off a pretence of mortality.

And after a troublingly short time, the paper neatly folded itself up again. Professor Quirrell plucked it from the air and tossed it to Harry, who caught it in sheer reflex; and then Professor Quirrell started walking again, and Harry automatically trudged after.

“No,” said Professor Quirrell, “that prophecy didn’t sound quite right to me either.”

Harry nodded, still stunned.

“The centaurs could have been put under an \emph{Imperius},” Professor Quirrell said, frowning, “\emph{that} seems understandable. What magic can make, magic can corrupt, and it is not unthinkable that the Great Seal of Gringotts could be twisted to another’s hand. The Unspeakable could have been impersonated with Polyjuice, likewise the Bavarian seer. And with \emph{enough} effort it might be possible to tamper with the proceedings of the Wizengamot. Do you have any idea how that was done?”

“I do not have one single plausible hypothesis,” said Harry. “I do know it was done on a total budget of forty Galleons.”

Professor Quirrell stopped short and whirled on Harry. His expression was now completely incredulous. “Forty Galleons will pay a competent ward-breaker to open a path into a home you wish to burgle! Forty \emph{thousand} Galleons \emph{might} pay a team of the greatest professional criminals in the world to tamper with the proceedings of the Wizengamot!”

Harry shrugged helplessly. “I’ll remember that the next time I want to save thirty-nine thousand, nine hundred and sixty Galleons by finding the right contractor.”

“I do not say this often,” said Professor Quirrell. “I am impressed.”

“Likewise,” said Harry.

“And who is this incredible Hogwarts student?”

“I’m afraid I couldn’t say.”

Somewhat to Harry’s surprise, Professor Quirrell made no objection to this.

They walked in the direction of the Gringotts building, thinking, for they were neither of them the sort of person who would give up on the problem without considering it for at least five minutes.

“I have a feeling,” Harry said finally, “that we’re coming at this from the wrong angle. There’s a tale I once heard about some students who came into a physics class, and the teacher showed them a large metal plate near a fire. She ordered them to feel the metal plate, and they felt that the metal nearer the fire was cooler, and the metal further away was warmer. And she said, write down your guess for why this happens. So some students wrote down ‘because of how the metal conducts heat’, and some students wrote down ‘because of how the air moves’, and no one said ‘this just seems impossible’, and the real answer was that before the students came into the room, the teacher turned the plate around.”

“Interesting,” said Professor Quirrell. “That does sound similar. Is there a moral?”

“That your strength as a rationalist is your ability to be more confused by fiction than by reality,” said Harry. “If you’re equally good at explaining any outcome, you have zero knowledge. The students thought they could use words like ‘because of heat conduction’ to explain anything, even a metal plate being cooler on the side nearer the fire. So they didn’t notice how confused they were, and that meant they couldn’t be more confused by falsehood than by truth. If you tell me that the centaurs were under the \emph{Imperius} Curse, I still have the feeling of something being not quite right. I notice that I’m still confused even after hearing your explanation.”

“Hm,” said Professor Quirrell.

They walked on further.

“I don’t suppose,” said Harry, “that it’s possible to \emph{actually} swap people into alternate universes? Like, this isn’t our own Rita Skeeter, or they temporarily sent her somewhere else?”

“If \emph{that} was possible,” Professor Quirrell said, his voice rather dry, “would I still be \emph{here?}”

And just as they were almost to the huge white front of the Gringotts building, Professor Quirrell said:

“Ah. Of \emph{course.} I see it now. Let me guess, the Weasley twins?”

“\emph{What?}” said Harry, his voice rising an octave.
“\emph{How?}”

“I’m afraid I couldn’t say.”

“…That is \emph{not} fair.”

“I think it is extremely fair,” said Professor Quirrell, and they entered through the bronze doors.

\later

The time was just before noon, and Harry and Professor Quirrell were seated at the foot and head of a wide, long, flat table, in a sumptuously appointed private room with thoroughly cushioned couches and chairs along the walls, and soft curtains hanging everywhere.

They were about to eat lunch in Mary’s Place, which Professor Quirrell had said was known to him as one of the best restaurants in Diagon Alley, especially for—his voice had dropped meaningfully—\emph{certain purposes.}

It was the nicest restaurant that Harry had ever been in, and it was really eating away at Harry that Professor Quirrell was treating \emph{him} to the meal.

The first part of the mission, to find an Occlumency instructor, had been a success. Professor Quirrell, smiling evilly, had told Griphook to recommend the best he knew, and not worry about the expense, since Dumbledore was paying it; and the goblin had smiled in return. There might have been a certain amount of smiling on Harry’s part as well.

The second part of the plan had been a complete failure.

Harry was not allowed to take money out of his vault without Headmaster Dumbledore or some other school official present, and Professor Quirrell had not been given the vault key. Harry’s Muggle parents could not authorize it because they were Muggles, and Muggles had around the same legal standing as children or kittens: they were cute, so if you tortured them in public you could get arrested, but they weren’t \emph{people}. Some reluctant provision had been made for recognizing the parents of Muggleborns as human in a limited sense, but Harry’s adoptive parents did not fall into that legal category.

It seemed that Harry was effectively an orphan in the eyes of the wizarding world. As such, the Headmaster of Hogwarts, or his deputies \emph{within} the school system, were Harry’s guardians until he graduated. Harry \emph{could} breathe without Dumbledore’s permission, but only so long as the Headmaster did not specifically prohibit it.

Harry had then asked if he could simply \emph{tell} Griphook how to diversify his investments beyond stacks of gold coins sitting in his vault.

Griphook had stared blankly and asked what ‘diversify’ meant.

Banks, it seemed, did not make investments. Banks stored your gold coins in secure vaults for an annual fee.

The wizarding world did not have a concept of stock. Or equity. Or corporations. Businesses were run by families out of their personal vaults.

Loans were made by rich people, not banks. Though Gringotts would witness the contract, for a fee, and enforce its collection, for a much larger fee.

Good rich people let their friends borrow money and pay it back whenever. \emph{Bad} rich people charged you \emph{interest.}

There was no secondary market in loans.

Evil rich people charged you annual interest rates of at least 20\%.

Harry had stood up, turned away, and rested his head against the wall.

Harry had asked if he needed the Headmaster’s permission before he could start a bank.

Professor Quirrell had interrupted at this point, saying that it was time for lunch, and swiftly conducted a fuming Harry out of the bronze doors of Gringotts, through Diagon Alley, and to a fine restaurant called Mary’s Place, where a room had been reserved for them. The owner had looked shocked at seeing Professor Quirrell accompanied by Harry Potter, but had conducted them to the room without complaint.

And Professor Quirrell had quite deliberately announced that he would pay the bill, seeming to rather enjoy the look on Harry’s face.

“No,” said Professor Quirrell to the waitress, “we will not require menus. I will have the daily special accompanied by a bottle of Chianti, and Mr~Potter will have the Diracawl soup to start, followed by a plate of Roopo balls, and treacle pudding for dessert.”

The waitress, clad in robes that still looked severe and formal while being rather shorter than usual, bowed respectfully and departed, shutting the door behind her.

Professor Quirrell waved a hand in the direction of the door, and a bolt slid shut. “Note the bolt on the inside. This room, Mr~Potter, is known as Mary’s Room. It happens to be proof against all scrying, and I do mean \emph{all;} Dumbledore himself could detect nothing of what happens here. Mary’s Room is used by two kinds of people. The first sort are engaged in illicit dalliances. And the second sort lead interesting lives.”

“\emph{Really,}” said Harry.

Professor Quirrell nodded.

Harry’s lips were parted in anticipation. “It would be a waste to just sit here and eat lunch, then, without doing anything special.”

Professor Quirrell grinned, then took out his wand and flicked it in the direction of the door. “Of course,” he said, “people who lead interesting lives take precautions more \emph{thorough} than the dalliers. I have just sealed us in. Nothing will now pass in or out of this room—through the crack under the door, for example. And…”

Professor Quirrell then spoke no fewer than four different Charms, none of which Harry recognized.

“Even that does not \emph{really} suffice,” said Professor Quirrell. “If we were doing anything of truly great import, it would be necessary to perform another twenty-three checks besides those. If, say, Rita Skeeter knew or guessed that we would come here, it is possible that she could be in this room wearing the true Cloak of Invisibility. Or she could be an Animagus with a tiny form, perhaps. There are tests to rule out such rare possibilities, but to perform all of them would be arduous. Still, I wonder if I should do them anyway, just so as not to teach you bad habits?” And Professor Quirrell tapped a finger on his cheek, looking abstracted.

“It’s fine,” Harry said, “I understand, and I’ll remember.” Though he was a little disappointed that they weren’t doing anything of truly great import.

“Very well,” Professor Quirrell said. He leaned back in his chair, smiling broadly. “You wrought quite well today, Mr~Potter. The basic notion was yours, I’m sure, even if you delegated the execution. I don’t think we’ll be hearing much more from Rita Skeeter after this. Lucius Malfoy will not be pleased with her failure. If she’s smart, she’ll flee the country the instant she realizes she’s been fooled.”

A sinking sensation began to dawn in Harry’s stomach. “Lucius was behind Rita Skeeter…?”

“Oh, you didn’t realize that?” said Professor Quirrell.

Harry hadn’t thought about what would happen to Rita Skeeter afterwards.

At all.

Not in the slightest.

But she would get fired from her job, \emph{of course} she would be fired, she might have children going through Hogwarts for all Harry knew, and now it was worse, much worse—

“Is Lucius going to have her killed?” Harry said in a barely audible voice. Somewhere in his head, the Sorting Hat was screaming at him.

Professor Quirrell smiled dryly. “If you have not dealt with journalists before, take it from me that the world gets a little brighter every time one dies.”

Harry jumped out of his chair with a convulsive movement, he had to find Rita Skeeter and warn her before it was too late—

“\emph{Sit down},” Professor Quirrell said sharply. “\emph{No}, Lucius won’t kill her. But Lucius makes life \emph{extremely} unpleasant for those who serve him ill. Miss~Skeeter will flee and start her life over with a new name. \emph{Sit down,} Mr~Potter; there is nothing you can do at this point, and you have a lesson to learn.”

Harry sat down, slowly. There was a disappointed, annoyed look on Professor Quirrell’s face that was doing more to stop him than the words.

“There are times,” Professor Quirrell said, his voice cutting, “when I worry that your brilliant Slytherin mind is simply wasted on you. Repeat after me. Rita Skeeter was a vile, disgusting woman.”

“Rita Skeeter was a vile, disgusting woman,” Harry said. He wasn’t comfortable saying it, but there didn’t seem to be any other possible actions, none at all.

“Rita Skeeter tried to destroy my reputation, but I executed an ingenious plan and destroyed \emph{her} reputation first.”

“Rita Skeeter challenged me. She lost the game, and I won.”

“Rita Skeeter was an obstacle to my future plans. I had no choice but to deal with her if I wanted those plans to succeed.”

“Rita Skeeter was my enemy.”

“I cannot possibly get anything done in life if I am not willing to defeat my enemies.”

“I have defeated one of my enemies today.”

“I am a good boy.”

“I deserve a special reward.”

“Ah,” said Professor Quirrell, who had been grinning a benevolent smile for the last few lines, “I see I have succeeded in catching your attention.”

That was true. And while Harry felt like he was being railroaded into something—no, that wasn’t just a feeling, he \emph{had} been railroaded—he couldn’t deny that saying those things, and seeing Professor Quirrell’s smile, \emph{did} make him feel better.

Professor Quirrell reached into his robes, the gesture slow and deliberately significant, and drew forth…

…a \emph{book}.

It was different from any book Harry had ever seen, the edges and corners visibly misshapen; \emph{rough-hewn} was the phrase that came to mind, like it had been hacked out of a book mine.

“What is it?” breathed Harry.

“A diary,” said Professor Quirrell.

“Whose?”

“That of a famous person.” Professor Quirrell was smiling broadly.

“Okay…”

Professor Quirrell’s expression became more serious. “Mr~Potter, one of the requisites for becoming a powerful wizard is an excellent memory. The key to a puzzle is often something you read twenty years ago in an old scroll, or a peculiar ring you saw on the finger of a man you met only once. I mention this to explain how I managed to remember this item, and the label attached to it, after meeting you a good deal later. You see, Mr~Potter, over the course of my life, I have viewed a number of private collections held by individuals who are, perhaps, not quite deserving of all that they have—”

“You \emph{stole} it?” Harry said incredulously.

“That is correct,” said Professor Quirrell. “Very recently, in fact. I think you will appreciate this particular item much more than the vile little man who held it for no other purpose than impressing his equally vile friends with its rarity.”

Harry was simply gaping now.

“But if you feel that my actions were incorrect, Mr~Potter, I suppose you needn’t accept your special present. Though of course I shan’t go to the trouble of putting it \emph{back}. So which is it to be?”

Professor Quirrell tossed the book from one hand to another, causing Harry to reach out involuntarily with a look of dismay.

“Oh,” said Professor Quirrell, “don’t worry about a little rough handling. You could toss this diary in a fireplace and it would emerge unscathed. In any case, I await your decision.”

Professor Quirrell casually threw the book up into the air and caught it again, grinning.

\emph{No,} said Gryffindor and Hufflepuff.

\emph{Yes}, said Ravenclaw. \emph{What part of the word ‘book’ did you two not understand?}

\emph{The theft part,} said Hufflepuff.

\emph{Oh, come on,} said Ravenclaw, \emph{you can’t seriously ask us to say no and spend the rest of our life wondering what it was.}

\emph{It sounds like a net positive from a utilitarian standpoint,} said Slytherin. \emph{Think of it as an economic transaction which generates gains from trade, only without the trade part. Plus, \emph{we} didn’t steal it and it won’t help anyone to have Professor Quirrell keep it.}

\emph{He’s trying to turn you Dark!} shrieked Gryffindor, and Hufflepuff nodded firmly.

\emph{Don’t be a naive little boy,} said Slytherin, \emph{he’s trying to teach you Slytherin.}

\emph{Yeah,} said Ravenclaw. \emph{Whoever owned the book originally was probably a Death Eater or something. It belongs with us.}

Harry’s mouth opened, then halted that way, an agonized look on his face.

Professor Quirrell seemed to be quite enjoying himself. He had balanced the book on its corner, on one finger, and was keeping it upright while humming a little tune.

There came a knock at the door.

The book vanished back into Professor Quirrell’s robes, and he rose up from his chair. Professor Quirrell started to walk over to the door—

—and staggered, suddenly lurching into the wall.

“It’s all right,” said Professor Quirrell’s voice, which suddenly sounded a lot weaker than usual. “Sit down, Mr~Potter, it’s just a dizzy spell. Sit down.”

Harry’s fingers gripped the edge of his chair, uncertain as to what he should do, what he \emph{could} do. Harry couldn’t even get too close to Professor Quirrell, not unless he wanted to defy that sense of Doom—

Professor Quirrell straightened, then, his breathing seeming a bit heavy, and opened the door.

The waitress came in, bearing a platter of food; and as she distributed the plates, Professor Quirrell walked slowly back to the table.

But by the time the waitress had bowed her way out, Professor Quirrell was sitting upright and smiling again.

Still, the brief episode of whatever-it-was had decided Harry. He couldn’t say no, not after Professor Quirrell had gone to that much trouble.

“Yes,” Harry said.

Professor Quirrell held up a cautioning finger, then took out his wand again, locked the door again, and repeated three of the same Charms from earlier.

Then Professor Quirrell took the book back out of his robes and tossed it to Harry, who almost dropped it into his soup.

Harry shot Professor Quirrell a look of helpless indignation. You weren’t supposed to \emph{do} that with books, enchanted or not.

Harry opened the book with ingrained, instinctive care. The pages seemed too thick, with a texture unlike either Muggle paper or wizarding parchment. And the contents were…

…blank?

“Am I supposed to be seeing—”

“Look nearer the beginning,” said Professor Quirrell, and Harry (again with that helpless, ingrained care) turned a block of pages back.

The lettering was obviously handwritten, and very hard to read, but Harry thought the words might be Latin.

“What \emph{is} this?” said Harry.

“That,” said Professor Quirrell, “is a record of the magical researches of a Muggleborn who never came to Hogwarts. He refused his letter, and conducted his own small investigations, which never did get very far without a wand. From the description on the placard, I expect that his name bears rather more significance to you than to me. That, Harry Potter, is the diary of Roger Bacon.”

Harry almost fainted.

Nestled up against the wall, where Professor Quirrell had stumbled, glistened the crushed remains of a beautiful blue beetle.

%  LocalWords:  rofessor Roopo
