% This file includes all the generic formatting for HPatMoR. This mostly entails configuring
% the memoir package, though “configuring” on occasion means “completely messing it up”.

\RequirePackage{fmtcount}
\RequirePackage{calc}

% Fonts used generally (specific fonts used only once or twice are not here).
\usepackage{xltxtra}
\defaultfontfeatures{Ligatures={TeX}}
\setmainfont[
  Extension=.otf
, UprightFont=*-Regular
, ItalicFont=*-Italic
, BoldFont=*-Bold
, BoldItalicFont=*-BoldItalic
, SmallCapsFont=AlegreyaSC-Regular
]{Alegreya}

\newfontface\hpfont[ExternalLocation]{Lumos}
\newfontface\ptsansi[ExternalLocation]{AlegreyaSans-Italic}

% Drop-caps at start of chapters
\renewcommand{\LettrineFontHook}{\hpfont}
\renewcommand{\LettrineTextFont}{}
\renewcommand{\DefaultLoversize}{.2}
\renewcommand{\DefaultLraise}{0.1}


\newcommand{\numberstringnumesp}[1]{
\ifthenelse{\equal{#1}{1}} {unu}{\ifthenelse{\equal{#1}{2}} {du}{\ifthenelse{\equal{#1}{3}} {tri}{\ifthenelse{\equal{#1}{4}} {kvar}{\ifthenelse{\equal{#1}{5}} {kvin}{\ifthenelse{\equal{#1}{6}} {ses}{\ifthenelse{\equal{#1}{7}} {sep}{\ifthenelse{\equal{#1}{8}} {ok}{\ifthenelse{\equal{#1}{9}} {naŭ}{\ifthenelse{\equal{#1}{10}} {dek}{\ifthenelse{\equal{#1}{11}} {dek unu}{\ifthenelse{\equal{#1}{12}} {dek du}{\ifthenelse{\equal{#1}{13}} {dek tri}{\ifthenelse{\equal{#1}{14}} {dek kvar}{\ifthenelse{\equal{#1}{15}} {dek kvin}{\ifthenelse{\equal{#1}{16}} {dek ses}{\ifthenelse{\equal{#1}{17}} {dek sep}{\ifthenelse{\equal{#1}{18}} {dek ok}{\ifthenelse{\equal{#1}{19}} {dek naŭ}{\ifthenelse{\equal{#1}{20}} {dudek}{\ifthenelse{\equal{#1}{21}} {dudek unu}{\ifthenelse{\equal{#1}{22}} {dudek du}{}}}}}}}}}}}}}}}}}}}}}}}

\newcommand{\numberstringnumespup}[1]{
\ifthenelse{\equal{#1}{1}} {UNU}{\ifthenelse{\equal{#1}{2}} {DU}{\ifthenelse{\equal{#1}{3}} {TRI}{\ifthenelse{\equal{#1}{4}} {KVAR}{\ifthenelse{\equal{#1}{5}} {KVIN}{\ifthenelse{\equal{#1}{6}} {SES}{\ifthenelse{\equal{#1}{7}} {SEP}{\ifthenelse{\equal{#1}{8}} {OK}{\ifthenelse{\equal{#1}{9}} {NAŬ}{\ifthenelse{\equal{#1}{10}} {DEK}{\ifthenelse{\equal{#1}{11}} {DEK UNU}{\ifthenelse{\equal{#1}{12}} {DEK DU}{\ifthenelse{\equal{#1}{13}} {DEK TRI}{\ifthenelse{\equal{#1}{14}} {DEK KVAR}{\ifthenelse{\equal{#1}{15}} {DEK KVIN}{\ifthenelse{\equal{#1}{16}} {DEK SES}{\ifthenelse{\equal{#1}{17}} {DEK SEP}{\ifthenelse{\equal{#1}{18}} {DEK OK}{\ifthenelse{\equal{#1}{19}} {DEK NAŬ}{\ifthenelse{\equal{#1}{20}} {DUDEK}{\ifthenelse{\equal{#1}{21}} {DUDEK UNU}{\ifthenelse{\equal{#1}{22}} {DUDEK DU}{}}}}}}}}}}}}}}}}}}}}}}}


% Extend lettrine cutout when more than one paragraph goes alongside the drop-cap
% Copied, with different macro names, from
% https://tex.stackexchange.com/questions/369868/using-lettrine-with-short-paragraphs
\newcount\hplettrinecount
\makeatletter
\def\hplettrineextrapara{%
\ifnum\prevgraf<\c@L@lines
\hplettrinecount\z@
\loop
\ifnum\hplettrinecount<\prevgraf
\advance\hplettrinecount\@ne
\afterassignment\hplettrinedima\count@\L@parshape\relax
\repeat
\parshape\L@parshape
\fi}
\def\hplettrinedima{\afterassignment\hplettrinedimb\dimen@}
\def\hplettrinedimb{\afterassignment\hplettrinedef\dimen@}
\def\hplettrinedef#1\relax{\edef\L@parshape{\the\numexpr\count@-1\relax\space #1}}
\makeatother
\newcommand{\lettrinepara}[3][]{\lettrine[nindent=0pt,#1]{#2}{#3}}

% Allow linebreaks after em-dash and hyphens, except when they’re followed by punctuation
\newXeTeXintercharclass \punctuationClass

\XeTeXcharclass `\’ \punctuationClass
\XeTeXcharclass `\‘ \punctuationClass
\XeTeXcharclass `\“ \punctuationClass
\XeTeXcharclass `\” \punctuationClass
\XeTeXcharclass `\. \punctuationClass
\XeTeXcharclass `\, \punctuationClass
\XeTeXcharclass `\: \punctuationClass
\XeTeXcharclass `\? \punctuationClass
\XeTeXcharclass `\! \punctuationClass
\XeTeXcharclass `\: \punctuationClass

\newXeTeXintercharclass \digitClass
\XeTeXcharclass `\0 \digitClass
\XeTeXcharclass `\1 \digitClass
\XeTeXcharclass `\2 \digitClass
\XeTeXcharclass `\3 \digitClass
\XeTeXcharclass `\4 \digitClass
\XeTeXcharclass `\5 \digitClass
\XeTeXcharclass `\6 \digitClass
\XeTeXcharclass `\7 \digitClass
\XeTeXcharclass `\8 \digitClass
\XeTeXcharclass `\9 \digitClass

\newXeTeXintercharclass \dashClass
\XeTeXcharclass `\— \dashClass % em
\XeTeXcharclass `\– \dashClass % en
\XeTeXcharclass `\- \dashClass % hyphen
\XeTeXcharclass `\… \dashClass

\XeTeXinterchartokenstate = 1

\def\morhyphenpenalty{75}
\exhyphenpenalty=10000

\XeTeXinterchartoks \dashClass 0 = {\hskip 0pt\penalty \morhyphenpenalty}

% Adjust space around lists
\setlength{\topsep}{.5\baselineskip plus 1\baselineskip minus .5\baselineskip}
\setlength{\partopsep}{.5\baselineskip plus 1\baselineskip minus .5\baselineskip}

% Miscellaneous global typesetting parameters
\frenchspacing
\setlength{\emergencystretch}{.06\textwidth}

% Try to avoid excessive hyphens
\doublehyphendemerits=30000
\finalhyphendemerits=30000
\adjdemerits=10000
\brokenpenalty10000\relax

% Make it easier to manage hyphenation
\makeatletter
\newcommand{\emdashhyp}{\leavevmode%
\prw@zbreak—\discretionary{}{}{}\prw@zbreak}
\makeatother

% Avoid widows and orphans
% https://mailman.ntg.nl/pipermail/ntg-context/2013/074250.html
\widowpenalty 10000
\clubpenalty 10000

% Various packages used
\usepackage[normalem]{ulem}
\usepackage{xfrac}
\usepackage{censor}
\usepackage[useregional]{datetime2}
\usepackage[nopagecolor=white,pagecolor=white]{pagecolor}
\usepackage{afterpage}
\usepackage{gitinfo2}
\usepackage{hyphenat}

